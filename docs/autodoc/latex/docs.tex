% Generated by https://pasdoc.github.io/PasDoc 0.16.0
\documentclass{report}
\usepackage{hyperref}
% WARNING: THIS SHOULD BE MODIFIED DEPENDING ON THE LETTER/A4 SIZE
\oddsidemargin 0cm
\evensidemargin 0cm
\marginparsep 0cm
\marginparwidth 0cm
\parindent 0cm
\setlength{\textwidth}{\paperwidth}
\addtolength{\textwidth}{-2in}


% Conditional define to determine if pdf output is used
\newif\ifpdf
\ifx\pdfoutput\undefined
\pdffalse
\else
\pdfoutput=1
\pdftrue
\fi

\ifpdf
  \usepackage[pdftex]{graphicx}
\else
  \usepackage[dvips]{graphicx}
\fi

% Write Document information for pdflatex/pdftex
\ifpdf
\pdfinfo{
 /Author     (Pasdoc)
 /Title      (PasDoc's autodoc)
}
\fi


% definitons for warning and note tag
\usepackage[most]{tcolorbox}
\newtcolorbox{tcbwarning}{
 breakable,
 enhanced jigsaw,
 top=0pt,
 bottom=0pt,
 titlerule=0pt,
 bottomtitle=0pt,
 rightrule=0pt,
 toprule=0pt,
 bottomrule=0pt,
 colback=white,
 arc=0pt,
 outer arc=0pt,
 title style={white},
 fonttitle=\color{black}\bfseries,
 left=8pt,
 colframe=red,
 title={Warning:},
}
\newtcolorbox{tcbnote}{
 breakable,
 enhanced jigsaw,
 top=0pt,
 bottom=0pt,
 titlerule=0pt,
 bottomtitle=0pt,
 rightrule=0pt,
 toprule=0pt,
 bottomrule=0pt,
 colback=white,
 arc=0pt,
 outer arc=0pt,
 title style={white},
 fonttitle=\color{black}\bfseries,
 left=8pt,
 colframe=yellow,
 title={Note:},
}

\begin{document}
\title{PasDoc's autodoc}
\author{Pasdoc}
\maketitle
\newpage
\label{toc}\tableofcontents
\newpage
% special variable used for calculating some widths.
\newlength{\tmplength}
\chapter{Pasdoc Sources Overview}
\label{introduction}
\index{introduction}
 

This is the documentation of the pasdoc sources, intended for pasdoc developers. For user's documentation see [\href{https://pasdoc.github.io/}{https://pasdoc.github.io/}].

Contents: 

General overview of the data flow in pasdoc:

\label{SecParsing}
\section{Parsing}


\begin{ttfamily}TTokenizer\end{ttfamily}(\ref{PasDoc_Tokenizer.TTokenizer}) reads the source file, and converts it to a series of \begin{ttfamily}TToken\end{ttfamily}(\ref{PasDoc_Tokenizer.TToken})s.

\begin{ttfamily}TScanner\end{ttfamily}(\ref{PasDoc_Scanner.TScanner}) uses an underlying \begin{ttfamily}TTokenizer\end{ttfamily}(\ref{PasDoc_Tokenizer.TTokenizer}) and also returns a series of \begin{ttfamily}TToken\end{ttfamily}(\ref{PasDoc_Tokenizer.TToken})s, but in addition it understands and interprets {\$}define, {\$}ifdef and similar compiler directives. While \begin{ttfamily}TTokenizer\end{ttfamily}(\ref{PasDoc_Tokenizer.TTokenizer}) simply returns all tokens, \begin{ttfamily}TScanner\end{ttfamily}(\ref{PasDoc_Scanner.TScanner}) returns only those tokens that are not "{\$}ifdefed out". E.g. if WIN32 is not defined then the \begin{ttfamily}TScanner\end{ttfamily}(\ref{PasDoc_Scanner.TScanner}) returns only tokens "\begin{ttfamily}const LineEnding = {\#}10;\end{ttfamily}" for the following code: \texttt{}\textbf{const}\texttt{~LineEnding~=~\textit{{\{}{\$}ifdef~WIN32{\}}}~{\#}13{\#}10~\textit{{\{}{\$}else{\}}}~{\#}10~\textit{{\{}{\$}endif{\}}};\\
}

Finally \begin{ttfamily}TParser\end{ttfamily}(\ref{PasDoc_Parser.TParser}) uses an underlying \begin{ttfamily}TScanner\end{ttfamily}(\ref{PasDoc_Scanner.TScanner}) and interprets the series of tokens, as e.g. "here I see a declaration of variable Foo, of type Integer". The Parser stores everything it reads in a \begin{ttfamily}TPasUnit\end{ttfamily}(\ref{PasDoc_Items.TPasUnit}) instance.

If you ever wrote a program that interprets a text language, you will see that there is nothing special here: We have a lexer (\begin{ttfamily}TScanner\end{ttfamily}(\ref{PasDoc_Scanner.TScanner}), a simplified lexer in \begin{ttfamily}TTokenizer\end{ttfamily}(\ref{PasDoc_Tokenizer.TTokenizer})) and a parser (\begin{ttfamily}TParser\end{ttfamily}(\ref{PasDoc_Parser.TParser})).

It is important to note that pasdoc's parser is somewhat unusual, compared to "normal" parsers that are used e.g. in Pascal compilers.

\begin{enumerate}
\setcounter{enumi}{0} \setcounter{enumii}{0} \setcounter{enumiii}{0} \setcounter{enumiv}{0} 
\item  Pasdoc's parser does not read the implementation section of a unit file (although this may change some day, see [\href{https://github.com/pasdoc/pasdoc/wiki/WantedFeaturesParsingImplementation}{https://github.com/pasdoc/pasdoc/wiki/WantedFeaturesParsingImplementation}]).
\setcounter{enumi}{1} \setcounter{enumii}{1} \setcounter{enumiii}{1} \setcounter{enumiv}{1} 
\item  Pasdoc's parser is "cheating": It does not really understand everything it reads. E.g. the parameter section of a procedure declaration is parsed "blindly", by simply reading tokens up to a matching closing parenthesis. Such cheating obviously simplifies the parser implementation, but it also makes pasdoc's parser "dumber", see [\href{https://github.com/pasdoc/pasdoc/wiki/ToDoParser}{https://github.com/pasdoc/pasdoc/wiki/ToDoParser}].
\setcounter{enumi}{2} \setcounter{enumii}{2} \setcounter{enumiii}{2} \setcounter{enumiv}{2} 
\item  Pasdoc's parser collects the comments before each declaration, since these comments must be converted and placed in the final documentation (while "normal" parsers usually treat comments as a meaningless white{-}space).
\end{enumerate}

\label{SecStoring}
\section{Storing}


The unit \begin{ttfamily}PasDoc{\_}Items\end{ttfamily}(\ref{PasDoc_Items}) provides a comfortable class hierarchy to store a parsed Pascal source tree. \begin{ttfamily}TPasUnit\end{ttfamily}(\ref{PasDoc_Items.TPasUnit}) is a "root class" (container{-}wise), it contains references to all other items within a unit, every item is some instance of \begin{ttfamily}TPasItem\end{ttfamily}(\ref{PasDoc_Items.TPasItem}).

\label{SecGenerators}
\section{Generators}


The last link in the chain are the generators. A generator uses the stored \begin{ttfamily}TPasItem\end{ttfamily}(\ref{PasDoc_Items.TPasItem}) tree and generates the final documentation. The base abstract class for a generator is \begin{ttfamily}TDocGenerator\end{ttfamily}(\ref{PasDoc_Gen.TDocGenerator}), this provides some general mechanisms used by all generators. From \begin{ttfamily}TDocGenerator\end{ttfamily}(\ref{PasDoc_Gen.TDocGenerator}) descend more specialized generator classes, like \begin{ttfamily}TGenericHTMLDocGenerator\end{ttfamily}(\ref{PasDoc_GenHtml.TGenericHTMLDocGenerator}), \begin{ttfamily}THTMLDocGenerator\end{ttfamily}(\ref{PasDoc_GenHtml.THTMLDocGenerator}), \begin{ttfamily}TTexDocGenerator\end{ttfamily}(\ref{PasDoc_GenLatex.TTexDocGenerator}) and others.

\label{SecNotes}
\section{Notes}


Note that the parser and the generators do not communicate with each other directly. The parser stores things in the \begin{ttfamily}TPasItem\end{ttfamily}(\ref{PasDoc_Items.TPasItem}) tree. Generators read and process the \begin{ttfamily}TPasItem\end{ttfamily}(\ref{PasDoc_Items.TPasItem}) tree.

So the parser cannot do any stupid thing like messing with some HTML{-}specific or LaTeX{-}specific issues of generating documentation. And the generator cannot deal with parsing Pascal source code.

Actually, this makes the implementation of the generator independent enough to be used in other cases, e.g. to generate an "introduction" file for the final documentation, like the one you are reading right now.\chapter{Unit PasDoc{\_}Aspell}
\label{PasDoc_Aspell}
\index{PasDoc{\_}Aspell}
\section{Description}
Spellchecking using Aspell.
\section{Uses}
\begin{itemize}
\item \begin{ttfamily}SysUtils\end{ttfamily}\item \begin{ttfamily}Classes\end{ttfamily}\item \begin{ttfamily}PasDoc{\_}ProcessLineTalk\end{ttfamily}(\ref{PasDoc_ProcessLineTalk})\item \begin{ttfamily}PasDoc{\_}ObjectVector\end{ttfamily}(\ref{PasDoc_ObjectVector})\item \begin{ttfamily}PasDoc{\_}Types\end{ttfamily}(\ref{PasDoc_Types})\end{itemize}
\section{Overview}
\begin{description}
\item[\texttt{\begin{ttfamily}TSpellingError\end{ttfamily} Class}]
\item[\texttt{\begin{ttfamily}TAspellProcess\end{ttfamily} Class}]This is a class to interface with aspell through pipe.
\end{description}
\section{Classes, Interfaces, Objects and Records}
\ifpdf
\subsection*{\large{\textbf{TSpellingError Class}}\normalsize\hspace{1ex}\hrulefill}
\else
\subsection*{TSpellingError Class}
\fi
\label{PasDoc_Aspell.TSpellingError}
\index{TSpellingError}
\subsubsection*{\large{\textbf{Hierarchy}}\normalsize\hspace{1ex}\hfill}
TSpellingError {$>$} TObject
%%%%Description
\subsubsection*{\large{\textbf{Fields}}\normalsize\hspace{1ex}\hfill}
\begin{list}{}{
\settowidth{\tmplength}{\textbf{Suggestions}}
\setlength{\itemindent}{0cm}
\setlength{\listparindent}{0cm}
\setlength{\leftmargin}{\evensidemargin}
\addtolength{\leftmargin}{\tmplength}
\settowidth{\labelsep}{X}
\addtolength{\leftmargin}{\labelsep}
\setlength{\labelwidth}{\tmplength}
}
\label{PasDoc_Aspell.TSpellingError-Word}
\index{Word}
\item[\textbf{Word}\hfill]
\ifpdf
\begin{flushleft}
\fi
\begin{ttfamily}
public Word: string;\end{ttfamily}

\ifpdf
\end{flushleft}
\fi


\par the mis{-}spelled word\label{PasDoc_Aspell.TSpellingError-Offset}
\index{Offset}
\item[\textbf{Offset}\hfill]
\ifpdf
\begin{flushleft}
\fi
\begin{ttfamily}
public Offset: Integer;\end{ttfamily}

\ifpdf
\end{flushleft}
\fi


\par offset inside the checked string\label{PasDoc_Aspell.TSpellingError-Suggestions}
\index{Suggestions}
\item[\textbf{Suggestions}\hfill]
\ifpdf
\begin{flushleft}
\fi
\begin{ttfamily}
public Suggestions: string;\end{ttfamily}

\ifpdf
\end{flushleft}
\fi


\par comma{-}separated list of suggestions\end{list}
\ifpdf
\subsection*{\large{\textbf{TAspellProcess Class}}\normalsize\hspace{1ex}\hrulefill}
\else
\subsection*{TAspellProcess Class}
\fi
\label{PasDoc_Aspell.TAspellProcess}
\index{TAspellProcess}
\subsubsection*{\large{\textbf{Hierarchy}}\normalsize\hspace{1ex}\hfill}
TAspellProcess {$>$} TObject
\subsubsection*{\large{\textbf{Description}}\normalsize\hspace{1ex}\hfill}
This is a class to interface with aspell through pipe. It uses underlying \begin{ttfamily}TProcessLineTalk\end{ttfamily}(\ref{PasDoc_ProcessLineTalk.TProcessLineTalk}) to execute and "talk" with aspell.\subsubsection*{\large{\textbf{Properties}}\normalsize\hspace{1ex}\hfill}
\begin{list}{}{
\settowidth{\tmplength}{\textbf{AspellLanguage}}
\setlength{\itemindent}{0cm}
\setlength{\listparindent}{0cm}
\setlength{\leftmargin}{\evensidemargin}
\addtolength{\leftmargin}{\tmplength}
\settowidth{\labelsep}{X}
\addtolength{\leftmargin}{\labelsep}
\setlength{\labelwidth}{\tmplength}
}
\label{PasDoc_Aspell.TAspellProcess-AspellMode}
\index{AspellMode}
\item[\textbf{AspellMode}\hfill]
\ifpdf
\begin{flushleft}
\fi
\begin{ttfamily}
public property AspellMode: string read FAspellMode;\end{ttfamily}

\ifpdf
\end{flushleft}
\fi


\par  \label{PasDoc_Aspell.TAspellProcess-AspellLanguage}
\index{AspellLanguage}
\item[\textbf{AspellLanguage}\hfill]
\ifpdf
\begin{flushleft}
\fi
\begin{ttfamily}
public property AspellLanguage: string read FAspellLanguage;\end{ttfamily}

\ifpdf
\end{flushleft}
\fi


\par  \label{PasDoc_Aspell.TAspellProcess-OnMessage}
\index{OnMessage}
\item[\textbf{OnMessage}\hfill]
\ifpdf
\begin{flushleft}
\fi
\begin{ttfamily}
public property OnMessage: TPasDocMessageEvent read FOnMessage write FOnMessage;\end{ttfamily}

\ifpdf
\end{flushleft}
\fi


\par  \end{list}
\subsubsection*{\large{\textbf{Methods}}\normalsize\hspace{1ex}\hfill}
\paragraph*{Create}\hspace*{\fill}

\label{PasDoc_Aspell.TAspellProcess-Create}
\index{Create}
\begin{list}{}{
\settowidth{\tmplength}{\textbf{Description}}
\setlength{\itemindent}{0cm}
\setlength{\listparindent}{0cm}
\setlength{\leftmargin}{\evensidemargin}
\addtolength{\leftmargin}{\tmplength}
\settowidth{\labelsep}{X}
\addtolength{\leftmargin}{\labelsep}
\setlength{\labelwidth}{\tmplength}
}
\item[\textbf{Declaration}\hfill]
\ifpdf
\begin{flushleft}
\fi
\begin{ttfamily}
public constructor Create(const AAspellMode, AAspellLanguage: string; AOnMessage: TPasDocMessageEvent);\end{ttfamily}

\ifpdf
\end{flushleft}
\fi

\par
\item[\textbf{Description}]
Constructor. Values for AspellMode and AspellLanguage are the same as for aspell {-}{-}mode and {-}{-}lang command{-}line options. You can pass here '', then we will not pass appropriate command{-}line option to aspell.

\end{list}
\paragraph*{Destroy}\hspace*{\fill}

\label{PasDoc_Aspell.TAspellProcess-Destroy}
\index{Destroy}
\begin{list}{}{
\settowidth{\tmplength}{\textbf{Description}}
\setlength{\itemindent}{0cm}
\setlength{\listparindent}{0cm}
\setlength{\leftmargin}{\evensidemargin}
\addtolength{\leftmargin}{\tmplength}
\settowidth{\labelsep}{X}
\addtolength{\leftmargin}{\labelsep}
\setlength{\labelwidth}{\tmplength}
}
\item[\textbf{Declaration}\hfill]
\ifpdf
\begin{flushleft}
\fi
\begin{ttfamily}
public destructor Destroy; override;\end{ttfamily}

\ifpdf
\end{flushleft}
\fi

\end{list}
\paragraph*{SetIgnoreWords}\hspace*{\fill}

\label{PasDoc_Aspell.TAspellProcess-SetIgnoreWords}
\index{SetIgnoreWords}
\begin{list}{}{
\settowidth{\tmplength}{\textbf{Description}}
\setlength{\itemindent}{0cm}
\setlength{\listparindent}{0cm}
\setlength{\leftmargin}{\evensidemargin}
\addtolength{\leftmargin}{\tmplength}
\settowidth{\labelsep}{X}
\addtolength{\leftmargin}{\labelsep}
\setlength{\labelwidth}{\tmplength}
}
\item[\textbf{Declaration}\hfill]
\ifpdf
\begin{flushleft}
\fi
\begin{ttfamily}
public procedure SetIgnoreWords(Value: TStringList);\end{ttfamily}

\ifpdf
\end{flushleft}
\fi

\end{list}
\paragraph*{CheckString}\hspace*{\fill}

\label{PasDoc_Aspell.TAspellProcess-CheckString}
\index{CheckString}
\begin{list}{}{
\settowidth{\tmplength}{\textbf{Description}}
\setlength{\itemindent}{0cm}
\setlength{\listparindent}{0cm}
\setlength{\leftmargin}{\evensidemargin}
\addtolength{\leftmargin}{\tmplength}
\settowidth{\labelsep}{X}
\addtolength{\leftmargin}{\labelsep}
\setlength{\labelwidth}{\tmplength}
}
\item[\textbf{Declaration}\hfill]
\ifpdf
\begin{flushleft}
\fi
\begin{ttfamily}
public procedure CheckString(const AString: string; const AErrors: TObjectVector);\end{ttfamily}

\ifpdf
\end{flushleft}
\fi

\par
\item[\textbf{Description}]
Spellchecks AString and returns result. Will create an array of TSpellingError objects, one entry for each misspelled word. Offsets of TSpellingErrors will be relative to AString.

\end{list}
\chapter{Unit PasDoc{\_}Base}
\label{PasDoc_Base}
\index{PasDoc{\_}Base}
\section{Description}
Contains the main TPasDoc component.\hfill\vspace*{1ex}

         

Unit name must be \begin{ttfamily}PasDoc{\_}Base\end{ttfamily} instead of just \begin{ttfamily}PasDoc\end{ttfamily} to not conflict with the name of base program name \begin{ttfamily}pasdoc.dpr\end{ttfamily}.
\section{Uses}
\begin{itemize}
\item \begin{ttfamily}SysUtils\end{ttfamily}\item \begin{ttfamily}Classes\end{ttfamily}\item \begin{ttfamily}PasDoc{\_}Items\end{ttfamily}(\ref{PasDoc_Items})\item \begin{ttfamily}PasDoc{\_}Languages\end{ttfamily}(\ref{PasDoc_Languages})\item \begin{ttfamily}PasDoc{\_}Gen\end{ttfamily}(\ref{PasDoc_Gen})\item \begin{ttfamily}PasDoc{\_}Types\end{ttfamily}(\ref{PasDoc_Types})\item \begin{ttfamily}PasDoc{\_}StringVector\end{ttfamily}(\ref{PasDoc_StringVector})\item \begin{ttfamily}PasDoc{\_}SortSettings\end{ttfamily}(\ref{PasDoc_SortSettings})\item \begin{ttfamily}PasDoc{\_}StreamUtils\end{ttfamily}(\ref{PasDoc_StreamUtils})\item \begin{ttfamily}PasDoc{\_}TagManager\end{ttfamily}(\ref{PasDoc_TagManager})\end{itemize}
\section{Overview}
\begin{description}
\item[\texttt{\begin{ttfamily}TPasDoc\end{ttfamily} Class}]The main object in the pasdoc application; first scans parameters, then parses files.
\end{description}
\section{Classes, Interfaces, Objects and Records}
\ifpdf
\subsection*{\large{\textbf{TPasDoc Class}}\normalsize\hspace{1ex}\hrulefill}
\else
\subsection*{TPasDoc Class}
\fi
\label{PasDoc_Base.TPasDoc}
\index{TPasDoc}
\subsubsection*{\large{\textbf{Hierarchy}}\normalsize\hspace{1ex}\hfill}
TPasDoc {$>$} TComponent
\subsubsection*{\large{\textbf{Description}}\normalsize\hspace{1ex}\hfill}
The main object in the pasdoc application; first scans parameters, then parses files. All parsed units are then given to documentation generator, which creates one or more documentation output files.\subsubsection*{\large{\textbf{Properties}}\normalsize\hspace{1ex}\hfill}
\begin{list}{}{
\settowidth{\tmplength}{\textbf{DescriptionFileNames}}
\setlength{\itemindent}{0cm}
\setlength{\listparindent}{0cm}
\setlength{\leftmargin}{\evensidemargin}
\addtolength{\leftmargin}{\tmplength}
\settowidth{\labelsep}{X}
\addtolength{\leftmargin}{\labelsep}
\setlength{\labelwidth}{\tmplength}
}
\label{PasDoc_Base.TPasDoc-Units}
\index{Units}
\item[\textbf{Units}\hfill]
\ifpdf
\begin{flushleft}
\fi
\begin{ttfamily}
public property Units: TPasUnits read FUnits;\end{ttfamily}

\ifpdf
\end{flushleft}
\fi


\par After \begin{ttfamily}Execute\end{ttfamily}(\ref{PasDoc_Base.TPasDoc-Execute}) has been called, \begin{ttfamily}Units\end{ttfamily} holds the units that have been parsed.\label{PasDoc_Base.TPasDoc-Conclusion}
\index{Conclusion}
\item[\textbf{Conclusion}\hfill]
\ifpdf
\begin{flushleft}
\fi
\begin{ttfamily}
public property Conclusion: TExternalItem read FConclusion;\end{ttfamily}

\ifpdf
\end{flushleft}
\fi


\par After \begin{ttfamily}Execute\end{ttfamily}(\ref{PasDoc_Base.TPasDoc-Execute}) has been called, \begin{ttfamily}Conclusion\end{ttfamily} holds the conclusion.\label{PasDoc_Base.TPasDoc-Introduction}
\index{Introduction}
\item[\textbf{Introduction}\hfill]
\ifpdf
\begin{flushleft}
\fi
\begin{ttfamily}
public property Introduction: TExternalItem read FIntroduction;\end{ttfamily}

\ifpdf
\end{flushleft}
\fi


\par After \begin{ttfamily}Execute\end{ttfamily}(\ref{PasDoc_Base.TPasDoc-Execute}) has been called, \begin{ttfamily}Introduction\end{ttfamily} holds the introduction.\label{PasDoc_Base.TPasDoc-AdditionalFiles}
\index{AdditionalFiles}
\item[\textbf{AdditionalFiles}\hfill]
\ifpdf
\begin{flushleft}
\fi
\begin{ttfamily}
public property AdditionalFiles: TExternalItemList read FAdditionalFiles;\end{ttfamily}

\ifpdf
\end{flushleft}
\fi


\par After \begin{ttfamily}Execute\end{ttfamily}(\ref{PasDoc_Base.TPasDoc-Execute}) has been called, \begin{ttfamily}AdditionalFiles\end{ttfamily} holds the additional external files.\label{PasDoc_Base.TPasDoc-DescriptionFileNames}
\index{DescriptionFileNames}
\item[\textbf{DescriptionFileNames}\hfill]
\ifpdf
\begin{flushleft}
\fi
\begin{ttfamily}
published property DescriptionFileNames: TStringVector
      read FDescriptionFileNames write SetDescriptionFileNames;\end{ttfamily}

\ifpdf
\end{flushleft}
\fi


\par  \label{PasDoc_Base.TPasDoc-Directives}
\index{Directives}
\item[\textbf{Directives}\hfill]
\ifpdf
\begin{flushleft}
\fi
\begin{ttfamily}
published property Directives: TStringVector read FDirectives write SetDirectives;\end{ttfamily}

\ifpdf
\end{flushleft}
\fi


\par  \label{PasDoc_Base.TPasDoc-IncludeDirectories}
\index{IncludeDirectories}
\item[\textbf{IncludeDirectories}\hfill]
\ifpdf
\begin{flushleft}
\fi
\begin{ttfamily}
published property IncludeDirectories: TStringVector read FIncludeDirectories write
      SetIncludeDirectories;\end{ttfamily}

\ifpdf
\end{flushleft}
\fi


\par  \label{PasDoc_Base.TPasDoc-OnWarning}
\index{OnWarning}
\item[\textbf{OnWarning}\hfill]
\ifpdf
\begin{flushleft}
\fi
\begin{ttfamily}
published property OnWarning: TPasDocMessageEvent read FOnMessage write FOnMessage stored false;\end{ttfamily}

\ifpdf
\end{flushleft}
\fi


\par This is deprecated name for \begin{ttfamily}OnMessage\end{ttfamily}(\ref{PasDoc_Base.TPasDoc-OnMessage})\label{PasDoc_Base.TPasDoc-OnMessage}
\index{OnMessage}
\item[\textbf{OnMessage}\hfill]
\ifpdf
\begin{flushleft}
\fi
\begin{ttfamily}
published property OnMessage: TPasDocMessageEvent read FOnMessage write FOnMessage;\end{ttfamily}

\ifpdf
\end{flushleft}
\fi


\par  \label{PasDoc_Base.TPasDoc-ProjectName}
\index{ProjectName}
\item[\textbf{ProjectName}\hfill]
\ifpdf
\begin{flushleft}
\fi
\begin{ttfamily}
published property ProjectName: string read FProjectName write FProjectName;\end{ttfamily}

\ifpdf
\end{flushleft}
\fi


\par The name PasDoc shall give to this documentation project, also used to name some of the output files.\label{PasDoc_Base.TPasDoc-SourceFileNames}
\index{SourceFileNames}
\item[\textbf{SourceFileNames}\hfill]
\ifpdf
\begin{flushleft}
\fi
\begin{ttfamily}
published property SourceFileNames: TStringVector read FSourceFileNames write
      SetSourceFileNames;\end{ttfamily}

\ifpdf
\end{flushleft}
\fi


\par  \label{PasDoc_Base.TPasDoc-Title}
\index{Title}
\item[\textbf{Title}\hfill]
\ifpdf
\begin{flushleft}
\fi
\begin{ttfamily}
published property Title: string read FTitle write FTitle;\end{ttfamily}

\ifpdf
\end{flushleft}
\fi


\par  \label{PasDoc_Base.TPasDoc-Verbosity}
\index{Verbosity}
\item[\textbf{Verbosity}\hfill]
\ifpdf
\begin{flushleft}
\fi
\begin{ttfamily}
published property Verbosity: Cardinal read FVerbosity write FVerbosity
      default DEFAULT{\_}VERBOSITY{\_}LEVEL;\end{ttfamily}

\ifpdf
\end{flushleft}
\fi


\par  \label{PasDoc_Base.TPasDoc-StarOnly}
\index{StarOnly}
\item[\textbf{StarOnly}\hfill]
\ifpdf
\begin{flushleft}
\fi
\begin{ttfamily}
published property StarOnly: boolean read GetStarOnly write SetStarOnly stored false;\end{ttfamily}

\ifpdf
\end{flushleft}
\fi


\par  \label{PasDoc_Base.TPasDoc-CommentMarkers}
\index{CommentMarkers}
\item[\textbf{CommentMarkers}\hfill]
\ifpdf
\begin{flushleft}
\fi
\begin{ttfamily}
published property CommentMarkers: TStringList read FCommentMarkers write SetCommentMarkers;\end{ttfamily}

\ifpdf
\end{flushleft}
\fi


\par  \label{PasDoc_Base.TPasDoc-IgnoreMarkers}
\index{IgnoreMarkers}
\item[\textbf{IgnoreMarkers}\hfill]
\ifpdf
\begin{flushleft}
\fi
\begin{ttfamily}
published property IgnoreMarkers: TStringList read FIgnoreMarkers write SetIgnoreMarkers;\end{ttfamily}

\ifpdf
\end{flushleft}
\fi


\par  \label{PasDoc_Base.TPasDoc-MarkerOptional}
\index{MarkerOptional}
\item[\textbf{MarkerOptional}\hfill]
\ifpdf
\begin{flushleft}
\fi
\begin{ttfamily}
published property MarkerOptional: boolean read FMarkerOptional write FMarkerOptional
      default false;\end{ttfamily}

\ifpdf
\end{flushleft}
\fi


\par  \label{PasDoc_Base.TPasDoc-IgnoreLeading}
\index{IgnoreLeading}
\item[\textbf{IgnoreLeading}\hfill]
\ifpdf
\begin{flushleft}
\fi
\begin{ttfamily}
published property IgnoreLeading: string read FIgnoreLeading write FIgnoreLeading;\end{ttfamily}

\ifpdf
\end{flushleft}
\fi


\par  \label{PasDoc_Base.TPasDoc-Generator}
\index{Generator}
\item[\textbf{Generator}\hfill]
\ifpdf
\begin{flushleft}
\fi
\begin{ttfamily}
published property Generator: TDocGenerator read FGenerator write SetGenerator;\end{ttfamily}

\ifpdf
\end{flushleft}
\fi


\par  \label{PasDoc_Base.TPasDoc-ShowVisibilities}
\index{ShowVisibilities}
\item[\textbf{ShowVisibilities}\hfill]
\ifpdf
\begin{flushleft}
\fi
\begin{ttfamily}
published property ShowVisibilities: TVisibilities read FShowVisibilities write FShowVisibilities;\end{ttfamily}

\ifpdf
\end{flushleft}
\fi


\par  \label{PasDoc_Base.TPasDoc-CacheDir}
\index{CacheDir}
\item[\textbf{CacheDir}\hfill]
\ifpdf
\begin{flushleft}
\fi
\begin{ttfamily}
published property CacheDir: string read FCacheDir write FCacheDir;\end{ttfamily}

\ifpdf
\end{flushleft}
\fi


\par  \label{PasDoc_Base.TPasDoc-SortSettings}
\index{SortSettings}
\item[\textbf{SortSettings}\hfill]
\ifpdf
\begin{flushleft}
\fi
\begin{ttfamily}
published property SortSettings: TSortSettings
      read FSortSettings write FSortSettings default [];\end{ttfamily}

\ifpdf
\end{flushleft}
\fi


\par This determines how items inside will be sorted. See [\href{https://github.com/pasdoc/pasdoc/wiki/SortOption}{https://github.com/pasdoc/pasdoc/wiki/SortOption}].\label{PasDoc_Base.TPasDoc-IntroductionFileName}
\index{IntroductionFileName}
\item[\textbf{IntroductionFileName}\hfill]
\ifpdf
\begin{flushleft}
\fi
\begin{ttfamily}
published property IntroductionFileName: string read FIntroductionFileName
      write FIntroductionFileName;\end{ttfamily}

\ifpdf
\end{flushleft}
\fi


\par  \label{PasDoc_Base.TPasDoc-ConclusionFileName}
\index{ConclusionFileName}
\item[\textbf{ConclusionFileName}\hfill]
\ifpdf
\begin{flushleft}
\fi
\begin{ttfamily}
published property ConclusionFileName: string read FConclusionFileName
      write FConclusionFileName;\end{ttfamily}

\ifpdf
\end{flushleft}
\fi


\par  \label{PasDoc_Base.TPasDoc-AdditionalFilesNames}
\index{AdditionalFilesNames}
\item[\textbf{AdditionalFilesNames}\hfill]
\ifpdf
\begin{flushleft}
\fi
\begin{ttfamily}
published property AdditionalFilesNames: TStringList read FAdditionalFilesNames;\end{ttfamily}

\ifpdf
\end{flushleft}
\fi


\par  \label{PasDoc_Base.TPasDoc-ImplicitVisibility}
\index{ImplicitVisibility}
\item[\textbf{ImplicitVisibility}\hfill]
\ifpdf
\begin{flushleft}
\fi
\begin{ttfamily}
published property ImplicitVisibility: TImplicitVisibility
      read FImplicitVisibility write FImplicitVisibility default ivPublic;\end{ttfamily}

\ifpdf
\end{flushleft}
\fi


\par See command{-}line option {-}{-}implicit{-}visibility documentation at [\href{https://github.com/pasdoc/pasdoc/wiki/ImplicitVisibilityOption}{https://github.com/pasdoc/pasdoc/wiki/ImplicitVisibilityOption}]. This will be passed to parser instance.\label{PasDoc_Base.TPasDoc-HandleMacros}
\index{HandleMacros}
\item[\textbf{HandleMacros}\hfill]
\ifpdf
\begin{flushleft}
\fi
\begin{ttfamily}
published property HandleMacros: boolean
      read FHandleMacros write FHandleMacros default true;\end{ttfamily}

\ifpdf
\end{flushleft}
\fi


\par  \label{PasDoc_Base.TPasDoc-AutoLink}
\index{AutoLink}
\item[\textbf{AutoLink}\hfill]
\ifpdf
\begin{flushleft}
\fi
\begin{ttfamily}
published property AutoLink: boolean
      read FAutoLink write FAutoLink default false;\end{ttfamily}

\ifpdf
\end{flushleft}
\fi


\par This controls auto{-}linking, see [\href{https://github.com/pasdoc/pasdoc/wiki/AutoLinkOption}{https://github.com/pasdoc/pasdoc/wiki/AutoLinkOption}]\label{PasDoc_Base.TPasDoc-AutoBackComments}
\index{AutoBackComments}
\item[\textbf{AutoBackComments}\hfill]
\ifpdf
\begin{flushleft}
\fi
\begin{ttfamily}
published property AutoBackComments: boolean
      read FAutoBackComments write FAutoBackComments default false;\end{ttfamily}

\ifpdf
\end{flushleft}
\fi


\par  \label{PasDoc_Base.TPasDoc-InfoMergeType}
\index{InfoMergeType}
\item[\textbf{InfoMergeType}\hfill]
\ifpdf
\begin{flushleft}
\fi
\begin{ttfamily}
published property InfoMergeType: TInfoMergeType
      read FInfoMergeType write FInfoMergeType;\end{ttfamily}

\ifpdf
\end{flushleft}
\fi


\par  \end{list}
\subsubsection*{\large{\textbf{Methods}}\normalsize\hspace{1ex}\hfill}
\paragraph*{RemoveExcludedItems}\hspace*{\fill}

\label{PasDoc_Base.TPasDoc-RemoveExcludedItems}
\index{RemoveExcludedItems}
\begin{list}{}{
\settowidth{\tmplength}{\textbf{Description}}
\setlength{\itemindent}{0cm}
\setlength{\listparindent}{0cm}
\setlength{\leftmargin}{\evensidemargin}
\addtolength{\leftmargin}{\tmplength}
\settowidth{\labelsep}{X}
\addtolength{\leftmargin}{\labelsep}
\setlength{\labelwidth}{\tmplength}
}
\item[\textbf{Declaration}\hfill]
\ifpdf
\begin{flushleft}
\fi
\begin{ttfamily}
protected procedure RemoveExcludedItems(const c: TPasItems);\end{ttfamily}

\ifpdf
\end{flushleft}
\fi

\par
\item[\textbf{Description}]
Searches the description of each TPasUnit item in the collection for an excluded tag. If one is found, the item is removed from the collection. If not, the fields, methods and properties collections are called with RemoveExcludedItems If the collection is empty after removal of all items, it is disposed of and the variable is set to nil.

\end{list}
\paragraph*{Notification}\hspace*{\fill}

\label{PasDoc_Base.TPasDoc-Notification}
\index{Notification}
\begin{list}{}{
\settowidth{\tmplength}{\textbf{Description}}
\setlength{\itemindent}{0cm}
\setlength{\listparindent}{0cm}
\setlength{\leftmargin}{\evensidemargin}
\addtolength{\leftmargin}{\tmplength}
\settowidth{\labelsep}{X}
\addtolength{\leftmargin}{\labelsep}
\setlength{\labelwidth}{\tmplength}
}
\item[\textbf{Declaration}\hfill]
\ifpdf
\begin{flushleft}
\fi
\begin{ttfamily}
protected procedure Notification(AComponent: TComponent; Operation: TOperation); override;\end{ttfamily}

\ifpdf
\end{flushleft}
\fi

\end{list}
\paragraph*{Create}\hspace*{\fill}

\label{PasDoc_Base.TPasDoc-Create}
\index{Create}
\begin{list}{}{
\settowidth{\tmplength}{\textbf{Description}}
\setlength{\itemindent}{0cm}
\setlength{\listparindent}{0cm}
\setlength{\leftmargin}{\evensidemargin}
\addtolength{\leftmargin}{\tmplength}
\settowidth{\labelsep}{X}
\addtolength{\leftmargin}{\labelsep}
\setlength{\labelwidth}{\tmplength}
}
\item[\textbf{Declaration}\hfill]
\ifpdf
\begin{flushleft}
\fi
\begin{ttfamily}
public constructor Create(AOwner: TComponent); override;\end{ttfamily}

\ifpdf
\end{flushleft}
\fi

\par
\item[\textbf{Description}]
Creates object and sets fields to default values.

\end{list}
\paragraph*{Destroy}\hspace*{\fill}

\label{PasDoc_Base.TPasDoc-Destroy}
\index{Destroy}
\begin{list}{}{
\settowidth{\tmplength}{\textbf{Description}}
\setlength{\itemindent}{0cm}
\setlength{\listparindent}{0cm}
\setlength{\leftmargin}{\evensidemargin}
\addtolength{\leftmargin}{\tmplength}
\settowidth{\labelsep}{X}
\addtolength{\leftmargin}{\labelsep}
\setlength{\labelwidth}{\tmplength}
}
\item[\textbf{Declaration}\hfill]
\ifpdf
\begin{flushleft}
\fi
\begin{ttfamily}
public destructor Destroy; override;\end{ttfamily}

\ifpdf
\end{flushleft}
\fi

\end{list}
\paragraph*{AddSourceFileNames}\hspace*{\fill}

\label{PasDoc_Base.TPasDoc-AddSourceFileNames}
\index{AddSourceFileNames}
\begin{list}{}{
\settowidth{\tmplength}{\textbf{Description}}
\setlength{\itemindent}{0cm}
\setlength{\listparindent}{0cm}
\setlength{\leftmargin}{\evensidemargin}
\addtolength{\leftmargin}{\tmplength}
\settowidth{\labelsep}{X}
\addtolength{\leftmargin}{\labelsep}
\setlength{\labelwidth}{\tmplength}
}
\item[\textbf{Declaration}\hfill]
\ifpdf
\begin{flushleft}
\fi
\begin{ttfamily}
public procedure AddSourceFileNames(const AFileNames: TStringList);\end{ttfamily}

\ifpdf
\end{flushleft}
\fi

\par
\item[\textbf{Description}]
Adds source filenames from a stringlist

\end{list}
\paragraph*{AddSourceFileNamesFromFile}\hspace*{\fill}

\label{PasDoc_Base.TPasDoc-AddSourceFileNamesFromFile}
\index{AddSourceFileNamesFromFile}
\begin{list}{}{
\settowidth{\tmplength}{\textbf{Description}}
\setlength{\itemindent}{0cm}
\setlength{\listparindent}{0cm}
\setlength{\leftmargin}{\evensidemargin}
\addtolength{\leftmargin}{\tmplength}
\settowidth{\labelsep}{X}
\addtolength{\leftmargin}{\labelsep}
\setlength{\labelwidth}{\tmplength}
}
\item[\textbf{Declaration}\hfill]
\ifpdf
\begin{flushleft}
\fi
\begin{ttfamily}
public procedure AddSourceFileNamesFromFile(const FileName: string; DashMeansStdin: boolean);\end{ttfamily}

\ifpdf
\end{flushleft}
\fi

\par
\item[\textbf{Description}]
Loads names of Pascal unit source code files from a text file. Adds all file names to \begin{ttfamily}SourceFileNames\end{ttfamily}(\ref{PasDoc_Base.TPasDoc-SourceFileNames}). If DashMeansStdin and AFileName = '{-}' then it will load filenames from stdin.

\end{list}
\paragraph*{DoError}\hspace*{\fill}

\label{PasDoc_Base.TPasDoc-DoError}
\index{DoError}
\begin{list}{}{
\settowidth{\tmplength}{\textbf{Description}}
\setlength{\itemindent}{0cm}
\setlength{\listparindent}{0cm}
\setlength{\leftmargin}{\evensidemargin}
\addtolength{\leftmargin}{\tmplength}
\settowidth{\labelsep}{X}
\addtolength{\leftmargin}{\labelsep}
\setlength{\labelwidth}{\tmplength}
}
\item[\textbf{Declaration}\hfill]
\ifpdf
\begin{flushleft}
\fi
\begin{ttfamily}
public procedure DoError(const AMessage: string; const AArguments: array of const; const AExitCode: Word);\end{ttfamily}

\ifpdf
\end{flushleft}
\fi

\par
\item[\textbf{Description}]
Raises an exception.

\end{list}
\paragraph*{DoMessage}\hspace*{\fill}

\label{PasDoc_Base.TPasDoc-DoMessage}
\index{DoMessage}
\begin{list}{}{
\settowidth{\tmplength}{\textbf{Description}}
\setlength{\itemindent}{0cm}
\setlength{\listparindent}{0cm}
\setlength{\leftmargin}{\evensidemargin}
\addtolength{\leftmargin}{\tmplength}
\settowidth{\labelsep}{X}
\addtolength{\leftmargin}{\labelsep}
\setlength{\labelwidth}{\tmplength}
}
\item[\textbf{Declaration}\hfill]
\ifpdf
\begin{flushleft}
\fi
\begin{ttfamily}
public procedure DoMessage(const AVerbosity: Cardinal; const AMessageType: TPasDocMessageType; const AMessage: string; const AArguments: array of const);\end{ttfamily}

\ifpdf
\end{flushleft}
\fi

\par
\item[\textbf{Description}]
Forwards a message to the \begin{ttfamily}OnMessage\end{ttfamily}(\ref{PasDoc_Base.TPasDoc-OnMessage}) event.

\end{list}
\paragraph*{GenMessage}\hspace*{\fill}

\label{PasDoc_Base.TPasDoc-GenMessage}
\index{GenMessage}
\begin{list}{}{
\settowidth{\tmplength}{\textbf{Description}}
\setlength{\itemindent}{0cm}
\setlength{\listparindent}{0cm}
\setlength{\leftmargin}{\evensidemargin}
\addtolength{\leftmargin}{\tmplength}
\settowidth{\labelsep}{X}
\addtolength{\leftmargin}{\labelsep}
\setlength{\labelwidth}{\tmplength}
}
\item[\textbf{Declaration}\hfill]
\ifpdf
\begin{flushleft}
\fi
\begin{ttfamily}
public procedure GenMessage(const MessageType: TPasDocMessageType; const AMessage: string; const AVerbosity: Cardinal);\end{ttfamily}

\ifpdf
\end{flushleft}
\fi

\par
\item[\textbf{Description}]
for Generator messages

\end{list}
\paragraph*{Execute}\hspace*{\fill}

\label{PasDoc_Base.TPasDoc-Execute}
\index{Execute}
\begin{list}{}{
\settowidth{\tmplength}{\textbf{Description}}
\setlength{\itemindent}{0cm}
\setlength{\listparindent}{0cm}
\setlength{\leftmargin}{\evensidemargin}
\addtolength{\leftmargin}{\tmplength}
\settowidth{\labelsep}{X}
\addtolength{\leftmargin}{\labelsep}
\setlength{\labelwidth}{\tmplength}
}
\item[\textbf{Declaration}\hfill]
\ifpdf
\begin{flushleft}
\fi
\begin{ttfamily}
public procedure Execute;\end{ttfamily}

\ifpdf
\end{flushleft}
\fi

\par
\item[\textbf{Description}]
Starts creating the documentation.

\end{list}
\section{Constants}
\ifpdf
\subsection*{\large{\textbf{DEFAULT{\_}VERBOSITY{\_}LEVEL}}\normalsize\hspace{1ex}\hrulefill}
\else
\subsection*{DEFAULT{\_}VERBOSITY{\_}LEVEL}
\fi
\label{PasDoc_Base-DEFAULT_VERBOSITY_LEVEL}
\index{DEFAULT{\_}VERBOSITY{\_}LEVEL}
\begin{list}{}{
\settowidth{\tmplength}{\textbf{Description}}
\setlength{\itemindent}{0cm}
\setlength{\listparindent}{0cm}
\setlength{\leftmargin}{\evensidemargin}
\addtolength{\leftmargin}{\tmplength}
\settowidth{\labelsep}{X}
\addtolength{\leftmargin}{\labelsep}
\setlength{\labelwidth}{\tmplength}
}
\item[\textbf{Declaration}\hfill]
\ifpdf
\begin{flushleft}
\fi
\begin{ttfamily}
DEFAULT{\_}VERBOSITY{\_}LEVEL = 2;\end{ttfamily}

\ifpdf
\end{flushleft}
\fi

\end{list}
\section{Authors}
\par
Johannes Berg {$<$}johannes@sipsolutions.de{$>$}

\par
Ralf Junker (delphi@zeitungsjunge.de)

\par
Erwin Scheuch-Heilig (ScheuchHeilig@t-online.de)

\par
Marco Schmidt (marcoschmidt@geocities.com)

\par
Michael van Canneyt (michael@tfdec1.fys.kuleuven.ac.be)

\par
Michalis Kamburelis

\par
Richard B. Winston {$<$}rbwinst@usgs.gov{$>$}

\par
Arno Garrels {$<$}first name.name@nospamgmx.de{$>$}

\section{Created}
\par
24 Sep 1999


\chapter{Unit PasDoc{\_}Gen}
\label{PasDoc_Gen}
\index{PasDoc{\_}Gen}
\section{Description}
basic doc generator object\hfill\vspace*{1ex}

             

\begin{ttfamily}PasDoc{\_}Gen\end{ttfamily} contains the basic documentation generator object \begin{ttfamily}TDocGenerator\end{ttfamily}(\ref{PasDoc_Gen.TDocGenerator}). It is not sufficient by itself but the basis for all generators that produce documentation in a specific format like HTML or LaTex. They override \begin{ttfamily}TDocGenerator\end{ttfamily}(\ref{PasDoc_Gen.TDocGenerator})'s virtual methods.
\section{Uses}
\begin{itemize}
\item \begin{ttfamily}PasDoc{\_}Items\end{ttfamily}(\ref{PasDoc_Items})\item \begin{ttfamily}PasDoc{\_}Languages\end{ttfamily}(\ref{PasDoc_Languages})\item \begin{ttfamily}PasDoc{\_}StringVector\end{ttfamily}(\ref{PasDoc_StringVector})\item \begin{ttfamily}PasDoc{\_}ObjectVector\end{ttfamily}(\ref{PasDoc_ObjectVector})\item \begin{ttfamily}PasDoc{\_}HierarchyTree\end{ttfamily}(\ref{PasDoc_HierarchyTree})\item \begin{ttfamily}PasDoc{\_}Types\end{ttfamily}(\ref{PasDoc_Types})\item \begin{ttfamily}Classes\end{ttfamily}\item \begin{ttfamily}PasDoc{\_}TagManager\end{ttfamily}(\ref{PasDoc_TagManager})\item \begin{ttfamily}PasDoc{\_}Aspell\end{ttfamily}(\ref{PasDoc_Aspell})\item \begin{ttfamily}PasDoc{\_}StreamUtils\end{ttfamily}(\ref{PasDoc_StreamUtils})\item \begin{ttfamily}PasDoc{\_}StringPairVector\end{ttfamily}(\ref{PasDoc_StringPairVector})\end{itemize}
\section{Overview}
\begin{description}
\item[\texttt{\begin{ttfamily}TOverviewFileInfo\end{ttfamily} Record}]
\item[\texttt{\begin{ttfamily}TListItemData\end{ttfamily} Class}]Collected information about @xxxList item.
\item[\texttt{\begin{ttfamily}TListData\end{ttfamily} Class}]Collected information about @xxxList content.
\item[\texttt{\begin{ttfamily}TRowData\end{ttfamily} Class}]Collected information about @row (or @rowHead).
\item[\texttt{\begin{ttfamily}TTableData\end{ttfamily} Class}]Collected information about @table.
\item[\texttt{\begin{ttfamily}TDocGenerator\end{ttfamily} Class}]basic documentation generator object
\end{description}
\section{Classes, Interfaces, Objects and Records}
\ifpdf
\subsection*{\large{\textbf{TOverviewFileInfo Record}}\normalsize\hspace{1ex}\hrulefill}
\else
\subsection*{TOverviewFileInfo Record}
\fi
\label{PasDoc_Gen.TOverviewFileInfo}
\index{TOverviewFileInfo}
%%%%Description
\subsubsection*{\large{\textbf{Fields}}\normalsize\hspace{1ex}\hfill}
\begin{list}{}{
\settowidth{\tmplength}{\textbf{TranslationHeadlineId}}
\setlength{\itemindent}{0cm}
\setlength{\listparindent}{0cm}
\setlength{\leftmargin}{\evensidemargin}
\addtolength{\leftmargin}{\tmplength}
\settowidth{\labelsep}{X}
\addtolength{\leftmargin}{\labelsep}
\setlength{\labelwidth}{\tmplength}
}
\label{PasDoc_Gen.TOverviewFileInfo-BaseFileName}
\index{BaseFileName}
\item[\textbf{BaseFileName}\hfill]
\ifpdf
\begin{flushleft}
\fi
\begin{ttfamily}
public BaseFileName: string;\end{ttfamily}

\ifpdf
\end{flushleft}
\fi


\par  \label{PasDoc_Gen.TOverviewFileInfo-TranslationId}
\index{TranslationId}
\item[\textbf{TranslationId}\hfill]
\ifpdf
\begin{flushleft}
\fi
\begin{ttfamily}
public TranslationId: TTranslationId;\end{ttfamily}

\ifpdf
\end{flushleft}
\fi


\par  \label{PasDoc_Gen.TOverviewFileInfo-TranslationHeadlineId}
\index{TranslationHeadlineId}
\item[\textbf{TranslationHeadlineId}\hfill]
\ifpdf
\begin{flushleft}
\fi
\begin{ttfamily}
public TranslationHeadlineId: TTranslationId;\end{ttfamily}

\ifpdf
\end{flushleft}
\fi


\par  \label{PasDoc_Gen.TOverviewFileInfo-NoItemsTranslationId}
\index{NoItemsTranslationId}
\item[\textbf{NoItemsTranslationId}\hfill]
\ifpdf
\begin{flushleft}
\fi
\begin{ttfamily}
public NoItemsTranslationId: TTranslationId;\end{ttfamily}

\ifpdf
\end{flushleft}
\fi


\par  \end{list}
\ifpdf
\subsection*{\large{\textbf{TListItemData Class}}\normalsize\hspace{1ex}\hrulefill}
\else
\subsection*{TListItemData Class}
\fi
\label{PasDoc_Gen.TListItemData}
\index{TListItemData}
\subsubsection*{\large{\textbf{Hierarchy}}\normalsize\hspace{1ex}\hfill}
TListItemData {$>$} TObject
\subsubsection*{\large{\textbf{Description}}\normalsize\hspace{1ex}\hfill}
Collected information about @xxxList item.\subsubsection*{\large{\textbf{Properties}}\normalsize\hspace{1ex}\hfill}
\begin{list}{}{
\settowidth{\tmplength}{\textbf{ItemLabel}}
\setlength{\itemindent}{0cm}
\setlength{\listparindent}{0cm}
\setlength{\leftmargin}{\evensidemargin}
\addtolength{\leftmargin}{\tmplength}
\settowidth{\labelsep}{X}
\addtolength{\leftmargin}{\labelsep}
\setlength{\labelwidth}{\tmplength}
}
\label{PasDoc_Gen.TListItemData-ItemLabel}
\index{ItemLabel}
\item[\textbf{ItemLabel}\hfill]
\ifpdf
\begin{flushleft}
\fi
\begin{ttfamily}
public property ItemLabel: string read FItemLabel;\end{ttfamily}

\ifpdf
\end{flushleft}
\fi


\par This is only for @definitionList: label for this list item, taken from @itemLabel. Already in the processed form. For other lists this will always be ''.\label{PasDoc_Gen.TListItemData-Text}
\index{Text}
\item[\textbf{Text}\hfill]
\ifpdf
\begin{flushleft}
\fi
\begin{ttfamily}
public property Text: string read FText;\end{ttfamily}

\ifpdf
\end{flushleft}
\fi


\par This is content of this item, taken from @item. Already in the processed form, after \begin{ttfamily}TDocGenerator.ConvertString\end{ttfamily}(\ref{PasDoc_Gen.TDocGenerator-ConvertString}) etc. Ready to be included in final documentation.\label{PasDoc_Gen.TListItemData-Index}
\index{Index}
\item[\textbf{Index}\hfill]
\ifpdf
\begin{flushleft}
\fi
\begin{ttfamily}
public property Index: Integer read FIndex;\end{ttfamily}

\ifpdf
\end{flushleft}
\fi


\par Number of this item. This should be used for @orderedList. When you iterate over \begin{ttfamily}TListData.Items\end{ttfamily}, you should be aware that Index of list item is \textit{not} necessarily equal to the position of item inside \begin{ttfamily}TListData.Items\end{ttfamily}. That's because of @itemSetNumber tag.

Normal list numbering (when no @itemSetNumber tag was used) starts from 1. Using @itemSetNumber user is able to change following item's Index.

For unordered and definition lists this is simpler: Index is always equal to the position within \begin{ttfamily}TListData.Items\end{ttfamily} (because @itemSetNumber is not allowed there). And usually you will just ignore Index of items on unordered and definition lists.\end{list}
\subsubsection*{\large{\textbf{Methods}}\normalsize\hspace{1ex}\hfill}
\paragraph*{Create}\hspace*{\fill}

\label{PasDoc_Gen.TListItemData-Create}
\index{Create}
\begin{list}{}{
\settowidth{\tmplength}{\textbf{Description}}
\setlength{\itemindent}{0cm}
\setlength{\listparindent}{0cm}
\setlength{\leftmargin}{\evensidemargin}
\addtolength{\leftmargin}{\tmplength}
\settowidth{\labelsep}{X}
\addtolength{\leftmargin}{\labelsep}
\setlength{\labelwidth}{\tmplength}
}
\item[\textbf{Declaration}\hfill]
\ifpdf
\begin{flushleft}
\fi
\begin{ttfamily}
public constructor Create(AItemLabel, AText: string; AIndex: Integer);\end{ttfamily}

\ifpdf
\end{flushleft}
\fi

\end{list}
\ifpdf
\subsection*{\large{\textbf{TListData Class}}\normalsize\hspace{1ex}\hrulefill}
\else
\subsection*{TListData Class}
\fi
\label{PasDoc_Gen.TListData}
\index{TListData}
\subsubsection*{\large{\textbf{Hierarchy}}\normalsize\hspace{1ex}\hfill}
TListData {$>$} \begin{ttfamily}TObjectVector\end{ttfamily}(\ref{PasDoc_ObjectVector.TObjectVector}) {$>$} 
TObjectList
\subsubsection*{\large{\textbf{Description}}\normalsize\hspace{1ex}\hfill}
Collected information about @xxxList content. Passed to \begin{ttfamily}TDocGenerator.FormatList\end{ttfamily}(\ref{PasDoc_Gen.TDocGenerator-FormatList}). Every item of this list should be non{-}nil instance of \begin{ttfamily}TListItemData\end{ttfamily}(\ref{PasDoc_Gen.TListItemData}).\subsubsection*{\large{\textbf{Properties}}\normalsize\hspace{1ex}\hfill}
\begin{list}{}{
\settowidth{\tmplength}{\textbf{ItemSpacing}}
\setlength{\itemindent}{0cm}
\setlength{\listparindent}{0cm}
\setlength{\leftmargin}{\evensidemargin}
\addtolength{\leftmargin}{\tmplength}
\settowidth{\labelsep}{X}
\addtolength{\leftmargin}{\labelsep}
\setlength{\labelwidth}{\tmplength}
}
\label{PasDoc_Gen.TListData-ItemSpacing}
\index{ItemSpacing}
\item[\textbf{ItemSpacing}\hfill]
\ifpdf
\begin{flushleft}
\fi
\begin{ttfamily}
public property ItemSpacing: TListItemSpacing read FItemSpacing;\end{ttfamily}

\ifpdf
\end{flushleft}
\fi


\par  \label{PasDoc_Gen.TListData-ListType}
\index{ListType}
\item[\textbf{ListType}\hfill]
\ifpdf
\begin{flushleft}
\fi
\begin{ttfamily}
public property ListType: TListType read FListType;\end{ttfamily}

\ifpdf
\end{flushleft}
\fi


\par  \end{list}
\subsubsection*{\large{\textbf{Methods}}\normalsize\hspace{1ex}\hfill}
\paragraph*{Create}\hspace*{\fill}

\label{PasDoc_Gen.TListData-Create}
\index{Create}
\begin{list}{}{
\settowidth{\tmplength}{\textbf{Description}}
\setlength{\itemindent}{0cm}
\setlength{\listparindent}{0cm}
\setlength{\leftmargin}{\evensidemargin}
\addtolength{\leftmargin}{\tmplength}
\settowidth{\labelsep}{X}
\addtolength{\leftmargin}{\labelsep}
\setlength{\labelwidth}{\tmplength}
}
\item[\textbf{Declaration}\hfill]
\ifpdf
\begin{flushleft}
\fi
\begin{ttfamily}
public constructor Create(const AOwnsObject: boolean); override;\end{ttfamily}

\ifpdf
\end{flushleft}
\fi

\end{list}
\ifpdf
\subsection*{\large{\textbf{TRowData Class}}\normalsize\hspace{1ex}\hrulefill}
\else
\subsection*{TRowData Class}
\fi
\label{PasDoc_Gen.TRowData}
\index{TRowData}
\subsubsection*{\large{\textbf{Hierarchy}}\normalsize\hspace{1ex}\hfill}
TRowData {$>$} TObject
\subsubsection*{\large{\textbf{Description}}\normalsize\hspace{1ex}\hfill}
Collected information about @row (or @rowHead).\subsubsection*{\large{\textbf{Fields}}\normalsize\hspace{1ex}\hfill}
\begin{list}{}{
\settowidth{\tmplength}{\textbf{Cells}}
\setlength{\itemindent}{0cm}
\setlength{\listparindent}{0cm}
\setlength{\leftmargin}{\evensidemargin}
\addtolength{\leftmargin}{\tmplength}
\settowidth{\labelsep}{X}
\addtolength{\leftmargin}{\labelsep}
\setlength{\labelwidth}{\tmplength}
}
\label{PasDoc_Gen.TRowData-Head}
\index{Head}
\item[\textbf{Head}\hfill]
\ifpdf
\begin{flushleft}
\fi
\begin{ttfamily}
public Head: boolean;\end{ttfamily}

\ifpdf
\end{flushleft}
\fi


\par \begin{ttfamily}True\end{ttfamily} if this is for @rowHead tag.\label{PasDoc_Gen.TRowData-Cells}
\index{Cells}
\item[\textbf{Cells}\hfill]
\ifpdf
\begin{flushleft}
\fi
\begin{ttfamily}
public Cells: TStringList;\end{ttfamily}

\ifpdf
\end{flushleft}
\fi


\par Each item on this list is already converted (with @{-}tags parsed, converted by ConvertString etc.) content of given cell tag.\end{list}
\subsubsection*{\large{\textbf{Methods}}\normalsize\hspace{1ex}\hfill}
\paragraph*{Create}\hspace*{\fill}

\label{PasDoc_Gen.TRowData-Create}
\index{Create}
\begin{list}{}{
\settowidth{\tmplength}{\textbf{Description}}
\setlength{\itemindent}{0cm}
\setlength{\listparindent}{0cm}
\setlength{\leftmargin}{\evensidemargin}
\addtolength{\leftmargin}{\tmplength}
\settowidth{\labelsep}{X}
\addtolength{\leftmargin}{\labelsep}
\setlength{\labelwidth}{\tmplength}
}
\item[\textbf{Declaration}\hfill]
\ifpdf
\begin{flushleft}
\fi
\begin{ttfamily}
public constructor Create;\end{ttfamily}

\ifpdf
\end{flushleft}
\fi

\end{list}
\paragraph*{Destroy}\hspace*{\fill}

\label{PasDoc_Gen.TRowData-Destroy}
\index{Destroy}
\begin{list}{}{
\settowidth{\tmplength}{\textbf{Description}}
\setlength{\itemindent}{0cm}
\setlength{\listparindent}{0cm}
\setlength{\leftmargin}{\evensidemargin}
\addtolength{\leftmargin}{\tmplength}
\settowidth{\labelsep}{X}
\addtolength{\leftmargin}{\labelsep}
\setlength{\labelwidth}{\tmplength}
}
\item[\textbf{Declaration}\hfill]
\ifpdf
\begin{flushleft}
\fi
\begin{ttfamily}
public destructor Destroy; override;\end{ttfamily}

\ifpdf
\end{flushleft}
\fi

\end{list}
\ifpdf
\subsection*{\large{\textbf{TTableData Class}}\normalsize\hspace{1ex}\hrulefill}
\else
\subsection*{TTableData Class}
\fi
\label{PasDoc_Gen.TTableData}
\index{TTableData}
\subsubsection*{\large{\textbf{Hierarchy}}\normalsize\hspace{1ex}\hfill}
TTableData {$>$} \begin{ttfamily}TObjectVector\end{ttfamily}(\ref{PasDoc_ObjectVector.TObjectVector}) {$>$} 
TObjectList
\subsubsection*{\large{\textbf{Description}}\normalsize\hspace{1ex}\hfill}
Collected information about @table. Passed to \begin{ttfamily}TDocGenerator.FormatTable\end{ttfamily}(\ref{PasDoc_Gen.TDocGenerator-FormatTable}). Every item of this list should be non{-}nil instance of \begin{ttfamily}TRowData\end{ttfamily}(\ref{PasDoc_Gen.TRowData}).\subsubsection*{\large{\textbf{Properties}}\normalsize\hspace{1ex}\hfill}
\begin{list}{}{
\settowidth{\tmplength}{\textbf{MaxCellCount}}
\setlength{\itemindent}{0cm}
\setlength{\listparindent}{0cm}
\setlength{\leftmargin}{\evensidemargin}
\addtolength{\leftmargin}{\tmplength}
\settowidth{\labelsep}{X}
\addtolength{\leftmargin}{\labelsep}
\setlength{\labelwidth}{\tmplength}
}
\label{PasDoc_Gen.TTableData-MaxCellCount}
\index{MaxCellCount}
\item[\textbf{MaxCellCount}\hfill]
\ifpdf
\begin{flushleft}
\fi
\begin{ttfamily}
public property MaxCellCount: Cardinal read FMaxCellCount;\end{ttfamily}

\ifpdf
\end{flushleft}
\fi


\par Maximum Cells.Count, considering all rows.\label{PasDoc_Gen.TTableData-MinCellCount}
\index{MinCellCount}
\item[\textbf{MinCellCount}\hfill]
\ifpdf
\begin{flushleft}
\fi
\begin{ttfamily}
public property MinCellCount: Cardinal read FMinCellCount;\end{ttfamily}

\ifpdf
\end{flushleft}
\fi


\par Minimum Cells.Count, considering all rows.\end{list}
\ifpdf
\subsection*{\large{\textbf{TDocGenerator Class}}\normalsize\hspace{1ex}\hrulefill}
\else
\subsection*{TDocGenerator Class}
\fi
\label{PasDoc_Gen.TDocGenerator}
\index{TDocGenerator}
\subsubsection*{\large{\textbf{Hierarchy}}\normalsize\hspace{1ex}\hfill}
TDocGenerator {$>$} TComponent
\subsubsection*{\large{\textbf{Description}}\normalsize\hspace{1ex}\hfill}
basic documentation generator object\hfill\vspace*{1ex}

 This abstract object will do the complete process of writing documentation files. It will be given the collection of units that was the result of the parsing process and a configuration object that was created from default values and program parameters. Depending on the output format, one or more files may be created (HTML will create several, Tex only one).\subsubsection*{\large{\textbf{Properties}}\normalsize\hspace{1ex}\hfill}
\begin{list}{}{
\settowidth{\tmplength}{\textbf{OutputGraphVizClassHierarchy}}
\setlength{\itemindent}{0cm}
\setlength{\listparindent}{0cm}
\setlength{\leftmargin}{\evensidemargin}
\addtolength{\leftmargin}{\tmplength}
\settowidth{\labelsep}{X}
\addtolength{\leftmargin}{\labelsep}
\setlength{\labelwidth}{\tmplength}
}
\label{PasDoc_Gen.TDocGenerator-CurrentStream}
\index{CurrentStream}
\item[\textbf{CurrentStream}\hfill]
\ifpdf
\begin{flushleft}
\fi
\begin{ttfamily}
protected property CurrentStream: TStream read FCurrentStream;\end{ttfamily}

\ifpdf
\end{flushleft}
\fi


\par  \label{PasDoc_Gen.TDocGenerator-Units}
\index{Units}
\item[\textbf{Units}\hfill]
\ifpdf
\begin{flushleft}
\fi
\begin{ttfamily}
public property Units: TPasUnits read FUnits write FUnits;\end{ttfamily}

\ifpdf
\end{flushleft}
\fi


\par  \label{PasDoc_Gen.TDocGenerator-Introduction}
\index{Introduction}
\item[\textbf{Introduction}\hfill]
\ifpdf
\begin{flushleft}
\fi
\begin{ttfamily}
public property Introduction: TExternalItem read FIntroduction
      write FIntroduction;\end{ttfamily}

\ifpdf
\end{flushleft}
\fi


\par  \label{PasDoc_Gen.TDocGenerator-Conclusion}
\index{Conclusion}
\item[\textbf{Conclusion}\hfill]
\ifpdf
\begin{flushleft}
\fi
\begin{ttfamily}
public property Conclusion: TExternalItem read FConclusion write FConclusion;\end{ttfamily}

\ifpdf
\end{flushleft}
\fi


\par  \label{PasDoc_Gen.TDocGenerator-AdditionalFiles}
\index{AdditionalFiles}
\item[\textbf{AdditionalFiles}\hfill]
\ifpdf
\begin{flushleft}
\fi
\begin{ttfamily}
public property AdditionalFiles: TExternalItemList read FAdditionalFiles write FAdditionalFiles;\end{ttfamily}

\ifpdf
\end{flushleft}
\fi


\par  \label{PasDoc_Gen.TDocGenerator-OnMessage}
\index{OnMessage}
\item[\textbf{OnMessage}\hfill]
\ifpdf
\begin{flushleft}
\fi
\begin{ttfamily}
public property OnMessage: TPasDocMessageEvent read FOnMessage write FOnMessage;\end{ttfamily}

\ifpdf
\end{flushleft}
\fi


\par Callback receiving messages from generator.

This is usually used internally by TPasDoc class, that assigns it's internal callback here when using this generator. Also, for the above reason, do not make this published.

See TPasDoc.OnMessage for something more useful for final programs.\label{PasDoc_Gen.TDocGenerator-Language}
\index{Language}
\item[\textbf{Language}\hfill]
\ifpdf
\begin{flushleft}
\fi
\begin{ttfamily}
published property Language: TLanguageID read GetLanguage write SetLanguage
      default DEFAULT{\_}LANGUAGE;\end{ttfamily}

\ifpdf
\end{flushleft}
\fi


\par the (human) output language of the documentation file(s)\label{PasDoc_Gen.TDocGenerator-ProjectName}
\index{ProjectName}
\item[\textbf{ProjectName}\hfill]
\ifpdf
\begin{flushleft}
\fi
\begin{ttfamily}
published property ProjectName: string read FProjectName write FProjectName;\end{ttfamily}

\ifpdf
\end{flushleft}
\fi


\par Name of the project to create.\label{PasDoc_Gen.TDocGenerator-ExcludeGenerator}
\index{ExcludeGenerator}
\item[\textbf{ExcludeGenerator}\hfill]
\ifpdf
\begin{flushleft}
\fi
\begin{ttfamily}
published property ExcludeGenerator: Boolean
      read FExcludeGenerator write FExcludeGenerator default false;\end{ttfamily}

\ifpdf
\end{flushleft}
\fi


\par "Generator info" are things that can change with each invocation of pasdoc, with different pasdoc binary etc.

This includes \begin{itemize}
\item pasdoc's compiler name and version,
\item pasdoc's version and time of compilation
\end{itemize} See [\href{https://github.com/pasdoc/pasdoc/wiki/ExcludeGeneratorOption}{https://github.com/pasdoc/pasdoc/wiki/ExcludeGeneratorOption}]. Default value is false (i.e. show them), as this information is generally considered useful.

Setting this to true is useful for automatically comparing two versions of pasdoc's output (e.g. when trying to automate pasdoc's tests).\label{PasDoc_Gen.TDocGenerator-IncludeCreationTime}
\index{IncludeCreationTime}
\item[\textbf{IncludeCreationTime}\hfill]
\ifpdf
\begin{flushleft}
\fi
\begin{ttfamily}
published property IncludeCreationTime: Boolean
      read FIncludeCreationTime write FIncludeCreationTime default false;\end{ttfamily}

\ifpdf
\end{flushleft}
\fi


\par Show creation time in the output.\label{PasDoc_Gen.TDocGenerator-UseLowercaseKeywords}
\index{UseLowercaseKeywords}
\item[\textbf{UseLowercaseKeywords}\hfill]
\ifpdf
\begin{flushleft}
\fi
\begin{ttfamily}
published property UseLowercaseKeywords: Boolean
      read FUseLowercaseKeywords write FUseLowercaseKeywords default false;\end{ttfamily}

\ifpdf
\end{flushleft}
\fi


\par Setting to define how literal tag keywords should appear in documentaion.\label{PasDoc_Gen.TDocGenerator-Title}
\index{Title}
\item[\textbf{Title}\hfill]
\ifpdf
\begin{flushleft}
\fi
\begin{ttfamily}
published property Title: string read FTitle write FTitle;\end{ttfamily}

\ifpdf
\end{flushleft}
\fi


\par Title of the documentation, supplied by user. May be empty. See \begin{ttfamily}TPasDoc.Title\end{ttfamily}(\ref{PasDoc_Base.TPasDoc-Title}).\label{PasDoc_Gen.TDocGenerator-DestinationDirectory}
\index{DestinationDirectory}
\item[\textbf{DestinationDirectory}\hfill]
\ifpdf
\begin{flushleft}
\fi
\begin{ttfamily}
published property DestinationDirectory: string read FDestDir write SetDestDir;\end{ttfamily}

\ifpdf
\end{flushleft}
\fi


\par Destination directory for documentation. Must include terminating forward slash or backslash so that valid file names can be created by concatenating DestinationDirectory and a pathless file name.\label{PasDoc_Gen.TDocGenerator-OutputGraphVizUses}
\index{OutputGraphVizUses}
\item[\textbf{OutputGraphVizUses}\hfill]
\ifpdf
\begin{flushleft}
\fi
\begin{ttfamily}
published property OutputGraphVizUses: boolean read FGraphVizUses write FGraphVizUses
      default false;\end{ttfamily}

\ifpdf
\end{flushleft}
\fi


\par generate a GraphViz diagram for the units dependencies\label{PasDoc_Gen.TDocGenerator-OutputGraphVizClassHierarchy}
\index{OutputGraphVizClassHierarchy}
\item[\textbf{OutputGraphVizClassHierarchy}\hfill]
\ifpdf
\begin{flushleft}
\fi
\begin{ttfamily}
published property OutputGraphVizClassHierarchy: boolean
      read FGraphVizClasses write FGraphVizClasses default false;\end{ttfamily}

\ifpdf
\end{flushleft}
\fi


\par generate a GraphViz diagram for the Class hierarchy\label{PasDoc_Gen.TDocGenerator-LinkGraphVizUses}
\index{LinkGraphVizUses}
\item[\textbf{LinkGraphVizUses}\hfill]
\ifpdf
\begin{flushleft}
\fi
\begin{ttfamily}
published property LinkGraphVizUses: string read FLinkGraphVizUses write FLinkGraphVizUses;\end{ttfamily}

\ifpdf
\end{flushleft}
\fi


\par link the GraphViz uses diagram\label{PasDoc_Gen.TDocGenerator-LinkGraphVizClasses}
\index{LinkGraphVizClasses}
\item[\textbf{LinkGraphVizClasses}\hfill]
\ifpdf
\begin{flushleft}
\fi
\begin{ttfamily}
published property LinkGraphVizClasses: string read FLinkGraphVizClasses write FLinkGraphVizClasses;\end{ttfamily}

\ifpdf
\end{flushleft}
\fi


\par link the GraphViz classes diagram\label{PasDoc_Gen.TDocGenerator-Abbreviations}
\index{Abbreviations}
\item[\textbf{Abbreviations}\hfill]
\ifpdf
\begin{flushleft}
\fi
\begin{ttfamily}
published property Abbreviations: TStringList read FAbbreviations write SetAbbreviations;\end{ttfamily}

\ifpdf
\end{flushleft}
\fi


\par  \label{PasDoc_Gen.TDocGenerator-CheckSpelling}
\index{CheckSpelling}
\item[\textbf{CheckSpelling}\hfill]
\ifpdf
\begin{flushleft}
\fi
\begin{ttfamily}
published property CheckSpelling: boolean read FCheckSpelling write FCheckSpelling
      default false;\end{ttfamily}

\ifpdf
\end{flushleft}
\fi


\par  \label{PasDoc_Gen.TDocGenerator-AspellLanguage}
\index{AspellLanguage}
\item[\textbf{AspellLanguage}\hfill]
\ifpdf
\begin{flushleft}
\fi
\begin{ttfamily}
published property AspellLanguage: string read FAspellLanguage write FAspellLanguage;\end{ttfamily}

\ifpdf
\end{flushleft}
\fi


\par  \label{PasDoc_Gen.TDocGenerator-SpellCheckIgnoreWords}
\index{SpellCheckIgnoreWords}
\item[\textbf{SpellCheckIgnoreWords}\hfill]
\ifpdf
\begin{flushleft}
\fi
\begin{ttfamily}
published property SpellCheckIgnoreWords: TStringList
      read FSpellCheckIgnoreWords write SetSpellCheckIgnoreWords;\end{ttfamily}

\ifpdf
\end{flushleft}
\fi


\par  \label{PasDoc_Gen.TDocGenerator-AutoAbstract}
\index{AutoAbstract}
\item[\textbf{AutoAbstract}\hfill]
\ifpdf
\begin{flushleft}
\fi
\begin{ttfamily}
published property AutoAbstract: boolean read FAutoAbstract write FAutoAbstract default false;\end{ttfamily}

\ifpdf
\end{flushleft}
\fi


\par The meaning of this is just like {-}{-}auto{-}abstract command{-}line option. It is used in \begin{ttfamily}ExpandDescriptions\end{ttfamily}(\ref{PasDoc_Gen.TDocGenerator-ExpandDescriptions}).\label{PasDoc_Gen.TDocGenerator-LinkLook}
\index{LinkLook}
\item[\textbf{LinkLook}\hfill]
\ifpdf
\begin{flushleft}
\fi
\begin{ttfamily}
published property LinkLook: TLinkLook read FLinkLook write FLinkLook default llDefault;\end{ttfamily}

\ifpdf
\end{flushleft}
\fi


\par This controls \begin{ttfamily}SearchLink\end{ttfamily}(\ref{PasDoc_Gen.TDocGenerator-SearchLink}) behavior, as described in [\href{https://github.com/pasdoc/pasdoc/wiki/LinkLookOption}{https://github.com/pasdoc/pasdoc/wiki/LinkLookOption}].\label{PasDoc_Gen.TDocGenerator-WriteUsesClause}
\index{WriteUsesClause}
\item[\textbf{WriteUsesClause}\hfill]
\ifpdf
\begin{flushleft}
\fi
\begin{ttfamily}
published property WriteUsesClause: boolean
      read FWriteUsesClause write FWriteUsesClause default false;\end{ttfamily}

\ifpdf
\end{flushleft}
\fi


\par  \label{PasDoc_Gen.TDocGenerator-AutoLink}
\index{AutoLink}
\item[\textbf{AutoLink}\hfill]
\ifpdf
\begin{flushleft}
\fi
\begin{ttfamily}
published property AutoLink: boolean
      read FAutoLink write FAutoLink default false;\end{ttfamily}

\ifpdf
\end{flushleft}
\fi


\par This controls auto{-}linking, see [\href{https://github.com/pasdoc/pasdoc/wiki/AutoLinkOption}{https://github.com/pasdoc/pasdoc/wiki/AutoLinkOption}]\label{PasDoc_Gen.TDocGenerator-AutoLinkExclude}
\index{AutoLinkExclude}
\item[\textbf{AutoLinkExclude}\hfill]
\ifpdf
\begin{flushleft}
\fi
\begin{ttfamily}
published property AutoLinkExclude: TStringList read FAutoLinkExclude;\end{ttfamily}

\ifpdf
\end{flushleft}
\fi


\par  \label{PasDoc_Gen.TDocGenerator-ExternalClassHierarchy}
\index{ExternalClassHierarchy}
\item[\textbf{ExternalClassHierarchy}\hfill]
\ifpdf
\begin{flushleft}
\fi
\begin{ttfamily}
published property ExternalClassHierarchy: TStrings
      read FExternalClassHierarchy write SetExternalClassHierarchy
      stored StoredExternalClassHierarchy;\end{ttfamily}

\ifpdf
\end{flushleft}
\fi


\par  \label{PasDoc_Gen.TDocGenerator-Markdown}
\index{Markdown}
\item[\textbf{Markdown}\hfill]
\ifpdf
\begin{flushleft}
\fi
\begin{ttfamily}
published property Markdown: boolean
      read FMarkdown write FMarkdown default false;\end{ttfamily}

\ifpdf
\end{flushleft}
\fi


\par  \end{list}
\subsubsection*{\large{\textbf{Fields}}\normalsize\hspace{1ex}\hfill}
\begin{list}{}{
\settowidth{\tmplength}{\textbf{FClassHierarchy}}
\setlength{\itemindent}{0cm}
\setlength{\listparindent}{0cm}
\setlength{\leftmargin}{\evensidemargin}
\addtolength{\leftmargin}{\tmplength}
\settowidth{\labelsep}{X}
\addtolength{\leftmargin}{\labelsep}
\setlength{\labelwidth}{\tmplength}
}
\label{PasDoc_Gen.TDocGenerator-FLanguage}
\index{FLanguage}
\item[\textbf{FLanguage}\hfill]
\ifpdf
\begin{flushleft}
\fi
\begin{ttfamily}
protected FLanguage: TPasDocLanguages;\end{ttfamily}

\ifpdf
\end{flushleft}
\fi


\par the (human) output language of the documentation file(s)\label{PasDoc_Gen.TDocGenerator-FClassHierarchy}
\index{FClassHierarchy}
\item[\textbf{FClassHierarchy}\hfill]
\ifpdf
\begin{flushleft}
\fi
\begin{ttfamily}
protected FClassHierarchy: TStringCardinalTree;\end{ttfamily}

\ifpdf
\end{flushleft}
\fi


\par  \label{PasDoc_Gen.TDocGenerator-FUnits}
\index{FUnits}
\item[\textbf{FUnits}\hfill]
\ifpdf
\begin{flushleft}
\fi
\begin{ttfamily}
protected FUnits: TPasUnits;\end{ttfamily}

\ifpdf
\end{flushleft}
\fi


\par list of all units that were successfully parsed\end{list}
\subsubsection*{\large{\textbf{Methods}}\normalsize\hspace{1ex}\hfill}
\paragraph*{DoError}\hspace*{\fill}

\label{PasDoc_Gen.TDocGenerator-DoError}
\index{DoError}
\begin{list}{}{
\settowidth{\tmplength}{\textbf{Description}}
\setlength{\itemindent}{0cm}
\setlength{\listparindent}{0cm}
\setlength{\leftmargin}{\evensidemargin}
\addtolength{\leftmargin}{\tmplength}
\settowidth{\labelsep}{X}
\addtolength{\leftmargin}{\labelsep}
\setlength{\labelwidth}{\tmplength}
}
\item[\textbf{Declaration}\hfill]
\ifpdf
\begin{flushleft}
\fi
\begin{ttfamily}
protected procedure DoError(const AMessage: string; const AArguments: array of const; const AExitCode: Word);\end{ttfamily}

\ifpdf
\end{flushleft}
\fi

\end{list}
\paragraph*{DoMessage}\hspace*{\fill}

\label{PasDoc_Gen.TDocGenerator-DoMessage}
\index{DoMessage}
\begin{list}{}{
\settowidth{\tmplength}{\textbf{Description}}
\setlength{\itemindent}{0cm}
\setlength{\listparindent}{0cm}
\setlength{\leftmargin}{\evensidemargin}
\addtolength{\leftmargin}{\tmplength}
\settowidth{\labelsep}{X}
\addtolength{\leftmargin}{\labelsep}
\setlength{\labelwidth}{\tmplength}
}
\item[\textbf{Declaration}\hfill]
\ifpdf
\begin{flushleft}
\fi
\begin{ttfamily}
protected procedure DoMessage(const AVerbosity: Cardinal; const MessageType: TPasDocMessageType; const AMessage: string; const AArguments: array of const);\end{ttfamily}

\ifpdf
\end{flushleft}
\fi

\end{list}
\paragraph*{CreateClassHierarchy}\hspace*{\fill}

\label{PasDoc_Gen.TDocGenerator-CreateClassHierarchy}
\index{CreateClassHierarchy}
\begin{list}{}{
\settowidth{\tmplength}{\textbf{Description}}
\setlength{\itemindent}{0cm}
\setlength{\listparindent}{0cm}
\setlength{\leftmargin}{\evensidemargin}
\addtolength{\leftmargin}{\tmplength}
\settowidth{\labelsep}{X}
\addtolength{\leftmargin}{\labelsep}
\setlength{\labelwidth}{\tmplength}
}
\item[\textbf{Declaration}\hfill]
\ifpdf
\begin{flushleft}
\fi
\begin{ttfamily}
protected procedure CreateClassHierarchy;\end{ttfamily}

\ifpdf
\end{flushleft}
\fi

\end{list}
\paragraph*{MakeItemLink}\hspace*{\fill}

\label{PasDoc_Gen.TDocGenerator-MakeItemLink}
\index{MakeItemLink}
\begin{list}{}{
\settowidth{\tmplength}{\textbf{Description}}
\setlength{\itemindent}{0cm}
\setlength{\listparindent}{0cm}
\setlength{\leftmargin}{\evensidemargin}
\addtolength{\leftmargin}{\tmplength}
\settowidth{\labelsep}{X}
\addtolength{\leftmargin}{\labelsep}
\setlength{\labelwidth}{\tmplength}
}
\item[\textbf{Declaration}\hfill]
\ifpdf
\begin{flushleft}
\fi
\begin{ttfamily}
protected function MakeItemLink(const Item: TBaseItem; const LinkCaption: string; const LinkContext: TLinkContext): string; virtual;\end{ttfamily}

\ifpdf
\end{flushleft}
\fi

\par
\item[\textbf{Description}]
Return a link to item Item which will be displayed as LinkCaption. Returned string may be directly inserted inside output documentation. LinkCaption will be always converted using ConvertString before writing, so don't worry about doing this yourself when calling this method.

LinkContext may be used in some descendants to present the link differently, see \begin{ttfamily}TLinkContext\end{ttfamily}(\ref{PasDoc_Gen-TLinkContext}) for it's meaning.

If some output format doesn't support this feature, it can return simply ConvertString(LinkCaption). This is the default implementation of this method in this class.

\end{list}
\paragraph*{WriteCodeWithLinksCommon}\hspace*{\fill}

\label{PasDoc_Gen.TDocGenerator-WriteCodeWithLinksCommon}
\index{WriteCodeWithLinksCommon}
\begin{list}{}{
\settowidth{\tmplength}{\textbf{Description}}
\setlength{\itemindent}{0cm}
\setlength{\listparindent}{0cm}
\setlength{\leftmargin}{\evensidemargin}
\addtolength{\leftmargin}{\tmplength}
\settowidth{\labelsep}{X}
\addtolength{\leftmargin}{\labelsep}
\setlength{\labelwidth}{\tmplength}
}
\item[\textbf{Declaration}\hfill]
\ifpdf
\begin{flushleft}
\fi
\begin{ttfamily}
protected procedure WriteCodeWithLinksCommon(const Item: TPasItem; const Code: string; WriteItemLink: boolean; const NameLinkBegin, NameLinkEnd: string);\end{ttfamily}

\ifpdf
\end{flushleft}
\fi

\par
\item[\textbf{Description}]
This writes Code as a Pascal code. Links inside the code are resolved from Item. If WriteItemLink then Item.Name is made a link. Item.Name is printed between NameLinkBegin and NameLinkEnd.

\end{list}
\paragraph*{CloseStream}\hspace*{\fill}

\label{PasDoc_Gen.TDocGenerator-CloseStream}
\index{CloseStream}
\begin{list}{}{
\settowidth{\tmplength}{\textbf{Description}}
\setlength{\itemindent}{0cm}
\setlength{\listparindent}{0cm}
\setlength{\leftmargin}{\evensidemargin}
\addtolength{\leftmargin}{\tmplength}
\settowidth{\labelsep}{X}
\addtolength{\leftmargin}{\labelsep}
\setlength{\labelwidth}{\tmplength}
}
\item[\textbf{Declaration}\hfill]
\ifpdf
\begin{flushleft}
\fi
\begin{ttfamily}
protected procedure CloseStream;\end{ttfamily}

\ifpdf
\end{flushleft}
\fi

\par
\item[\textbf{Description}]
If field \begin{ttfamily}CurrentStream\end{ttfamily}(\ref{PasDoc_Gen.TDocGenerator-CurrentStream}) is assigned, it is disposed and set to nil.

\end{list}
\paragraph*{CodeString}\hspace*{\fill}

\label{PasDoc_Gen.TDocGenerator-CodeString}
\index{CodeString}
\begin{list}{}{
\settowidth{\tmplength}{\textbf{Description}}
\setlength{\itemindent}{0cm}
\setlength{\listparindent}{0cm}
\setlength{\leftmargin}{\evensidemargin}
\addtolength{\leftmargin}{\tmplength}
\settowidth{\labelsep}{X}
\addtolength{\leftmargin}{\labelsep}
\setlength{\labelwidth}{\tmplength}
}
\item[\textbf{Declaration}\hfill]
\ifpdf
\begin{flushleft}
\fi
\begin{ttfamily}
protected function CodeString(const s: string): string; virtual; abstract;\end{ttfamily}

\ifpdf
\end{flushleft}
\fi

\par
\item[\textbf{Description}]
Makes a String look like a coded String, i.e. {$<$}CODE{$>$}TheString{$<$}/CODE{$>$} in Html.\hfill\vspace*{1ex}

  \par
\item[\textbf{Parameters}]
\begin{description}
\item[s] is the string to format
\end{description}
\item[\textbf{Returns}]the formatted string


\end{list}
\paragraph*{ConvertString}\hspace*{\fill}

\label{PasDoc_Gen.TDocGenerator-ConvertString}
\index{ConvertString}
\begin{list}{}{
\settowidth{\tmplength}{\textbf{Description}}
\setlength{\itemindent}{0cm}
\setlength{\listparindent}{0cm}
\setlength{\leftmargin}{\evensidemargin}
\addtolength{\leftmargin}{\tmplength}
\settowidth{\labelsep}{X}
\addtolength{\leftmargin}{\labelsep}
\setlength{\labelwidth}{\tmplength}
}
\item[\textbf{Declaration}\hfill]
\ifpdf
\begin{flushleft}
\fi
\begin{ttfamily}
protected function ConvertString(const s: string): string; virtual; abstract;\end{ttfamily}

\ifpdf
\end{flushleft}
\fi

\par
\item[\textbf{Description}]
Converts for each character in S, thus assembling a String that is returned and can be written to the documentation file.

The @ character should not be converted, this will be done later on.

\end{list}
\paragraph*{ConvertChar}\hspace*{\fill}

\label{PasDoc_Gen.TDocGenerator-ConvertChar}
\index{ConvertChar}
\begin{list}{}{
\settowidth{\tmplength}{\textbf{Description}}
\setlength{\itemindent}{0cm}
\setlength{\listparindent}{0cm}
\setlength{\leftmargin}{\evensidemargin}
\addtolength{\leftmargin}{\tmplength}
\settowidth{\labelsep}{X}
\addtolength{\leftmargin}{\labelsep}
\setlength{\labelwidth}{\tmplength}
}
\item[\textbf{Declaration}\hfill]
\ifpdf
\begin{flushleft}
\fi
\begin{ttfamily}
protected function ConvertChar(c: char): string; virtual; abstract;\end{ttfamily}

\ifpdf
\end{flushleft}
\fi

\par
\item[\textbf{Description}]
Converts a character to its converted form. This method should always be called to add characters to a string.

@ should also be converted by this routine.

\end{list}
\paragraph*{CreateLink}\hspace*{\fill}

\label{PasDoc_Gen.TDocGenerator-CreateLink}
\index{CreateLink}
\begin{list}{}{
\settowidth{\tmplength}{\textbf{Description}}
\setlength{\itemindent}{0cm}
\setlength{\listparindent}{0cm}
\setlength{\leftmargin}{\evensidemargin}
\addtolength{\leftmargin}{\tmplength}
\settowidth{\labelsep}{X}
\addtolength{\leftmargin}{\labelsep}
\setlength{\labelwidth}{\tmplength}
}
\item[\textbf{Declaration}\hfill]
\ifpdf
\begin{flushleft}
\fi
\begin{ttfamily}
protected function CreateLink(const Item: TBaseItem): string; virtual;\end{ttfamily}

\ifpdf
\end{flushleft}
\fi

\par
\item[\textbf{Description}]
This function is supposed to return a reference to an item, that is the name combined with some linking information like a hyperlink element in HTML or a page number in Tex.

\end{list}
\paragraph*{CreateStream}\hspace*{\fill}

\label{PasDoc_Gen.TDocGenerator-CreateStream}
\index{CreateStream}
\begin{list}{}{
\settowidth{\tmplength}{\textbf{Description}}
\setlength{\itemindent}{0cm}
\setlength{\listparindent}{0cm}
\setlength{\leftmargin}{\evensidemargin}
\addtolength{\leftmargin}{\tmplength}
\settowidth{\labelsep}{X}
\addtolength{\leftmargin}{\labelsep}
\setlength{\labelwidth}{\tmplength}
}
\item[\textbf{Declaration}\hfill]
\ifpdf
\begin{flushleft}
\fi
\begin{ttfamily}
protected function CreateStream(const AName: string): Boolean;\end{ttfamily}

\ifpdf
\end{flushleft}
\fi

\par
\item[\textbf{Description}]
Open output stream in the destination directory. If \begin{ttfamily}CurrentStream\end{ttfamily}(\ref{PasDoc_Gen.TDocGenerator-CurrentStream}) still exists ({$<$}{$>$} nil), it is closed. Then, a new output stream in the destination directory is created and assigned to \begin{ttfamily}CurrentStream\end{ttfamily}(\ref{PasDoc_Gen.TDocGenerator-CurrentStream}). The file is overwritten if exists.

Use this only for text files that you want to write using WriteXxx methods of this class (like WriteConverted). There's no point to use if for other files.

Returns \begin{ttfamily}True\end{ttfamily} if creation was successful, \begin{ttfamily}False\end{ttfamily} otherwise. When it returns \begin{ttfamily}False\end{ttfamily}, the error message was already shown by DoMessage.

\end{list}
\paragraph*{ExtractEmailAddress}\hspace*{\fill}

\label{PasDoc_Gen.TDocGenerator-ExtractEmailAddress}
\index{ExtractEmailAddress}
\begin{list}{}{
\settowidth{\tmplength}{\textbf{Description}}
\setlength{\itemindent}{0cm}
\setlength{\listparindent}{0cm}
\setlength{\leftmargin}{\evensidemargin}
\addtolength{\leftmargin}{\tmplength}
\settowidth{\labelsep}{X}
\addtolength{\leftmargin}{\labelsep}
\setlength{\labelwidth}{\tmplength}
}
\item[\textbf{Declaration}\hfill]
\ifpdf
\begin{flushleft}
\fi
\begin{ttfamily}
protected function ExtractEmailAddress(s: string; out S1, S2, EmailAddress: string): Boolean;\end{ttfamily}

\ifpdf
\end{flushleft}
\fi

\par
\item[\textbf{Description}]
Searches for an email address in String S. Searches for first appearance of the @ character

\end{list}
\paragraph*{FixEmailaddressWithoutMailTo}\hspace*{\fill}

\label{PasDoc_Gen.TDocGenerator-FixEmailaddressWithoutMailTo}
\index{FixEmailaddressWithoutMailTo}
\begin{list}{}{
\settowidth{\tmplength}{\textbf{Description}}
\setlength{\itemindent}{0cm}
\setlength{\listparindent}{0cm}
\setlength{\leftmargin}{\evensidemargin}
\addtolength{\leftmargin}{\tmplength}
\settowidth{\labelsep}{X}
\addtolength{\leftmargin}{\labelsep}
\setlength{\labelwidth}{\tmplength}
}
\item[\textbf{Declaration}\hfill]
\ifpdf
\begin{flushleft}
\fi
\begin{ttfamily}
protected function FixEmailaddressWithoutMailTo(const PossibleEmailAddress: String): String;\end{ttfamily}

\ifpdf
\end{flushleft}
\fi

\par
\item[\textbf{Description}]
Searches for an email address in PossibleEmailAddress and appends mailto: if it's an email address and mailto: wasn't provided. Otherwise it simply returns the input.

Needed to link email addresses properly which doesn't start with mailto:

\end{list}
\paragraph*{ExtractWebAddress}\hspace*{\fill}

\label{PasDoc_Gen.TDocGenerator-ExtractWebAddress}
\index{ExtractWebAddress}
\begin{list}{}{
\settowidth{\tmplength}{\textbf{Description}}
\setlength{\itemindent}{0cm}
\setlength{\listparindent}{0cm}
\setlength{\leftmargin}{\evensidemargin}
\addtolength{\leftmargin}{\tmplength}
\settowidth{\labelsep}{X}
\addtolength{\leftmargin}{\labelsep}
\setlength{\labelwidth}{\tmplength}
}
\item[\textbf{Declaration}\hfill]
\ifpdf
\begin{flushleft}
\fi
\begin{ttfamily}
protected function ExtractWebAddress(s: string; out S1, S2, WebAddress: string): Boolean;\end{ttfamily}

\ifpdf
\end{flushleft}
\fi

\par
\item[\textbf{Description}]
Searches for a web address in String S. It must either contain a \href{http://}{http://} or start with www.

\end{list}
\paragraph*{FindGlobal}\hspace*{\fill}

\label{PasDoc_Gen.TDocGenerator-FindGlobal}
\index{FindGlobal}
\begin{list}{}{
\settowidth{\tmplength}{\textbf{Description}}
\setlength{\itemindent}{0cm}
\setlength{\listparindent}{0cm}
\setlength{\leftmargin}{\evensidemargin}
\addtolength{\leftmargin}{\tmplength}
\settowidth{\labelsep}{X}
\addtolength{\leftmargin}{\labelsep}
\setlength{\labelwidth}{\tmplength}
}
\item[\textbf{Declaration}\hfill]
\ifpdf
\begin{flushleft}
\fi
\begin{ttfamily}
protected function FindGlobal(const NameParts: TNameParts): TBaseItem;\end{ttfamily}

\ifpdf
\end{flushleft}
\fi

\par
\item[\textbf{Description}]
Searches all items in all units (given by field \begin{ttfamily}Units\end{ttfamily}(\ref{PasDoc_Gen.TDocGenerator-Units})) for item with NameParts. Returns a pointer to the item on success, nil otherwise.

\end{list}
\paragraph*{FindGlobalPasItem}\hspace*{\fill}

\label{PasDoc_Gen.TDocGenerator-FindGlobalPasItem}
\index{FindGlobalPasItem}
\begin{list}{}{
\settowidth{\tmplength}{\textbf{Description}}
\setlength{\itemindent}{0cm}
\setlength{\listparindent}{0cm}
\setlength{\leftmargin}{\evensidemargin}
\addtolength{\leftmargin}{\tmplength}
\settowidth{\labelsep}{X}
\addtolength{\leftmargin}{\labelsep}
\setlength{\labelwidth}{\tmplength}
}
\item[\textbf{Declaration}\hfill]
\ifpdf
\begin{flushleft}
\fi
\begin{ttfamily}
protected function FindGlobalPasItem(const NameParts: TNameParts): TPasItem; overload;\end{ttfamily}

\ifpdf
\end{flushleft}
\fi

\par
\item[\textbf{Description}]
Find a Pascal item, searching global namespace. Returns \begin{ttfamily}Nil\end{ttfamily} if not found.

\end{list}
\paragraph*{FindGlobalPasItem}\hspace*{\fill}

\label{PasDoc_Gen.TDocGenerator-FindGlobalPasItem}
\index{FindGlobalPasItem}
\begin{list}{}{
\settowidth{\tmplength}{\textbf{Description}}
\setlength{\itemindent}{0cm}
\setlength{\listparindent}{0cm}
\setlength{\leftmargin}{\evensidemargin}
\addtolength{\leftmargin}{\tmplength}
\settowidth{\labelsep}{X}
\addtolength{\leftmargin}{\labelsep}
\setlength{\labelwidth}{\tmplength}
}
\item[\textbf{Declaration}\hfill]
\ifpdf
\begin{flushleft}
\fi
\begin{ttfamily}
protected function FindGlobalPasItem(const ItemName: String): TPasItem; overload;\end{ttfamily}

\ifpdf
\end{flushleft}
\fi

\par
\item[\textbf{Description}]
Find a Pascal item, searching global namespace. Assumes that Name is only one component (not something with dots inside). Returns \begin{ttfamily}Nil\end{ttfamily} if not found.

\end{list}
\paragraph*{GetClassDirectiveName}\hspace*{\fill}

\label{PasDoc_Gen.TDocGenerator-GetClassDirectiveName}
\index{GetClassDirectiveName}
\begin{list}{}{
\settowidth{\tmplength}{\textbf{Description}}
\setlength{\itemindent}{0cm}
\setlength{\listparindent}{0cm}
\setlength{\leftmargin}{\evensidemargin}
\addtolength{\leftmargin}{\tmplength}
\settowidth{\labelsep}{X}
\addtolength{\leftmargin}{\labelsep}
\setlength{\labelwidth}{\tmplength}
}
\item[\textbf{Declaration}\hfill]
\ifpdf
\begin{flushleft}
\fi
\begin{ttfamily}
protected function GetClassDirectiveName(Directive: TClassDirective): string;\end{ttfamily}

\ifpdf
\end{flushleft}
\fi

\par
\item[\textbf{Description}]
\begin{ttfamily}GetClassDirectiveName\end{ttfamily} returns ' abstract', or ' sealed' for classes that abstract or sealed respectively. \begin{ttfamily}GetClassDirectiveName\end{ttfamily} is used by \begin{ttfamily}TTexDocGenerator\end{ttfamily}(\ref{PasDoc_GenLatex.TTexDocGenerator}) and \begin{ttfamily}TGenericHTMLDocGenerator\end{ttfamily}(\ref{PasDoc_GenHtml.TGenericHTMLDocGenerator}) in writing the declaration of the class.

\end{list}
\paragraph*{GetCIOTypeName}\hspace*{\fill}

\label{PasDoc_Gen.TDocGenerator-GetCIOTypeName}
\index{GetCIOTypeName}
\begin{list}{}{
\settowidth{\tmplength}{\textbf{Description}}
\setlength{\itemindent}{0cm}
\setlength{\listparindent}{0cm}
\setlength{\leftmargin}{\evensidemargin}
\addtolength{\leftmargin}{\tmplength}
\settowidth{\labelsep}{X}
\addtolength{\leftmargin}{\labelsep}
\setlength{\labelwidth}{\tmplength}
}
\item[\textbf{Declaration}\hfill]
\ifpdf
\begin{flushleft}
\fi
\begin{ttfamily}
protected function GetCIOTypeName(MyType: TCIOType): string;\end{ttfamily}

\ifpdf
\end{flushleft}
\fi

\par
\item[\textbf{Description}]
\begin{ttfamily}GetCIOTypeName\end{ttfamily} writes a translation of MyType based on the current language. However, 'record' and 'packed record' are not translated.

\end{list}
\paragraph*{LoadDescriptionFile}\hspace*{\fill}

\label{PasDoc_Gen.TDocGenerator-LoadDescriptionFile}
\index{LoadDescriptionFile}
\begin{list}{}{
\settowidth{\tmplength}{\textbf{Description}}
\setlength{\itemindent}{0cm}
\setlength{\listparindent}{0cm}
\setlength{\leftmargin}{\evensidemargin}
\addtolength{\leftmargin}{\tmplength}
\settowidth{\labelsep}{X}
\addtolength{\leftmargin}{\labelsep}
\setlength{\labelwidth}{\tmplength}
}
\item[\textbf{Declaration}\hfill]
\ifpdf
\begin{flushleft}
\fi
\begin{ttfamily}
protected procedure LoadDescriptionFile(n: string);\end{ttfamily}

\ifpdf
\end{flushleft}
\fi

\par
\item[\textbf{Description}]
Loads descriptions from file N and replaces or fills the corresponding comment sections of items.

\end{list}
\paragraph*{SearchItem}\hspace*{\fill}

\label{PasDoc_Gen.TDocGenerator-SearchItem}
\index{SearchItem}
\begin{list}{}{
\settowidth{\tmplength}{\textbf{Description}}
\setlength{\itemindent}{0cm}
\setlength{\listparindent}{0cm}
\setlength{\leftmargin}{\evensidemargin}
\addtolength{\leftmargin}{\tmplength}
\settowidth{\labelsep}{X}
\addtolength{\leftmargin}{\labelsep}
\setlength{\labelwidth}{\tmplength}
}
\item[\textbf{Declaration}\hfill]
\ifpdf
\begin{flushleft}
\fi
\begin{ttfamily}
protected function SearchItem(s: string; const Item: TBaseItem; WarningIfNotSplittable: boolean): TBaseItem;\end{ttfamily}

\ifpdf
\end{flushleft}
\fi

\par
\item[\textbf{Description}]
Searches for item with name S.

If S is not splittable by SplitNameParts, returns nil. If WarningIfNotSplittable, additionally does DoMessage with appropriate warning.

Else (if S is "splittable"), seeks for S (first trying Item.FindName, if Item is not nil, then trying FindGlobal). Returns nil if not found.

\end{list}
\paragraph*{SearchLink}\hspace*{\fill}

\label{PasDoc_Gen.TDocGenerator-SearchLink}
\index{SearchLink}
\begin{list}{}{
\settowidth{\tmplength}{\textbf{Description}}
\setlength{\itemindent}{0cm}
\setlength{\listparindent}{0cm}
\setlength{\leftmargin}{\evensidemargin}
\addtolength{\leftmargin}{\tmplength}
\settowidth{\labelsep}{X}
\addtolength{\leftmargin}{\labelsep}
\setlength{\labelwidth}{\tmplength}
}
\item[\textbf{Declaration}\hfill]
\ifpdf
\begin{flushleft}
\fi
\begin{ttfamily}
protected function SearchLink(s: string; const Item: TBaseItem; const LinkDisplay: string; const WarningIfLinkNotFound: boolean; out FoundItem: TBaseItem): string; overload;\end{ttfamily}

\ifpdf
\end{flushleft}
\fi

\par
\item[\textbf{Description}]
Searches for an item of name S which was linked in the description of Item. Starts search within item, then does a search on all items in all units using \begin{ttfamily}FindGlobal\end{ttfamily}(\ref{PasDoc_Gen.TDocGenerator-FindGlobal}). Returns a link as String on success.

If S is not splittable by SplitNameParts, it always does DoMessage with appropriate warning and returns something like 'UNKNOWN' (no matter what is the value of WarningIfLinkNotFound). FoundItem will be set to nil in this case.

When item will not be found then: \begin{itemize}
\item  if WarningIfLinkNotFound is true then it returns CodeString(ConvertString(S)) and makes DoMessage with appropriate warning.
\item else it returns '' (and does not do any DoMessage)
\end{itemize}

If LinkDisplay is not '', then it specifies explicite the display text for link. Else how exactly link does look like is controlled by \begin{ttfamily}LinkLook\end{ttfamily}(\ref{PasDoc_Gen.TDocGenerator-LinkLook}) property.

\par
\item[\textbf{Parameters}]
\begin{description}
\item[FoundItem] is the found item instance or nil if not found.
\end{description}


\end{list}
\paragraph*{SearchLink}\hspace*{\fill}

\label{PasDoc_Gen.TDocGenerator-SearchLink}
\index{SearchLink}
\begin{list}{}{
\settowidth{\tmplength}{\textbf{Description}}
\setlength{\itemindent}{0cm}
\setlength{\listparindent}{0cm}
\setlength{\leftmargin}{\evensidemargin}
\addtolength{\leftmargin}{\tmplength}
\settowidth{\labelsep}{X}
\addtolength{\leftmargin}{\labelsep}
\setlength{\labelwidth}{\tmplength}
}
\item[\textbf{Declaration}\hfill]
\ifpdf
\begin{flushleft}
\fi
\begin{ttfamily}
protected function SearchLink(s: string; const Item: TBaseItem; const LinkDisplay: string; const WarningIfLinkNotFound: boolean): string; overload;\end{ttfamily}

\ifpdf
\end{flushleft}
\fi

\par
\item[\textbf{Description}]
Just like previous overloaded version, but this doesn't return FoundItem (in case you don't need it).

\end{list}
\paragraph*{StoreDescription}\hspace*{\fill}

\label{PasDoc_Gen.TDocGenerator-StoreDescription}
\index{StoreDescription}
\begin{list}{}{
\settowidth{\tmplength}{\textbf{Description}}
\setlength{\itemindent}{0cm}
\setlength{\listparindent}{0cm}
\setlength{\leftmargin}{\evensidemargin}
\addtolength{\leftmargin}{\tmplength}
\settowidth{\labelsep}{X}
\addtolength{\leftmargin}{\labelsep}
\setlength{\labelwidth}{\tmplength}
}
\item[\textbf{Declaration}\hfill]
\ifpdf
\begin{flushleft}
\fi
\begin{ttfamily}
protected procedure StoreDescription(ItemName: string; var t: string);\end{ttfamily}

\ifpdf
\end{flushleft}
\fi

\end{list}
\paragraph*{WriteConverted}\hspace*{\fill}

\label{PasDoc_Gen.TDocGenerator-WriteConverted}
\index{WriteConverted}
\begin{list}{}{
\settowidth{\tmplength}{\textbf{Description}}
\setlength{\itemindent}{0cm}
\setlength{\listparindent}{0cm}
\setlength{\leftmargin}{\evensidemargin}
\addtolength{\leftmargin}{\tmplength}
\settowidth{\labelsep}{X}
\addtolength{\leftmargin}{\labelsep}
\setlength{\labelwidth}{\tmplength}
}
\item[\textbf{Declaration}\hfill]
\ifpdf
\begin{flushleft}
\fi
\begin{ttfamily}
protected procedure WriteConverted(const s: string; Newline: boolean); overload;\end{ttfamily}

\ifpdf
\end{flushleft}
\fi

\par
\item[\textbf{Description}]
Writes S to CurrentStream, converting it using \begin{ttfamily}ConvertString\end{ttfamily}(\ref{PasDoc_Gen.TDocGenerator-ConvertString}). Then optionally writes LineEnding.

\end{list}
\paragraph*{WriteConverted}\hspace*{\fill}

\label{PasDoc_Gen.TDocGenerator-WriteConverted}
\index{WriteConverted}
\begin{list}{}{
\settowidth{\tmplength}{\textbf{Description}}
\setlength{\itemindent}{0cm}
\setlength{\listparindent}{0cm}
\setlength{\leftmargin}{\evensidemargin}
\addtolength{\leftmargin}{\tmplength}
\settowidth{\labelsep}{X}
\addtolength{\leftmargin}{\labelsep}
\setlength{\labelwidth}{\tmplength}
}
\item[\textbf{Declaration}\hfill]
\ifpdf
\begin{flushleft}
\fi
\begin{ttfamily}
protected procedure WriteConverted(const s: string); overload;\end{ttfamily}

\ifpdf
\end{flushleft}
\fi

\par
\item[\textbf{Description}]
Writes S to CurrentStream, converting it using \begin{ttfamily}ConvertString\end{ttfamily}(\ref{PasDoc_Gen.TDocGenerator-ConvertString}). No LineEnding at the end.

\end{list}
\paragraph*{WriteConvertedLine}\hspace*{\fill}

\label{PasDoc_Gen.TDocGenerator-WriteConvertedLine}
\index{WriteConvertedLine}
\begin{list}{}{
\settowidth{\tmplength}{\textbf{Description}}
\setlength{\itemindent}{0cm}
\setlength{\listparindent}{0cm}
\setlength{\leftmargin}{\evensidemargin}
\addtolength{\leftmargin}{\tmplength}
\settowidth{\labelsep}{X}
\addtolength{\leftmargin}{\labelsep}
\setlength{\labelwidth}{\tmplength}
}
\item[\textbf{Declaration}\hfill]
\ifpdf
\begin{flushleft}
\fi
\begin{ttfamily}
protected procedure WriteConvertedLine(const s: string);\end{ttfamily}

\ifpdf
\end{flushleft}
\fi

\par
\item[\textbf{Description}]
Writes S to CurrentStream, converting it using \begin{ttfamily}ConvertString\end{ttfamily}(\ref{PasDoc_Gen.TDocGenerator-ConvertString}). Then writes LineEnding.

\end{list}
\paragraph*{WriteDirect}\hspace*{\fill}

\label{PasDoc_Gen.TDocGenerator-WriteDirect}
\index{WriteDirect}
\begin{list}{}{
\settowidth{\tmplength}{\textbf{Description}}
\setlength{\itemindent}{0cm}
\setlength{\listparindent}{0cm}
\setlength{\leftmargin}{\evensidemargin}
\addtolength{\leftmargin}{\tmplength}
\settowidth{\labelsep}{X}
\addtolength{\leftmargin}{\labelsep}
\setlength{\labelwidth}{\tmplength}
}
\item[\textbf{Declaration}\hfill]
\ifpdf
\begin{flushleft}
\fi
\begin{ttfamily}
protected procedure WriteDirect(const t: string; Newline: boolean); overload;\end{ttfamily}

\ifpdf
\end{flushleft}
\fi

\par
\item[\textbf{Description}]
Simply writes T to CurrentStream, with optional LineEnding.

\end{list}
\paragraph*{WriteDirect}\hspace*{\fill}

\label{PasDoc_Gen.TDocGenerator-WriteDirect}
\index{WriteDirect}
\begin{list}{}{
\settowidth{\tmplength}{\textbf{Description}}
\setlength{\itemindent}{0cm}
\setlength{\listparindent}{0cm}
\setlength{\leftmargin}{\evensidemargin}
\addtolength{\leftmargin}{\tmplength}
\settowidth{\labelsep}{X}
\addtolength{\leftmargin}{\labelsep}
\setlength{\labelwidth}{\tmplength}
}
\item[\textbf{Declaration}\hfill]
\ifpdf
\begin{flushleft}
\fi
\begin{ttfamily}
protected procedure WriteDirect(const t: string); overload;\end{ttfamily}

\ifpdf
\end{flushleft}
\fi

\par
\item[\textbf{Description}]
Simply writes T to CurrentStream.

\end{list}
\paragraph*{WriteDirectLine}\hspace*{\fill}

\label{PasDoc_Gen.TDocGenerator-WriteDirectLine}
\index{WriteDirectLine}
\begin{list}{}{
\settowidth{\tmplength}{\textbf{Description}}
\setlength{\itemindent}{0cm}
\setlength{\listparindent}{0cm}
\setlength{\leftmargin}{\evensidemargin}
\addtolength{\leftmargin}{\tmplength}
\settowidth{\labelsep}{X}
\addtolength{\leftmargin}{\labelsep}
\setlength{\labelwidth}{\tmplength}
}
\item[\textbf{Declaration}\hfill]
\ifpdf
\begin{flushleft}
\fi
\begin{ttfamily}
protected procedure WriteDirectLine(const t: string);\end{ttfamily}

\ifpdf
\end{flushleft}
\fi

\par
\item[\textbf{Description}]
Simply writes T followed by LineEnding to CurrentStream.

\end{list}
\paragraph*{WriteUnit}\hspace*{\fill}

\label{PasDoc_Gen.TDocGenerator-WriteUnit}
\index{WriteUnit}
\begin{list}{}{
\settowidth{\tmplength}{\textbf{Description}}
\setlength{\itemindent}{0cm}
\setlength{\listparindent}{0cm}
\setlength{\leftmargin}{\evensidemargin}
\addtolength{\leftmargin}{\tmplength}
\settowidth{\labelsep}{X}
\addtolength{\leftmargin}{\labelsep}
\setlength{\labelwidth}{\tmplength}
}
\item[\textbf{Declaration}\hfill]
\ifpdf
\begin{flushleft}
\fi
\begin{ttfamily}
protected procedure WriteUnit(const HL: integer; const U: TPasUnit); virtual; abstract;\end{ttfamily}

\ifpdf
\end{flushleft}
\fi

\par
\item[\textbf{Description}]
Abstract method that writes all documentation for a single unit U to output, starting at heading level HL. Implementation must be provided by descendant objects and is dependent on output format.

\end{list}
\paragraph*{WriteUnits}\hspace*{\fill}

\label{PasDoc_Gen.TDocGenerator-WriteUnits}
\index{WriteUnits}
\begin{list}{}{
\settowidth{\tmplength}{\textbf{Description}}
\setlength{\itemindent}{0cm}
\setlength{\listparindent}{0cm}
\setlength{\leftmargin}{\evensidemargin}
\addtolength{\leftmargin}{\tmplength}
\settowidth{\labelsep}{X}
\addtolength{\leftmargin}{\labelsep}
\setlength{\labelwidth}{\tmplength}
}
\item[\textbf{Declaration}\hfill]
\ifpdf
\begin{flushleft}
\fi
\begin{ttfamily}
protected procedure WriteUnits(const HL: integer);\end{ttfamily}

\ifpdf
\end{flushleft}
\fi

\par
\item[\textbf{Description}]
Writes documentation for all units, calling \begin{ttfamily}WriteUnit\end{ttfamily}(\ref{PasDoc_Gen.TDocGenerator-WriteUnit}) for each unit.

\end{list}
\paragraph*{WriteStartOfCode}\hspace*{\fill}

\label{PasDoc_Gen.TDocGenerator-WriteStartOfCode}
\index{WriteStartOfCode}
\begin{list}{}{
\settowidth{\tmplength}{\textbf{Description}}
\setlength{\itemindent}{0cm}
\setlength{\listparindent}{0cm}
\setlength{\leftmargin}{\evensidemargin}
\addtolength{\leftmargin}{\tmplength}
\settowidth{\labelsep}{X}
\addtolength{\leftmargin}{\labelsep}
\setlength{\labelwidth}{\tmplength}
}
\item[\textbf{Declaration}\hfill]
\ifpdf
\begin{flushleft}
\fi
\begin{ttfamily}
protected procedure WriteStartOfCode; virtual;\end{ttfamily}

\ifpdf
\end{flushleft}
\fi

\end{list}
\paragraph*{WriteEndOfCode}\hspace*{\fill}

\label{PasDoc_Gen.TDocGenerator-WriteEndOfCode}
\index{WriteEndOfCode}
\begin{list}{}{
\settowidth{\tmplength}{\textbf{Description}}
\setlength{\itemindent}{0cm}
\setlength{\listparindent}{0cm}
\setlength{\leftmargin}{\evensidemargin}
\addtolength{\leftmargin}{\tmplength}
\settowidth{\labelsep}{X}
\addtolength{\leftmargin}{\labelsep}
\setlength{\labelwidth}{\tmplength}
}
\item[\textbf{Declaration}\hfill]
\ifpdf
\begin{flushleft}
\fi
\begin{ttfamily}
protected procedure WriteEndOfCode; virtual;\end{ttfamily}

\ifpdf
\end{flushleft}
\fi

\end{list}
\paragraph*{WriteGVUses}\hspace*{\fill}

\label{PasDoc_Gen.TDocGenerator-WriteGVUses}
\index{WriteGVUses}
\begin{list}{}{
\settowidth{\tmplength}{\textbf{Description}}
\setlength{\itemindent}{0cm}
\setlength{\listparindent}{0cm}
\setlength{\leftmargin}{\evensidemargin}
\addtolength{\leftmargin}{\tmplength}
\settowidth{\labelsep}{X}
\addtolength{\leftmargin}{\labelsep}
\setlength{\labelwidth}{\tmplength}
}
\item[\textbf{Declaration}\hfill]
\ifpdf
\begin{flushleft}
\fi
\begin{ttfamily}
protected procedure WriteGVUses;\end{ttfamily}

\ifpdf
\end{flushleft}
\fi

\par
\item[\textbf{Description}]
output graphviz uses tree

\end{list}
\paragraph*{WriteGVClasses}\hspace*{\fill}

\label{PasDoc_Gen.TDocGenerator-WriteGVClasses}
\index{WriteGVClasses}
\begin{list}{}{
\settowidth{\tmplength}{\textbf{Description}}
\setlength{\itemindent}{0cm}
\setlength{\listparindent}{0cm}
\setlength{\leftmargin}{\evensidemargin}
\addtolength{\leftmargin}{\tmplength}
\settowidth{\labelsep}{X}
\addtolength{\leftmargin}{\labelsep}
\setlength{\labelwidth}{\tmplength}
}
\item[\textbf{Declaration}\hfill]
\ifpdf
\begin{flushleft}
\fi
\begin{ttfamily}
protected procedure WriteGVClasses;\end{ttfamily}

\ifpdf
\end{flushleft}
\fi

\par
\item[\textbf{Description}]
output graphviz class tree

\end{list}
\paragraph*{StartSpellChecking}\hspace*{\fill}

\label{PasDoc_Gen.TDocGenerator-StartSpellChecking}
\index{StartSpellChecking}
\begin{list}{}{
\settowidth{\tmplength}{\textbf{Description}}
\setlength{\itemindent}{0cm}
\setlength{\listparindent}{0cm}
\setlength{\leftmargin}{\evensidemargin}
\addtolength{\leftmargin}{\tmplength}
\settowidth{\labelsep}{X}
\addtolength{\leftmargin}{\labelsep}
\setlength{\labelwidth}{\tmplength}
}
\item[\textbf{Declaration}\hfill]
\ifpdf
\begin{flushleft}
\fi
\begin{ttfamily}
protected procedure StartSpellChecking(const AMode: string);\end{ttfamily}

\ifpdf
\end{flushleft}
\fi

\par
\item[\textbf{Description}]
starts the spell checker

\end{list}
\paragraph*{CheckString}\hspace*{\fill}

\label{PasDoc_Gen.TDocGenerator-CheckString}
\index{CheckString}
\begin{list}{}{
\settowidth{\tmplength}{\textbf{Description}}
\setlength{\itemindent}{0cm}
\setlength{\listparindent}{0cm}
\setlength{\leftmargin}{\evensidemargin}
\addtolength{\leftmargin}{\tmplength}
\settowidth{\labelsep}{X}
\addtolength{\leftmargin}{\labelsep}
\setlength{\labelwidth}{\tmplength}
}
\item[\textbf{Declaration}\hfill]
\ifpdf
\begin{flushleft}
\fi
\begin{ttfamily}
protected procedure CheckString(const AString: string; const AErrors: TObjectVector);\end{ttfamily}

\ifpdf
\end{flushleft}
\fi

\par
\item[\textbf{Description}]
If CheckSpelling and spell checking was successfully started, this will run \begin{ttfamily}FAspellProcess.CheckString\end{ttfamily}(\ref{PasDoc_Aspell.TAspellProcess-CheckString}) and will report all errors using DoMessage with mtWarning.

Otherwise this just clears AErrors, which means that no errors were found.

\end{list}
\paragraph*{EndSpellChecking}\hspace*{\fill}

\label{PasDoc_Gen.TDocGenerator-EndSpellChecking}
\index{EndSpellChecking}
\begin{list}{}{
\settowidth{\tmplength}{\textbf{Description}}
\setlength{\itemindent}{0cm}
\setlength{\listparindent}{0cm}
\setlength{\leftmargin}{\evensidemargin}
\addtolength{\leftmargin}{\tmplength}
\settowidth{\labelsep}{X}
\addtolength{\leftmargin}{\labelsep}
\setlength{\labelwidth}{\tmplength}
}
\item[\textbf{Declaration}\hfill]
\ifpdf
\begin{flushleft}
\fi
\begin{ttfamily}
protected procedure EndSpellChecking;\end{ttfamily}

\ifpdf
\end{flushleft}
\fi

\par
\item[\textbf{Description}]
closes the spellchecker

\end{list}
\paragraph*{FormatPascalCode}\hspace*{\fill}

\label{PasDoc_Gen.TDocGenerator-FormatPascalCode}
\index{FormatPascalCode}
\begin{list}{}{
\settowidth{\tmplength}{\textbf{Description}}
\setlength{\itemindent}{0cm}
\setlength{\listparindent}{0cm}
\setlength{\leftmargin}{\evensidemargin}
\addtolength{\leftmargin}{\tmplength}
\settowidth{\labelsep}{X}
\addtolength{\leftmargin}{\labelsep}
\setlength{\labelwidth}{\tmplength}
}
\item[\textbf{Declaration}\hfill]
\ifpdf
\begin{flushleft}
\fi
\begin{ttfamily}
protected function FormatPascalCode(const Line: string): string; virtual;\end{ttfamily}

\ifpdf
\end{flushleft}
\fi

\par
\item[\textbf{Description}]
FormatPascalCode will cause Line to be formatted in the way that Pascal code is formatted in Delphi. Note that given Line is taken directly from what user put inside \texttt{}, it is not even processed by ConvertString. You should process it with ConvertString if you want.

\end{list}
\paragraph*{FormatNormalCode}\hspace*{\fill}

\label{PasDoc_Gen.TDocGenerator-FormatNormalCode}
\index{FormatNormalCode}
\begin{list}{}{
\settowidth{\tmplength}{\textbf{Description}}
\setlength{\itemindent}{0cm}
\setlength{\listparindent}{0cm}
\setlength{\leftmargin}{\evensidemargin}
\addtolength{\leftmargin}{\tmplength}
\settowidth{\labelsep}{X}
\addtolength{\leftmargin}{\labelsep}
\setlength{\labelwidth}{\tmplength}
}
\item[\textbf{Declaration}\hfill]
\ifpdf
\begin{flushleft}
\fi
\begin{ttfamily}
protected function FormatNormalCode(AString: string): string; virtual;\end{ttfamily}

\ifpdf
\end{flushleft}
\fi

\par
\item[\textbf{Description}]
This will cause AString to be formatted in the way that normal Pascal statements (not keywords, strings, comments, etc.) look in Delphi.

\end{list}
\paragraph*{FormatComment}\hspace*{\fill}

\label{PasDoc_Gen.TDocGenerator-FormatComment}
\index{FormatComment}
\begin{list}{}{
\settowidth{\tmplength}{\textbf{Description}}
\setlength{\itemindent}{0cm}
\setlength{\listparindent}{0cm}
\setlength{\leftmargin}{\evensidemargin}
\addtolength{\leftmargin}{\tmplength}
\settowidth{\labelsep}{X}
\addtolength{\leftmargin}{\labelsep}
\setlength{\labelwidth}{\tmplength}
}
\item[\textbf{Declaration}\hfill]
\ifpdf
\begin{flushleft}
\fi
\begin{ttfamily}
protected function FormatComment(AString: string): string; virtual;\end{ttfamily}

\ifpdf
\end{flushleft}
\fi

\par
\item[\textbf{Description}]
FormatComment will cause AString to be formatted in the way that comments other than compiler directives are formatted in Delphi. See: \begin{ttfamily}FormatCompilerComment\end{ttfamily}(\ref{PasDoc_Gen.TDocGenerator-FormatCompilerComment}).

\end{list}
\paragraph*{FormatHex}\hspace*{\fill}

\label{PasDoc_Gen.TDocGenerator-FormatHex}
\index{FormatHex}
\begin{list}{}{
\settowidth{\tmplength}{\textbf{Description}}
\setlength{\itemindent}{0cm}
\setlength{\listparindent}{0cm}
\setlength{\leftmargin}{\evensidemargin}
\addtolength{\leftmargin}{\tmplength}
\settowidth{\labelsep}{X}
\addtolength{\leftmargin}{\labelsep}
\setlength{\labelwidth}{\tmplength}
}
\item[\textbf{Declaration}\hfill]
\ifpdf
\begin{flushleft}
\fi
\begin{ttfamily}
protected function FormatHex(AString: string): string; virtual;\end{ttfamily}

\ifpdf
\end{flushleft}
\fi

\par
\item[\textbf{Description}]
FormatHex will cause AString to be formatted in the way that Hex are formatted in Delphi.

\end{list}
\paragraph*{FormatNumeric}\hspace*{\fill}

\label{PasDoc_Gen.TDocGenerator-FormatNumeric}
\index{FormatNumeric}
\begin{list}{}{
\settowidth{\tmplength}{\textbf{Description}}
\setlength{\itemindent}{0cm}
\setlength{\listparindent}{0cm}
\setlength{\leftmargin}{\evensidemargin}
\addtolength{\leftmargin}{\tmplength}
\settowidth{\labelsep}{X}
\addtolength{\leftmargin}{\labelsep}
\setlength{\labelwidth}{\tmplength}
}
\item[\textbf{Declaration}\hfill]
\ifpdf
\begin{flushleft}
\fi
\begin{ttfamily}
protected function FormatNumeric(AString: string): string; virtual;\end{ttfamily}

\ifpdf
\end{flushleft}
\fi

\par
\item[\textbf{Description}]
FormatNumeric will cause AString to be formatted in the way that Numeric are formatted in Delphi.

\end{list}
\paragraph*{FormatFloat}\hspace*{\fill}

\label{PasDoc_Gen.TDocGenerator-FormatFloat}
\index{FormatFloat}
\begin{list}{}{
\settowidth{\tmplength}{\textbf{Description}}
\setlength{\itemindent}{0cm}
\setlength{\listparindent}{0cm}
\setlength{\leftmargin}{\evensidemargin}
\addtolength{\leftmargin}{\tmplength}
\settowidth{\labelsep}{X}
\addtolength{\leftmargin}{\labelsep}
\setlength{\labelwidth}{\tmplength}
}
\item[\textbf{Declaration}\hfill]
\ifpdf
\begin{flushleft}
\fi
\begin{ttfamily}
protected function FormatFloat(AString: string): string; virtual;\end{ttfamily}

\ifpdf
\end{flushleft}
\fi

\par
\item[\textbf{Description}]
FormatFloat will cause AString to be formatted in the way that Float are formatted in Delphi.

\end{list}
\paragraph*{FormatString}\hspace*{\fill}

\label{PasDoc_Gen.TDocGenerator-FormatString}
\index{FormatString}
\begin{list}{}{
\settowidth{\tmplength}{\textbf{Description}}
\setlength{\itemindent}{0cm}
\setlength{\listparindent}{0cm}
\setlength{\leftmargin}{\evensidemargin}
\addtolength{\leftmargin}{\tmplength}
\settowidth{\labelsep}{X}
\addtolength{\leftmargin}{\labelsep}
\setlength{\labelwidth}{\tmplength}
}
\item[\textbf{Declaration}\hfill]
\ifpdf
\begin{flushleft}
\fi
\begin{ttfamily}
protected function FormatString(AString: string): string; virtual;\end{ttfamily}

\ifpdf
\end{flushleft}
\fi

\par
\item[\textbf{Description}]
FormatString will cause AString to be formatted in the way that strings are formatted in Delphi.

\end{list}
\paragraph*{FormatKeyWord}\hspace*{\fill}

\label{PasDoc_Gen.TDocGenerator-FormatKeyWord}
\index{FormatKeyWord}
\begin{list}{}{
\settowidth{\tmplength}{\textbf{Description}}
\setlength{\itemindent}{0cm}
\setlength{\listparindent}{0cm}
\setlength{\leftmargin}{\evensidemargin}
\addtolength{\leftmargin}{\tmplength}
\settowidth{\labelsep}{X}
\addtolength{\leftmargin}{\labelsep}
\setlength{\labelwidth}{\tmplength}
}
\item[\textbf{Declaration}\hfill]
\ifpdf
\begin{flushleft}
\fi
\begin{ttfamily}
protected function FormatKeyWord(AString: string): string; virtual;\end{ttfamily}

\ifpdf
\end{flushleft}
\fi

\par
\item[\textbf{Description}]
FormatKeyWord will cause AString to be formatted in the way that reserved words are formatted in Delphi.

\end{list}
\paragraph*{FormatCompilerComment}\hspace*{\fill}

\label{PasDoc_Gen.TDocGenerator-FormatCompilerComment}
\index{FormatCompilerComment}
\begin{list}{}{
\settowidth{\tmplength}{\textbf{Description}}
\setlength{\itemindent}{0cm}
\setlength{\listparindent}{0cm}
\setlength{\leftmargin}{\evensidemargin}
\addtolength{\leftmargin}{\tmplength}
\settowidth{\labelsep}{X}
\addtolength{\leftmargin}{\labelsep}
\setlength{\labelwidth}{\tmplength}
}
\item[\textbf{Declaration}\hfill]
\ifpdf
\begin{flushleft}
\fi
\begin{ttfamily}
protected function FormatCompilerComment(AString: string): string; virtual;\end{ttfamily}

\ifpdf
\end{flushleft}
\fi

\par
\item[\textbf{Description}]
FormatCompilerComment will cause AString to be formatted in the way that compiler directives are formatted in Delphi.

\end{list}
\paragraph*{Paragraph}\hspace*{\fill}

\label{PasDoc_Gen.TDocGenerator-Paragraph}
\index{Paragraph}
\begin{list}{}{
\settowidth{\tmplength}{\textbf{Description}}
\setlength{\itemindent}{0cm}
\setlength{\listparindent}{0cm}
\setlength{\leftmargin}{\evensidemargin}
\addtolength{\leftmargin}{\tmplength}
\settowidth{\labelsep}{X}
\addtolength{\leftmargin}{\labelsep}
\setlength{\labelwidth}{\tmplength}
}
\item[\textbf{Declaration}\hfill]
\ifpdf
\begin{flushleft}
\fi
\begin{ttfamily}
protected function Paragraph: string; virtual;\end{ttfamily}

\ifpdf
\end{flushleft}
\fi

\par
\item[\textbf{Description}]
This is paragraph marker in output documentation.

Default implementation in this class simply returns ' ' (one space).

\end{list}
\paragraph*{ShortDash}\hspace*{\fill}

\label{PasDoc_Gen.TDocGenerator-ShortDash}
\index{ShortDash}
\begin{list}{}{
\settowidth{\tmplength}{\textbf{Description}}
\setlength{\itemindent}{0cm}
\setlength{\listparindent}{0cm}
\setlength{\leftmargin}{\evensidemargin}
\addtolength{\leftmargin}{\tmplength}
\settowidth{\labelsep}{X}
\addtolength{\leftmargin}{\labelsep}
\setlength{\labelwidth}{\tmplength}
}
\item[\textbf{Declaration}\hfill]
\ifpdf
\begin{flushleft}
\fi
\begin{ttfamily}
protected function ShortDash: string; virtual;\end{ttfamily}

\ifpdf
\end{flushleft}
\fi

\par
\item[\textbf{Description}]
See \begin{ttfamily}TTagManager.ShortDash\end{ttfamily}(\ref{PasDoc_TagManager.TTagManager-ShortDash}). Default implementation in this class returns '{-}'.

\end{list}
\paragraph*{EnDash}\hspace*{\fill}

\label{PasDoc_Gen.TDocGenerator-EnDash}
\index{EnDash}
\begin{list}{}{
\settowidth{\tmplength}{\textbf{Description}}
\setlength{\itemindent}{0cm}
\setlength{\listparindent}{0cm}
\setlength{\leftmargin}{\evensidemargin}
\addtolength{\leftmargin}{\tmplength}
\settowidth{\labelsep}{X}
\addtolength{\leftmargin}{\labelsep}
\setlength{\labelwidth}{\tmplength}
}
\item[\textbf{Declaration}\hfill]
\ifpdf
\begin{flushleft}
\fi
\begin{ttfamily}
protected function EnDash: string; virtual;\end{ttfamily}

\ifpdf
\end{flushleft}
\fi

\par
\item[\textbf{Description}]
See \begin{ttfamily}TTagManager.EnDash\end{ttfamily}(\ref{PasDoc_TagManager.TTagManager-EnDash}). Default implementation in this class returns '{-}{-}'.

\end{list}
\paragraph*{EmDash}\hspace*{\fill}

\label{PasDoc_Gen.TDocGenerator-EmDash}
\index{EmDash}
\begin{list}{}{
\settowidth{\tmplength}{\textbf{Description}}
\setlength{\itemindent}{0cm}
\setlength{\listparindent}{0cm}
\setlength{\leftmargin}{\evensidemargin}
\addtolength{\leftmargin}{\tmplength}
\settowidth{\labelsep}{X}
\addtolength{\leftmargin}{\labelsep}
\setlength{\labelwidth}{\tmplength}
}
\item[\textbf{Declaration}\hfill]
\ifpdf
\begin{flushleft}
\fi
\begin{ttfamily}
protected function EmDash: string; virtual;\end{ttfamily}

\ifpdf
\end{flushleft}
\fi

\par
\item[\textbf{Description}]
See \begin{ttfamily}TTagManager.EmDash\end{ttfamily}(\ref{PasDoc_TagManager.TTagManager-EmDash}). Default implementation in this class returns '{-}{-}{-}'.

\end{list}
\paragraph*{HtmlString}\hspace*{\fill}

\label{PasDoc_Gen.TDocGenerator-HtmlString}
\index{HtmlString}
\begin{list}{}{
\settowidth{\tmplength}{\textbf{Description}}
\setlength{\itemindent}{0cm}
\setlength{\listparindent}{0cm}
\setlength{\leftmargin}{\evensidemargin}
\addtolength{\leftmargin}{\tmplength}
\settowidth{\labelsep}{X}
\addtolength{\leftmargin}{\labelsep}
\setlength{\labelwidth}{\tmplength}
}
\item[\textbf{Declaration}\hfill]
\ifpdf
\begin{flushleft}
\fi
\begin{ttfamily}
protected function HtmlString(const S: string): string; virtual;\end{ttfamily}

\ifpdf
\end{flushleft}
\fi

\par
\item[\textbf{Description}]
S is guaranteed (guaranteed by the user) to be correct html content, this is taken directly from parameters of  Override this function to decide what to put in output on such thing.

Note that S is not processed in any way, even with ConvertString. So you're able to copy user's input inside @html() verbatim to the output.

The default implementation is this class simply discards it, i.e. returns always ''. Generators that know what to do with HTML can override this with simple "Result := S".

\end{list}
\paragraph*{LatexString}\hspace*{\fill}

\label{PasDoc_Gen.TDocGenerator-LatexString}
\index{LatexString}
\begin{list}{}{
\settowidth{\tmplength}{\textbf{Description}}
\setlength{\itemindent}{0cm}
\setlength{\listparindent}{0cm}
\setlength{\leftmargin}{\evensidemargin}
\addtolength{\leftmargin}{\tmplength}
\settowidth{\labelsep}{X}
\addtolength{\leftmargin}{\labelsep}
\setlength{\labelwidth}{\tmplength}
}
\item[\textbf{Declaration}\hfill]
\ifpdf
\begin{flushleft}
\fi
\begin{ttfamily}
protected function LatexString(const S: string): string; virtual;\end{ttfamily}

\ifpdf
\end{flushleft}
\fi

\par
\item[\textbf{Description}]
This is equivalent of \begin{ttfamily}HtmlString\end{ttfamily}(\ref{PasDoc_Gen.TDocGenerator-HtmlString}) for @latex tag.

The default implementation is this class simply discards it, i.e. returns always ''. Generators that know what to do with raw LaTeX markup can override this with simple "Result := S".

\end{list}
\paragraph*{LineBreak}\hspace*{\fill}

\label{PasDoc_Gen.TDocGenerator-LineBreak}
\index{LineBreak}
\begin{list}{}{
\settowidth{\tmplength}{\textbf{Description}}
\setlength{\itemindent}{0cm}
\setlength{\listparindent}{0cm}
\setlength{\leftmargin}{\evensidemargin}
\addtolength{\leftmargin}{\tmplength}
\settowidth{\labelsep}{X}
\addtolength{\leftmargin}{\labelsep}
\setlength{\labelwidth}{\tmplength}
}
\item[\textbf{Declaration}\hfill]
\ifpdf
\begin{flushleft}
\fi
\begin{ttfamily}
protected function LineBreak: string; virtual;\end{ttfamily}

\ifpdf
\end{flushleft}
\fi

\par
\item[\textbf{Description}]
This returns markup that forces line break in given output format (e.g. '{$<$}br{$>$}' in html or '{\textbackslash}{\textbackslash}' in LaTeX).\hfill\vspace*{1ex}



It is used on \\{} tag (but may also be used on other occasions in the future).

In this class it returns '', because it's valid for an output generator to simply ignore \\{} tags if linebreaks can't be expressed in given output format.

\end{list}
\paragraph*{URLLink}\hspace*{\fill}

\label{PasDoc_Gen.TDocGenerator-URLLink}
\index{URLLink}
\begin{list}{}{
\settowidth{\tmplength}{\textbf{Description}}
\setlength{\itemindent}{0cm}
\setlength{\listparindent}{0cm}
\setlength{\leftmargin}{\evensidemargin}
\addtolength{\leftmargin}{\tmplength}
\settowidth{\labelsep}{X}
\addtolength{\leftmargin}{\labelsep}
\setlength{\labelwidth}{\tmplength}
}
\item[\textbf{Declaration}\hfill]
\ifpdf
\begin{flushleft}
\fi
\begin{ttfamily}
protected function URLLink(const URL: string): string; overload; virtual;\end{ttfamily}

\ifpdf
\end{flushleft}
\fi

\par
\item[\textbf{Description}]
This should return markup upon finding URL in description. E.g. HTML generator will want to wrap this in {$<$}a href="..."{$>$}...{$<$}/a{$>$}.

Note that passed here URL is \textit{not} processed by \begin{ttfamily}ConvertString\end{ttfamily}(\ref{PasDoc_Gen.TDocGenerator-ConvertString}) (because sometimes it could be undesirable). If you want you can process URL with ConvertString when overriding this method.

Default implementation in this class simply returns ConvertString(URL). This is good if your documentation format does not support anything like URL links.

\end{list}
\paragraph*{URLLink}\hspace*{\fill}

\label{PasDoc_Gen.TDocGenerator-URLLink}
\index{URLLink}
\begin{list}{}{
\settowidth{\tmplength}{\textbf{Description}}
\setlength{\itemindent}{0cm}
\setlength{\listparindent}{0cm}
\setlength{\leftmargin}{\evensidemargin}
\addtolength{\leftmargin}{\tmplength}
\settowidth{\labelsep}{X}
\addtolength{\leftmargin}{\labelsep}
\setlength{\labelwidth}{\tmplength}
}
\item[\textbf{Declaration}\hfill]
\ifpdf
\begin{flushleft}
\fi
\begin{ttfamily}
protected function URLLink(const URL, LinkDisplay: string): string; overload; virtual;\end{ttfamily}

\ifpdf
\end{flushleft}
\fi

\par
\item[\textbf{Description}]
This returns the Text which will be shown for an URL tag.

URL is a link to a website or e{-}mail address. LinkDisplay is an optional parameter which will be used as the display name of the URL.

\end{list}
\paragraph*{WriteExternal}\hspace*{\fill}

\label{PasDoc_Gen.TDocGenerator-WriteExternal}
\index{WriteExternal}
\begin{list}{}{
\settowidth{\tmplength}{\textbf{Description}}
\setlength{\itemindent}{0cm}
\setlength{\listparindent}{0cm}
\setlength{\leftmargin}{\evensidemargin}
\addtolength{\leftmargin}{\tmplength}
\settowidth{\labelsep}{X}
\addtolength{\leftmargin}{\labelsep}
\setlength{\labelwidth}{\tmplength}
}
\item[\textbf{Declaration}\hfill]
\ifpdf
\begin{flushleft}
\fi
\begin{ttfamily}
protected procedure WriteExternal(const ExternalItem: TExternalItem; const Id: TTranslationID);\end{ttfamily}

\ifpdf
\end{flushleft}
\fi

\par
\item[\textbf{Description}]
\begin{ttfamily}WriteExternal\end{ttfamily} is used to write the introduction and conclusion of the project.

\end{list}
\paragraph*{WriteExternalCore}\hspace*{\fill}

\label{PasDoc_Gen.TDocGenerator-WriteExternalCore}
\index{WriteExternalCore}
\begin{list}{}{
\settowidth{\tmplength}{\textbf{Description}}
\setlength{\itemindent}{0cm}
\setlength{\listparindent}{0cm}
\setlength{\leftmargin}{\evensidemargin}
\addtolength{\leftmargin}{\tmplength}
\settowidth{\labelsep}{X}
\addtolength{\leftmargin}{\labelsep}
\setlength{\labelwidth}{\tmplength}
}
\item[\textbf{Declaration}\hfill]
\ifpdf
\begin{flushleft}
\fi
\begin{ttfamily}
protected procedure WriteExternalCore(const ExternalItem: TExternalItem; const Id: TTranslationID); virtual; abstract;\end{ttfamily}

\ifpdf
\end{flushleft}
\fi

\par
\item[\textbf{Description}]
This is called from \begin{ttfamily}WriteExternal\end{ttfamily}(\ref{PasDoc_Gen.TDocGenerator-WriteExternal}) when ExternalItem.Title and ShortTitle are already set, message about generating appropriate item is printed etc. This should write ExternalItem, including ExternalItem.DetailedDescription, ExternalItem.Authors, ExternalItem.Created, ExternalItem.LastMod.

\end{list}
\paragraph*{WriteConclusion}\hspace*{\fill}

\label{PasDoc_Gen.TDocGenerator-WriteConclusion}
\index{WriteConclusion}
\begin{list}{}{
\settowidth{\tmplength}{\textbf{Description}}
\setlength{\itemindent}{0cm}
\setlength{\listparindent}{0cm}
\setlength{\leftmargin}{\evensidemargin}
\addtolength{\leftmargin}{\tmplength}
\settowidth{\labelsep}{X}
\addtolength{\leftmargin}{\labelsep}
\setlength{\labelwidth}{\tmplength}
}
\item[\textbf{Declaration}\hfill]
\ifpdf
\begin{flushleft}
\fi
\begin{ttfamily}
protected procedure WriteConclusion;\end{ttfamily}

\ifpdf
\end{flushleft}
\fi

\par
\item[\textbf{Description}]
\begin{ttfamily}WriteConclusion\end{ttfamily} writes a conclusion for the project. See \begin{ttfamily}WriteExternal\end{ttfamily}(\ref{PasDoc_Gen.TDocGenerator-WriteExternal}).

\end{list}
\paragraph*{WriteIntroduction}\hspace*{\fill}

\label{PasDoc_Gen.TDocGenerator-WriteIntroduction}
\index{WriteIntroduction}
\begin{list}{}{
\settowidth{\tmplength}{\textbf{Description}}
\setlength{\itemindent}{0cm}
\setlength{\listparindent}{0cm}
\setlength{\leftmargin}{\evensidemargin}
\addtolength{\leftmargin}{\tmplength}
\settowidth{\labelsep}{X}
\addtolength{\leftmargin}{\labelsep}
\setlength{\labelwidth}{\tmplength}
}
\item[\textbf{Declaration}\hfill]
\ifpdf
\begin{flushleft}
\fi
\begin{ttfamily}
protected procedure WriteIntroduction;\end{ttfamily}

\ifpdf
\end{flushleft}
\fi

\par
\item[\textbf{Description}]
\begin{ttfamily}WriteIntroduction\end{ttfamily} writes an introduction for the project. See \begin{ttfamily}WriteExternal\end{ttfamily}(\ref{PasDoc_Gen.TDocGenerator-WriteExternal}).

\end{list}
\paragraph*{WriteAdditionalFiles}\hspace*{\fill}

\label{PasDoc_Gen.TDocGenerator-WriteAdditionalFiles}
\index{WriteAdditionalFiles}
\begin{list}{}{
\settowidth{\tmplength}{\textbf{Description}}
\setlength{\itemindent}{0cm}
\setlength{\listparindent}{0cm}
\setlength{\leftmargin}{\evensidemargin}
\addtolength{\leftmargin}{\tmplength}
\settowidth{\labelsep}{X}
\addtolength{\leftmargin}{\labelsep}
\setlength{\labelwidth}{\tmplength}
}
\item[\textbf{Declaration}\hfill]
\ifpdf
\begin{flushleft}
\fi
\begin{ttfamily}
protected procedure WriteAdditionalFiles;\end{ttfamily}

\ifpdf
\end{flushleft}
\fi

\par
\item[\textbf{Description}]
\begin{ttfamily}WriteAdditionalFiles\end{ttfamily} writes the other files for the project. See \begin{ttfamily}WriteExternal\end{ttfamily}(\ref{PasDoc_Gen.TDocGenerator-WriteExternal}).

\end{list}
\paragraph*{FormatSection}\hspace*{\fill}

\label{PasDoc_Gen.TDocGenerator-FormatSection}
\index{FormatSection}
\begin{list}{}{
\settowidth{\tmplength}{\textbf{Description}}
\setlength{\itemindent}{0cm}
\setlength{\listparindent}{0cm}
\setlength{\leftmargin}{\evensidemargin}
\addtolength{\leftmargin}{\tmplength}
\settowidth{\labelsep}{X}
\addtolength{\leftmargin}{\labelsep}
\setlength{\labelwidth}{\tmplength}
}
\item[\textbf{Declaration}\hfill]
\ifpdf
\begin{flushleft}
\fi
\begin{ttfamily}
protected function FormatSection(HL: integer; const Anchor: string; const Caption: string): string; virtual; abstract;\end{ttfamily}

\ifpdf
\end{flushleft}
\fi

\par
\item[\textbf{Description}]
\begin{ttfamily}FormatSection\end{ttfamily} writes a section heading and a link{-}anchor;

\end{list}
\paragraph*{FormatAnchor}\hspace*{\fill}

\label{PasDoc_Gen.TDocGenerator-FormatAnchor}
\index{FormatAnchor}
\begin{list}{}{
\settowidth{\tmplength}{\textbf{Description}}
\setlength{\itemindent}{0cm}
\setlength{\listparindent}{0cm}
\setlength{\leftmargin}{\evensidemargin}
\addtolength{\leftmargin}{\tmplength}
\settowidth{\labelsep}{X}
\addtolength{\leftmargin}{\labelsep}
\setlength{\labelwidth}{\tmplength}
}
\item[\textbf{Declaration}\hfill]
\ifpdf
\begin{flushleft}
\fi
\begin{ttfamily}
protected function FormatAnchor(const Anchor: string): string; virtual; abstract;\end{ttfamily}

\ifpdf
\end{flushleft}
\fi

\par
\item[\textbf{Description}]
\begin{ttfamily}FormatAnchor\end{ttfamily} writes a link{-}anchor;

\end{list}
\paragraph*{FormatBold}\hspace*{\fill}

\label{PasDoc_Gen.TDocGenerator-FormatBold}
\index{FormatBold}
\begin{list}{}{
\settowidth{\tmplength}{\textbf{Description}}
\setlength{\itemindent}{0cm}
\setlength{\listparindent}{0cm}
\setlength{\leftmargin}{\evensidemargin}
\addtolength{\leftmargin}{\tmplength}
\settowidth{\labelsep}{X}
\addtolength{\leftmargin}{\labelsep}
\setlength{\labelwidth}{\tmplength}
}
\item[\textbf{Declaration}\hfill]
\ifpdf
\begin{flushleft}
\fi
\begin{ttfamily}
protected function FormatBold(const Text: string): string; virtual;\end{ttfamily}

\ifpdf
\end{flushleft}
\fi

\par
\item[\textbf{Description}]
This returns Text formatted using bold font.

Given Text is already in the final output format (with characters converted using \begin{ttfamily}ConvertString\end{ttfamily}(\ref{PasDoc_Gen.TDocGenerator-ConvertString}), @{-}tags expanded etc.).

Implementation of this method in this class simply returns \begin{ttfamily}Result := Text\end{ttfamily}. Output generators that can somehow express bold formatting (or at least emphasis of some text) should override this.

\item[\textbf{See also}]
\begin{description}
\item[\begin{ttfamily}FormatItalic\end{ttfamily}(\ref{PasDoc_Gen.TDocGenerator-FormatItalic})] 
This returns Text formatted using italic font.
\end{description}


\end{list}
\paragraph*{FormatItalic}\hspace*{\fill}

\label{PasDoc_Gen.TDocGenerator-FormatItalic}
\index{FormatItalic}
\begin{list}{}{
\settowidth{\tmplength}{\textbf{Description}}
\setlength{\itemindent}{0cm}
\setlength{\listparindent}{0cm}
\setlength{\leftmargin}{\evensidemargin}
\addtolength{\leftmargin}{\tmplength}
\settowidth{\labelsep}{X}
\addtolength{\leftmargin}{\labelsep}
\setlength{\labelwidth}{\tmplength}
}
\item[\textbf{Declaration}\hfill]
\ifpdf
\begin{flushleft}
\fi
\begin{ttfamily}
protected function FormatItalic(const Text: string): string; virtual;\end{ttfamily}

\ifpdf
\end{flushleft}
\fi

\par
\item[\textbf{Description}]
This returns Text formatted using italic font. Analogous to \begin{ttfamily}FormatBold\end{ttfamily}(\ref{PasDoc_Gen.TDocGenerator-FormatBold}).

\end{list}
\paragraph*{FormatWarning}\hspace*{\fill}

\label{PasDoc_Gen.TDocGenerator-FormatWarning}
\index{FormatWarning}
\begin{list}{}{
\settowidth{\tmplength}{\textbf{Description}}
\setlength{\itemindent}{0cm}
\setlength{\listparindent}{0cm}
\setlength{\leftmargin}{\evensidemargin}
\addtolength{\leftmargin}{\tmplength}
\settowidth{\labelsep}{X}
\addtolength{\leftmargin}{\labelsep}
\setlength{\labelwidth}{\tmplength}
}
\item[\textbf{Declaration}\hfill]
\ifpdf
\begin{flushleft}
\fi
\begin{ttfamily}
protected function FormatWarning(const Text: string): string; virtual;\end{ttfamily}

\ifpdf
\end{flushleft}
\fi

\par
\item[\textbf{Description}]
This returns Text using bold font by calling FormatBold(Text).

\end{list}
\paragraph*{FormatNote}\hspace*{\fill}

\label{PasDoc_Gen.TDocGenerator-FormatNote}
\index{FormatNote}
\begin{list}{}{
\settowidth{\tmplength}{\textbf{Description}}
\setlength{\itemindent}{0cm}
\setlength{\listparindent}{0cm}
\setlength{\leftmargin}{\evensidemargin}
\addtolength{\leftmargin}{\tmplength}
\settowidth{\labelsep}{X}
\addtolength{\leftmargin}{\labelsep}
\setlength{\labelwidth}{\tmplength}
}
\item[\textbf{Declaration}\hfill]
\ifpdf
\begin{flushleft}
\fi
\begin{ttfamily}
protected function FormatNote(const Text: string): string; virtual;\end{ttfamily}

\ifpdf
\end{flushleft}
\fi

\par
\item[\textbf{Description}]
This returns Text using italic font by calling FormatItalic(Text).

\end{list}
\paragraph*{FormatPreformatted}\hspace*{\fill}

\label{PasDoc_Gen.TDocGenerator-FormatPreformatted}
\index{FormatPreformatted}
\begin{list}{}{
\settowidth{\tmplength}{\textbf{Description}}
\setlength{\itemindent}{0cm}
\setlength{\listparindent}{0cm}
\setlength{\leftmargin}{\evensidemargin}
\addtolength{\leftmargin}{\tmplength}
\settowidth{\labelsep}{X}
\addtolength{\leftmargin}{\labelsep}
\setlength{\labelwidth}{\tmplength}
}
\item[\textbf{Declaration}\hfill]
\ifpdf
\begin{flushleft}
\fi
\begin{ttfamily}
protected function FormatPreformatted(const Text: string): string; virtual;\end{ttfamily}

\ifpdf
\end{flushleft}
\fi

\par
\item[\textbf{Description}]
This returns Text preserving spaces and line breaks. Note that Text passed here is not yet converted with ConvertString. The implementation of this method in this class just returns ConvertString(Text).

\end{list}
\paragraph*{FormatImage}\hspace*{\fill}

\label{PasDoc_Gen.TDocGenerator-FormatImage}
\index{FormatImage}
\begin{list}{}{
\settowidth{\tmplength}{\textbf{Description}}
\setlength{\itemindent}{0cm}
\setlength{\listparindent}{0cm}
\setlength{\leftmargin}{\evensidemargin}
\addtolength{\leftmargin}{\tmplength}
\settowidth{\labelsep}{X}
\addtolength{\leftmargin}{\labelsep}
\setlength{\labelwidth}{\tmplength}
}
\item[\textbf{Declaration}\hfill]
\ifpdf
\begin{flushleft}
\fi
\begin{ttfamily}
protected function FormatImage(FileNames: TStringList): string; virtual;\end{ttfamily}

\ifpdf
\end{flushleft}
\fi

\par
\item[\textbf{Description}]
Return markup to show an image. FileNames is a list of possible filenames of the image. FileNames always contains at least one item (i.e. FileNames.Count {$>$}= 1), never contains empty lines (i.e. Trim(FileNames[I]) {$<$}{$>$} ''), and contains only absolute filenames.

E.g. HTML generator will want to choose the best format for HTML, then somehow copy the image from FileNames[Chosen] and wrap this in {$<$}img src="..."{$>$}.

Implementation of this method in this class simply shows \begin{ttfamily}FileNames[0]\end{ttfamily}. Output generators should override this.

\end{list}
\paragraph*{FormatList}\hspace*{\fill}

\label{PasDoc_Gen.TDocGenerator-FormatList}
\index{FormatList}
\begin{list}{}{
\settowidth{\tmplength}{\textbf{Description}}
\setlength{\itemindent}{0cm}
\setlength{\listparindent}{0cm}
\setlength{\leftmargin}{\evensidemargin}
\addtolength{\leftmargin}{\tmplength}
\settowidth{\labelsep}{X}
\addtolength{\leftmargin}{\labelsep}
\setlength{\labelwidth}{\tmplength}
}
\item[\textbf{Declaration}\hfill]
\ifpdf
\begin{flushleft}
\fi
\begin{ttfamily}
protected function FormatList(ListData: TListData): string; virtual; abstract;\end{ttfamily}

\ifpdf
\end{flushleft}
\fi

\par
\item[\textbf{Description}]
Format a list from given ListData.

\end{list}
\paragraph*{FormatTable}\hspace*{\fill}

\label{PasDoc_Gen.TDocGenerator-FormatTable}
\index{FormatTable}
\begin{list}{}{
\settowidth{\tmplength}{\textbf{Description}}
\setlength{\itemindent}{0cm}
\setlength{\listparindent}{0cm}
\setlength{\leftmargin}{\evensidemargin}
\addtolength{\leftmargin}{\tmplength}
\settowidth{\labelsep}{X}
\addtolength{\leftmargin}{\labelsep}
\setlength{\labelwidth}{\tmplength}
}
\item[\textbf{Declaration}\hfill]
\ifpdf
\begin{flushleft}
\fi
\begin{ttfamily}
protected function FormatTable(Table: TTableData): string; virtual; abstract;\end{ttfamily}

\ifpdf
\end{flushleft}
\fi

\par
\item[\textbf{Description}]
This should return appropriate content for given Table. It's guaranteed that the Table passed here will have at least one row and in each row there will be at least one cell, so you don't have to check it within descendants.

\end{list}
\paragraph*{FormatTableOfContents}\hspace*{\fill}

\label{PasDoc_Gen.TDocGenerator-FormatTableOfContents}
\index{FormatTableOfContents}
\begin{list}{}{
\settowidth{\tmplength}{\textbf{Description}}
\setlength{\itemindent}{0cm}
\setlength{\listparindent}{0cm}
\setlength{\leftmargin}{\evensidemargin}
\addtolength{\leftmargin}{\tmplength}
\settowidth{\labelsep}{X}
\addtolength{\leftmargin}{\labelsep}
\setlength{\labelwidth}{\tmplength}
}
\item[\textbf{Declaration}\hfill]
\ifpdf
\begin{flushleft}
\fi
\begin{ttfamily}
protected function FormatTableOfContents(Sections: TStringPairVector): string; virtual;\end{ttfamily}

\ifpdf
\end{flushleft}
\fi

\par
\item[\textbf{Description}]
Override this if you want to insert something on @tableOfContents tag. As a parameter you get already prepared tree of sections that your table of contents should show. Each item of Sections is a section on the level 1. Item's Name is section name, item's Value is section caption, item's Data is a TStringPairVector instance that describes subsections (on level 2) below this section. And so on, recursively.

Sections given here are never nil, and item's Data is never nil. But of course they may contain 0 items, and this should be a signal to you that given section doesn't have any subsections.

Default implementation of this method in this class just returns empty string.

\end{list}
\paragraph*{BuildLinks}\hspace*{\fill}

\label{PasDoc_Gen.TDocGenerator-BuildLinks}
\index{BuildLinks}
\begin{list}{}{
\settowidth{\tmplength}{\textbf{Description}}
\setlength{\itemindent}{0cm}
\setlength{\listparindent}{0cm}
\setlength{\leftmargin}{\evensidemargin}
\addtolength{\leftmargin}{\tmplength}
\settowidth{\labelsep}{X}
\addtolength{\leftmargin}{\labelsep}
\setlength{\labelwidth}{\tmplength}
}
\item[\textbf{Declaration}\hfill]
\ifpdf
\begin{flushleft}
\fi
\begin{ttfamily}
public procedure BuildLinks; virtual;\end{ttfamily}

\ifpdf
\end{flushleft}
\fi

\par
\item[\textbf{Description}]
Creates anchors and links for all items in all units.

\end{list}
\paragraph*{ExpandDescriptions}\hspace*{\fill}

\label{PasDoc_Gen.TDocGenerator-ExpandDescriptions}
\index{ExpandDescriptions}
\begin{list}{}{
\settowidth{\tmplength}{\textbf{Description}}
\setlength{\itemindent}{0cm}
\setlength{\listparindent}{0cm}
\setlength{\leftmargin}{\evensidemargin}
\addtolength{\leftmargin}{\tmplength}
\settowidth{\labelsep}{X}
\addtolength{\leftmargin}{\labelsep}
\setlength{\labelwidth}{\tmplength}
}
\item[\textbf{Declaration}\hfill]
\ifpdf
\begin{flushleft}
\fi
\begin{ttfamily}
public procedure ExpandDescriptions;\end{ttfamily}

\ifpdf
\end{flushleft}
\fi

\par
\item[\textbf{Description}]
Expands description for each item in each unit of \begin{ttfamily}Units\end{ttfamily}(\ref{PasDoc_Gen.TDocGenerator-Units}). "Expands description" means that TTagManager.Execute is called, and item's DetailedDescription, AbstractDescription, AbstractDescriptionWasAutomatic (and many others, set by @{-}tags handlers) properties are calculated.

\end{list}
\paragraph*{GetFileExtension}\hspace*{\fill}

\label{PasDoc_Gen.TDocGenerator-GetFileExtension}
\index{GetFileExtension}
\begin{list}{}{
\settowidth{\tmplength}{\textbf{Description}}
\setlength{\itemindent}{0cm}
\setlength{\listparindent}{0cm}
\setlength{\leftmargin}{\evensidemargin}
\addtolength{\leftmargin}{\tmplength}
\settowidth{\labelsep}{X}
\addtolength{\leftmargin}{\labelsep}
\setlength{\labelwidth}{\tmplength}
}
\item[\textbf{Declaration}\hfill]
\ifpdf
\begin{flushleft}
\fi
\begin{ttfamily}
public function GetFileExtension: string; virtual; abstract;\end{ttfamily}

\ifpdf
\end{flushleft}
\fi

\par
\item[\textbf{Description}]
Abstract function that provides file extension for documentation format. Must be overwritten by descendants.

\end{list}
\paragraph*{LoadDescriptionFiles}\hspace*{\fill}

\label{PasDoc_Gen.TDocGenerator-LoadDescriptionFiles}
\index{LoadDescriptionFiles}
\begin{list}{}{
\settowidth{\tmplength}{\textbf{Description}}
\setlength{\itemindent}{0cm}
\setlength{\listparindent}{0cm}
\setlength{\leftmargin}{\evensidemargin}
\addtolength{\leftmargin}{\tmplength}
\settowidth{\labelsep}{X}
\addtolength{\leftmargin}{\labelsep}
\setlength{\labelwidth}{\tmplength}
}
\item[\textbf{Declaration}\hfill]
\ifpdf
\begin{flushleft}
\fi
\begin{ttfamily}
public procedure LoadDescriptionFiles(const c: TStringVector);\end{ttfamily}

\ifpdf
\end{flushleft}
\fi

\par
\item[\textbf{Description}]
Assumes C contains file names as PString variables. Calls \begin{ttfamily}LoadDescriptionFile\end{ttfamily}(\ref{PasDoc_Gen.TDocGenerator-LoadDescriptionFile}) with each file name.

\end{list}
\paragraph*{WriteDocumentation}\hspace*{\fill}

\label{PasDoc_Gen.TDocGenerator-WriteDocumentation}
\index{WriteDocumentation}
\begin{list}{}{
\settowidth{\tmplength}{\textbf{Description}}
\setlength{\itemindent}{0cm}
\setlength{\listparindent}{0cm}
\setlength{\leftmargin}{\evensidemargin}
\addtolength{\leftmargin}{\tmplength}
\settowidth{\labelsep}{X}
\addtolength{\leftmargin}{\labelsep}
\setlength{\labelwidth}{\tmplength}
}
\item[\textbf{Declaration}\hfill]
\ifpdf
\begin{flushleft}
\fi
\begin{ttfamily}
public procedure WriteDocumentation; virtual;\end{ttfamily}

\ifpdf
\end{flushleft}
\fi

\par
\item[\textbf{Description}]
Must be overwritten, writes all documentation. Will create either a single file or one file for each unit and each class, interface or object, depending on output format.

\end{list}
\paragraph*{Create}\hspace*{\fill}

\label{PasDoc_Gen.TDocGenerator-Create}
\index{Create}
\begin{list}{}{
\settowidth{\tmplength}{\textbf{Description}}
\setlength{\itemindent}{0cm}
\setlength{\listparindent}{0cm}
\setlength{\leftmargin}{\evensidemargin}
\addtolength{\leftmargin}{\tmplength}
\settowidth{\labelsep}{X}
\addtolength{\leftmargin}{\labelsep}
\setlength{\labelwidth}{\tmplength}
}
\item[\textbf{Declaration}\hfill]
\ifpdf
\begin{flushleft}
\fi
\begin{ttfamily}
public constructor Create(AOwner: TComponent); override;\end{ttfamily}

\ifpdf
\end{flushleft}
\fi

\end{list}
\paragraph*{Destroy}\hspace*{\fill}

\label{PasDoc_Gen.TDocGenerator-Destroy}
\index{Destroy}
\begin{list}{}{
\settowidth{\tmplength}{\textbf{Description}}
\setlength{\itemindent}{0cm}
\setlength{\listparindent}{0cm}
\setlength{\leftmargin}{\evensidemargin}
\addtolength{\leftmargin}{\tmplength}
\settowidth{\labelsep}{X}
\addtolength{\leftmargin}{\labelsep}
\setlength{\labelwidth}{\tmplength}
}
\item[\textbf{Declaration}\hfill]
\ifpdf
\begin{flushleft}
\fi
\begin{ttfamily}
public destructor Destroy; override;\end{ttfamily}

\ifpdf
\end{flushleft}
\fi

\end{list}
\paragraph*{ParseAbbreviationsFile}\hspace*{\fill}

\label{PasDoc_Gen.TDocGenerator-ParseAbbreviationsFile}
\index{ParseAbbreviationsFile}
\begin{list}{}{
\settowidth{\tmplength}{\textbf{Description}}
\setlength{\itemindent}{0cm}
\setlength{\listparindent}{0cm}
\setlength{\leftmargin}{\evensidemargin}
\addtolength{\leftmargin}{\tmplength}
\settowidth{\labelsep}{X}
\addtolength{\leftmargin}{\labelsep}
\setlength{\labelwidth}{\tmplength}
}
\item[\textbf{Declaration}\hfill]
\ifpdf
\begin{flushleft}
\fi
\begin{ttfamily}
public procedure ParseAbbreviationsFile(const AFileName: string);\end{ttfamily}

\ifpdf
\end{flushleft}
\fi

\end{list}
\section{Types}
\ifpdf
\subsection*{\large{\textbf{TOverviewFile}}\normalsize\hspace{1ex}\hrulefill}
\else
\subsection*{TOverviewFile}
\fi
\label{PasDoc_Gen-TOverviewFile}
\index{TOverviewFile}
\begin{list}{}{
\settowidth{\tmplength}{\textbf{Description}}
\setlength{\itemindent}{0cm}
\setlength{\listparindent}{0cm}
\setlength{\leftmargin}{\evensidemargin}
\addtolength{\leftmargin}{\tmplength}
\settowidth{\labelsep}{X}
\addtolength{\leftmargin}{\labelsep}
\setlength{\labelwidth}{\tmplength}
}
\item[\textbf{Declaration}\hfill]
\ifpdf
\begin{flushleft}
\fi
\begin{ttfamily}
TOverviewFile = (...);\end{ttfamily}

\ifpdf
\end{flushleft}
\fi

\par
\item[\textbf{Description}]
Overview files that pasdoc generates for multiple{-}document{-}formats like HTML (see \begin{ttfamily}TGenericHTMLDocGenerator\end{ttfamily}(\ref{PasDoc_GenHtml.TGenericHTMLDocGenerator})).

But not all of them are supposed to be generated by pasdoc, some must be generated by external programs by user, e.g. uses and class diagrams must be made by user using programs such as GraphViz. See type TCreatedOverviewFile for subrange type of TOverviewFile that specifies only overview files that are really supposed to be made by pasdoc.\item[\textbf{Values}]
\begin{description}
\item[\texttt{ofUnits}] \label{PasDoc_Gen-ofUnits}
\index{}
 
\item[\texttt{ofClassHierarchy}] \label{PasDoc_Gen-ofClassHierarchy}
\index{}
 
\item[\texttt{ofCios}] \label{PasDoc_Gen-ofCios}
\index{}
 
\item[\texttt{ofTypes}] \label{PasDoc_Gen-ofTypes}
\index{}
 
\item[\texttt{ofVariables}] \label{PasDoc_Gen-ofVariables}
\index{}
 
\item[\texttt{ofConstants}] \label{PasDoc_Gen-ofConstants}
\index{}
 
\item[\texttt{ofFunctionsAndProcedures}] \label{PasDoc_Gen-ofFunctionsAndProcedures}
\index{}
 
\item[\texttt{ofIdentifiers}] \label{PasDoc_Gen-ofIdentifiers}
\index{}
 
\item[\texttt{ofGraphVizUses}] \label{PasDoc_Gen-ofGraphVizUses}
\index{}
 
\item[\texttt{ofGraphVizClasses}] \label{PasDoc_Gen-ofGraphVizClasses}
\index{}
 
\end{description}


\end{list}
\ifpdf
\subsection*{\large{\textbf{TCreatedOverviewFile}}\normalsize\hspace{1ex}\hrulefill}
\else
\subsection*{TCreatedOverviewFile}
\fi
\label{PasDoc_Gen-TCreatedOverviewFile}
\index{TCreatedOverviewFile}
\begin{list}{}{
\settowidth{\tmplength}{\textbf{Description}}
\setlength{\itemindent}{0cm}
\setlength{\listparindent}{0cm}
\setlength{\leftmargin}{\evensidemargin}
\addtolength{\leftmargin}{\tmplength}
\settowidth{\labelsep}{X}
\addtolength{\leftmargin}{\labelsep}
\setlength{\labelwidth}{\tmplength}
}
\item[\textbf{Declaration}\hfill]
\ifpdf
\begin{flushleft}
\fi
\begin{ttfamily}
TCreatedOverviewFile = Low(TOverviewFile) .. ofIdentifiers;\end{ttfamily}

\ifpdf
\end{flushleft}
\fi

\end{list}
\ifpdf
\subsection*{\large{\textbf{TLinkLook}}\normalsize\hspace{1ex}\hrulefill}
\else
\subsection*{TLinkLook}
\fi
\label{PasDoc_Gen-TLinkLook}
\index{TLinkLook}
\begin{list}{}{
\settowidth{\tmplength}{\textbf{Description}}
\setlength{\itemindent}{0cm}
\setlength{\listparindent}{0cm}
\setlength{\leftmargin}{\evensidemargin}
\addtolength{\leftmargin}{\tmplength}
\settowidth{\labelsep}{X}
\addtolength{\leftmargin}{\labelsep}
\setlength{\labelwidth}{\tmplength}
}
\item[\textbf{Declaration}\hfill]
\ifpdf
\begin{flushleft}
\fi
\begin{ttfamily}
TLinkLook = (...);\end{ttfamily}

\ifpdf
\end{flushleft}
\fi

\par
\item[\textbf{Description}]
 \item[\textbf{Values}]
\begin{description}
\item[\texttt{llDefault}] \label{PasDoc_Gen-llDefault}
\index{}
 
\item[\texttt{llFull}] \label{PasDoc_Gen-llFull}
\index{}
 
\item[\texttt{llStripped}] \label{PasDoc_Gen-llStripped}
\index{}
 
\end{description}


\end{list}
\ifpdf
\subsection*{\large{\textbf{TLinkContext}}\normalsize\hspace{1ex}\hrulefill}
\else
\subsection*{TLinkContext}
\fi
\label{PasDoc_Gen-TLinkContext}
\index{TLinkContext}
\begin{list}{}{
\settowidth{\tmplength}{\textbf{Description}}
\setlength{\itemindent}{0cm}
\setlength{\listparindent}{0cm}
\setlength{\leftmargin}{\evensidemargin}
\addtolength{\leftmargin}{\tmplength}
\settowidth{\labelsep}{X}
\addtolength{\leftmargin}{\labelsep}
\setlength{\labelwidth}{\tmplength}
}
\item[\textbf{Declaration}\hfill]
\ifpdf
\begin{flushleft}
\fi
\begin{ttfamily}
TLinkContext = (...);\end{ttfamily}

\ifpdf
\end{flushleft}
\fi

\par
\item[\textbf{Description}]
This is used by \begin{ttfamily}TDocGenerator.MakeItemLink\end{ttfamily}(\ref{PasDoc_Gen.TDocGenerator-MakeItemLink})\item[\textbf{Values}]
\begin{description}
\item[\texttt{lcCode}] \label{PasDoc_Gen-lcCode}
\index{}
This means that link is inside some larger code piece, e.g. within FullDeclaration of some item etc. This means that we \textit{may} be inside a context where used font has constant width.
\item[\texttt{lcNormal}] \label{PasDoc_Gen-lcNormal}
\index{}
This means that link is inside some "normal" description text.
\end{description}


\end{list}
\ifpdf
\subsection*{\large{\textbf{TListType}}\normalsize\hspace{1ex}\hrulefill}
\else
\subsection*{TListType}
\fi
\label{PasDoc_Gen-TListType}
\index{TListType}
\begin{list}{}{
\settowidth{\tmplength}{\textbf{Description}}
\setlength{\itemindent}{0cm}
\setlength{\listparindent}{0cm}
\setlength{\leftmargin}{\evensidemargin}
\addtolength{\leftmargin}{\tmplength}
\settowidth{\labelsep}{X}
\addtolength{\leftmargin}{\labelsep}
\setlength{\labelwidth}{\tmplength}
}
\item[\textbf{Declaration}\hfill]
\ifpdf
\begin{flushleft}
\fi
\begin{ttfamily}
TListType = (...);\end{ttfamily}

\ifpdf
\end{flushleft}
\fi

\par
\item[\textbf{Description}]
 \item[\textbf{Values}]
\begin{description}
\item[\texttt{ltUnordered}] \label{PasDoc_Gen-ltUnordered}
\index{}
 
\item[\texttt{ltOrdered}] \label{PasDoc_Gen-ltOrdered}
\index{}
 
\item[\texttt{ltDefinition}] \label{PasDoc_Gen-ltDefinition}
\index{}
 
\end{description}


\end{list}
\ifpdf
\subsection*{\large{\textbf{TListItemSpacing}}\normalsize\hspace{1ex}\hrulefill}
\else
\subsection*{TListItemSpacing}
\fi
\label{PasDoc_Gen-TListItemSpacing}
\index{TListItemSpacing}
\begin{list}{}{
\settowidth{\tmplength}{\textbf{Description}}
\setlength{\itemindent}{0cm}
\setlength{\listparindent}{0cm}
\setlength{\leftmargin}{\evensidemargin}
\addtolength{\leftmargin}{\tmplength}
\settowidth{\labelsep}{X}
\addtolength{\leftmargin}{\labelsep}
\setlength{\labelwidth}{\tmplength}
}
\item[\textbf{Declaration}\hfill]
\ifpdf
\begin{flushleft}
\fi
\begin{ttfamily}
TListItemSpacing = (...);\end{ttfamily}

\ifpdf
\end{flushleft}
\fi

\par
\item[\textbf{Description}]
 \item[\textbf{Values}]
\begin{description}
\item[\texttt{lisCompact}] \label{PasDoc_Gen-lisCompact}
\index{}
 
\item[\texttt{lisParagraph}] \label{PasDoc_Gen-lisParagraph}
\index{}
 
\end{description}


\end{list}
\section{Constants}
\ifpdf
\subsection*{\large{\textbf{OverviewFilesInfo}}\normalsize\hspace{1ex}\hrulefill}
\else
\subsection*{OverviewFilesInfo}
\fi
\label{PasDoc_Gen-OverviewFilesInfo}
\index{OverviewFilesInfo}
\begin{list}{}{
\settowidth{\tmplength}{\textbf{Description}}
\setlength{\itemindent}{0cm}
\setlength{\listparindent}{0cm}
\setlength{\leftmargin}{\evensidemargin}
\addtolength{\leftmargin}{\tmplength}
\settowidth{\labelsep}{X}
\addtolength{\leftmargin}{\labelsep}
\setlength{\labelwidth}{\tmplength}
}
\item[\textbf{Declaration}\hfill]
\ifpdf
\begin{flushleft}
\fi
\begin{ttfamily}
OverviewFilesInfo: array[TOverviewFile] of TOverviewFileInfo = (
    (BaseFileName: 'AllUnits'      ; TranslationId: trUnits                 ; TranslationHeadlineId: trHeadlineUnits                 ; NoItemsTranslationId: trNone   ; ),
    (BaseFileName: 'ClassHierarchy'; TranslationId: trClassHierarchy        ; TranslationHeadlineId: trClassHierarchy ; NoItemsTranslationId: trNoCIOs           ; ),
    (BaseFileName: 'AllClasses'    ; TranslationId: trCio                   ; TranslationHeadlineId: trHeadlineCio                   ; NoItemsTranslationId: trNoCIOs           ; ),
    (BaseFileName: 'AllTypes'      ; TranslationId: trTypes                 ; TranslationHeadlineId: trHeadlineTypes                 ; NoItemsTranslationId: trNoTypes          ; ),
    (BaseFileName: 'AllVariables'  ; TranslationId: trVariables             ; TranslationHeadlineId: trHeadlineVariables             ; NoItemsTranslationId: trNoVariables      ; ),
    (BaseFileName: 'AllConstants'  ; TranslationId: trConstants             ; TranslationHeadlineId: trHeadlineConstants             ; NoItemsTranslationId: trNoConstants      ; ),
    (BaseFileName: 'AllFunctions'  ; TranslationId: trFunctionsAndProcedures; TranslationHeadlineId: trHeadlineFunctionsAndProcedures; NoItemsTranslationId: trNoFunctions      ; ),
    (BaseFileName: 'AllIdentifiers'; TranslationId: trIdentifiers           ; TranslationHeadlineId: trHeadlineIdentifiers           ; NoItemsTranslationId: trNoIdentifiers    ; ),
    (BaseFileName: 'GVUses'        ; TranslationId: trGvUses                ; TranslationHeadlineId: trGvUses         ; NoItemsTranslationId: trNone   ; ),
    (BaseFileName: 'GVClasses'     ; TranslationId: trGvClasses             ; TranslationHeadlineId: trGvClasses      ; NoItemsTranslationId: trNoCIOs ; )
  );\end{ttfamily}

\ifpdf
\end{flushleft}
\fi

\end{list}
\ifpdf
\subsection*{\large{\textbf{LowCreatedOverviewFile}}\normalsize\hspace{1ex}\hrulefill}
\else
\subsection*{LowCreatedOverviewFile}
\fi
\label{PasDoc_Gen-LowCreatedOverviewFile}
\index{LowCreatedOverviewFile}
\begin{list}{}{
\settowidth{\tmplength}{\textbf{Description}}
\setlength{\itemindent}{0cm}
\setlength{\listparindent}{0cm}
\setlength{\leftmargin}{\evensidemargin}
\addtolength{\leftmargin}{\tmplength}
\settowidth{\labelsep}{X}
\addtolength{\leftmargin}{\labelsep}
\setlength{\labelwidth}{\tmplength}
}
\item[\textbf{Declaration}\hfill]
\ifpdf
\begin{flushleft}
\fi
\begin{ttfamily}
LowCreatedOverviewFile = Low(TCreatedOverviewFile);\end{ttfamily}

\ifpdf
\end{flushleft}
\fi

\par
\item[\textbf{Description}]
Using High(TCreatedOverviewFile) or High(Overview) where Overview: TCreatedOverviewFile in PasDoc{\_}GenHtml produces internal error in FPC 2.0.0. Same for Low(TCreatedOverviewFile).

This is submitted as FPC bug 4140, [\href{http://www.freepascal.org/bugs/showrec.php3?ID=4140}{http://www.freepascal.org/bugs/showrec.php3?ID=4140}]. Fixed in FPC 2.0.1 and FPC 2.1.1.

\end{list}
\ifpdf
\subsection*{\large{\textbf{HighCreatedOverviewFile}}\normalsize\hspace{1ex}\hrulefill}
\else
\subsection*{HighCreatedOverviewFile}
\fi
\label{PasDoc_Gen-HighCreatedOverviewFile}
\index{HighCreatedOverviewFile}
\begin{list}{}{
\settowidth{\tmplength}{\textbf{Description}}
\setlength{\itemindent}{0cm}
\setlength{\listparindent}{0cm}
\setlength{\leftmargin}{\evensidemargin}
\addtolength{\leftmargin}{\tmplength}
\settowidth{\labelsep}{X}
\addtolength{\leftmargin}{\labelsep}
\setlength{\labelwidth}{\tmplength}
}
\item[\textbf{Declaration}\hfill]
\ifpdf
\begin{flushleft}
\fi
\begin{ttfamily}
HighCreatedOverviewFile = High(TCreatedOverviewFile);\end{ttfamily}

\ifpdf
\end{flushleft}
\fi

\end{list}
\section{Authors}
\par
Johannes Berg {$<$}johannes@sipsolutions.de{$>$}

\par
Ralf Junker (delphi@zeitungsjunge.de)

\par
Ivan Montes Velencoso (senbei@teleline.es)

\par
Marco Schmidt (marcoschmidt@geocities.com)

\par
Philippe Jean Dit Bailleul (jdb@abacom.com)

\par
Rodrigo Urubatan Ferreira Jardim (rodrigo@netscape.net)

\par
Grzegorz Skoczylas {$<$}gskoczylas@rekord.pl{$>$}

\par
Pierre Woestyn {$<$}pwoestyn@users.sourceforge.net{$>$}

\par
Michalis Kamburelis

\par
Richard B. Winston {$<$}rbwinst@usgs.gov{$>$}

\par
Ascanio Pressato

\par
Arno Garrels {$<$}first name.name@nospamgmx.de{$>$}

\section{Created}
\par
30 Aug 1998


\chapter{Unit PasDoc{\_}GenHtml}
\label{PasDoc_GenHtml}
\index{PasDoc{\_}GenHtml}
\section{Description}
Provides HTML document generator object.\hfill\vspace*{1ex}

              

Implements an object to generate HTML documentation, overriding many of \begin{ttfamily}TDocGenerator\end{ttfamily}(\ref{PasDoc_Gen.TDocGenerator})'s virtual methods.
\section{Uses}
\begin{itemize}
\item \begin{ttfamily}PasDoc{\_}Utils\end{ttfamily}(\ref{PasDoc_Utils})\item \begin{ttfamily}PasDoc{\_}Gen\end{ttfamily}(\ref{PasDoc_Gen})\item \begin{ttfamily}PasDoc{\_}Items\end{ttfamily}(\ref{PasDoc_Items})\item \begin{ttfamily}PasDoc{\_}Languages\end{ttfamily}(\ref{PasDoc_Languages})\item \begin{ttfamily}PasDoc{\_}StringVector\end{ttfamily}(\ref{PasDoc_StringVector})\item \begin{ttfamily}PasDoc{\_}Types\end{ttfamily}(\ref{PasDoc_Types})\item \begin{ttfamily}Classes\end{ttfamily}\item \begin{ttfamily}PasDoc{\_}StringPairVector\end{ttfamily}(\ref{PasDoc_StringPairVector})\end{itemize}
\section{Overview}
\begin{description}
\item[\texttt{\begin{ttfamily}TGenericHTMLDocGenerator\end{ttfamily} Class}]generates HTML documentation
\item[\texttt{\begin{ttfamily}THTMLDocGenerator\end{ttfamily} Class}]Right now this is the same thing as TGenericHTMLDocGenerator.
\end{description}
\section{Classes, Interfaces, Objects and Records}
\ifpdf
\subsection*{\large{\textbf{TGenericHTMLDocGenerator Class}}\normalsize\hspace{1ex}\hrulefill}
\else
\subsection*{TGenericHTMLDocGenerator Class}
\fi
\label{PasDoc_GenHtml.TGenericHTMLDocGenerator}
\index{TGenericHTMLDocGenerator}
\subsubsection*{\large{\textbf{Hierarchy}}\normalsize\hspace{1ex}\hfill}
TGenericHTMLDocGenerator {$>$} \begin{ttfamily}TDocGenerator\end{ttfamily}(\ref{PasDoc_Gen.TDocGenerator}) {$>$} 
TComponent
\subsubsection*{\large{\textbf{Description}}\normalsize\hspace{1ex}\hfill}
generates HTML documentation\hfill\vspace*{1ex}

 Extends \begin{ttfamily}TDocGenerator\end{ttfamily}(\ref{PasDoc_Gen.TDocGenerator}) and overwrites many of its methods to generate output in HTML (HyperText Markup Language) format.\subsubsection*{\large{\textbf{Properties}}\normalsize\hspace{1ex}\hfill}
\begin{list}{}{
\settowidth{\tmplength}{\textbf{NumericFilenames}}
\setlength{\itemindent}{0cm}
\setlength{\listparindent}{0cm}
\setlength{\leftmargin}{\evensidemargin}
\addtolength{\leftmargin}{\tmplength}
\settowidth{\labelsep}{X}
\addtolength{\leftmargin}{\labelsep}
\setlength{\labelwidth}{\tmplength}
}
\label{PasDoc_GenHtml.TGenericHTMLDocGenerator-Header}
\index{Header}
\item[\textbf{Header}\hfill]
\ifpdf
\begin{flushleft}
\fi
\begin{ttfamily}
published property Header: string read FHeader write FHeader;\end{ttfamily}

\ifpdf
\end{flushleft}
\fi


\par some HTML code to be written as header for every page\label{PasDoc_GenHtml.TGenericHTMLDocGenerator-Footer}
\index{Footer}
\item[\textbf{Footer}\hfill]
\ifpdf
\begin{flushleft}
\fi
\begin{ttfamily}
published property Footer: string read FFooter write FFooter;\end{ttfamily}

\ifpdf
\end{flushleft}
\fi


\par some HTML code to be written as footer for every page\label{PasDoc_GenHtml.TGenericHTMLDocGenerator-HtmlBodyBegin}
\index{HtmlBodyBegin}
\item[\textbf{HtmlBodyBegin}\hfill]
\ifpdf
\begin{flushleft}
\fi
\begin{ttfamily}
published property HtmlBodyBegin: string read FHtmlBodyBegin write FHtmlBodyBegin;\end{ttfamily}

\ifpdf
\end{flushleft}
\fi


\par  \label{PasDoc_GenHtml.TGenericHTMLDocGenerator-HtmlBodyEnd}
\index{HtmlBodyEnd}
\item[\textbf{HtmlBodyEnd}\hfill]
\ifpdf
\begin{flushleft}
\fi
\begin{ttfamily}
published property HtmlBodyEnd: string read FHtmlBodyEnd write FHtmlBodyEnd;\end{ttfamily}

\ifpdf
\end{flushleft}
\fi


\par  \label{PasDoc_GenHtml.TGenericHTMLDocGenerator-HtmlHead}
\index{HtmlHead}
\item[\textbf{HtmlHead}\hfill]
\ifpdf
\begin{flushleft}
\fi
\begin{ttfamily}
published property HtmlHead: string read FHtmlHead write FHtmlHead;\end{ttfamily}

\ifpdf
\end{flushleft}
\fi


\par  \label{PasDoc_GenHtml.TGenericHTMLDocGenerator-CSS}
\index{CSS}
\item[\textbf{CSS}\hfill]
\ifpdf
\begin{flushleft}
\fi
\begin{ttfamily}
published property CSS: string read FCSS write FCSS;\end{ttfamily}

\ifpdf
\end{flushleft}
\fi


\par the content of the cascading stylesheet\label{PasDoc_GenHtml.TGenericHTMLDocGenerator-NumericFilenames}
\index{NumericFilenames}
\item[\textbf{NumericFilenames}\hfill]
\ifpdf
\begin{flushleft}
\fi
\begin{ttfamily}
published property NumericFilenames: boolean read FNumericFilenames write FNumericFilenames
      default false;\end{ttfamily}

\ifpdf
\end{flushleft}
\fi


\par if set to true, numeric filenames will be used rather than names with multiple dots\label{PasDoc_GenHtml.TGenericHTMLDocGenerator-UseTipueSearch}
\index{UseTipueSearch}
\item[\textbf{UseTipueSearch}\hfill]
\ifpdf
\begin{flushleft}
\fi
\begin{ttfamily}
published property UseTipueSearch: boolean read FUseTipueSearch write FUseTipueSearch
      default False;\end{ttfamily}

\ifpdf
\end{flushleft}
\fi


\par Enable Tiptue fulltext search. See [\href{https://github.com/pasdoc/pasdoc/wiki/UseTipueSearchOption}{https://github.com/pasdoc/pasdoc/wiki/UseTipueSearchOption}]\end{list}
\subsubsection*{\large{\textbf{Methods}}\normalsize\hspace{1ex}\hfill}
\paragraph*{MakeHead}\hspace*{\fill}

\label{PasDoc_GenHtml.TGenericHTMLDocGenerator-MakeHead}
\index{MakeHead}
\begin{list}{}{
\settowidth{\tmplength}{\textbf{Description}}
\setlength{\itemindent}{0cm}
\setlength{\listparindent}{0cm}
\setlength{\leftmargin}{\evensidemargin}
\addtolength{\leftmargin}{\tmplength}
\settowidth{\labelsep}{X}
\addtolength{\leftmargin}{\labelsep}
\setlength{\labelwidth}{\tmplength}
}
\item[\textbf{Declaration}\hfill]
\ifpdf
\begin{flushleft}
\fi
\begin{ttfamily}
protected function MakeHead: string;\end{ttfamily}

\ifpdf
\end{flushleft}
\fi

\par
\item[\textbf{Description}]
Return common HTML content that goes inside {$<$}head{$>$}.

\end{list}
\paragraph*{MakeBodyBegin}\hspace*{\fill}

\label{PasDoc_GenHtml.TGenericHTMLDocGenerator-MakeBodyBegin}
\index{MakeBodyBegin}
\begin{list}{}{
\settowidth{\tmplength}{\textbf{Description}}
\setlength{\itemindent}{0cm}
\setlength{\listparindent}{0cm}
\setlength{\leftmargin}{\evensidemargin}
\addtolength{\leftmargin}{\tmplength}
\settowidth{\labelsep}{X}
\addtolength{\leftmargin}{\labelsep}
\setlength{\labelwidth}{\tmplength}
}
\item[\textbf{Declaration}\hfill]
\ifpdf
\begin{flushleft}
\fi
\begin{ttfamily}
protected function MakeBodyBegin: string; virtual;\end{ttfamily}

\ifpdf
\end{flushleft}
\fi

\par
\item[\textbf{Description}]
Return common HTML content that goes right after {$<$}body{$>$}.

\end{list}
\paragraph*{MakeBodyEnd}\hspace*{\fill}

\label{PasDoc_GenHtml.TGenericHTMLDocGenerator-MakeBodyEnd}
\index{MakeBodyEnd}
\begin{list}{}{
\settowidth{\tmplength}{\textbf{Description}}
\setlength{\itemindent}{0cm}
\setlength{\listparindent}{0cm}
\setlength{\leftmargin}{\evensidemargin}
\addtolength{\leftmargin}{\tmplength}
\settowidth{\labelsep}{X}
\addtolength{\leftmargin}{\labelsep}
\setlength{\labelwidth}{\tmplength}
}
\item[\textbf{Declaration}\hfill]
\ifpdf
\begin{flushleft}
\fi
\begin{ttfamily}
protected function MakeBodyEnd: string; virtual;\end{ttfamily}

\ifpdf
\end{flushleft}
\fi

\par
\item[\textbf{Description}]
Return common HTML content that goes right before {$<$}/body{$>$}.

\end{list}
\paragraph*{ConvertString}\hspace*{\fill}

\label{PasDoc_GenHtml.TGenericHTMLDocGenerator-ConvertString}
\index{ConvertString}
\begin{list}{}{
\settowidth{\tmplength}{\textbf{Description}}
\setlength{\itemindent}{0cm}
\setlength{\listparindent}{0cm}
\setlength{\leftmargin}{\evensidemargin}
\addtolength{\leftmargin}{\tmplength}
\settowidth{\labelsep}{X}
\addtolength{\leftmargin}{\labelsep}
\setlength{\labelwidth}{\tmplength}
}
\item[\textbf{Declaration}\hfill]
\ifpdf
\begin{flushleft}
\fi
\begin{ttfamily}
protected function ConvertString(const s: string): string; override;\end{ttfamily}

\ifpdf
\end{flushleft}
\fi

\end{list}
\paragraph*{ConvertChar}\hspace*{\fill}

\label{PasDoc_GenHtml.TGenericHTMLDocGenerator-ConvertChar}
\index{ConvertChar}
\begin{list}{}{
\settowidth{\tmplength}{\textbf{Description}}
\setlength{\itemindent}{0cm}
\setlength{\listparindent}{0cm}
\setlength{\leftmargin}{\evensidemargin}
\addtolength{\leftmargin}{\tmplength}
\settowidth{\labelsep}{X}
\addtolength{\leftmargin}{\labelsep}
\setlength{\labelwidth}{\tmplength}
}
\item[\textbf{Declaration}\hfill]
\ifpdf
\begin{flushleft}
\fi
\begin{ttfamily}
protected function ConvertChar(c: char): string; override;\end{ttfamily}

\ifpdf
\end{flushleft}
\fi

\par
\item[\textbf{Description}]
Called by \begin{ttfamily}ConvertString\end{ttfamily}(\ref{PasDoc_GenHtml.TGenericHTMLDocGenerator-ConvertString}) to convert a character. Will convert special characters to their html escape sequence {-}{$>$} test

\end{list}
\paragraph*{WriteUnit}\hspace*{\fill}

\label{PasDoc_GenHtml.TGenericHTMLDocGenerator-WriteUnit}
\index{WriteUnit}
\begin{list}{}{
\settowidth{\tmplength}{\textbf{Description}}
\setlength{\itemindent}{0cm}
\setlength{\listparindent}{0cm}
\setlength{\leftmargin}{\evensidemargin}
\addtolength{\leftmargin}{\tmplength}
\settowidth{\labelsep}{X}
\addtolength{\leftmargin}{\labelsep}
\setlength{\labelwidth}{\tmplength}
}
\item[\textbf{Declaration}\hfill]
\ifpdf
\begin{flushleft}
\fi
\begin{ttfamily}
protected procedure WriteUnit(const HL: integer; const U: TPasUnit); override;\end{ttfamily}

\ifpdf
\end{flushleft}
\fi

\end{list}
\paragraph*{HtmlString}\hspace*{\fill}

\label{PasDoc_GenHtml.TGenericHTMLDocGenerator-HtmlString}
\index{HtmlString}
\begin{list}{}{
\settowidth{\tmplength}{\textbf{Description}}
\setlength{\itemindent}{0cm}
\setlength{\listparindent}{0cm}
\setlength{\leftmargin}{\evensidemargin}
\addtolength{\leftmargin}{\tmplength}
\settowidth{\labelsep}{X}
\addtolength{\leftmargin}{\labelsep}
\setlength{\labelwidth}{\tmplength}
}
\item[\textbf{Declaration}\hfill]
\ifpdf
\begin{flushleft}
\fi
\begin{ttfamily}
protected function HtmlString(const S: string): string; override;\end{ttfamily}

\ifpdf
\end{flushleft}
\fi

\par
\item[\textbf{Description}]
overrides \begin{ttfamily}TDocGenerator.HtmlString\end{ttfamily}(\ref{PasDoc_Gen.TDocGenerator-HtmlString}).HtmlString to return the string verbatim (\begin{ttfamily}TDocGenerator.HtmlString\end{ttfamily}(\ref{PasDoc_Gen.TDocGenerator-HtmlString}) discards those strings)

\end{list}
\paragraph*{FormatPascalCode}\hspace*{\fill}

\label{PasDoc_GenHtml.TGenericHTMLDocGenerator-FormatPascalCode}
\index{FormatPascalCode}
\begin{list}{}{
\settowidth{\tmplength}{\textbf{Description}}
\setlength{\itemindent}{0cm}
\setlength{\listparindent}{0cm}
\setlength{\leftmargin}{\evensidemargin}
\addtolength{\leftmargin}{\tmplength}
\settowidth{\labelsep}{X}
\addtolength{\leftmargin}{\labelsep}
\setlength{\labelwidth}{\tmplength}
}
\item[\textbf{Declaration}\hfill]
\ifpdf
\begin{flushleft}
\fi
\begin{ttfamily}
protected function FormatPascalCode(const Line: string): string; override;\end{ttfamily}

\ifpdf
\end{flushleft}
\fi

\par
\item[\textbf{Description}]
FormatPascalCode will cause Line to be formatted in the way that Pascal code is formatted in Delphi.

\end{list}
\paragraph*{FormatComment}\hspace*{\fill}

\label{PasDoc_GenHtml.TGenericHTMLDocGenerator-FormatComment}
\index{FormatComment}
\begin{list}{}{
\settowidth{\tmplength}{\textbf{Description}}
\setlength{\itemindent}{0cm}
\setlength{\listparindent}{0cm}
\setlength{\leftmargin}{\evensidemargin}
\addtolength{\leftmargin}{\tmplength}
\settowidth{\labelsep}{X}
\addtolength{\leftmargin}{\labelsep}
\setlength{\labelwidth}{\tmplength}
}
\item[\textbf{Declaration}\hfill]
\ifpdf
\begin{flushleft}
\fi
\begin{ttfamily}
protected function FormatComment(AString: string): string; override;\end{ttfamily}

\ifpdf
\end{flushleft}
\fi

\par
\item[\textbf{Description}]
FormatComment will cause AString to be formatted in the way that comments other than compiler directives are formatted in Delphi. See: \begin{ttfamily}FormatCompilerComment\end{ttfamily}(\ref{PasDoc_GenHtml.TGenericHTMLDocGenerator-FormatCompilerComment}).

\end{list}
\paragraph*{FormatHex}\hspace*{\fill}

\label{PasDoc_GenHtml.TGenericHTMLDocGenerator-FormatHex}
\index{FormatHex}
\begin{list}{}{
\settowidth{\tmplength}{\textbf{Description}}
\setlength{\itemindent}{0cm}
\setlength{\listparindent}{0cm}
\setlength{\leftmargin}{\evensidemargin}
\addtolength{\leftmargin}{\tmplength}
\settowidth{\labelsep}{X}
\addtolength{\leftmargin}{\labelsep}
\setlength{\labelwidth}{\tmplength}
}
\item[\textbf{Declaration}\hfill]
\ifpdf
\begin{flushleft}
\fi
\begin{ttfamily}
protected function FormatHex(AString: string): string; override;\end{ttfamily}

\ifpdf
\end{flushleft}
\fi

\par
\item[\textbf{Description}]
FormatHex will cause AString to be formatted in the way that Hex are formatted in Delphi.

\end{list}
\paragraph*{FormatNumeric}\hspace*{\fill}

\label{PasDoc_GenHtml.TGenericHTMLDocGenerator-FormatNumeric}
\index{FormatNumeric}
\begin{list}{}{
\settowidth{\tmplength}{\textbf{Description}}
\setlength{\itemindent}{0cm}
\setlength{\listparindent}{0cm}
\setlength{\leftmargin}{\evensidemargin}
\addtolength{\leftmargin}{\tmplength}
\settowidth{\labelsep}{X}
\addtolength{\leftmargin}{\labelsep}
\setlength{\labelwidth}{\tmplength}
}
\item[\textbf{Declaration}\hfill]
\ifpdf
\begin{flushleft}
\fi
\begin{ttfamily}
protected function FormatNumeric(AString: string): string; override;\end{ttfamily}

\ifpdf
\end{flushleft}
\fi

\par
\item[\textbf{Description}]
FormatNumeric will cause AString to be formatted in the way that Numeric are formatted in Delphi.

\end{list}
\paragraph*{FormatFloat}\hspace*{\fill}

\label{PasDoc_GenHtml.TGenericHTMLDocGenerator-FormatFloat}
\index{FormatFloat}
\begin{list}{}{
\settowidth{\tmplength}{\textbf{Description}}
\setlength{\itemindent}{0cm}
\setlength{\listparindent}{0cm}
\setlength{\leftmargin}{\evensidemargin}
\addtolength{\leftmargin}{\tmplength}
\settowidth{\labelsep}{X}
\addtolength{\leftmargin}{\labelsep}
\setlength{\labelwidth}{\tmplength}
}
\item[\textbf{Declaration}\hfill]
\ifpdf
\begin{flushleft}
\fi
\begin{ttfamily}
protected function FormatFloat(AString: string): string; override;\end{ttfamily}

\ifpdf
\end{flushleft}
\fi

\par
\item[\textbf{Description}]
FormatFloat will cause AString to be formatted in the way that Float are formatted in Delphi.

\end{list}
\paragraph*{FormatString}\hspace*{\fill}

\label{PasDoc_GenHtml.TGenericHTMLDocGenerator-FormatString}
\index{FormatString}
\begin{list}{}{
\settowidth{\tmplength}{\textbf{Description}}
\setlength{\itemindent}{0cm}
\setlength{\listparindent}{0cm}
\setlength{\leftmargin}{\evensidemargin}
\addtolength{\leftmargin}{\tmplength}
\settowidth{\labelsep}{X}
\addtolength{\leftmargin}{\labelsep}
\setlength{\labelwidth}{\tmplength}
}
\item[\textbf{Declaration}\hfill]
\ifpdf
\begin{flushleft}
\fi
\begin{ttfamily}
protected function FormatString(AString: string): string; override;\end{ttfamily}

\ifpdf
\end{flushleft}
\fi

\par
\item[\textbf{Description}]
FormatKeyWord will cause AString to be formatted in the way that strings are formatted in Delphi.

\end{list}
\paragraph*{FormatKeyWord}\hspace*{\fill}

\label{PasDoc_GenHtml.TGenericHTMLDocGenerator-FormatKeyWord}
\index{FormatKeyWord}
\begin{list}{}{
\settowidth{\tmplength}{\textbf{Description}}
\setlength{\itemindent}{0cm}
\setlength{\listparindent}{0cm}
\setlength{\leftmargin}{\evensidemargin}
\addtolength{\leftmargin}{\tmplength}
\settowidth{\labelsep}{X}
\addtolength{\leftmargin}{\labelsep}
\setlength{\labelwidth}{\tmplength}
}
\item[\textbf{Declaration}\hfill]
\ifpdf
\begin{flushleft}
\fi
\begin{ttfamily}
protected function FormatKeyWord(AString: string): string; override;\end{ttfamily}

\ifpdf
\end{flushleft}
\fi

\par
\item[\textbf{Description}]
FormatKeyWord will cause AString to be formatted in the way that reserved words are formatted in Delphi.

\end{list}
\paragraph*{FormatCompilerComment}\hspace*{\fill}

\label{PasDoc_GenHtml.TGenericHTMLDocGenerator-FormatCompilerComment}
\index{FormatCompilerComment}
\begin{list}{}{
\settowidth{\tmplength}{\textbf{Description}}
\setlength{\itemindent}{0cm}
\setlength{\listparindent}{0cm}
\setlength{\leftmargin}{\evensidemargin}
\addtolength{\leftmargin}{\tmplength}
\settowidth{\labelsep}{X}
\addtolength{\leftmargin}{\labelsep}
\setlength{\labelwidth}{\tmplength}
}
\item[\textbf{Declaration}\hfill]
\ifpdf
\begin{flushleft}
\fi
\begin{ttfamily}
protected function FormatCompilerComment(AString: string): string; override;\end{ttfamily}

\ifpdf
\end{flushleft}
\fi

\par
\item[\textbf{Description}]
FormatCompilerComment will cause AString to be formatted in the way that compiler directives are formatted in Delphi.

\end{list}
\paragraph*{CodeString}\hspace*{\fill}

\label{PasDoc_GenHtml.TGenericHTMLDocGenerator-CodeString}
\index{CodeString}
\begin{list}{}{
\settowidth{\tmplength}{\textbf{Description}}
\setlength{\itemindent}{0cm}
\setlength{\listparindent}{0cm}
\setlength{\leftmargin}{\evensidemargin}
\addtolength{\leftmargin}{\tmplength}
\settowidth{\labelsep}{X}
\addtolength{\leftmargin}{\labelsep}
\setlength{\labelwidth}{\tmplength}
}
\item[\textbf{Declaration}\hfill]
\ifpdf
\begin{flushleft}
\fi
\begin{ttfamily}
protected function CodeString(const s: string): string; override;\end{ttfamily}

\ifpdf
\end{flushleft}
\fi

\par
\item[\textbf{Description}]
Makes a String look like a coded String, i.e. {$<$}CODE{$>$}TheString{$<$}/CODE{$>$} in Html.

\end{list}
\paragraph*{CreateLink}\hspace*{\fill}

\label{PasDoc_GenHtml.TGenericHTMLDocGenerator-CreateLink}
\index{CreateLink}
\begin{list}{}{
\settowidth{\tmplength}{\textbf{Description}}
\setlength{\itemindent}{0cm}
\setlength{\listparindent}{0cm}
\setlength{\leftmargin}{\evensidemargin}
\addtolength{\leftmargin}{\tmplength}
\settowidth{\labelsep}{X}
\addtolength{\leftmargin}{\labelsep}
\setlength{\labelwidth}{\tmplength}
}
\item[\textbf{Declaration}\hfill]
\ifpdf
\begin{flushleft}
\fi
\begin{ttfamily}
protected function CreateLink(const Item: TBaseItem): string; override;\end{ttfamily}

\ifpdf
\end{flushleft}
\fi

\par
\item[\textbf{Description}]
Returns a link to an anchor within a document. HTML simply concatenates the strings with a "{\#}" character between them.

\end{list}
\paragraph*{WriteStartOfCode}\hspace*{\fill}

\label{PasDoc_GenHtml.TGenericHTMLDocGenerator-WriteStartOfCode}
\index{WriteStartOfCode}
\begin{list}{}{
\settowidth{\tmplength}{\textbf{Description}}
\setlength{\itemindent}{0cm}
\setlength{\listparindent}{0cm}
\setlength{\leftmargin}{\evensidemargin}
\addtolength{\leftmargin}{\tmplength}
\settowidth{\labelsep}{X}
\addtolength{\leftmargin}{\labelsep}
\setlength{\labelwidth}{\tmplength}
}
\item[\textbf{Declaration}\hfill]
\ifpdf
\begin{flushleft}
\fi
\begin{ttfamily}
protected procedure WriteStartOfCode; override;\end{ttfamily}

\ifpdf
\end{flushleft}
\fi

\end{list}
\paragraph*{WriteEndOfCode}\hspace*{\fill}

\label{PasDoc_GenHtml.TGenericHTMLDocGenerator-WriteEndOfCode}
\index{WriteEndOfCode}
\begin{list}{}{
\settowidth{\tmplength}{\textbf{Description}}
\setlength{\itemindent}{0cm}
\setlength{\listparindent}{0cm}
\setlength{\leftmargin}{\evensidemargin}
\addtolength{\leftmargin}{\tmplength}
\settowidth{\labelsep}{X}
\addtolength{\leftmargin}{\labelsep}
\setlength{\labelwidth}{\tmplength}
}
\item[\textbf{Declaration}\hfill]
\ifpdf
\begin{flushleft}
\fi
\begin{ttfamily}
protected procedure WriteEndOfCode; override;\end{ttfamily}

\ifpdf
\end{flushleft}
\fi

\end{list}
\paragraph*{WriteAnchor}\hspace*{\fill}

\label{PasDoc_GenHtml.TGenericHTMLDocGenerator-WriteAnchor}
\index{WriteAnchor}
\begin{list}{}{
\settowidth{\tmplength}{\textbf{Description}}
\setlength{\itemindent}{0cm}
\setlength{\listparindent}{0cm}
\setlength{\leftmargin}{\evensidemargin}
\addtolength{\leftmargin}{\tmplength}
\settowidth{\labelsep}{X}
\addtolength{\leftmargin}{\labelsep}
\setlength{\labelwidth}{\tmplength}
}
\item[\textbf{Declaration}\hfill]
\ifpdf
\begin{flushleft}
\fi
\begin{ttfamily}
protected procedure WriteAnchor(const AName: string); overload;\end{ttfamily}

\ifpdf
\end{flushleft}
\fi

\end{list}
\paragraph*{WriteAnchor}\hspace*{\fill}

\label{PasDoc_GenHtml.TGenericHTMLDocGenerator-WriteAnchor}
\index{WriteAnchor}
\begin{list}{}{
\settowidth{\tmplength}{\textbf{Description}}
\setlength{\itemindent}{0cm}
\setlength{\listparindent}{0cm}
\setlength{\leftmargin}{\evensidemargin}
\addtolength{\leftmargin}{\tmplength}
\settowidth{\labelsep}{X}
\addtolength{\leftmargin}{\labelsep}
\setlength{\labelwidth}{\tmplength}
}
\item[\textbf{Declaration}\hfill]
\ifpdf
\begin{flushleft}
\fi
\begin{ttfamily}
protected procedure WriteAnchor(const AName, Caption: string); overload;\end{ttfamily}

\ifpdf
\end{flushleft}
\fi

\par
\item[\textbf{Description}]
Write an anchor. Note that the Caption is assumed to be already processed with the \begin{ttfamily}ConvertString\end{ttfamily}(\ref{PasDoc_GenHtml.TGenericHTMLDocGenerator-ConvertString}).

\end{list}
\paragraph*{Paragraph}\hspace*{\fill}

\label{PasDoc_GenHtml.TGenericHTMLDocGenerator-Paragraph}
\index{Paragraph}
\begin{list}{}{
\settowidth{\tmplength}{\textbf{Description}}
\setlength{\itemindent}{0cm}
\setlength{\listparindent}{0cm}
\setlength{\leftmargin}{\evensidemargin}
\addtolength{\leftmargin}{\tmplength}
\settowidth{\labelsep}{X}
\addtolength{\leftmargin}{\labelsep}
\setlength{\labelwidth}{\tmplength}
}
\item[\textbf{Declaration}\hfill]
\ifpdf
\begin{flushleft}
\fi
\begin{ttfamily}
protected function Paragraph: string; override;\end{ttfamily}

\ifpdf
\end{flushleft}
\fi

\end{list}
\paragraph*{EnDash}\hspace*{\fill}

\label{PasDoc_GenHtml.TGenericHTMLDocGenerator-EnDash}
\index{EnDash}
\begin{list}{}{
\settowidth{\tmplength}{\textbf{Description}}
\setlength{\itemindent}{0cm}
\setlength{\listparindent}{0cm}
\setlength{\leftmargin}{\evensidemargin}
\addtolength{\leftmargin}{\tmplength}
\settowidth{\labelsep}{X}
\addtolength{\leftmargin}{\labelsep}
\setlength{\labelwidth}{\tmplength}
}
\item[\textbf{Declaration}\hfill]
\ifpdf
\begin{flushleft}
\fi
\begin{ttfamily}
protected function EnDash: string; override;\end{ttfamily}

\ifpdf
\end{flushleft}
\fi

\end{list}
\paragraph*{EmDash}\hspace*{\fill}

\label{PasDoc_GenHtml.TGenericHTMLDocGenerator-EmDash}
\index{EmDash}
\begin{list}{}{
\settowidth{\tmplength}{\textbf{Description}}
\setlength{\itemindent}{0cm}
\setlength{\listparindent}{0cm}
\setlength{\leftmargin}{\evensidemargin}
\addtolength{\leftmargin}{\tmplength}
\settowidth{\labelsep}{X}
\addtolength{\leftmargin}{\labelsep}
\setlength{\labelwidth}{\tmplength}
}
\item[\textbf{Declaration}\hfill]
\ifpdf
\begin{flushleft}
\fi
\begin{ttfamily}
protected function EmDash: string; override;\end{ttfamily}

\ifpdf
\end{flushleft}
\fi

\end{list}
\paragraph*{LineBreak}\hspace*{\fill}

\label{PasDoc_GenHtml.TGenericHTMLDocGenerator-LineBreak}
\index{LineBreak}
\begin{list}{}{
\settowidth{\tmplength}{\textbf{Description}}
\setlength{\itemindent}{0cm}
\setlength{\listparindent}{0cm}
\setlength{\leftmargin}{\evensidemargin}
\addtolength{\leftmargin}{\tmplength}
\settowidth{\labelsep}{X}
\addtolength{\leftmargin}{\labelsep}
\setlength{\labelwidth}{\tmplength}
}
\item[\textbf{Declaration}\hfill]
\ifpdf
\begin{flushleft}
\fi
\begin{ttfamily}
protected function LineBreak: string; override;\end{ttfamily}

\ifpdf
\end{flushleft}
\fi

\end{list}
\paragraph*{URLLink}\hspace*{\fill}

\label{PasDoc_GenHtml.TGenericHTMLDocGenerator-URLLink}
\index{URLLink}
\begin{list}{}{
\settowidth{\tmplength}{\textbf{Description}}
\setlength{\itemindent}{0cm}
\setlength{\listparindent}{0cm}
\setlength{\leftmargin}{\evensidemargin}
\addtolength{\leftmargin}{\tmplength}
\settowidth{\labelsep}{X}
\addtolength{\leftmargin}{\labelsep}
\setlength{\labelwidth}{\tmplength}
}
\item[\textbf{Declaration}\hfill]
\ifpdf
\begin{flushleft}
\fi
\begin{ttfamily}
protected function URLLink(const URL: string): string; override;\end{ttfamily}

\ifpdf
\end{flushleft}
\fi

\end{list}
\paragraph*{URLLink}\hspace*{\fill}

\label{PasDoc_GenHtml.TGenericHTMLDocGenerator-URLLink}
\index{URLLink}
\begin{list}{}{
\settowidth{\tmplength}{\textbf{Description}}
\setlength{\itemindent}{0cm}
\setlength{\listparindent}{0cm}
\setlength{\leftmargin}{\evensidemargin}
\addtolength{\leftmargin}{\tmplength}
\settowidth{\labelsep}{X}
\addtolength{\leftmargin}{\labelsep}
\setlength{\labelwidth}{\tmplength}
}
\item[\textbf{Declaration}\hfill]
\ifpdf
\begin{flushleft}
\fi
\begin{ttfamily}
protected function URLLink(const URL, LinkDisplay: string): string; override;\end{ttfamily}

\ifpdf
\end{flushleft}
\fi

\end{list}
\paragraph*{WriteExternalCore}\hspace*{\fill}

\label{PasDoc_GenHtml.TGenericHTMLDocGenerator-WriteExternalCore}
\index{WriteExternalCore}
\begin{list}{}{
\settowidth{\tmplength}{\textbf{Description}}
\setlength{\itemindent}{0cm}
\setlength{\listparindent}{0cm}
\setlength{\leftmargin}{\evensidemargin}
\addtolength{\leftmargin}{\tmplength}
\settowidth{\labelsep}{X}
\addtolength{\leftmargin}{\labelsep}
\setlength{\labelwidth}{\tmplength}
}
\item[\textbf{Declaration}\hfill]
\ifpdf
\begin{flushleft}
\fi
\begin{ttfamily}
protected procedure WriteExternalCore(const ExternalItem: TExternalItem; const Id: TTranslationID); override;\end{ttfamily}

\ifpdf
\end{flushleft}
\fi

\end{list}
\paragraph*{MakeItemLink}\hspace*{\fill}

\label{PasDoc_GenHtml.TGenericHTMLDocGenerator-MakeItemLink}
\index{MakeItemLink}
\begin{list}{}{
\settowidth{\tmplength}{\textbf{Description}}
\setlength{\itemindent}{0cm}
\setlength{\listparindent}{0cm}
\setlength{\leftmargin}{\evensidemargin}
\addtolength{\leftmargin}{\tmplength}
\settowidth{\labelsep}{X}
\addtolength{\leftmargin}{\labelsep}
\setlength{\labelwidth}{\tmplength}
}
\item[\textbf{Declaration}\hfill]
\ifpdf
\begin{flushleft}
\fi
\begin{ttfamily}
protected function MakeItemLink(const Item: TBaseItem; const LinkCaption: string; const LinkContext: TLinkContext): string; override;\end{ttfamily}

\ifpdf
\end{flushleft}
\fi

\end{list}
\paragraph*{EscapeURL}\hspace*{\fill}

\label{PasDoc_GenHtml.TGenericHTMLDocGenerator-EscapeURL}
\index{EscapeURL}
\begin{list}{}{
\settowidth{\tmplength}{\textbf{Description}}
\setlength{\itemindent}{0cm}
\setlength{\listparindent}{0cm}
\setlength{\leftmargin}{\evensidemargin}
\addtolength{\leftmargin}{\tmplength}
\settowidth{\labelsep}{X}
\addtolength{\leftmargin}{\labelsep}
\setlength{\labelwidth}{\tmplength}
}
\item[\textbf{Declaration}\hfill]
\ifpdf
\begin{flushleft}
\fi
\begin{ttfamily}
protected function EscapeURL(const AString: string): string; virtual;\end{ttfamily}

\ifpdf
\end{flushleft}
\fi

\end{list}
\paragraph*{FormatSection}\hspace*{\fill}

\label{PasDoc_GenHtml.TGenericHTMLDocGenerator-FormatSection}
\index{FormatSection}
\begin{list}{}{
\settowidth{\tmplength}{\textbf{Description}}
\setlength{\itemindent}{0cm}
\setlength{\listparindent}{0cm}
\setlength{\leftmargin}{\evensidemargin}
\addtolength{\leftmargin}{\tmplength}
\settowidth{\labelsep}{X}
\addtolength{\leftmargin}{\labelsep}
\setlength{\labelwidth}{\tmplength}
}
\item[\textbf{Declaration}\hfill]
\ifpdf
\begin{flushleft}
\fi
\begin{ttfamily}
protected function FormatSection(HL: integer; const Anchor: string; const Caption: string): string; override;\end{ttfamily}

\ifpdf
\end{flushleft}
\fi

\end{list}
\paragraph*{FormatAnchor}\hspace*{\fill}

\label{PasDoc_GenHtml.TGenericHTMLDocGenerator-FormatAnchor}
\index{FormatAnchor}
\begin{list}{}{
\settowidth{\tmplength}{\textbf{Description}}
\setlength{\itemindent}{0cm}
\setlength{\listparindent}{0cm}
\setlength{\leftmargin}{\evensidemargin}
\addtolength{\leftmargin}{\tmplength}
\settowidth{\labelsep}{X}
\addtolength{\leftmargin}{\labelsep}
\setlength{\labelwidth}{\tmplength}
}
\item[\textbf{Declaration}\hfill]
\ifpdf
\begin{flushleft}
\fi
\begin{ttfamily}
protected function FormatAnchor(const Anchor: string): string; override;\end{ttfamily}

\ifpdf
\end{flushleft}
\fi

\end{list}
\paragraph*{FormatBold}\hspace*{\fill}

\label{PasDoc_GenHtml.TGenericHTMLDocGenerator-FormatBold}
\index{FormatBold}
\begin{list}{}{
\settowidth{\tmplength}{\textbf{Description}}
\setlength{\itemindent}{0cm}
\setlength{\listparindent}{0cm}
\setlength{\leftmargin}{\evensidemargin}
\addtolength{\leftmargin}{\tmplength}
\settowidth{\labelsep}{X}
\addtolength{\leftmargin}{\labelsep}
\setlength{\labelwidth}{\tmplength}
}
\item[\textbf{Declaration}\hfill]
\ifpdf
\begin{flushleft}
\fi
\begin{ttfamily}
protected function FormatBold(const Text: string): string; override;\end{ttfamily}

\ifpdf
\end{flushleft}
\fi

\end{list}
\paragraph*{FormatItalic}\hspace*{\fill}

\label{PasDoc_GenHtml.TGenericHTMLDocGenerator-FormatItalic}
\index{FormatItalic}
\begin{list}{}{
\settowidth{\tmplength}{\textbf{Description}}
\setlength{\itemindent}{0cm}
\setlength{\listparindent}{0cm}
\setlength{\leftmargin}{\evensidemargin}
\addtolength{\leftmargin}{\tmplength}
\settowidth{\labelsep}{X}
\addtolength{\leftmargin}{\labelsep}
\setlength{\labelwidth}{\tmplength}
}
\item[\textbf{Declaration}\hfill]
\ifpdf
\begin{flushleft}
\fi
\begin{ttfamily}
protected function FormatItalic(const Text: string): string; override;\end{ttfamily}

\ifpdf
\end{flushleft}
\fi

\end{list}
\paragraph*{FormatWarning}\hspace*{\fill}

\label{PasDoc_GenHtml.TGenericHTMLDocGenerator-FormatWarning}
\index{FormatWarning}
\begin{list}{}{
\settowidth{\tmplength}{\textbf{Description}}
\setlength{\itemindent}{0cm}
\setlength{\listparindent}{0cm}
\setlength{\leftmargin}{\evensidemargin}
\addtolength{\leftmargin}{\tmplength}
\settowidth{\labelsep}{X}
\addtolength{\leftmargin}{\labelsep}
\setlength{\labelwidth}{\tmplength}
}
\item[\textbf{Declaration}\hfill]
\ifpdf
\begin{flushleft}
\fi
\begin{ttfamily}
protected function FormatWarning(const Text: string): string; override;\end{ttfamily}

\ifpdf
\end{flushleft}
\fi

\end{list}
\paragraph*{FormatNote}\hspace*{\fill}

\label{PasDoc_GenHtml.TGenericHTMLDocGenerator-FormatNote}
\index{FormatNote}
\begin{list}{}{
\settowidth{\tmplength}{\textbf{Description}}
\setlength{\itemindent}{0cm}
\setlength{\listparindent}{0cm}
\setlength{\leftmargin}{\evensidemargin}
\addtolength{\leftmargin}{\tmplength}
\settowidth{\labelsep}{X}
\addtolength{\leftmargin}{\labelsep}
\setlength{\labelwidth}{\tmplength}
}
\item[\textbf{Declaration}\hfill]
\ifpdf
\begin{flushleft}
\fi
\begin{ttfamily}
protected function FormatNote(const Text: string): string; override;\end{ttfamily}

\ifpdf
\end{flushleft}
\fi

\end{list}
\paragraph*{FormatPreformatted}\hspace*{\fill}

\label{PasDoc_GenHtml.TGenericHTMLDocGenerator-FormatPreformatted}
\index{FormatPreformatted}
\begin{list}{}{
\settowidth{\tmplength}{\textbf{Description}}
\setlength{\itemindent}{0cm}
\setlength{\listparindent}{0cm}
\setlength{\leftmargin}{\evensidemargin}
\addtolength{\leftmargin}{\tmplength}
\settowidth{\labelsep}{X}
\addtolength{\leftmargin}{\labelsep}
\setlength{\labelwidth}{\tmplength}
}
\item[\textbf{Declaration}\hfill]
\ifpdf
\begin{flushleft}
\fi
\begin{ttfamily}
protected function FormatPreformatted(const Text: string): string; override;\end{ttfamily}

\ifpdf
\end{flushleft}
\fi

\end{list}
\paragraph*{FormatImage}\hspace*{\fill}

\label{PasDoc_GenHtml.TGenericHTMLDocGenerator-FormatImage}
\index{FormatImage}
\begin{list}{}{
\settowidth{\tmplength}{\textbf{Description}}
\setlength{\itemindent}{0cm}
\setlength{\listparindent}{0cm}
\setlength{\leftmargin}{\evensidemargin}
\addtolength{\leftmargin}{\tmplength}
\settowidth{\labelsep}{X}
\addtolength{\leftmargin}{\labelsep}
\setlength{\labelwidth}{\tmplength}
}
\item[\textbf{Declaration}\hfill]
\ifpdf
\begin{flushleft}
\fi
\begin{ttfamily}
protected function FormatImage(FileNames: TStringList): string; override;\end{ttfamily}

\ifpdf
\end{flushleft}
\fi

\end{list}
\paragraph*{FormatList}\hspace*{\fill}

\label{PasDoc_GenHtml.TGenericHTMLDocGenerator-FormatList}
\index{FormatList}
\begin{list}{}{
\settowidth{\tmplength}{\textbf{Description}}
\setlength{\itemindent}{0cm}
\setlength{\listparindent}{0cm}
\setlength{\leftmargin}{\evensidemargin}
\addtolength{\leftmargin}{\tmplength}
\settowidth{\labelsep}{X}
\addtolength{\leftmargin}{\labelsep}
\setlength{\labelwidth}{\tmplength}
}
\item[\textbf{Declaration}\hfill]
\ifpdf
\begin{flushleft}
\fi
\begin{ttfamily}
protected function FormatList(ListData: TListData): string; override;\end{ttfamily}

\ifpdf
\end{flushleft}
\fi

\end{list}
\paragraph*{FormatTable}\hspace*{\fill}

\label{PasDoc_GenHtml.TGenericHTMLDocGenerator-FormatTable}
\index{FormatTable}
\begin{list}{}{
\settowidth{\tmplength}{\textbf{Description}}
\setlength{\itemindent}{0cm}
\setlength{\listparindent}{0cm}
\setlength{\leftmargin}{\evensidemargin}
\addtolength{\leftmargin}{\tmplength}
\settowidth{\labelsep}{X}
\addtolength{\leftmargin}{\labelsep}
\setlength{\labelwidth}{\tmplength}
}
\item[\textbf{Declaration}\hfill]
\ifpdf
\begin{flushleft}
\fi
\begin{ttfamily}
protected function FormatTable(Table: TTableData): string; override;\end{ttfamily}

\ifpdf
\end{flushleft}
\fi

\end{list}
\paragraph*{FormatTableOfContents}\hspace*{\fill}

\label{PasDoc_GenHtml.TGenericHTMLDocGenerator-FormatTableOfContents}
\index{FormatTableOfContents}
\begin{list}{}{
\settowidth{\tmplength}{\textbf{Description}}
\setlength{\itemindent}{0cm}
\setlength{\listparindent}{0cm}
\setlength{\leftmargin}{\evensidemargin}
\addtolength{\leftmargin}{\tmplength}
\settowidth{\labelsep}{X}
\addtolength{\leftmargin}{\labelsep}
\setlength{\labelwidth}{\tmplength}
}
\item[\textbf{Declaration}\hfill]
\ifpdf
\begin{flushleft}
\fi
\begin{ttfamily}
protected function FormatTableOfContents(Sections: TStringPairVector): string; override;\end{ttfamily}

\ifpdf
\end{flushleft}
\fi

\end{list}
\paragraph*{Create}\hspace*{\fill}

\label{PasDoc_GenHtml.TGenericHTMLDocGenerator-Create}
\index{Create}
\begin{list}{}{
\settowidth{\tmplength}{\textbf{Description}}
\setlength{\itemindent}{0cm}
\setlength{\listparindent}{0cm}
\setlength{\leftmargin}{\evensidemargin}
\addtolength{\leftmargin}{\tmplength}
\settowidth{\labelsep}{X}
\addtolength{\leftmargin}{\labelsep}
\setlength{\labelwidth}{\tmplength}
}
\item[\textbf{Declaration}\hfill]
\ifpdf
\begin{flushleft}
\fi
\begin{ttfamily}
public constructor Create(AOwner: TComponent); override;\end{ttfamily}

\ifpdf
\end{flushleft}
\fi

\end{list}
\paragraph*{Destroy}\hspace*{\fill}

\label{PasDoc_GenHtml.TGenericHTMLDocGenerator-Destroy}
\index{Destroy}
\begin{list}{}{
\settowidth{\tmplength}{\textbf{Description}}
\setlength{\itemindent}{0cm}
\setlength{\listparindent}{0cm}
\setlength{\leftmargin}{\evensidemargin}
\addtolength{\leftmargin}{\tmplength}
\settowidth{\labelsep}{X}
\addtolength{\leftmargin}{\labelsep}
\setlength{\labelwidth}{\tmplength}
}
\item[\textbf{Declaration}\hfill]
\ifpdf
\begin{flushleft}
\fi
\begin{ttfamily}
public destructor Destroy; override;\end{ttfamily}

\ifpdf
\end{flushleft}
\fi

\end{list}
\paragraph*{GetFileExtension}\hspace*{\fill}

\label{PasDoc_GenHtml.TGenericHTMLDocGenerator-GetFileExtension}
\index{GetFileExtension}
\begin{list}{}{
\settowidth{\tmplength}{\textbf{Description}}
\setlength{\itemindent}{0cm}
\setlength{\listparindent}{0cm}
\setlength{\leftmargin}{\evensidemargin}
\addtolength{\leftmargin}{\tmplength}
\settowidth{\labelsep}{X}
\addtolength{\leftmargin}{\labelsep}
\setlength{\labelwidth}{\tmplength}
}
\item[\textbf{Declaration}\hfill]
\ifpdf
\begin{flushleft}
\fi
\begin{ttfamily}
public function GetFileExtension: string; override;\end{ttfamily}

\ifpdf
\end{flushleft}
\fi

\par
\item[\textbf{Description}]
Returns HTML file extension ".htm".

\end{list}
\paragraph*{WriteDocumentation}\hspace*{\fill}

\label{PasDoc_GenHtml.TGenericHTMLDocGenerator-WriteDocumentation}
\index{WriteDocumentation}
\begin{list}{}{
\settowidth{\tmplength}{\textbf{Description}}
\setlength{\itemindent}{0cm}
\setlength{\listparindent}{0cm}
\setlength{\leftmargin}{\evensidemargin}
\addtolength{\leftmargin}{\tmplength}
\settowidth{\labelsep}{X}
\addtolength{\leftmargin}{\labelsep}
\setlength{\labelwidth}{\tmplength}
}
\item[\textbf{Declaration}\hfill]
\ifpdf
\begin{flushleft}
\fi
\begin{ttfamily}
public procedure WriteDocumentation; override;\end{ttfamily}

\ifpdf
\end{flushleft}
\fi

\par
\item[\textbf{Description}]
The method that does everything {-} writes documentation for all units and creates overview files.

\end{list}
\ifpdf
\subsection*{\large{\textbf{THTMLDocGenerator Class}}\normalsize\hspace{1ex}\hrulefill}
\else
\subsection*{THTMLDocGenerator Class}
\fi
\label{PasDoc_GenHtml.THTMLDocGenerator}
\index{THTMLDocGenerator}
\subsubsection*{\large{\textbf{Hierarchy}}\normalsize\hspace{1ex}\hfill}
THTMLDocGenerator {$>$} \begin{ttfamily}TGenericHTMLDocGenerator\end{ttfamily}(\ref{PasDoc_GenHtml.TGenericHTMLDocGenerator}) {$>$} \begin{ttfamily}TDocGenerator\end{ttfamily}(\ref{PasDoc_Gen.TDocGenerator}) {$>$} 
TComponent
\subsubsection*{\large{\textbf{Description}}\normalsize\hspace{1ex}\hfill}
Right now this is the same thing as TGenericHTMLDocGenerator. In the future it may be extended to include some things not needed for HtmlHelp generator.\subsubsection*{\large{\textbf{Methods}}\normalsize\hspace{1ex}\hfill}
\paragraph*{MakeBodyBegin}\hspace*{\fill}

\label{PasDoc_GenHtml.THTMLDocGenerator-MakeBodyBegin}
\index{MakeBodyBegin}
\begin{list}{}{
\settowidth{\tmplength}{\textbf{Description}}
\setlength{\itemindent}{0cm}
\setlength{\listparindent}{0cm}
\setlength{\leftmargin}{\evensidemargin}
\addtolength{\leftmargin}{\tmplength}
\settowidth{\labelsep}{X}
\addtolength{\leftmargin}{\labelsep}
\setlength{\labelwidth}{\tmplength}
}
\item[\textbf{Declaration}\hfill]
\ifpdf
\begin{flushleft}
\fi
\begin{ttfamily}
protected function MakeBodyBegin: string; override;\end{ttfamily}

\ifpdf
\end{flushleft}
\fi

\end{list}
\paragraph*{MakeBodyEnd}\hspace*{\fill}

\label{PasDoc_GenHtml.THTMLDocGenerator-MakeBodyEnd}
\index{MakeBodyEnd}
\begin{list}{}{
\settowidth{\tmplength}{\textbf{Description}}
\setlength{\itemindent}{0cm}
\setlength{\listparindent}{0cm}
\setlength{\leftmargin}{\evensidemargin}
\addtolength{\leftmargin}{\tmplength}
\settowidth{\labelsep}{X}
\addtolength{\leftmargin}{\labelsep}
\setlength{\labelwidth}{\tmplength}
}
\item[\textbf{Declaration}\hfill]
\ifpdf
\begin{flushleft}
\fi
\begin{ttfamily}
protected function MakeBodyEnd: string; override;\end{ttfamily}

\ifpdf
\end{flushleft}
\fi

\end{list}
\section{Constants}
\ifpdf
\subsection*{\large{\textbf{DefaultPasdocCss}}\normalsize\hspace{1ex}\hrulefill}
\else
\subsection*{DefaultPasdocCss}
\fi
\label{PasDoc_GenHtml-DefaultPasdocCss}
\index{DefaultPasdocCss}
\begin{list}{}{
\settowidth{\tmplength}{\textbf{Description}}
\setlength{\itemindent}{0cm}
\setlength{\listparindent}{0cm}
\setlength{\leftmargin}{\evensidemargin}
\addtolength{\leftmargin}{\tmplength}
\settowidth{\labelsep}{X}
\addtolength{\leftmargin}{\labelsep}
\setlength{\labelwidth}{\tmplength}
}
\item[\textbf{Declaration}\hfill]
\ifpdf
\begin{flushleft}
\fi
\begin{ttfamily}
DefaultPasdocCss = 

'/*' + LineEnding +
'  Copyright 1998-2018 PasDoc developers.' + LineEnding +
'' + LineEnding +
'  This file is part of "PasDoc".' + LineEnding +
'' + LineEnding +
'  "PasDoc" is free software; you can redistribute it and/or modify' + LineEnding +
'  it under the terms of the GNU General Public License as published by' + LineEnding +
'  the Free Software Foundation; either version 2 of the License, or' + LineEnding +
'  (at your option) any later version.' + LineEnding +
'' + LineEnding +
'  "PasDoc" is distributed in the hope that it will be useful,' + LineEnding +
'  but WITHOUT ANY WARRANTY; without even the implied warranty of' + LineEnding +
'  MERCHANTABILITY or FITNESS FOR A PARTICULAR PURPOSE.  See the' + LineEnding +
'  GNU General Public License for more details.' + LineEnding +
'' + LineEnding +
'  You should have received a copy of the GNU General Public License' + LineEnding +
'  along with "PasDoc"; if not, write to the Free Software' + LineEnding +
'  Foundation, Inc., 51 Franklin Street, Fifth Floor, Boston, MA 02110-1301, USA' + LineEnding +
'' + LineEnding +
'  ----------------------------------------------------------------------------' + LineEnding +
'*/' + LineEnding +
'' + LineEnding +
'body, html {\{}' + LineEnding +
'  margin: 0;' + LineEnding +
'  padding: 0;' + LineEnding +
'{\}}' + LineEnding +
'' + LineEnding +
'body {\{}' + LineEnding +
'  font-family: Verdana,Arial;' + LineEnding +
'  color: black;' + LineEnding +
'  background-color: white;' + LineEnding +
'{\}}' + LineEnding +
'' + LineEnding +
'.container {\{}' + LineEnding +
'  width: 100{\%};' + LineEnding +
'  height: 100{\%};' + LineEnding +
'  border-spacing: 0;' + LineEnding +
'{\}}' + LineEnding +
'' + LineEnding +
'.navigation {\{}' + LineEnding +
'  float: left;' + LineEnding +
'  width: 20em; /* must match .content margin-left */' + LineEnding +
'  height: 100{\%};' + LineEnding +
'  color: white;' + LineEnding +
'  background-color: {\#}787878;' + LineEnding +
'  position: fixed;' + LineEnding +
'  margin: 0;' + LineEnding +
'  box-sizing: border-box; /* without this, you could not have padding here, it would overlap with .content, causing errors on narrow screens */' + LineEnding +
'  padding: 1em;' + LineEnding +
'{\}}' + LineEnding +
'.navigation ul {\{}' + LineEnding +
'  margin: 0em;' + LineEnding +
'  padding: 0em;' + LineEnding +
'{\}}' + LineEnding +
'.navigation li {\{}' + LineEnding +
'  list-style-type: none;' + LineEnding +
'  margin: 0.2em 0em 0em 0em;' + LineEnding +
'  padding: 0.25em;' + LineEnding +
'{\}}' + LineEnding +
'.navigation h2 {\{}' + LineEnding +
'  text-align: center;' + LineEnding +
'  margin: 0em;' + LineEnding +
'  padding: 0.5em;' + LineEnding +
'{\}}' + LineEnding +
'' + LineEnding +
'.content {\{}' + LineEnding +
'  margin-left: 20em; /* must match .navigation width */' + LineEnding +
'  box-sizing: border-box; /* without this, you could not have padding here, it would overlap with .navigation, causing errors on narrow screens */' + LineEnding +
'  padding: 1em;' + LineEnding +
'{\}}' + LineEnding +
'.content h1 {\{}' + LineEnding +
'  margin-top: 0;' + LineEnding +
'{\}}' + LineEnding +
'' + LineEnding +
'.appinfo {\{}' + LineEnding +
'  float: right;' + LineEnding +
'  text-align: right;' + LineEnding +
'  margin-bottom: 1em;' + LineEnding +
'{\}}' + LineEnding +
'' + LineEnding +
'img {\{} border:0px; {\}}' + LineEnding +
'' + LineEnding +
'hr {\{}' + LineEnding +
'  border-bottom: medium none;' + LineEnding +
'  border-top: thin solid {\#}888;' + LineEnding +
'{\}}' + LineEnding +
'' + LineEnding +
'a:link {\{}color:{\#}C91E0C; text-decoration: none; {\}}' + LineEnding +
'a:visited {\{}color:{\#}7E5C31; text-decoration: none; {\}}' + LineEnding +
'a:hover {\{}text-decoration: underline; {\}}' + LineEnding +
'a:active {\{}text-decoration: underline; {\}}' + LineEnding +
'' + LineEnding +
'.navigation a:link {\{} color: white; text-decoration: none; {\}}' + LineEnding +
'.navigation a:visited {\{} color: white; text-decoration: none; {\}}' + LineEnding +
'.navigation a:hover {\{} color: white; font-weight: bold; text-decoration: none; {\}}' + LineEnding +
'.navigation a:active {\{} color: white; text-decoration: none; {\}}' + LineEnding +
'' + LineEnding +
'a.bold:link {\{}color:{\#}C91E0C; text-decoration: none; font-weight:bold; {\}}' + LineEnding +
'a.bold:visited {\{}color:{\#}7E5C31; text-decoration: none; font-weight:bold; {\}}' + LineEnding +
'a.bold:hover {\{}text-decoration: underline; font-weight:bold; {\}}' + LineEnding +
'a.bold:active {\{}text-decoration: underline; font-weight:bold; {\}}' + LineEnding +
'' + LineEnding +
'a.section {\{}color: green; text-decoration: none; font-weight: bold; {\}}' + LineEnding +
'a.section:hover {\{}color: green; text-decoration: underline; font-weight: bold; {\}}' + LineEnding +
'' + LineEnding +
'ul.useslist a:link {\{}color:{\#}C91E0C; text-decoration: none; font-weight:bold; {\}}' + LineEnding +
'ul.useslist a:visited {\{}color:{\#}7E5C31; text-decoration: none; font-weight:bold; {\}}' + LineEnding +
'ul.useslist a:hover {\{}text-decoration: underline; font-weight:bold; {\}}' + LineEnding +
'ul.useslist a:active {\{}text-decoration: underline; font-weight:bold; {\}}' + LineEnding +
'' + LineEnding +
'ul.hierarchy {\{} list-style-type:none; {\}}' + LineEnding +
'ul.hierarchylevel {\{} list-style-type:none; {\}}' + LineEnding +
'' + LineEnding +
'p.unitlink a:link {\{}color:{\#}C91E0C; text-decoration: none; font-weight:bold; {\}}' + LineEnding +
'p.unitlink a:visited {\{}color:{\#}7E5C31; text-decoration: none; font-weight:bold; {\}}' + LineEnding +
'p.unitlink a:hover {\{}text-decoration: underline; font-weight:bold; {\}}' + LineEnding +
'p.unitlink a:active {\{}text-decoration: underline; font-weight:bold; {\}}' + LineEnding +
'' + LineEnding +
'tr.list {\{} background: {\#}FFBF44; {\}}' + LineEnding +
'tr.list2 {\{} background: {\#}FFC982; {\}}' + LineEnding +
'tr.listheader {\{} background: {\#}C91E0C; color: white; {\}}' + LineEnding +
'' + LineEnding +
'table.wide{\_}list {\{} border-spacing:2px; width:100{\%}; {\}}' + LineEnding +
'table.wide{\_}list td {\{} vertical-align:top; padding:4px; {\}}' + LineEnding +
'' + LineEnding +
'table.markerlegend {\{} width:auto; {\}}' + LineEnding +
'table.markerlegend td.legendmarker {\{} text-align:center; {\}}' + LineEnding +
'' + LineEnding +
'.sections {\{} background:white; {\}}' + LineEnding +
'.sections .one{\_}section {\{}' + LineEnding +
'  background:lightgray;' + LineEnding +
'  display: inline-block;' + LineEnding +
'  margin: 0.2em;' + LineEnding +
'  padding: 0.5em 1em;' + LineEnding +
'{\}}' + LineEnding +
'' + LineEnding +
'table.summary td.itemcode {\{} width:100{\%}; {\}}' + LineEnding +
'table.detail td.itemcode {\{} width:100{\%}; {\}}' + LineEnding +
'' + LineEnding +
'td.itemname {\{}white-space:nowrap; {\}}' + LineEnding +
'td.itemunit {\{}white-space:nowrap; {\}}' + LineEnding +
'td.itemdesc {\{} width:100{\%}; {\}}' + LineEnding +
'' + LineEnding +
'div.nodescription {\{} color:red; {\}}' + LineEnding +
'dl.parameters dt {\{}' + LineEnding +
'  color:blue;' + LineEnding +
'{\}}' + LineEnding +
'' + LineEnding +
'code {\{}' + LineEnding +
'  font-family: monospace;' + LineEnding +
'  font-size:1.2em;' + LineEnding +
'{\}}' + LineEnding +
'' + LineEnding +
'/* style for warning and note tag */' + LineEnding +
'dl.tag.warning {\{}' + LineEnding +
'  margin-left:-2px;' + LineEnding +
'  padding-left: 3px;' + LineEnding +
'  border-left:4px solid;' + LineEnding +
'  border-color: {\#}FF0000;' + LineEnding +
'{\}}' + LineEnding +
'dl.tag.note {\{}' + LineEnding +
'  margin-left:-2px;' + LineEnding +
'  padding-left: 3px;' + LineEnding +
'  border-left:4px solid;' + LineEnding +
'  border-color: {\#}D0C000;' + LineEnding +
'{\}}' + LineEnding +
'' + LineEnding +
'/* Various browsers have various default styles for {$<$}h6{$>$},' + LineEnding +
'   sometimes ugly for our purposes, so it''s best to set things' + LineEnding +
'   like font-size and font-weight in out pasdoc.css explicitly. */' + LineEnding +
'h6.description{\_}section {\{}' + LineEnding +
'  /* font-size 100{\%} means that it has the same font size as the' + LineEnding +
'     parent element, i.e. normal description text */' + LineEnding +
'  font-size: 100{\%};' + LineEnding +
'  font-weight: bold;' + LineEnding +
'  /* By default browsers usually have some large margin-bottom and' + LineEnding +
'     margin-top for {$<$}h1-6{$>$} tags. In our case, margin-bottom is' + LineEnding +
'     unnecessary, we want to visually show that description{\_}section' + LineEnding +
'     is closely related to content below. In this situation' + LineEnding +
'     (where the font size is just as a normal text), smaller bottom' + LineEnding +
'     margin seems to look good. */' + LineEnding +
'  margin-top: 1.4em;' + LineEnding +
'  margin-bottom: 0em;' + LineEnding +
'{\}}' + LineEnding +
'' + LineEnding +
'/* Style applied to Pascal code in documentation' + LineEnding +
'   (e.g. produced by @longcode tag) {\}} */' + LineEnding +
'.longcode {\{}' + LineEnding +
'  font-family: monospace;' + LineEnding +
'  font-size: 1.2em;' + LineEnding +
'  background-color: {\#}eee;' + LineEnding +
'  padding: 0.5em;' + LineEnding +
'  border: thin solid {\#}ccc;' + LineEnding +
'{\}}' + LineEnding +
'span.pascal{\_}string {\{} color: {\#}000080; {\}}' + LineEnding +
'span.pascal{\_}keyword {\{} font-weight: bolder; {\}}' + LineEnding +
'span.pascal{\_}comment {\{} color: {\#}000080; font-style: italic; {\}}' + LineEnding +
'span.pascal{\_}compiler{\_}comment {\{} color: {\#}008000; {\}}' + LineEnding +
'span.pascal{\_}numeric {\{} {\}}' + LineEnding +
'span.pascal{\_}hex {\{} {\}}' + LineEnding +
'' + LineEnding +
'p.hint{\_}directive {\{} color: red; {\}}' + LineEnding +
'' + LineEnding +
'input{\#}search{\_}text {\{} {\}}' + LineEnding +
'input{\#}search{\_}submit{\_}button {\{} {\}}' + LineEnding +
'' + LineEnding +
'acronym.mispelling {\{} background-color: {\#}f00; {\}}' + LineEnding +
'' + LineEnding +
'/* Actually this reduces vertical space between *every* paragraph' + LineEnding +
'   inside list with @itemSpacing(compact).' + LineEnding +
'   While we would like to reduce this space only for the' + LineEnding +
'   top of 1st and bottom of last paragraph within each list item.' + LineEnding +
'   But, well, user probably will not do any paragraph breaks' + LineEnding +
'   within a list with @itemSpacing(compact) anyway, so it''s' + LineEnding +
'   acceptable solution. */' + LineEnding +
'ul.compact{\_}spacing p {\{} margin-top: 0em; margin-bottom: 0em; {\}}' + LineEnding +
'ol.compact{\_}spacing p {\{} margin-top: 0em; margin-bottom: 0em; {\}}' + LineEnding +
'dl.compact{\_}spacing p {\{} margin-top: 0em; margin-bottom: 0em; {\}}' + LineEnding +
'' + LineEnding +
'/* Style for table created by @table tags:' + LineEnding +
'   just some thin border.' + LineEnding +
'' + LineEnding +
'   This way we have some borders around the cells' + LineEnding +
'   (so cells are visibly separated), but the border' + LineEnding +
'   "blends with the background" so it doesn''t look too ugly.' + LineEnding +
'   Hopefully it looks satisfactory in most cases and for most' + LineEnding +
'   people.' + LineEnding +
'' + LineEnding +
'   We add padding for cells, otherwise they look too close.' + LineEnding +
'   This is normal thing to do when border-collapse is set to' + LineEnding +
'   collapse (because this eliminates spacing between cells).' + LineEnding +
'*/' + LineEnding +
'table.table{\_}tag {\{} border-collapse: collapse; {\}}' + LineEnding +
'table.table{\_}tag td {\{} border: 1pt solid gray; padding: 0.3em; {\}}' + LineEnding +
'table.table{\_}tag th {\{} border: 1pt solid gray; padding: 0.3em; {\}}' + LineEnding +
'' + LineEnding +
'table.detail {\{}' + LineEnding +
'  border: 1pt solid gray;' + LineEnding +
'  margin-top: 0.3em;' + LineEnding +
'  margin-bottom: 0.3em;' + LineEnding +
'{\}}' + LineEnding +
'' + LineEnding +
'.search-form {\{} white-space: nowrap; {\}}' + LineEnding +
'.search-input input {\{} max-width: 80{\%}; {\}} /* this provides some safe space to always fit even on very narrow screens */' + LineEnding +
'.search-input input, .search-button {\{} display: inline-block; vertical-align: middle; {\}}' + LineEnding +
'.search-input {\{} display: inline-block; {\}}' + LineEnding +
'' + LineEnding +
'/* Do not make extra vertical space at the beginning/end of table cells.' + LineEnding +
'   We need "{$>$}" selector, to not change paragraphs inside lists inside' + LineEnding +
'   table cells. */' + LineEnding +
'table.table{\_}tag td {$>$} p:first-child,' + LineEnding +
'table.table{\_}tag th {$>$} p:first-child,' + LineEnding +
'       td.itemdesc {$>$} p:first-child {\{} margin-top: 0em; {\}}' + LineEnding +
'' + LineEnding +
'table.table{\_}tag td {$>$} p:last-child,' + LineEnding +
'table.table{\_}tag th {$>$} p:last-child,' + LineEnding +
'       td.itemdesc {$>$} p:last-child {\{} margin-bottom: 0em; {\}}' + LineEnding +
''
;\end{ttfamily}

\ifpdf
\end{flushleft}
\fi

\end{list}
\section{Authors}
\par
Johannes Berg {$<$}johannes@sipsolutions.de{$>$}

\par
Ralf Junker (delphi@zeitungsjunge.de)

\par
Alexander Lisnevsky (alisnevsky@yandex.ru)

\par
Erwin Scheuch-Heilig (ScheuchHeilig@t-online.de)

\par
Marco Schmidt (marcoschmidt@geocities.com)

\par
Hendy Irawan (ceefour@gauldong.net)

\par
Wim van der Vegt (wvd{\_}vegt@knoware.nl)

\par
Thomas Mueller (www.dummzeuch.de)

\par
David Berg (HTML Layout) {$<$}david@sipsolutions.de{$>$}

\par
Grzegorz Skoczylas {$<$}gskoczylas@rekord.pl{$>$}

\par
Michalis Kamburelis

\par
Richard B. Winston {$<$}rbwinst@usgs.gov{$>$}

\par
Ascanio Pressato

\par
Arno Garrels {$<$}first name.name@nospamgmx.de{$>$}

\chapter{Unit PasDoc{\_}GenHtmlHelp}
\label{PasDoc_GenHtmlHelp}
\index{PasDoc{\_}GenHtmlHelp}
\section{Description}
Generate HtmlHelp output.
\section{Uses}
\begin{itemize}
\item \begin{ttfamily}PasDoc{\_}GenHtml\end{ttfamily}(\ref{PasDoc_GenHtml})\item \begin{ttfamily}PasDoc{\_}Utils\end{ttfamily}(\ref{PasDoc_Utils})\item \begin{ttfamily}PasDoc{\_}SortSettings\end{ttfamily}(\ref{PasDoc_SortSettings})\end{itemize}
\section{Overview}
\begin{description}
\item[\texttt{\begin{ttfamily}THTMLHelpDocGenerator\end{ttfamily} Class}]
\end{description}
\section{Classes, Interfaces, Objects and Records}
\ifpdf
\subsection*{\large{\textbf{THTMLHelpDocGenerator Class}}\normalsize\hspace{1ex}\hrulefill}
\else
\subsection*{THTMLHelpDocGenerator Class}
\fi
\label{PasDoc_GenHtmlHelp.THTMLHelpDocGenerator}
\index{THTMLHelpDocGenerator}
\subsubsection*{\large{\textbf{Hierarchy}}\normalsize\hspace{1ex}\hfill}
THTMLHelpDocGenerator {$>$} \begin{ttfamily}TGenericHTMLDocGenerator\end{ttfamily}(\ref{PasDoc_GenHtml.TGenericHTMLDocGenerator}) {$>$} \begin{ttfamily}TDocGenerator\end{ttfamily}(\ref{PasDoc_Gen.TDocGenerator}) {$>$} 
TComponent
\subsubsection*{\large{\textbf{Description}}\normalsize\hspace{1ex}\hfill}
no description available, TGenericHTMLDocGenerator description followsgenerates HTML documentation\hfill\vspace*{1ex}

 Extends \begin{ttfamily}TDocGenerator\end{ttfamily}(\ref{PasDoc_Gen.TDocGenerator}) and overwrites many of its methods to generate output in HTML (HyperText Markup Language) format.\subsubsection*{\large{\textbf{Properties}}\normalsize\hspace{1ex}\hfill}
\begin{list}{}{
\settowidth{\tmplength}{\textbf{ContentsFile}}
\setlength{\itemindent}{0cm}
\setlength{\listparindent}{0cm}
\setlength{\leftmargin}{\evensidemargin}
\addtolength{\leftmargin}{\tmplength}
\settowidth{\labelsep}{X}
\addtolength{\leftmargin}{\labelsep}
\setlength{\labelwidth}{\tmplength}
}
\label{PasDoc_GenHtmlHelp.THTMLHelpDocGenerator-ContentsFile}
\index{ContentsFile}
\item[\textbf{ContentsFile}\hfill]
\ifpdf
\begin{flushleft}
\fi
\begin{ttfamily}
published property ContentsFile: string read FContentsFile write FContentsFile;\end{ttfamily}

\ifpdf
\end{flushleft}
\fi


\par Contains Name of a file to read HtmlHelp Contents from. If empty, create default contents file.\end{list}
\subsubsection*{\large{\textbf{Methods}}\normalsize\hspace{1ex}\hfill}
\paragraph*{WriteDocumentation}\hspace*{\fill}

\label{PasDoc_GenHtmlHelp.THTMLHelpDocGenerator-WriteDocumentation}
\index{WriteDocumentation}
\begin{list}{}{
\settowidth{\tmplength}{\textbf{Description}}
\setlength{\itemindent}{0cm}
\setlength{\listparindent}{0cm}
\setlength{\leftmargin}{\evensidemargin}
\addtolength{\leftmargin}{\tmplength}
\settowidth{\labelsep}{X}
\addtolength{\leftmargin}{\labelsep}
\setlength{\labelwidth}{\tmplength}
}
\item[\textbf{Declaration}\hfill]
\ifpdf
\begin{flushleft}
\fi
\begin{ttfamily}
public procedure WriteDocumentation; override;\end{ttfamily}

\ifpdf
\end{flushleft}
\fi

\end{list}
\chapter{Unit PasDoc{\_}GenLatex}
\label{PasDoc_GenLatex}
\index{PasDoc{\_}GenLatex}
\section{Description}
Provides Latex document generator object.\hfill\vspace*{1ex}



Implements an object to generate latex documentation, overriding many of \begin{ttfamily}TDocGenerator\end{ttfamily}(\ref{PasDoc_Gen.TDocGenerator})'s virtual methods.
\section{Uses}
\begin{itemize}
\item \begin{ttfamily}PasDoc{\_}Gen\end{ttfamily}(\ref{PasDoc_Gen})\item \begin{ttfamily}PasDoc{\_}Items\end{ttfamily}(\ref{PasDoc_Items})\item \begin{ttfamily}PasDoc{\_}Languages\end{ttfamily}(\ref{PasDoc_Languages})\item \begin{ttfamily}PasDoc{\_}StringVector\end{ttfamily}(\ref{PasDoc_StringVector})\item \begin{ttfamily}PasDoc{\_}Types\end{ttfamily}(\ref{PasDoc_Types})\item \begin{ttfamily}Classes\end{ttfamily}\end{itemize}
\section{Overview}
\begin{description}
\item[\texttt{\begin{ttfamily}TTexDocGenerator\end{ttfamily} Class}]generates latex documentation
\end{description}
\section{Classes, Interfaces, Objects and Records}
\ifpdf
\subsection*{\large{\textbf{TTexDocGenerator Class}}\normalsize\hspace{1ex}\hrulefill}
\else
\subsection*{TTexDocGenerator Class}
\fi
\label{PasDoc_GenLatex.TTexDocGenerator}
\index{TTexDocGenerator}
\subsubsection*{\large{\textbf{Hierarchy}}\normalsize\hspace{1ex}\hfill}
TTexDocGenerator {$>$} \begin{ttfamily}TDocGenerator\end{ttfamily}(\ref{PasDoc_Gen.TDocGenerator}) {$>$} 
TComponent
\subsubsection*{\large{\textbf{Description}}\normalsize\hspace{1ex}\hfill}
generates latex documentation\hfill\vspace*{1ex}

 Extends \begin{ttfamily}TDocGenerator\end{ttfamily}(\ref{PasDoc_Gen.TDocGenerator}) and overwrites many of its methods to generate output in LaTex format.\subsubsection*{\large{\textbf{Properties}}\normalsize\hspace{1ex}\hfill}
\begin{list}{}{
\settowidth{\tmplength}{\textbf{Latex2rtf}}
\setlength{\itemindent}{0cm}
\setlength{\listparindent}{0cm}
\setlength{\leftmargin}{\evensidemargin}
\addtolength{\leftmargin}{\tmplength}
\settowidth{\labelsep}{X}
\addtolength{\leftmargin}{\labelsep}
\setlength{\labelwidth}{\tmplength}
}
\label{PasDoc_GenLatex.TTexDocGenerator-Latex2rtf}
\index{Latex2rtf}
\item[\textbf{Latex2rtf}\hfill]
\ifpdf
\begin{flushleft}
\fi
\begin{ttfamily}
published property Latex2rtf: boolean read FLatex2rtf write FLatex2rtf default false;\end{ttfamily}

\ifpdf
\end{flushleft}
\fi


\par Indicate if the output must be simplified for latex2rtf\label{PasDoc_GenLatex.TTexDocGenerator-LatexHead}
\index{LatexHead}
\item[\textbf{LatexHead}\hfill]
\ifpdf
\begin{flushleft}
\fi
\begin{ttfamily}
published property LatexHead: TStrings read FLatexHead write SetLatexHead;\end{ttfamily}

\ifpdf
\end{flushleft}
\fi


\par The strings in \begin{ttfamily}LatexHead\end{ttfamily} are inserted directly into the preamble of the LaTeX document. Therefore they must be valid LaTeX code.\end{list}
\subsubsection*{\large{\textbf{Methods}}\normalsize\hspace{1ex}\hfill}
\paragraph*{ConvertString}\hspace*{\fill}

\label{PasDoc_GenLatex.TTexDocGenerator-ConvertString}
\index{ConvertString}
\begin{list}{}{
\settowidth{\tmplength}{\textbf{Description}}
\setlength{\itemindent}{0cm}
\setlength{\listparindent}{0cm}
\setlength{\leftmargin}{\evensidemargin}
\addtolength{\leftmargin}{\tmplength}
\settowidth{\labelsep}{X}
\addtolength{\leftmargin}{\labelsep}
\setlength{\labelwidth}{\tmplength}
}
\item[\textbf{Declaration}\hfill]
\ifpdf
\begin{flushleft}
\fi
\begin{ttfamily}
protected function ConvertString(const s: string): string; override;\end{ttfamily}

\ifpdf
\end{flushleft}
\fi

\end{list}
\paragraph*{ConvertChar}\hspace*{\fill}

\label{PasDoc_GenLatex.TTexDocGenerator-ConvertChar}
\index{ConvertChar}
\begin{list}{}{
\settowidth{\tmplength}{\textbf{Description}}
\setlength{\itemindent}{0cm}
\setlength{\listparindent}{0cm}
\setlength{\leftmargin}{\evensidemargin}
\addtolength{\leftmargin}{\tmplength}
\settowidth{\labelsep}{X}
\addtolength{\leftmargin}{\labelsep}
\setlength{\labelwidth}{\tmplength}
}
\item[\textbf{Declaration}\hfill]
\ifpdf
\begin{flushleft}
\fi
\begin{ttfamily}
protected function ConvertChar(c: char): String; override;\end{ttfamily}

\ifpdf
\end{flushleft}
\fi

\par
\item[\textbf{Description}]
Called by \begin{ttfamily}ConvertString\end{ttfamily}(\ref{PasDoc_GenLatex.TTexDocGenerator-ConvertString}) to convert a character. Will convert special characters to their html escape sequence {-}{$>$} test

\end{list}
\paragraph*{WriteUnit}\hspace*{\fill}

\label{PasDoc_GenLatex.TTexDocGenerator-WriteUnit}
\index{WriteUnit}
\begin{list}{}{
\settowidth{\tmplength}{\textbf{Description}}
\setlength{\itemindent}{0cm}
\setlength{\listparindent}{0cm}
\setlength{\leftmargin}{\evensidemargin}
\addtolength{\leftmargin}{\tmplength}
\settowidth{\labelsep}{X}
\addtolength{\leftmargin}{\labelsep}
\setlength{\labelwidth}{\tmplength}
}
\item[\textbf{Declaration}\hfill]
\ifpdf
\begin{flushleft}
\fi
\begin{ttfamily}
protected procedure WriteUnit(const HL: integer; const U: TPasUnit); override;\end{ttfamily}

\ifpdf
\end{flushleft}
\fi

\end{list}
\paragraph*{LatexString}\hspace*{\fill}

\label{PasDoc_GenLatex.TTexDocGenerator-LatexString}
\index{LatexString}
\begin{list}{}{
\settowidth{\tmplength}{\textbf{Description}}
\setlength{\itemindent}{0cm}
\setlength{\listparindent}{0cm}
\setlength{\leftmargin}{\evensidemargin}
\addtolength{\leftmargin}{\tmplength}
\settowidth{\labelsep}{X}
\addtolength{\leftmargin}{\labelsep}
\setlength{\labelwidth}{\tmplength}
}
\item[\textbf{Declaration}\hfill]
\ifpdf
\begin{flushleft}
\fi
\begin{ttfamily}
protected function LatexString(const S: string): string; override;\end{ttfamily}

\ifpdf
\end{flushleft}
\fi

\end{list}
\paragraph*{CodeString}\hspace*{\fill}

\label{PasDoc_GenLatex.TTexDocGenerator-CodeString}
\index{CodeString}
\begin{list}{}{
\settowidth{\tmplength}{\textbf{Description}}
\setlength{\itemindent}{0cm}
\setlength{\listparindent}{0cm}
\setlength{\leftmargin}{\evensidemargin}
\addtolength{\leftmargin}{\tmplength}
\settowidth{\labelsep}{X}
\addtolength{\leftmargin}{\labelsep}
\setlength{\labelwidth}{\tmplength}
}
\item[\textbf{Declaration}\hfill]
\ifpdf
\begin{flushleft}
\fi
\begin{ttfamily}
protected function CodeString(const s: string): string; override;\end{ttfamily}

\ifpdf
\end{flushleft}
\fi

\par
\item[\textbf{Description}]
Makes a String look like a coded String, i.e. '{\textbackslash}begin{\{}ttfamily{\}}TheString{\textbackslash}end{\{}ttfamily{\}}' in LaTeX. {\}}

\end{list}
\paragraph*{CreateLink}\hspace*{\fill}

\label{PasDoc_GenLatex.TTexDocGenerator-CreateLink}
\index{CreateLink}
\begin{list}{}{
\settowidth{\tmplength}{\textbf{Description}}
\setlength{\itemindent}{0cm}
\setlength{\listparindent}{0cm}
\setlength{\leftmargin}{\evensidemargin}
\addtolength{\leftmargin}{\tmplength}
\settowidth{\labelsep}{X}
\addtolength{\leftmargin}{\labelsep}
\setlength{\labelwidth}{\tmplength}
}
\item[\textbf{Declaration}\hfill]
\ifpdf
\begin{flushleft}
\fi
\begin{ttfamily}
protected function CreateLink(const Item: TBaseItem): string; override;\end{ttfamily}

\ifpdf
\end{flushleft}
\fi

\par
\item[\textbf{Description}]
Returns a link to an anchor within a document. LaTeX simply concatenates the strings with either a "{-}" or "." character between them.

\end{list}
\paragraph*{WriteStartOfCode}\hspace*{\fill}

\label{PasDoc_GenLatex.TTexDocGenerator-WriteStartOfCode}
\index{WriteStartOfCode}
\begin{list}{}{
\settowidth{\tmplength}{\textbf{Description}}
\setlength{\itemindent}{0cm}
\setlength{\listparindent}{0cm}
\setlength{\leftmargin}{\evensidemargin}
\addtolength{\leftmargin}{\tmplength}
\settowidth{\labelsep}{X}
\addtolength{\leftmargin}{\labelsep}
\setlength{\labelwidth}{\tmplength}
}
\item[\textbf{Declaration}\hfill]
\ifpdf
\begin{flushleft}
\fi
\begin{ttfamily}
protected procedure WriteStartOfCode; override;\end{ttfamily}

\ifpdf
\end{flushleft}
\fi

\end{list}
\paragraph*{WriteEndOfCode}\hspace*{\fill}

\label{PasDoc_GenLatex.TTexDocGenerator-WriteEndOfCode}
\index{WriteEndOfCode}
\begin{list}{}{
\settowidth{\tmplength}{\textbf{Description}}
\setlength{\itemindent}{0cm}
\setlength{\listparindent}{0cm}
\setlength{\leftmargin}{\evensidemargin}
\addtolength{\leftmargin}{\tmplength}
\settowidth{\labelsep}{X}
\addtolength{\leftmargin}{\labelsep}
\setlength{\labelwidth}{\tmplength}
}
\item[\textbf{Declaration}\hfill]
\ifpdf
\begin{flushleft}
\fi
\begin{ttfamily}
protected procedure WriteEndOfCode; override;\end{ttfamily}

\ifpdf
\end{flushleft}
\fi

\end{list}
\paragraph*{Paragraph}\hspace*{\fill}

\label{PasDoc_GenLatex.TTexDocGenerator-Paragraph}
\index{Paragraph}
\begin{list}{}{
\settowidth{\tmplength}{\textbf{Description}}
\setlength{\itemindent}{0cm}
\setlength{\listparindent}{0cm}
\setlength{\leftmargin}{\evensidemargin}
\addtolength{\leftmargin}{\tmplength}
\settowidth{\labelsep}{X}
\addtolength{\leftmargin}{\labelsep}
\setlength{\labelwidth}{\tmplength}
}
\item[\textbf{Declaration}\hfill]
\ifpdf
\begin{flushleft}
\fi
\begin{ttfamily}
protected function Paragraph: string; override;\end{ttfamily}

\ifpdf
\end{flushleft}
\fi

\end{list}
\paragraph*{ShortDash}\hspace*{\fill}

\label{PasDoc_GenLatex.TTexDocGenerator-ShortDash}
\index{ShortDash}
\begin{list}{}{
\settowidth{\tmplength}{\textbf{Description}}
\setlength{\itemindent}{0cm}
\setlength{\listparindent}{0cm}
\setlength{\leftmargin}{\evensidemargin}
\addtolength{\leftmargin}{\tmplength}
\settowidth{\labelsep}{X}
\addtolength{\leftmargin}{\labelsep}
\setlength{\labelwidth}{\tmplength}
}
\item[\textbf{Declaration}\hfill]
\ifpdf
\begin{flushleft}
\fi
\begin{ttfamily}
protected function ShortDash: string; override;\end{ttfamily}

\ifpdf
\end{flushleft}
\fi

\end{list}
\paragraph*{LineBreak}\hspace*{\fill}

\label{PasDoc_GenLatex.TTexDocGenerator-LineBreak}
\index{LineBreak}
\begin{list}{}{
\settowidth{\tmplength}{\textbf{Description}}
\setlength{\itemindent}{0cm}
\setlength{\listparindent}{0cm}
\setlength{\leftmargin}{\evensidemargin}
\addtolength{\leftmargin}{\tmplength}
\settowidth{\labelsep}{X}
\addtolength{\leftmargin}{\labelsep}
\setlength{\labelwidth}{\tmplength}
}
\item[\textbf{Declaration}\hfill]
\ifpdf
\begin{flushleft}
\fi
\begin{ttfamily}
protected function LineBreak: string; override;\end{ttfamily}

\ifpdf
\end{flushleft}
\fi

\end{list}
\paragraph*{URLLink}\hspace*{\fill}

\label{PasDoc_GenLatex.TTexDocGenerator-URLLink}
\index{URLLink}
\begin{list}{}{
\settowidth{\tmplength}{\textbf{Description}}
\setlength{\itemindent}{0cm}
\setlength{\listparindent}{0cm}
\setlength{\leftmargin}{\evensidemargin}
\addtolength{\leftmargin}{\tmplength}
\settowidth{\labelsep}{X}
\addtolength{\leftmargin}{\labelsep}
\setlength{\labelwidth}{\tmplength}
}
\item[\textbf{Declaration}\hfill]
\ifpdf
\begin{flushleft}
\fi
\begin{ttfamily}
protected function URLLink(const URL: string): string; override;\end{ttfamily}

\ifpdf
\end{flushleft}
\fi

\end{list}
\paragraph*{URLLink}\hspace*{\fill}

\label{PasDoc_GenLatex.TTexDocGenerator-URLLink}
\index{URLLink}
\begin{list}{}{
\settowidth{\tmplength}{\textbf{Description}}
\setlength{\itemindent}{0cm}
\setlength{\listparindent}{0cm}
\setlength{\leftmargin}{\evensidemargin}
\addtolength{\leftmargin}{\tmplength}
\settowidth{\labelsep}{X}
\addtolength{\leftmargin}{\labelsep}
\setlength{\labelwidth}{\tmplength}
}
\item[\textbf{Declaration}\hfill]
\ifpdf
\begin{flushleft}
\fi
\begin{ttfamily}
protected function URLLink(const URL, LinkDisplay: string): string; override;\end{ttfamily}

\ifpdf
\end{flushleft}
\fi

\end{list}
\paragraph*{WriteExternalCore}\hspace*{\fill}

\label{PasDoc_GenLatex.TTexDocGenerator-WriteExternalCore}
\index{WriteExternalCore}
\begin{list}{}{
\settowidth{\tmplength}{\textbf{Description}}
\setlength{\itemindent}{0cm}
\setlength{\listparindent}{0cm}
\setlength{\leftmargin}{\evensidemargin}
\addtolength{\leftmargin}{\tmplength}
\settowidth{\labelsep}{X}
\addtolength{\leftmargin}{\labelsep}
\setlength{\labelwidth}{\tmplength}
}
\item[\textbf{Declaration}\hfill]
\ifpdf
\begin{flushleft}
\fi
\begin{ttfamily}
protected procedure WriteExternalCore(const ExternalItem: TExternalItem; const Id: TTranslationID); override;\end{ttfamily}

\ifpdf
\end{flushleft}
\fi

\end{list}
\paragraph*{FormatKeyWord}\hspace*{\fill}

\label{PasDoc_GenLatex.TTexDocGenerator-FormatKeyWord}
\index{FormatKeyWord}
\begin{list}{}{
\settowidth{\tmplength}{\textbf{Description}}
\setlength{\itemindent}{0cm}
\setlength{\listparindent}{0cm}
\setlength{\leftmargin}{\evensidemargin}
\addtolength{\leftmargin}{\tmplength}
\settowidth{\labelsep}{X}
\addtolength{\leftmargin}{\labelsep}
\setlength{\labelwidth}{\tmplength}
}
\item[\textbf{Declaration}\hfill]
\ifpdf
\begin{flushleft}
\fi
\begin{ttfamily}
protected function FormatKeyWord(AString: string): string; override;\end{ttfamily}

\ifpdf
\end{flushleft}
\fi

\par
\item[\textbf{Description}]
\begin{ttfamily}FormatKeyWord\end{ttfamily} is called from within \begin{ttfamily}FormatPascalCode\end{ttfamily}(\ref{PasDoc_GenLatex.TTexDocGenerator-FormatPascalCode}) to return AString in a bold font.

\end{list}
\paragraph*{FormatCompilerComment}\hspace*{\fill}

\label{PasDoc_GenLatex.TTexDocGenerator-FormatCompilerComment}
\index{FormatCompilerComment}
\begin{list}{}{
\settowidth{\tmplength}{\textbf{Description}}
\setlength{\itemindent}{0cm}
\setlength{\listparindent}{0cm}
\setlength{\leftmargin}{\evensidemargin}
\addtolength{\leftmargin}{\tmplength}
\settowidth{\labelsep}{X}
\addtolength{\leftmargin}{\labelsep}
\setlength{\labelwidth}{\tmplength}
}
\item[\textbf{Declaration}\hfill]
\ifpdf
\begin{flushleft}
\fi
\begin{ttfamily}
protected function FormatCompilerComment(AString: string): string; override;\end{ttfamily}

\ifpdf
\end{flushleft}
\fi

\par
\item[\textbf{Description}]
\begin{ttfamily}FormatCompilerComment\end{ttfamily} is called from within \begin{ttfamily}FormatPascalCode\end{ttfamily}(\ref{PasDoc_GenLatex.TTexDocGenerator-FormatPascalCode}) to return AString in italics.

\end{list}
\paragraph*{FormatComment}\hspace*{\fill}

\label{PasDoc_GenLatex.TTexDocGenerator-FormatComment}
\index{FormatComment}
\begin{list}{}{
\settowidth{\tmplength}{\textbf{Description}}
\setlength{\itemindent}{0cm}
\setlength{\listparindent}{0cm}
\setlength{\leftmargin}{\evensidemargin}
\addtolength{\leftmargin}{\tmplength}
\settowidth{\labelsep}{X}
\addtolength{\leftmargin}{\labelsep}
\setlength{\labelwidth}{\tmplength}
}
\item[\textbf{Declaration}\hfill]
\ifpdf
\begin{flushleft}
\fi
\begin{ttfamily}
protected function FormatComment(AString: string): string; override;\end{ttfamily}

\ifpdf
\end{flushleft}
\fi

\par
\item[\textbf{Description}]
\begin{ttfamily}FormatComment\end{ttfamily} is called from within \begin{ttfamily}FormatPascalCode\end{ttfamily}(\ref{PasDoc_GenLatex.TTexDocGenerator-FormatPascalCode}) to return AString in italics.

\end{list}
\paragraph*{FormatAnchor}\hspace*{\fill}

\label{PasDoc_GenLatex.TTexDocGenerator-FormatAnchor}
\index{FormatAnchor}
\begin{list}{}{
\settowidth{\tmplength}{\textbf{Description}}
\setlength{\itemindent}{0cm}
\setlength{\listparindent}{0cm}
\setlength{\leftmargin}{\evensidemargin}
\addtolength{\leftmargin}{\tmplength}
\settowidth{\labelsep}{X}
\addtolength{\leftmargin}{\labelsep}
\setlength{\labelwidth}{\tmplength}
}
\item[\textbf{Declaration}\hfill]
\ifpdf
\begin{flushleft}
\fi
\begin{ttfamily}
protected function FormatAnchor(const Anchor: string): string; override;\end{ttfamily}

\ifpdf
\end{flushleft}
\fi

\end{list}
\paragraph*{MakeItemLink}\hspace*{\fill}

\label{PasDoc_GenLatex.TTexDocGenerator-MakeItemLink}
\index{MakeItemLink}
\begin{list}{}{
\settowidth{\tmplength}{\textbf{Description}}
\setlength{\itemindent}{0cm}
\setlength{\listparindent}{0cm}
\setlength{\leftmargin}{\evensidemargin}
\addtolength{\leftmargin}{\tmplength}
\settowidth{\labelsep}{X}
\addtolength{\leftmargin}{\labelsep}
\setlength{\labelwidth}{\tmplength}
}
\item[\textbf{Declaration}\hfill]
\ifpdf
\begin{flushleft}
\fi
\begin{ttfamily}
protected function MakeItemLink(const Item: TBaseItem; const LinkCaption: string; const LinkContext: TLinkContext): string; override;\end{ttfamily}

\ifpdf
\end{flushleft}
\fi

\end{list}
\paragraph*{FormatBold}\hspace*{\fill}

\label{PasDoc_GenLatex.TTexDocGenerator-FormatBold}
\index{FormatBold}
\begin{list}{}{
\settowidth{\tmplength}{\textbf{Description}}
\setlength{\itemindent}{0cm}
\setlength{\listparindent}{0cm}
\setlength{\leftmargin}{\evensidemargin}
\addtolength{\leftmargin}{\tmplength}
\settowidth{\labelsep}{X}
\addtolength{\leftmargin}{\labelsep}
\setlength{\labelwidth}{\tmplength}
}
\item[\textbf{Declaration}\hfill]
\ifpdf
\begin{flushleft}
\fi
\begin{ttfamily}
protected function FormatBold(const Text: string): string; override;\end{ttfamily}

\ifpdf
\end{flushleft}
\fi

\end{list}
\paragraph*{FormatItalic}\hspace*{\fill}

\label{PasDoc_GenLatex.TTexDocGenerator-FormatItalic}
\index{FormatItalic}
\begin{list}{}{
\settowidth{\tmplength}{\textbf{Description}}
\setlength{\itemindent}{0cm}
\setlength{\listparindent}{0cm}
\setlength{\leftmargin}{\evensidemargin}
\addtolength{\leftmargin}{\tmplength}
\settowidth{\labelsep}{X}
\addtolength{\leftmargin}{\labelsep}
\setlength{\labelwidth}{\tmplength}
}
\item[\textbf{Declaration}\hfill]
\ifpdf
\begin{flushleft}
\fi
\begin{ttfamily}
protected function FormatItalic(const Text: string): string; override;\end{ttfamily}

\ifpdf
\end{flushleft}
\fi

\end{list}
\paragraph*{FormatWarning}\hspace*{\fill}

\label{PasDoc_GenLatex.TTexDocGenerator-FormatWarning}
\index{FormatWarning}
\begin{list}{}{
\settowidth{\tmplength}{\textbf{Description}}
\setlength{\itemindent}{0cm}
\setlength{\listparindent}{0cm}
\setlength{\leftmargin}{\evensidemargin}
\addtolength{\leftmargin}{\tmplength}
\settowidth{\labelsep}{X}
\addtolength{\leftmargin}{\labelsep}
\setlength{\labelwidth}{\tmplength}
}
\item[\textbf{Declaration}\hfill]
\ifpdf
\begin{flushleft}
\fi
\begin{ttfamily}
protected function FormatWarning(const Text: string): string; override;\end{ttfamily}

\ifpdf
\end{flushleft}
\fi

\end{list}
\paragraph*{FormatNote}\hspace*{\fill}

\label{PasDoc_GenLatex.TTexDocGenerator-FormatNote}
\index{FormatNote}
\begin{list}{}{
\settowidth{\tmplength}{\textbf{Description}}
\setlength{\itemindent}{0cm}
\setlength{\listparindent}{0cm}
\setlength{\leftmargin}{\evensidemargin}
\addtolength{\leftmargin}{\tmplength}
\settowidth{\labelsep}{X}
\addtolength{\leftmargin}{\labelsep}
\setlength{\labelwidth}{\tmplength}
}
\item[\textbf{Declaration}\hfill]
\ifpdf
\begin{flushleft}
\fi
\begin{ttfamily}
protected function FormatNote(const Text: string): string; override;\end{ttfamily}

\ifpdf
\end{flushleft}
\fi

\end{list}
\paragraph*{FormatPreformatted}\hspace*{\fill}

\label{PasDoc_GenLatex.TTexDocGenerator-FormatPreformatted}
\index{FormatPreformatted}
\begin{list}{}{
\settowidth{\tmplength}{\textbf{Description}}
\setlength{\itemindent}{0cm}
\setlength{\listparindent}{0cm}
\setlength{\leftmargin}{\evensidemargin}
\addtolength{\leftmargin}{\tmplength}
\settowidth{\labelsep}{X}
\addtolength{\leftmargin}{\labelsep}
\setlength{\labelwidth}{\tmplength}
}
\item[\textbf{Declaration}\hfill]
\ifpdf
\begin{flushleft}
\fi
\begin{ttfamily}
protected function FormatPreformatted(const Text: string): string; override;\end{ttfamily}

\ifpdf
\end{flushleft}
\fi

\end{list}
\paragraph*{FormatImage}\hspace*{\fill}

\label{PasDoc_GenLatex.TTexDocGenerator-FormatImage}
\index{FormatImage}
\begin{list}{}{
\settowidth{\tmplength}{\textbf{Description}}
\setlength{\itemindent}{0cm}
\setlength{\listparindent}{0cm}
\setlength{\leftmargin}{\evensidemargin}
\addtolength{\leftmargin}{\tmplength}
\settowidth{\labelsep}{X}
\addtolength{\leftmargin}{\labelsep}
\setlength{\labelwidth}{\tmplength}
}
\item[\textbf{Declaration}\hfill]
\ifpdf
\begin{flushleft}
\fi
\begin{ttfamily}
protected function FormatImage(FileNames: TStringList): string; override;\end{ttfamily}

\ifpdf
\end{flushleft}
\fi

\end{list}
\paragraph*{FormatList}\hspace*{\fill}

\label{PasDoc_GenLatex.TTexDocGenerator-FormatList}
\index{FormatList}
\begin{list}{}{
\settowidth{\tmplength}{\textbf{Description}}
\setlength{\itemindent}{0cm}
\setlength{\listparindent}{0cm}
\setlength{\leftmargin}{\evensidemargin}
\addtolength{\leftmargin}{\tmplength}
\settowidth{\labelsep}{X}
\addtolength{\leftmargin}{\labelsep}
\setlength{\labelwidth}{\tmplength}
}
\item[\textbf{Declaration}\hfill]
\ifpdf
\begin{flushleft}
\fi
\begin{ttfamily}
protected function FormatList(ListData: TListData): string; override;\end{ttfamily}

\ifpdf
\end{flushleft}
\fi

\end{list}
\paragraph*{FormatTable}\hspace*{\fill}

\label{PasDoc_GenLatex.TTexDocGenerator-FormatTable}
\index{FormatTable}
\begin{list}{}{
\settowidth{\tmplength}{\textbf{Description}}
\setlength{\itemindent}{0cm}
\setlength{\listparindent}{0cm}
\setlength{\leftmargin}{\evensidemargin}
\addtolength{\leftmargin}{\tmplength}
\settowidth{\labelsep}{X}
\addtolength{\leftmargin}{\labelsep}
\setlength{\labelwidth}{\tmplength}
}
\item[\textbf{Declaration}\hfill]
\ifpdf
\begin{flushleft}
\fi
\begin{ttfamily}
protected function FormatTable(Table: TTableData): string; override;\end{ttfamily}

\ifpdf
\end{flushleft}
\fi

\end{list}
\paragraph*{FormatPascalCode}\hspace*{\fill}

\label{PasDoc_GenLatex.TTexDocGenerator-FormatPascalCode}
\index{FormatPascalCode}
\begin{list}{}{
\settowidth{\tmplength}{\textbf{Description}}
\setlength{\itemindent}{0cm}
\setlength{\listparindent}{0cm}
\setlength{\leftmargin}{\evensidemargin}
\addtolength{\leftmargin}{\tmplength}
\settowidth{\labelsep}{X}
\addtolength{\leftmargin}{\labelsep}
\setlength{\labelwidth}{\tmplength}
}
\item[\textbf{Declaration}\hfill]
\ifpdf
\begin{flushleft}
\fi
\begin{ttfamily}
public function FormatPascalCode(const Line: string): string; override;\end{ttfamily}

\ifpdf
\end{flushleft}
\fi

\par
\item[\textbf{Description}]
\begin{ttfamily}FormatPascalCode\end{ttfamily} is intended to format Line as if it were Object Pascal code in Delphi or Lazarus. However, unlike Lazarus and Delphi, colored text is not used because printing colored text tends to be much more expensive than printing all black text.

\end{list}
\paragraph*{GetFileExtension}\hspace*{\fill}

\label{PasDoc_GenLatex.TTexDocGenerator-GetFileExtension}
\index{GetFileExtension}
\begin{list}{}{
\settowidth{\tmplength}{\textbf{Description}}
\setlength{\itemindent}{0cm}
\setlength{\listparindent}{0cm}
\setlength{\leftmargin}{\evensidemargin}
\addtolength{\leftmargin}{\tmplength}
\settowidth{\labelsep}{X}
\addtolength{\leftmargin}{\labelsep}
\setlength{\labelwidth}{\tmplength}
}
\item[\textbf{Declaration}\hfill]
\ifpdf
\begin{flushleft}
\fi
\begin{ttfamily}
public function GetFileExtension: string; override;\end{ttfamily}

\ifpdf
\end{flushleft}
\fi

\par
\item[\textbf{Description}]
Returns Latex file extension ".tex".

\end{list}
\paragraph*{WriteDocumentation}\hspace*{\fill}

\label{PasDoc_GenLatex.TTexDocGenerator-WriteDocumentation}
\index{WriteDocumentation}
\begin{list}{}{
\settowidth{\tmplength}{\textbf{Description}}
\setlength{\itemindent}{0cm}
\setlength{\listparindent}{0cm}
\setlength{\leftmargin}{\evensidemargin}
\addtolength{\leftmargin}{\tmplength}
\settowidth{\labelsep}{X}
\addtolength{\leftmargin}{\labelsep}
\setlength{\labelwidth}{\tmplength}
}
\item[\textbf{Declaration}\hfill]
\ifpdf
\begin{flushleft}
\fi
\begin{ttfamily}
public procedure WriteDocumentation; override;\end{ttfamily}

\ifpdf
\end{flushleft}
\fi

\par
\item[\textbf{Description}]
The method that does everything --- writes documentation for all units and creates overview files.

\end{list}
\paragraph*{Create}\hspace*{\fill}

\label{PasDoc_GenLatex.TTexDocGenerator-Create}
\index{Create}
\begin{list}{}{
\settowidth{\tmplength}{\textbf{Description}}
\setlength{\itemindent}{0cm}
\setlength{\listparindent}{0cm}
\setlength{\leftmargin}{\evensidemargin}
\addtolength{\leftmargin}{\tmplength}
\settowidth{\labelsep}{X}
\addtolength{\leftmargin}{\labelsep}
\setlength{\labelwidth}{\tmplength}
}
\item[\textbf{Declaration}\hfill]
\ifpdf
\begin{flushleft}
\fi
\begin{ttfamily}
public constructor Create(AOwner: TComponent); override;\end{ttfamily}

\ifpdf
\end{flushleft}
\fi

\end{list}
\paragraph*{Destroy}\hspace*{\fill}

\label{PasDoc_GenLatex.TTexDocGenerator-Destroy}
\index{Destroy}
\begin{list}{}{
\settowidth{\tmplength}{\textbf{Description}}
\setlength{\itemindent}{0cm}
\setlength{\listparindent}{0cm}
\setlength{\leftmargin}{\evensidemargin}
\addtolength{\leftmargin}{\tmplength}
\settowidth{\labelsep}{X}
\addtolength{\leftmargin}{\labelsep}
\setlength{\labelwidth}{\tmplength}
}
\item[\textbf{Declaration}\hfill]
\ifpdf
\begin{flushleft}
\fi
\begin{ttfamily}
public destructor Destroy; override;\end{ttfamily}

\ifpdf
\end{flushleft}
\fi

\end{list}
\paragraph*{EscapeURL}\hspace*{\fill}

\label{PasDoc_GenLatex.TTexDocGenerator-EscapeURL}
\index{EscapeURL}
\begin{list}{}{
\settowidth{\tmplength}{\textbf{Description}}
\setlength{\itemindent}{0cm}
\setlength{\listparindent}{0cm}
\setlength{\leftmargin}{\evensidemargin}
\addtolength{\leftmargin}{\tmplength}
\settowidth{\labelsep}{X}
\addtolength{\leftmargin}{\labelsep}
\setlength{\labelwidth}{\tmplength}
}
\item[\textbf{Declaration}\hfill]
\ifpdf
\begin{flushleft}
\fi
\begin{ttfamily}
public function EscapeURL(const AString: string): string; virtual;\end{ttfamily}

\ifpdf
\end{flushleft}
\fi

\end{list}
\paragraph*{FormatSection}\hspace*{\fill}

\label{PasDoc_GenLatex.TTexDocGenerator-FormatSection}
\index{FormatSection}
\begin{list}{}{
\settowidth{\tmplength}{\textbf{Description}}
\setlength{\itemindent}{0cm}
\setlength{\listparindent}{0cm}
\setlength{\leftmargin}{\evensidemargin}
\addtolength{\leftmargin}{\tmplength}
\settowidth{\labelsep}{X}
\addtolength{\leftmargin}{\labelsep}
\setlength{\labelwidth}{\tmplength}
}
\item[\textbf{Declaration}\hfill]
\ifpdf
\begin{flushleft}
\fi
\begin{ttfamily}
public function FormatSection(HL: integer; const Anchor: string; const Caption: string): string; override;\end{ttfamily}

\ifpdf
\end{flushleft}
\fi

\end{list}
\chapter{Unit PasDoc{\_}GenSimpleXML}
\label{PasDoc_GenSimpleXML}
\index{PasDoc{\_}GenSimpleXML}
\section{Description}
SimpleXML output generator.
\section{Uses}
\begin{itemize}
\item \begin{ttfamily}PasDoc{\_}Utils\end{ttfamily}(\ref{PasDoc_Utils})\item \begin{ttfamily}PasDoc{\_}Gen\end{ttfamily}(\ref{PasDoc_Gen})\item \begin{ttfamily}PasDoc{\_}Items\end{ttfamily}(\ref{PasDoc_Items})\item \begin{ttfamily}PasDoc{\_}Languages\end{ttfamily}(\ref{PasDoc_Languages})\item \begin{ttfamily}PasDoc{\_}StringVector\end{ttfamily}(\ref{PasDoc_StringVector})\item \begin{ttfamily}PasDoc{\_}Types\end{ttfamily}(\ref{PasDoc_Types})\item \begin{ttfamily}Classes\end{ttfamily}\item \begin{ttfamily}PasDoc{\_}StringPairVector\end{ttfamily}(\ref{PasDoc_StringPairVector})\end{itemize}
\section{Overview}
\begin{description}
\item[\texttt{\begin{ttfamily}TSimpleXMLDocGenerator\end{ttfamily} Class}]
\end{description}
\section{Classes, Interfaces, Objects and Records}
\ifpdf
\subsection*{\large{\textbf{TSimpleXMLDocGenerator Class}}\normalsize\hspace{1ex}\hrulefill}
\else
\subsection*{TSimpleXMLDocGenerator Class}
\fi
\label{PasDoc_GenSimpleXML.TSimpleXMLDocGenerator}
\index{TSimpleXMLDocGenerator}
\subsubsection*{\large{\textbf{Hierarchy}}\normalsize\hspace{1ex}\hfill}
TSimpleXMLDocGenerator {$>$} \begin{ttfamily}TDocGenerator\end{ttfamily}(\ref{PasDoc_Gen.TDocGenerator}) {$>$} 
TComponent
\subsubsection*{\large{\textbf{Description}}\normalsize\hspace{1ex}\hfill}
no description available, TDocGenerator description followsbasic documentation generator object\hfill\vspace*{1ex}

 This abstract object will do the complete process of writing documentation files. It will be given the collection of units that was the result of the parsing process and a configuration object that was created from default values and program parameters. Depending on the output format, one or more files may be created (HTML will create several, Tex only one).\subsubsection*{\large{\textbf{Methods}}\normalsize\hspace{1ex}\hfill}
\paragraph*{CodeString}\hspace*{\fill}

\label{PasDoc_GenSimpleXML.TSimpleXMLDocGenerator-CodeString}
\index{CodeString}
\begin{list}{}{
\settowidth{\tmplength}{\textbf{Description}}
\setlength{\itemindent}{0cm}
\setlength{\listparindent}{0cm}
\setlength{\leftmargin}{\evensidemargin}
\addtolength{\leftmargin}{\tmplength}
\settowidth{\labelsep}{X}
\addtolength{\leftmargin}{\labelsep}
\setlength{\labelwidth}{\tmplength}
}
\item[\textbf{Declaration}\hfill]
\ifpdf
\begin{flushleft}
\fi
\begin{ttfamily}
protected function CodeString(const s: string): string; override;\end{ttfamily}

\ifpdf
\end{flushleft}
\fi

\end{list}
\paragraph*{ConvertString}\hspace*{\fill}

\label{PasDoc_GenSimpleXML.TSimpleXMLDocGenerator-ConvertString}
\index{ConvertString}
\begin{list}{}{
\settowidth{\tmplength}{\textbf{Description}}
\setlength{\itemindent}{0cm}
\setlength{\listparindent}{0cm}
\setlength{\leftmargin}{\evensidemargin}
\addtolength{\leftmargin}{\tmplength}
\settowidth{\labelsep}{X}
\addtolength{\leftmargin}{\labelsep}
\setlength{\labelwidth}{\tmplength}
}
\item[\textbf{Declaration}\hfill]
\ifpdf
\begin{flushleft}
\fi
\begin{ttfamily}
protected function ConvertString(const s: string): string; override;\end{ttfamily}

\ifpdf
\end{flushleft}
\fi

\end{list}
\paragraph*{ConvertChar}\hspace*{\fill}

\label{PasDoc_GenSimpleXML.TSimpleXMLDocGenerator-ConvertChar}
\index{ConvertChar}
\begin{list}{}{
\settowidth{\tmplength}{\textbf{Description}}
\setlength{\itemindent}{0cm}
\setlength{\listparindent}{0cm}
\setlength{\leftmargin}{\evensidemargin}
\addtolength{\leftmargin}{\tmplength}
\settowidth{\labelsep}{X}
\addtolength{\leftmargin}{\labelsep}
\setlength{\labelwidth}{\tmplength}
}
\item[\textbf{Declaration}\hfill]
\ifpdf
\begin{flushleft}
\fi
\begin{ttfamily}
protected function ConvertChar(c: char): string; override;\end{ttfamily}

\ifpdf
\end{flushleft}
\fi

\end{list}
\paragraph*{WriteUnit}\hspace*{\fill}

\label{PasDoc_GenSimpleXML.TSimpleXMLDocGenerator-WriteUnit}
\index{WriteUnit}
\begin{list}{}{
\settowidth{\tmplength}{\textbf{Description}}
\setlength{\itemindent}{0cm}
\setlength{\listparindent}{0cm}
\setlength{\leftmargin}{\evensidemargin}
\addtolength{\leftmargin}{\tmplength}
\settowidth{\labelsep}{X}
\addtolength{\leftmargin}{\labelsep}
\setlength{\labelwidth}{\tmplength}
}
\item[\textbf{Declaration}\hfill]
\ifpdf
\begin{flushleft}
\fi
\begin{ttfamily}
protected procedure WriteUnit(const HL: integer; const U: TPasUnit); override;\end{ttfamily}

\ifpdf
\end{flushleft}
\fi

\end{list}
\paragraph*{WriteExternalCore}\hspace*{\fill}

\label{PasDoc_GenSimpleXML.TSimpleXMLDocGenerator-WriteExternalCore}
\index{WriteExternalCore}
\begin{list}{}{
\settowidth{\tmplength}{\textbf{Description}}
\setlength{\itemindent}{0cm}
\setlength{\listparindent}{0cm}
\setlength{\leftmargin}{\evensidemargin}
\addtolength{\leftmargin}{\tmplength}
\settowidth{\labelsep}{X}
\addtolength{\leftmargin}{\labelsep}
\setlength{\labelwidth}{\tmplength}
}
\item[\textbf{Declaration}\hfill]
\ifpdf
\begin{flushleft}
\fi
\begin{ttfamily}
protected procedure WriteExternalCore(const ExternalItem: TExternalItem; const Id: TTranslationID); override;\end{ttfamily}

\ifpdf
\end{flushleft}
\fi

\end{list}
\paragraph*{FormatSection}\hspace*{\fill}

\label{PasDoc_GenSimpleXML.TSimpleXMLDocGenerator-FormatSection}
\index{FormatSection}
\begin{list}{}{
\settowidth{\tmplength}{\textbf{Description}}
\setlength{\itemindent}{0cm}
\setlength{\listparindent}{0cm}
\setlength{\leftmargin}{\evensidemargin}
\addtolength{\leftmargin}{\tmplength}
\settowidth{\labelsep}{X}
\addtolength{\leftmargin}{\labelsep}
\setlength{\labelwidth}{\tmplength}
}
\item[\textbf{Declaration}\hfill]
\ifpdf
\begin{flushleft}
\fi
\begin{ttfamily}
protected function FormatSection(HL: integer; const Anchor: string; const Caption: string): string; override;\end{ttfamily}

\ifpdf
\end{flushleft}
\fi

\end{list}
\paragraph*{FormatAnchor}\hspace*{\fill}

\label{PasDoc_GenSimpleXML.TSimpleXMLDocGenerator-FormatAnchor}
\index{FormatAnchor}
\begin{list}{}{
\settowidth{\tmplength}{\textbf{Description}}
\setlength{\itemindent}{0cm}
\setlength{\listparindent}{0cm}
\setlength{\leftmargin}{\evensidemargin}
\addtolength{\leftmargin}{\tmplength}
\settowidth{\labelsep}{X}
\addtolength{\leftmargin}{\labelsep}
\setlength{\labelwidth}{\tmplength}
}
\item[\textbf{Declaration}\hfill]
\ifpdf
\begin{flushleft}
\fi
\begin{ttfamily}
protected function FormatAnchor(const Anchor: string): string; override;\end{ttfamily}

\ifpdf
\end{flushleft}
\fi

\end{list}
\paragraph*{FormatTable}\hspace*{\fill}

\label{PasDoc_GenSimpleXML.TSimpleXMLDocGenerator-FormatTable}
\index{FormatTable}
\begin{list}{}{
\settowidth{\tmplength}{\textbf{Description}}
\setlength{\itemindent}{0cm}
\setlength{\listparindent}{0cm}
\setlength{\leftmargin}{\evensidemargin}
\addtolength{\leftmargin}{\tmplength}
\settowidth{\labelsep}{X}
\addtolength{\leftmargin}{\labelsep}
\setlength{\labelwidth}{\tmplength}
}
\item[\textbf{Declaration}\hfill]
\ifpdf
\begin{flushleft}
\fi
\begin{ttfamily}
protected function FormatTable(Table: TTableData): string; override;\end{ttfamily}

\ifpdf
\end{flushleft}
\fi

\end{list}
\paragraph*{FormatList}\hspace*{\fill}

\label{PasDoc_GenSimpleXML.TSimpleXMLDocGenerator-FormatList}
\index{FormatList}
\begin{list}{}{
\settowidth{\tmplength}{\textbf{Description}}
\setlength{\itemindent}{0cm}
\setlength{\listparindent}{0cm}
\setlength{\leftmargin}{\evensidemargin}
\addtolength{\leftmargin}{\tmplength}
\settowidth{\labelsep}{X}
\addtolength{\leftmargin}{\labelsep}
\setlength{\labelwidth}{\tmplength}
}
\item[\textbf{Declaration}\hfill]
\ifpdf
\begin{flushleft}
\fi
\begin{ttfamily}
protected function FormatList(ListData: TListData): string; override;\end{ttfamily}

\ifpdf
\end{flushleft}
\fi

\end{list}
\paragraph*{FormatBold}\hspace*{\fill}

\label{PasDoc_GenSimpleXML.TSimpleXMLDocGenerator-FormatBold}
\index{FormatBold}
\begin{list}{}{
\settowidth{\tmplength}{\textbf{Description}}
\setlength{\itemindent}{0cm}
\setlength{\listparindent}{0cm}
\setlength{\leftmargin}{\evensidemargin}
\addtolength{\leftmargin}{\tmplength}
\settowidth{\labelsep}{X}
\addtolength{\leftmargin}{\labelsep}
\setlength{\labelwidth}{\tmplength}
}
\item[\textbf{Declaration}\hfill]
\ifpdf
\begin{flushleft}
\fi
\begin{ttfamily}
protected function FormatBold(const Text: string): string; override;\end{ttfamily}

\ifpdf
\end{flushleft}
\fi

\end{list}
\paragraph*{FormatItalic}\hspace*{\fill}

\label{PasDoc_GenSimpleXML.TSimpleXMLDocGenerator-FormatItalic}
\index{FormatItalic}
\begin{list}{}{
\settowidth{\tmplength}{\textbf{Description}}
\setlength{\itemindent}{0cm}
\setlength{\listparindent}{0cm}
\setlength{\leftmargin}{\evensidemargin}
\addtolength{\leftmargin}{\tmplength}
\settowidth{\labelsep}{X}
\addtolength{\leftmargin}{\labelsep}
\setlength{\labelwidth}{\tmplength}
}
\item[\textbf{Declaration}\hfill]
\ifpdf
\begin{flushleft}
\fi
\begin{ttfamily}
protected function FormatItalic(const Text: string): string; override;\end{ttfamily}

\ifpdf
\end{flushleft}
\fi

\end{list}
\paragraph*{WriteDocumentation}\hspace*{\fill}

\label{PasDoc_GenSimpleXML.TSimpleXMLDocGenerator-WriteDocumentation}
\index{WriteDocumentation}
\begin{list}{}{
\settowidth{\tmplength}{\textbf{Description}}
\setlength{\itemindent}{0cm}
\setlength{\listparindent}{0cm}
\setlength{\leftmargin}{\evensidemargin}
\addtolength{\leftmargin}{\tmplength}
\settowidth{\labelsep}{X}
\addtolength{\leftmargin}{\labelsep}
\setlength{\labelwidth}{\tmplength}
}
\item[\textbf{Declaration}\hfill]
\ifpdf
\begin{flushleft}
\fi
\begin{ttfamily}
public procedure WriteDocumentation; override;\end{ttfamily}

\ifpdf
\end{flushleft}
\fi

\end{list}
\paragraph*{GetFileExtension}\hspace*{\fill}

\label{PasDoc_GenSimpleXML.TSimpleXMLDocGenerator-GetFileExtension}
\index{GetFileExtension}
\begin{list}{}{
\settowidth{\tmplength}{\textbf{Description}}
\setlength{\itemindent}{0cm}
\setlength{\listparindent}{0cm}
\setlength{\leftmargin}{\evensidemargin}
\addtolength{\leftmargin}{\tmplength}
\settowidth{\labelsep}{X}
\addtolength{\leftmargin}{\labelsep}
\setlength{\labelwidth}{\tmplength}
}
\item[\textbf{Declaration}\hfill]
\ifpdf
\begin{flushleft}
\fi
\begin{ttfamily}
public function GetFileExtension: string; override;\end{ttfamily}

\ifpdf
\end{flushleft}
\fi

\end{list}
\chapter{Unit PasDoc{\_}Hashes}
\label{PasDoc_Hashes}
\index{PasDoc{\_}Hashes}
\section{Description}
This unit implements an associative array. Before writing this unit, I've always missed Perl commands like \begin{ttfamily}{\$}h{\{}abc{\}}='def'\end{ttfamily} in Pascal.



Version 0.9.1 (works fine, don't know a bug, but 1.0? No, error checks are missing!)

\textit{ This library is free software; you can redistribute it and/or modify it under the terms of the GNU Library General Public License as published by the Free Software Foundation; either version 2 of the License, or (at your option) any later version.}

\textit{ This library is distributed in the hope that it will be useful, but WITHOUT ANY WARRANTY; without even the implied warranty of MERCHANTABILITY or FITNESS FOR A PARTICULAR PURPOSE. See the GNU Library General Public License for more details.}

\textit{ You should have received a copy of the GNU Library General Public License along with this library; if not, write to the Free Software Foundation, Inc., 51 Franklin Street, Fifth Floor, Boston, MA 02110{-}1301, USA}

Thanks to: \begin{itemize}
\item Larry Wall for perl! And because I found a way how to implement a hash in perl's source code (hv.c and hv.h). This is not a direct translation from C to Pascal, but the algortithms are more or less the same.
\end{itemize}

Be warned: \begin{itemize}
\item There is NOT a single ERROR CHECK in this unit. So expect anything! Especially there are NO checks on NEW and GETMEM functions --- this might be dangerous on machines with low memory.
\end{itemize}

Programmer's information: \begin{itemize}
\item you need Freepascal (\href{http://www.freepascal.org}{http://www.freepascal.org}) or Delphi (\href{http://www.borland.com}{http://www.borland.com}) to compile this unit
\item I recommend that you use Ansistrings {\{}{\$}H+{\}} to be able to use keys longer than 255 chars
\end{itemize}

How to use this unit:

\begin{verbatim}
Simply put this unit in your uses line. You can use a new class - THash.

Initialize a hash (assuming "var h: THash;"):
h:=THash.Create;

Save a String:
h.SetString('key','value');          //perl: $h{key}='value'

Get the String back:
string_var:=h.GetString('key');      //perl: $string_var=$h{key}
returns '' if 'key' is not set

Test if a key has been set:
if h.KeyExists('key') then...        //perl: if (exists $h{key}) ...
returns a boolean

Delete a key
h.DeleteKey('key');                  //perl: delete $h{key};

Which keys do exist?
stringlist:=h.Keys;                  //perl: @list=keys %h;
returns a TStringList

Which keys do exist beginning with a special string?
stinglist:=h.Keys('abc');
returns all keys beginning with 'abc'  //perl: @list=grep /^abc/, keys %h;

How many keys are there?
number_of_keys:=h.Count;             //perl: $number=scalar keys %hash;

How many keys fit in memory allocated by THash?
c:=h.Capacity; (property)
THash automatically increases h.Capacity if needed.
This property is similar to Delphi's TList.Capacity property.
Note #1: You can't decrease h.Capacity.
Note #2: Capacity must be 2**n -- Create sets Capacity:=8;
         The same: Capacity:=17; , Capacity:=32;

I know there will be 4097 key/values in my hash. I don't want
the hash's capacity to be 8192 (wasting 50% ram). What to do?
h.MaxCapacity:=4096; => Capacity will never be > 4096.
Note: You can store more than MaxCapacity key/values in the
      hash (as many as you want) but Count should be >= Capacity
      for best performance.
MaxCapacity is -1 by default, meaning no limit.

Delete the hash
h.Free;    OR
h.Destroy;

Instead of just strings you can also save objects in my hash -
anything that is a pointer can be saved. Similar to SetString
and GetString  there are SetObject  and GetObject. The latter
returns nil if the key is unknown.
You can use both Set/GetString and Set/GetObject for a single
key string - no problem. But if DeleteKey is called, both the
string and the pointer are lost.
If you want to store a pointer  and a string, it is faster to
call  SetStringObject(key,string,pointer)  than SetString and
SetObject. The same is true getting the data back - GetString
and GetObject are  significantly slower  then a singe call to
GetStringObject(key, var string, var pointer).

Happy programming!\end{verbatim}
\section{Uses}
\begin{itemize}
\item \begin{ttfamily}SysUtils\end{ttfamily}\item \begin{ttfamily}Classes\end{ttfamily}\end{itemize}
\section{Overview}
\begin{description}
\item[\texttt{\begin{ttfamily}THashEntry\end{ttfamily} Record}]
\item[\texttt{\begin{ttfamily}THash\end{ttfamily} Class}]
\item[\texttt{\begin{ttfamily}TObjectHash\end{ttfamily} Class}]
\end{description}
\section{Classes, Interfaces, Objects and Records}
\ifpdf
\subsection*{\large{\textbf{THashEntry Record}}\normalsize\hspace{1ex}\hrulefill}
\else
\subsection*{THashEntry Record}
\fi
\label{PasDoc_Hashes.THashEntry}
\index{THashEntry}
%%%%Description
\subsubsection*{\large{\textbf{Fields}}\normalsize\hspace{1ex}\hfill}
\begin{list}{}{
\settowidth{\tmplength}{\textbf{value}}
\setlength{\itemindent}{0cm}
\setlength{\listparindent}{0cm}
\setlength{\leftmargin}{\evensidemargin}
\addtolength{\leftmargin}{\tmplength}
\settowidth{\labelsep}{X}
\addtolength{\leftmargin}{\labelsep}
\setlength{\labelwidth}{\tmplength}
}
\label{PasDoc_Hashes.THashEntry-next}
\index{next}
\item[\textbf{next}\hfill]
\ifpdf
\begin{flushleft}
\fi
\begin{ttfamily}
public next: PHashEntry;\end{ttfamily}

\ifpdf
\end{flushleft}
\fi


\par  \label{PasDoc_Hashes.THashEntry-hash}
\index{hash}
\item[\textbf{hash}\hfill]
\ifpdf
\begin{flushleft}
\fi
\begin{ttfamily}
public hash: Integer;\end{ttfamily}

\ifpdf
\end{flushleft}
\fi


\par  \label{PasDoc_Hashes.THashEntry-key}
\index{key}
\item[\textbf{key}\hfill]
\ifpdf
\begin{flushleft}
\fi
\begin{ttfamily}
public key: String;\end{ttfamily}

\ifpdf
\end{flushleft}
\fi


\par  \label{PasDoc_Hashes.THashEntry-value}
\index{value}
\item[\textbf{value}\hfill]
\ifpdf
\begin{flushleft}
\fi
\begin{ttfamily}
public value: String;\end{ttfamily}

\ifpdf
\end{flushleft}
\fi


\par  \label{PasDoc_Hashes.THashEntry-data}
\index{data}
\item[\textbf{data}\hfill]
\ifpdf
\begin{flushleft}
\fi
\begin{ttfamily}
public data: Pointer;\end{ttfamily}

\ifpdf
\end{flushleft}
\fi


\par  \end{list}
\ifpdf
\subsection*{\large{\textbf{THash Class}}\normalsize\hspace{1ex}\hrulefill}
\else
\subsection*{THash Class}
\fi
\label{PasDoc_Hashes.THash}
\index{THash}
\subsubsection*{\large{\textbf{Hierarchy}}\normalsize\hspace{1ex}\hfill}
THash {$>$} TObject
%%%%Description
\subsubsection*{\large{\textbf{Properties}}\normalsize\hspace{1ex}\hfill}
\begin{list}{}{
\settowidth{\tmplength}{\textbf{MaxCapacity}}
\setlength{\itemindent}{0cm}
\setlength{\listparindent}{0cm}
\setlength{\leftmargin}{\evensidemargin}
\addtolength{\leftmargin}{\tmplength}
\settowidth{\labelsep}{X}
\addtolength{\leftmargin}{\labelsep}
\setlength{\labelwidth}{\tmplength}
}
\label{PasDoc_Hashes.THash-Count}
\index{Count}
\item[\textbf{Count}\hfill]
\ifpdf
\begin{flushleft}
\fi
\begin{ttfamily}
public property Count: Integer read FeldBelegt;\end{ttfamily}

\ifpdf
\end{flushleft}
\fi


\par  \label{PasDoc_Hashes.THash-Capacity}
\index{Capacity}
\item[\textbf{Capacity}\hfill]
\ifpdf
\begin{flushleft}
\fi
\begin{ttfamily}
public property Capacity: Integer read GetCapacity write SetCapacity;\end{ttfamily}

\ifpdf
\end{flushleft}
\fi


\par  \label{PasDoc_Hashes.THash-MaxCapacity}
\index{MaxCapacity}
\item[\textbf{MaxCapacity}\hfill]
\ifpdf
\begin{flushleft}
\fi
\begin{ttfamily}
public property MaxCapacity: Integer read FMaxCapacity write SetMaxCapacity;\end{ttfamily}

\ifpdf
\end{flushleft}
\fi


\par  \end{list}
\subsubsection*{\large{\textbf{Methods}}\normalsize\hspace{1ex}\hfill}
\paragraph*{Create}\hspace*{\fill}

\label{PasDoc_Hashes.THash-Create}
\index{Create}
\begin{list}{}{
\settowidth{\tmplength}{\textbf{Description}}
\setlength{\itemindent}{0cm}
\setlength{\listparindent}{0cm}
\setlength{\leftmargin}{\evensidemargin}
\addtolength{\leftmargin}{\tmplength}
\settowidth{\labelsep}{X}
\addtolength{\leftmargin}{\labelsep}
\setlength{\labelwidth}{\tmplength}
}
\item[\textbf{Declaration}\hfill]
\ifpdf
\begin{flushleft}
\fi
\begin{ttfamily}
public constructor Create;\end{ttfamily}

\ifpdf
\end{flushleft}
\fi

\end{list}
\paragraph*{Destroy}\hspace*{\fill}

\label{PasDoc_Hashes.THash-Destroy}
\index{Destroy}
\begin{list}{}{
\settowidth{\tmplength}{\textbf{Description}}
\setlength{\itemindent}{0cm}
\setlength{\listparindent}{0cm}
\setlength{\leftmargin}{\evensidemargin}
\addtolength{\leftmargin}{\tmplength}
\settowidth{\labelsep}{X}
\addtolength{\leftmargin}{\labelsep}
\setlength{\labelwidth}{\tmplength}
}
\item[\textbf{Declaration}\hfill]
\ifpdf
\begin{flushleft}
\fi
\begin{ttfamily}
public destructor Destroy; override;\end{ttfamily}

\ifpdf
\end{flushleft}
\fi

\end{list}
\paragraph*{SetObject}\hspace*{\fill}

\label{PasDoc_Hashes.THash-SetObject}
\index{SetObject}
\begin{list}{}{
\settowidth{\tmplength}{\textbf{Description}}
\setlength{\itemindent}{0cm}
\setlength{\listparindent}{0cm}
\setlength{\leftmargin}{\evensidemargin}
\addtolength{\leftmargin}{\tmplength}
\settowidth{\labelsep}{X}
\addtolength{\leftmargin}{\labelsep}
\setlength{\labelwidth}{\tmplength}
}
\item[\textbf{Declaration}\hfill]
\ifpdf
\begin{flushleft}
\fi
\begin{ttfamily}
public procedure SetObject({\_}key: String; data: Pointer);\end{ttfamily}

\ifpdf
\end{flushleft}
\fi

\end{list}
\paragraph*{SetString}\hspace*{\fill}

\label{PasDoc_Hashes.THash-SetString}
\index{SetString}
\begin{list}{}{
\settowidth{\tmplength}{\textbf{Description}}
\setlength{\itemindent}{0cm}
\setlength{\listparindent}{0cm}
\setlength{\leftmargin}{\evensidemargin}
\addtolength{\leftmargin}{\tmplength}
\settowidth{\labelsep}{X}
\addtolength{\leftmargin}{\labelsep}
\setlength{\labelwidth}{\tmplength}
}
\item[\textbf{Declaration}\hfill]
\ifpdf
\begin{flushleft}
\fi
\begin{ttfamily}
public procedure SetString({\_}key: String; data: String);\end{ttfamily}

\ifpdf
\end{flushleft}
\fi

\end{list}
\paragraph*{SetStringObject}\hspace*{\fill}

\label{PasDoc_Hashes.THash-SetStringObject}
\index{SetStringObject}
\begin{list}{}{
\settowidth{\tmplength}{\textbf{Description}}
\setlength{\itemindent}{0cm}
\setlength{\listparindent}{0cm}
\setlength{\leftmargin}{\evensidemargin}
\addtolength{\leftmargin}{\tmplength}
\settowidth{\labelsep}{X}
\addtolength{\leftmargin}{\labelsep}
\setlength{\labelwidth}{\tmplength}
}
\item[\textbf{Declaration}\hfill]
\ifpdf
\begin{flushleft}
\fi
\begin{ttfamily}
public procedure SetStringObject({\_}key: String; s: String; p: Pointer);\end{ttfamily}

\ifpdf
\end{flushleft}
\fi

\end{list}
\paragraph*{GetObject}\hspace*{\fill}

\label{PasDoc_Hashes.THash-GetObject}
\index{GetObject}
\begin{list}{}{
\settowidth{\tmplength}{\textbf{Description}}
\setlength{\itemindent}{0cm}
\setlength{\listparindent}{0cm}
\setlength{\leftmargin}{\evensidemargin}
\addtolength{\leftmargin}{\tmplength}
\settowidth{\labelsep}{X}
\addtolength{\leftmargin}{\labelsep}
\setlength{\labelwidth}{\tmplength}
}
\item[\textbf{Declaration}\hfill]
\ifpdf
\begin{flushleft}
\fi
\begin{ttfamily}
public function GetObject({\_}key: String): Pointer;\end{ttfamily}

\ifpdf
\end{flushleft}
\fi

\end{list}
\paragraph*{GetString}\hspace*{\fill}

\label{PasDoc_Hashes.THash-GetString}
\index{GetString}
\begin{list}{}{
\settowidth{\tmplength}{\textbf{Description}}
\setlength{\itemindent}{0cm}
\setlength{\listparindent}{0cm}
\setlength{\leftmargin}{\evensidemargin}
\addtolength{\leftmargin}{\tmplength}
\settowidth{\labelsep}{X}
\addtolength{\leftmargin}{\labelsep}
\setlength{\labelwidth}{\tmplength}
}
\item[\textbf{Declaration}\hfill]
\ifpdf
\begin{flushleft}
\fi
\begin{ttfamily}
public function GetString({\_}key: String): String;\end{ttfamily}

\ifpdf
\end{flushleft}
\fi

\end{list}
\paragraph*{GetStringObject}\hspace*{\fill}

\label{PasDoc_Hashes.THash-GetStringObject}
\index{GetStringObject}
\begin{list}{}{
\settowidth{\tmplength}{\textbf{Description}}
\setlength{\itemindent}{0cm}
\setlength{\listparindent}{0cm}
\setlength{\leftmargin}{\evensidemargin}
\addtolength{\leftmargin}{\tmplength}
\settowidth{\labelsep}{X}
\addtolength{\leftmargin}{\labelsep}
\setlength{\labelwidth}{\tmplength}
}
\item[\textbf{Declaration}\hfill]
\ifpdf
\begin{flushleft}
\fi
\begin{ttfamily}
public procedure GetStringObject({\_}key: String; var s: String; var p: Pointer);\end{ttfamily}

\ifpdf
\end{flushleft}
\fi

\end{list}
\paragraph*{KeyExists}\hspace*{\fill}

\label{PasDoc_Hashes.THash-KeyExists}
\index{KeyExists}
\begin{list}{}{
\settowidth{\tmplength}{\textbf{Description}}
\setlength{\itemindent}{0cm}
\setlength{\listparindent}{0cm}
\setlength{\leftmargin}{\evensidemargin}
\addtolength{\leftmargin}{\tmplength}
\settowidth{\labelsep}{X}
\addtolength{\leftmargin}{\labelsep}
\setlength{\labelwidth}{\tmplength}
}
\item[\textbf{Declaration}\hfill]
\ifpdf
\begin{flushleft}
\fi
\begin{ttfamily}
public function KeyExists({\_}key: String): Boolean;\end{ttfamily}

\ifpdf
\end{flushleft}
\fi

\end{list}
\paragraph*{DeleteKey}\hspace*{\fill}

\label{PasDoc_Hashes.THash-DeleteKey}
\index{DeleteKey}
\begin{list}{}{
\settowidth{\tmplength}{\textbf{Description}}
\setlength{\itemindent}{0cm}
\setlength{\listparindent}{0cm}
\setlength{\leftmargin}{\evensidemargin}
\addtolength{\leftmargin}{\tmplength}
\settowidth{\labelsep}{X}
\addtolength{\leftmargin}{\labelsep}
\setlength{\labelwidth}{\tmplength}
}
\item[\textbf{Declaration}\hfill]
\ifpdf
\begin{flushleft}
\fi
\begin{ttfamily}
public procedure DeleteKey({\_}key: String);\end{ttfamily}

\ifpdf
\end{flushleft}
\fi

\end{list}
\paragraph*{Keys}\hspace*{\fill}

\label{PasDoc_Hashes.THash-Keys}
\index{Keys}
\begin{list}{}{
\settowidth{\tmplength}{\textbf{Description}}
\setlength{\itemindent}{0cm}
\setlength{\listparindent}{0cm}
\setlength{\leftmargin}{\evensidemargin}
\addtolength{\leftmargin}{\tmplength}
\settowidth{\labelsep}{X}
\addtolength{\leftmargin}{\labelsep}
\setlength{\labelwidth}{\tmplength}
}
\item[\textbf{Declaration}\hfill]
\ifpdf
\begin{flushleft}
\fi
\begin{ttfamily}
public function Keys: TStringList; overload;\end{ttfamily}

\ifpdf
\end{flushleft}
\fi

\end{list}
\paragraph*{Keys}\hspace*{\fill}

\label{PasDoc_Hashes.THash-Keys}
\index{Keys}
\begin{list}{}{
\settowidth{\tmplength}{\textbf{Description}}
\setlength{\itemindent}{0cm}
\setlength{\listparindent}{0cm}
\setlength{\leftmargin}{\evensidemargin}
\addtolength{\leftmargin}{\tmplength}
\settowidth{\labelsep}{X}
\addtolength{\leftmargin}{\labelsep}
\setlength{\labelwidth}{\tmplength}
}
\item[\textbf{Declaration}\hfill]
\ifpdf
\begin{flushleft}
\fi
\begin{ttfamily}
public function Keys(beginning: String): TStringList; overload;\end{ttfamily}

\ifpdf
\end{flushleft}
\fi

\end{list}
\ifpdf
\subsection*{\large{\textbf{TObjectHash Class}}\normalsize\hspace{1ex}\hrulefill}
\else
\subsection*{TObjectHash Class}
\fi
\label{PasDoc_Hashes.TObjectHash}
\index{TObjectHash}
\subsubsection*{\large{\textbf{Hierarchy}}\normalsize\hspace{1ex}\hfill}
TObjectHash {$>$} \begin{ttfamily}THash\end{ttfamily}(\ref{PasDoc_Hashes.THash}) {$>$} 
TObject
%%%%Description
\subsubsection*{\large{\textbf{Properties}}\normalsize\hspace{1ex}\hfill}
\begin{list}{}{
\settowidth{\tmplength}{\textbf{Items}}
\setlength{\itemindent}{0cm}
\setlength{\listparindent}{0cm}
\setlength{\leftmargin}{\evensidemargin}
\addtolength{\leftmargin}{\tmplength}
\settowidth{\labelsep}{X}
\addtolength{\leftmargin}{\labelsep}
\setlength{\labelwidth}{\tmplength}
}
\label{PasDoc_Hashes.TObjectHash-Items}
\index{Items}
\item[\textbf{Items}\hfill]
\ifpdf
\begin{flushleft}
\fi
\begin{ttfamily}
public property Items[{\_}key:string]: Pointer read GetObject write SetObject;\end{ttfamily}

\ifpdf
\end{flushleft}
\fi


\par  \end{list}
\subsubsection*{\large{\textbf{Methods}}\normalsize\hspace{1ex}\hfill}
\paragraph*{Delete}\hspace*{\fill}

\label{PasDoc_Hashes.TObjectHash-Delete}
\index{Delete}
\begin{list}{}{
\settowidth{\tmplength}{\textbf{Description}}
\setlength{\itemindent}{0cm}
\setlength{\listparindent}{0cm}
\setlength{\leftmargin}{\evensidemargin}
\addtolength{\leftmargin}{\tmplength}
\settowidth{\labelsep}{X}
\addtolength{\leftmargin}{\labelsep}
\setlength{\labelwidth}{\tmplength}
}
\item[\textbf{Declaration}\hfill]
\ifpdf
\begin{flushleft}
\fi
\begin{ttfamily}
public procedure Delete({\_}key: String);\end{ttfamily}

\ifpdf
\end{flushleft}
\fi

\end{list}
\section{Types}
\ifpdf
\subsection*{\large{\textbf{PPHashEntry}}\normalsize\hspace{1ex}\hrulefill}
\else
\subsection*{PPHashEntry}
\fi
\label{PasDoc_Hashes-PPHashEntry}
\index{PPHashEntry}
\begin{list}{}{
\settowidth{\tmplength}{\textbf{Description}}
\setlength{\itemindent}{0cm}
\setlength{\listparindent}{0cm}
\setlength{\leftmargin}{\evensidemargin}
\addtolength{\leftmargin}{\tmplength}
\settowidth{\labelsep}{X}
\addtolength{\leftmargin}{\labelsep}
\setlength{\labelwidth}{\tmplength}
}
\item[\textbf{Declaration}\hfill]
\ifpdf
\begin{flushleft}
\fi
\begin{ttfamily}
PPHashEntry={\^{}}PHashEntry;\end{ttfamily}

\ifpdf
\end{flushleft}
\fi

\end{list}
\ifpdf
\subsection*{\large{\textbf{PHashEntry}}\normalsize\hspace{1ex}\hrulefill}
\else
\subsection*{PHashEntry}
\fi
\label{PasDoc_Hashes-PHashEntry}
\index{PHashEntry}
\begin{list}{}{
\settowidth{\tmplength}{\textbf{Description}}
\setlength{\itemindent}{0cm}
\setlength{\listparindent}{0cm}
\setlength{\leftmargin}{\evensidemargin}
\addtolength{\leftmargin}{\tmplength}
\settowidth{\labelsep}{X}
\addtolength{\leftmargin}{\labelsep}
\setlength{\labelwidth}{\tmplength}
}
\item[\textbf{Declaration}\hfill]
\ifpdf
\begin{flushleft}
\fi
\begin{ttfamily}
PHashEntry={\^{}}THashEntry;\end{ttfamily}

\ifpdf
\end{flushleft}
\fi

\end{list}
\ifpdf
\subsection*{\large{\textbf{TFakeArray}}\normalsize\hspace{1ex}\hrulefill}
\else
\subsection*{TFakeArray}
\fi
\label{PasDoc_Hashes-TFakeArray}
\index{TFakeArray}
\begin{list}{}{
\settowidth{\tmplength}{\textbf{Description}}
\setlength{\itemindent}{0cm}
\setlength{\listparindent}{0cm}
\setlength{\leftmargin}{\evensidemargin}
\addtolength{\leftmargin}{\tmplength}
\settowidth{\labelsep}{X}
\addtolength{\leftmargin}{\labelsep}
\setlength{\labelwidth}{\tmplength}
}
\item[\textbf{Declaration}\hfill]
\ifpdf
\begin{flushleft}
\fi
\begin{ttfamily}
TFakeArray=array[0..0] of PHashEntry;\end{ttfamily}

\ifpdf
\end{flushleft}
\fi

\par
\item[\textbf{Description}]
in FPC, I can simply use PPHashEntry as an array of PHashEntry {-} Delphi doesn't allow that. I need this stupid array[0..0] definition! From Delphi4, I could use a dynamic array.

\end{list}
\ifpdf
\subsection*{\large{\textbf{PFakeArray}}\normalsize\hspace{1ex}\hrulefill}
\else
\subsection*{PFakeArray}
\fi
\label{PasDoc_Hashes-PFakeArray}
\index{PFakeArray}
\begin{list}{}{
\settowidth{\tmplength}{\textbf{Description}}
\setlength{\itemindent}{0cm}
\setlength{\listparindent}{0cm}
\setlength{\leftmargin}{\evensidemargin}
\addtolength{\leftmargin}{\tmplength}
\settowidth{\labelsep}{X}
\addtolength{\leftmargin}{\labelsep}
\setlength{\labelwidth}{\tmplength}
}
\item[\textbf{Declaration}\hfill]
\ifpdf
\begin{flushleft}
\fi
\begin{ttfamily}
PFakeArray={\^{}}TFakeArray;\end{ttfamily}

\ifpdf
\end{flushleft}
\fi

\end{list}
\section{Author}
\par
Copyright (C) 2001-2014  Wolf Behrenhoff {$<$}wolf@behrenhoff.de{$>$} and PasDoc developers

\chapter{Unit PasDoc{\_}HierarchyTree}
\label{PasDoc_HierarchyTree}
\index{PasDoc{\_}HierarchyTree}
\section{Description}
 a n{-}ary tree for PasItems --- for use in Class Hierarchy
\section{Uses}
\begin{itemize}
\item \begin{ttfamily}Classes\end{ttfamily}\item \begin{ttfamily}PasDoc{\_}Items\end{ttfamily}(\ref{PasDoc_Items})\end{itemize}
\section{Overview}
\begin{description}
\item[\texttt{\begin{ttfamily}TPasItemNode\end{ttfamily} Class}]
\item[\texttt{\begin{ttfamily}TStringCardinalTree\end{ttfamily} Class}]
\end{description}
\begin{description}
\item[\texttt{NewStringCardinalTree}]
\end{description}
\section{Classes, Interfaces, Objects and Records}
\ifpdf
\subsection*{\large{\textbf{TPasItemNode Class}}\normalsize\hspace{1ex}\hrulefill}
\else
\subsection*{TPasItemNode Class}
\fi
\label{PasDoc_HierarchyTree.TPasItemNode}
\index{TPasItemNode}
\subsubsection*{\large{\textbf{Hierarchy}}\normalsize\hspace{1ex}\hfill}
TPasItemNode {$>$} TObject
%%%%Description
\subsubsection*{\large{\textbf{Properties}}\normalsize\hspace{1ex}\hfill}
\begin{list}{}{
\settowidth{\tmplength}{\textbf{Parent}}
\setlength{\itemindent}{0cm}
\setlength{\listparindent}{0cm}
\setlength{\leftmargin}{\evensidemargin}
\addtolength{\leftmargin}{\tmplength}
\settowidth{\labelsep}{X}
\addtolength{\leftmargin}{\labelsep}
\setlength{\labelwidth}{\tmplength}
}
\label{PasDoc_HierarchyTree.TPasItemNode-Name}
\index{Name}
\item[\textbf{Name}\hfill]
\ifpdf
\begin{flushleft}
\fi
\begin{ttfamily}
public property Name: string read GetName;\end{ttfamily}

\ifpdf
\end{flushleft}
\fi


\par  \label{PasDoc_HierarchyTree.TPasItemNode-Item}
\index{Item}
\item[\textbf{Item}\hfill]
\ifpdf
\begin{flushleft}
\fi
\begin{ttfamily}
public property Item: TPasItem read FItem;\end{ttfamily}

\ifpdf
\end{flushleft}
\fi


\par  \label{PasDoc_HierarchyTree.TPasItemNode-Parent}
\index{Parent}
\item[\textbf{Parent}\hfill]
\ifpdf
\begin{flushleft}
\fi
\begin{ttfamily}
public property Parent: TPasItemNode read FParent;\end{ttfamily}

\ifpdf
\end{flushleft}
\fi


\par  \end{list}
\subsubsection*{\large{\textbf{Fields}}\normalsize\hspace{1ex}\hfill}
\begin{list}{}{
\settowidth{\tmplength}{\textbf{FChildren}}
\setlength{\itemindent}{0cm}
\setlength{\listparindent}{0cm}
\setlength{\leftmargin}{\evensidemargin}
\addtolength{\leftmargin}{\tmplength}
\settowidth{\labelsep}{X}
\addtolength{\leftmargin}{\labelsep}
\setlength{\labelwidth}{\tmplength}
}
\label{PasDoc_HierarchyTree.TPasItemNode-FChildren}
\index{FChildren}
\item[\textbf{FChildren}\hfill]
\ifpdf
\begin{flushleft}
\fi
\begin{ttfamily}
protected FChildren: TList;\end{ttfamily}

\ifpdf
\end{flushleft}
\fi


\par  \label{PasDoc_HierarchyTree.TPasItemNode-FParent}
\index{FParent}
\item[\textbf{FParent}\hfill]
\ifpdf
\begin{flushleft}
\fi
\begin{ttfamily}
protected FParent: TPasItemNode;\end{ttfamily}

\ifpdf
\end{flushleft}
\fi


\par  \label{PasDoc_HierarchyTree.TPasItemNode-FItem}
\index{FItem}
\item[\textbf{FItem}\hfill]
\ifpdf
\begin{flushleft}
\fi
\begin{ttfamily}
protected FItem: TPasItem;\end{ttfamily}

\ifpdf
\end{flushleft}
\fi


\par  \label{PasDoc_HierarchyTree.TPasItemNode-FName}
\index{FName}
\item[\textbf{FName}\hfill]
\ifpdf
\begin{flushleft}
\fi
\begin{ttfamily}
protected FName: string;\end{ttfamily}

\ifpdf
\end{flushleft}
\fi


\par  \end{list}
\subsubsection*{\large{\textbf{Methods}}\normalsize\hspace{1ex}\hfill}
\paragraph*{GetName}\hspace*{\fill}

\label{PasDoc_HierarchyTree.TPasItemNode-GetName}
\index{GetName}
\begin{list}{}{
\settowidth{\tmplength}{\textbf{Description}}
\setlength{\itemindent}{0cm}
\setlength{\listparindent}{0cm}
\setlength{\leftmargin}{\evensidemargin}
\addtolength{\leftmargin}{\tmplength}
\settowidth{\labelsep}{X}
\addtolength{\leftmargin}{\labelsep}
\setlength{\labelwidth}{\tmplength}
}
\item[\textbf{Declaration}\hfill]
\ifpdf
\begin{flushleft}
\fi
\begin{ttfamily}
protected function GetName: string;\end{ttfamily}

\ifpdf
\end{flushleft}
\fi

\end{list}
\paragraph*{AddChild}\hspace*{\fill}

\label{PasDoc_HierarchyTree.TPasItemNode-AddChild}
\index{AddChild}
\begin{list}{}{
\settowidth{\tmplength}{\textbf{Description}}
\setlength{\itemindent}{0cm}
\setlength{\listparindent}{0cm}
\setlength{\leftmargin}{\evensidemargin}
\addtolength{\leftmargin}{\tmplength}
\settowidth{\labelsep}{X}
\addtolength{\leftmargin}{\labelsep}
\setlength{\labelwidth}{\tmplength}
}
\item[\textbf{Declaration}\hfill]
\ifpdf
\begin{flushleft}
\fi
\begin{ttfamily}
protected procedure AddChild(const Child: TPasItemNode); overload;\end{ttfamily}

\ifpdf
\end{flushleft}
\fi

\end{list}
\paragraph*{AddChild}\hspace*{\fill}

\label{PasDoc_HierarchyTree.TPasItemNode-AddChild}
\index{AddChild}
\begin{list}{}{
\settowidth{\tmplength}{\textbf{Description}}
\setlength{\itemindent}{0cm}
\setlength{\listparindent}{0cm}
\setlength{\leftmargin}{\evensidemargin}
\addtolength{\leftmargin}{\tmplength}
\settowidth{\labelsep}{X}
\addtolength{\leftmargin}{\labelsep}
\setlength{\labelwidth}{\tmplength}
}
\item[\textbf{Declaration}\hfill]
\ifpdf
\begin{flushleft}
\fi
\begin{ttfamily}
protected function AddChild(const AName: string): TPasItemNode; overload;\end{ttfamily}

\ifpdf
\end{flushleft}
\fi

\end{list}
\paragraph*{AddChild}\hspace*{\fill}

\label{PasDoc_HierarchyTree.TPasItemNode-AddChild}
\index{AddChild}
\begin{list}{}{
\settowidth{\tmplength}{\textbf{Description}}
\setlength{\itemindent}{0cm}
\setlength{\listparindent}{0cm}
\setlength{\leftmargin}{\evensidemargin}
\addtolength{\leftmargin}{\tmplength}
\settowidth{\labelsep}{X}
\addtolength{\leftmargin}{\labelsep}
\setlength{\labelwidth}{\tmplength}
}
\item[\textbf{Declaration}\hfill]
\ifpdf
\begin{flushleft}
\fi
\begin{ttfamily}
protected function AddChild(const AItem: TPasItem): TPasItemNode; overload;\end{ttfamily}

\ifpdf
\end{flushleft}
\fi

\end{list}
\paragraph*{FindItem}\hspace*{\fill}

\label{PasDoc_HierarchyTree.TPasItemNode-FindItem}
\index{FindItem}
\begin{list}{}{
\settowidth{\tmplength}{\textbf{Description}}
\setlength{\itemindent}{0cm}
\setlength{\listparindent}{0cm}
\setlength{\leftmargin}{\evensidemargin}
\addtolength{\leftmargin}{\tmplength}
\settowidth{\labelsep}{X}
\addtolength{\leftmargin}{\labelsep}
\setlength{\labelwidth}{\tmplength}
}
\item[\textbf{Declaration}\hfill]
\ifpdf
\begin{flushleft}
\fi
\begin{ttfamily}
protected function FindItem(const AName: string): TPasItemNode;\end{ttfamily}

\ifpdf
\end{flushleft}
\fi

\end{list}
\paragraph*{Adopt}\hspace*{\fill}

\label{PasDoc_HierarchyTree.TPasItemNode-Adopt}
\index{Adopt}
\begin{list}{}{
\settowidth{\tmplength}{\textbf{Description}}
\setlength{\itemindent}{0cm}
\setlength{\listparindent}{0cm}
\setlength{\leftmargin}{\evensidemargin}
\addtolength{\leftmargin}{\tmplength}
\settowidth{\labelsep}{X}
\addtolength{\leftmargin}{\labelsep}
\setlength{\labelwidth}{\tmplength}
}
\item[\textbf{Declaration}\hfill]
\ifpdf
\begin{flushleft}
\fi
\begin{ttfamily}
protected procedure Adopt(const AChild: TPasItemNode);\end{ttfamily}

\ifpdf
\end{flushleft}
\fi

\end{list}
\paragraph*{Orphan}\hspace*{\fill}

\label{PasDoc_HierarchyTree.TPasItemNode-Orphan}
\index{Orphan}
\begin{list}{}{
\settowidth{\tmplength}{\textbf{Description}}
\setlength{\itemindent}{0cm}
\setlength{\listparindent}{0cm}
\setlength{\leftmargin}{\evensidemargin}
\addtolength{\leftmargin}{\tmplength}
\settowidth{\labelsep}{X}
\addtolength{\leftmargin}{\labelsep}
\setlength{\labelwidth}{\tmplength}
}
\item[\textbf{Declaration}\hfill]
\ifpdf
\begin{flushleft}
\fi
\begin{ttfamily}
protected function Orphan(const AChild: TPasItemNode): boolean;\end{ttfamily}

\ifpdf
\end{flushleft}
\fi

\end{list}
\paragraph*{Sort}\hspace*{\fill}

\label{PasDoc_HierarchyTree.TPasItemNode-Sort}
\index{Sort}
\begin{list}{}{
\settowidth{\tmplength}{\textbf{Description}}
\setlength{\itemindent}{0cm}
\setlength{\listparindent}{0cm}
\setlength{\leftmargin}{\evensidemargin}
\addtolength{\leftmargin}{\tmplength}
\settowidth{\labelsep}{X}
\addtolength{\leftmargin}{\labelsep}
\setlength{\labelwidth}{\tmplength}
}
\item[\textbf{Declaration}\hfill]
\ifpdf
\begin{flushleft}
\fi
\begin{ttfamily}
protected procedure Sort;\end{ttfamily}

\ifpdf
\end{flushleft}
\fi

\end{list}
\paragraph*{Create}\hspace*{\fill}

\label{PasDoc_HierarchyTree.TPasItemNode-Create}
\index{Create}
\begin{list}{}{
\settowidth{\tmplength}{\textbf{Description}}
\setlength{\itemindent}{0cm}
\setlength{\listparindent}{0cm}
\setlength{\leftmargin}{\evensidemargin}
\addtolength{\leftmargin}{\tmplength}
\settowidth{\labelsep}{X}
\addtolength{\leftmargin}{\labelsep}
\setlength{\labelwidth}{\tmplength}
}
\item[\textbf{Declaration}\hfill]
\ifpdf
\begin{flushleft}
\fi
\begin{ttfamily}
public constructor Create;\end{ttfamily}

\ifpdf
\end{flushleft}
\fi

\end{list}
\paragraph*{Destroy}\hspace*{\fill}

\label{PasDoc_HierarchyTree.TPasItemNode-Destroy}
\index{Destroy}
\begin{list}{}{
\settowidth{\tmplength}{\textbf{Description}}
\setlength{\itemindent}{0cm}
\setlength{\listparindent}{0cm}
\setlength{\leftmargin}{\evensidemargin}
\addtolength{\leftmargin}{\tmplength}
\settowidth{\labelsep}{X}
\addtolength{\leftmargin}{\labelsep}
\setlength{\labelwidth}{\tmplength}
}
\item[\textbf{Declaration}\hfill]
\ifpdf
\begin{flushleft}
\fi
\begin{ttfamily}
public destructor Destroy; override;\end{ttfamily}

\ifpdf
\end{flushleft}
\fi

\end{list}
\paragraph*{Level}\hspace*{\fill}

\label{PasDoc_HierarchyTree.TPasItemNode-Level}
\index{Level}
\begin{list}{}{
\settowidth{\tmplength}{\textbf{Description}}
\setlength{\itemindent}{0cm}
\setlength{\listparindent}{0cm}
\setlength{\leftmargin}{\evensidemargin}
\addtolength{\leftmargin}{\tmplength}
\settowidth{\labelsep}{X}
\addtolength{\leftmargin}{\labelsep}
\setlength{\labelwidth}{\tmplength}
}
\item[\textbf{Declaration}\hfill]
\ifpdf
\begin{flushleft}
\fi
\begin{ttfamily}
public function Level: Integer;\end{ttfamily}

\ifpdf
\end{flushleft}
\fi

\end{list}
\ifpdf
\subsection*{\large{\textbf{TStringCardinalTree Class}}\normalsize\hspace{1ex}\hrulefill}
\else
\subsection*{TStringCardinalTree Class}
\fi
\label{PasDoc_HierarchyTree.TStringCardinalTree}
\index{TStringCardinalTree}
\subsubsection*{\large{\textbf{Hierarchy}}\normalsize\hspace{1ex}\hfill}
TStringCardinalTree {$>$} TObject
%%%%Description
\subsubsection*{\large{\textbf{Properties}}\normalsize\hspace{1ex}\hfill}
\begin{list}{}{
\settowidth{\tmplength}{\textbf{FirstItem}}
\setlength{\itemindent}{0cm}
\setlength{\listparindent}{0cm}
\setlength{\leftmargin}{\evensidemargin}
\addtolength{\leftmargin}{\tmplength}
\settowidth{\labelsep}{X}
\addtolength{\leftmargin}{\labelsep}
\setlength{\labelwidth}{\tmplength}
}
\label{PasDoc_HierarchyTree.TStringCardinalTree-IsEmpty}
\index{IsEmpty}
\item[\textbf{IsEmpty}\hfill]
\ifpdf
\begin{flushleft}
\fi
\begin{ttfamily}
public property IsEmpty: boolean read GetIsEmpty;\end{ttfamily}

\ifpdf
\end{flushleft}
\fi


\par  \label{PasDoc_HierarchyTree.TStringCardinalTree-FirstItem}
\index{FirstItem}
\item[\textbf{FirstItem}\hfill]
\ifpdf
\begin{flushleft}
\fi
\begin{ttfamily}
public property FirstItem: TPasItemNode read GetFirstItem;\end{ttfamily}

\ifpdf
\end{flushleft}
\fi


\par  \end{list}
\subsubsection*{\large{\textbf{Fields}}\normalsize\hspace{1ex}\hfill}
\begin{list}{}{
\settowidth{\tmplength}{\textbf{FRoot}}
\setlength{\itemindent}{0cm}
\setlength{\listparindent}{0cm}
\setlength{\leftmargin}{\evensidemargin}
\addtolength{\leftmargin}{\tmplength}
\settowidth{\labelsep}{X}
\addtolength{\leftmargin}{\labelsep}
\setlength{\labelwidth}{\tmplength}
}
\label{PasDoc_HierarchyTree.TStringCardinalTree-FRoot}
\index{FRoot}
\item[\textbf{FRoot}\hfill]
\ifpdf
\begin{flushleft}
\fi
\begin{ttfamily}
protected FRoot: TPasItemNode;\end{ttfamily}

\ifpdf
\end{flushleft}
\fi


\par  \end{list}
\subsubsection*{\large{\textbf{Methods}}\normalsize\hspace{1ex}\hfill}
\paragraph*{GetIsEmpty}\hspace*{\fill}

\label{PasDoc_HierarchyTree.TStringCardinalTree-GetIsEmpty}
\index{GetIsEmpty}
\begin{list}{}{
\settowidth{\tmplength}{\textbf{Description}}
\setlength{\itemindent}{0cm}
\setlength{\listparindent}{0cm}
\setlength{\leftmargin}{\evensidemargin}
\addtolength{\leftmargin}{\tmplength}
\settowidth{\labelsep}{X}
\addtolength{\leftmargin}{\labelsep}
\setlength{\labelwidth}{\tmplength}
}
\item[\textbf{Declaration}\hfill]
\ifpdf
\begin{flushleft}
\fi
\begin{ttfamily}
protected function GetIsEmpty: boolean;\end{ttfamily}

\ifpdf
\end{flushleft}
\fi

\end{list}
\paragraph*{GetFirstItem}\hspace*{\fill}

\label{PasDoc_HierarchyTree.TStringCardinalTree-GetFirstItem}
\index{GetFirstItem}
\begin{list}{}{
\settowidth{\tmplength}{\textbf{Description}}
\setlength{\itemindent}{0cm}
\setlength{\listparindent}{0cm}
\setlength{\leftmargin}{\evensidemargin}
\addtolength{\leftmargin}{\tmplength}
\settowidth{\labelsep}{X}
\addtolength{\leftmargin}{\labelsep}
\setlength{\labelwidth}{\tmplength}
}
\item[\textbf{Declaration}\hfill]
\ifpdf
\begin{flushleft}
\fi
\begin{ttfamily}
protected function GetFirstItem: TPasItemNode;\end{ttfamily}

\ifpdf
\end{flushleft}
\fi

\end{list}
\paragraph*{NeedRoot}\hspace*{\fill}

\label{PasDoc_HierarchyTree.TStringCardinalTree-NeedRoot}
\index{NeedRoot}
\begin{list}{}{
\settowidth{\tmplength}{\textbf{Description}}
\setlength{\itemindent}{0cm}
\setlength{\listparindent}{0cm}
\setlength{\leftmargin}{\evensidemargin}
\addtolength{\leftmargin}{\tmplength}
\settowidth{\labelsep}{X}
\addtolength{\leftmargin}{\labelsep}
\setlength{\labelwidth}{\tmplength}
}
\item[\textbf{Declaration}\hfill]
\ifpdf
\begin{flushleft}
\fi
\begin{ttfamily}
protected procedure NeedRoot;\end{ttfamily}

\ifpdf
\end{flushleft}
\fi

\end{list}
\paragraph*{ItemOfName}\hspace*{\fill}

\label{PasDoc_HierarchyTree.TStringCardinalTree-ItemOfName}
\index{ItemOfName}
\begin{list}{}{
\settowidth{\tmplength}{\textbf{Description}}
\setlength{\itemindent}{0cm}
\setlength{\listparindent}{0cm}
\setlength{\leftmargin}{\evensidemargin}
\addtolength{\leftmargin}{\tmplength}
\settowidth{\labelsep}{X}
\addtolength{\leftmargin}{\labelsep}
\setlength{\labelwidth}{\tmplength}
}
\item[\textbf{Declaration}\hfill]
\ifpdf
\begin{flushleft}
\fi
\begin{ttfamily}
public function ItemOfName(const AName: string): TPasItemNode;\end{ttfamily}

\ifpdf
\end{flushleft}
\fi

\end{list}
\paragraph*{InsertName}\hspace*{\fill}

\label{PasDoc_HierarchyTree.TStringCardinalTree-InsertName}
\index{InsertName}
\begin{list}{}{
\settowidth{\tmplength}{\textbf{Description}}
\setlength{\itemindent}{0cm}
\setlength{\listparindent}{0cm}
\setlength{\leftmargin}{\evensidemargin}
\addtolength{\leftmargin}{\tmplength}
\settowidth{\labelsep}{X}
\addtolength{\leftmargin}{\labelsep}
\setlength{\labelwidth}{\tmplength}
}
\item[\textbf{Declaration}\hfill]
\ifpdf
\begin{flushleft}
\fi
\begin{ttfamily}
public function InsertName(const AName: string): TPasItemNode; overload;\end{ttfamily}

\ifpdf
\end{flushleft}
\fi

\end{list}
\paragraph*{InsertItem}\hspace*{\fill}

\label{PasDoc_HierarchyTree.TStringCardinalTree-InsertItem}
\index{InsertItem}
\begin{list}{}{
\settowidth{\tmplength}{\textbf{Description}}
\setlength{\itemindent}{0cm}
\setlength{\listparindent}{0cm}
\setlength{\leftmargin}{\evensidemargin}
\addtolength{\leftmargin}{\tmplength}
\settowidth{\labelsep}{X}
\addtolength{\leftmargin}{\labelsep}
\setlength{\labelwidth}{\tmplength}
}
\item[\textbf{Declaration}\hfill]
\ifpdf
\begin{flushleft}
\fi
\begin{ttfamily}
public function InsertItem(const AItem: TPasItem): TPasItemNode; overload;\end{ttfamily}

\ifpdf
\end{flushleft}
\fi

\end{list}
\paragraph*{InsertParented}\hspace*{\fill}

\label{PasDoc_HierarchyTree.TStringCardinalTree-InsertParented}
\index{InsertParented}
\begin{list}{}{
\settowidth{\tmplength}{\textbf{Description}}
\setlength{\itemindent}{0cm}
\setlength{\listparindent}{0cm}
\setlength{\leftmargin}{\evensidemargin}
\addtolength{\leftmargin}{\tmplength}
\settowidth{\labelsep}{X}
\addtolength{\leftmargin}{\labelsep}
\setlength{\labelwidth}{\tmplength}
}
\item[\textbf{Declaration}\hfill]
\ifpdf
\begin{flushleft}
\fi
\begin{ttfamily}
public function InsertParented(const AParent: TPasItemNode; const AItem: TPasItem): TPasItemNode; overload;\end{ttfamily}

\ifpdf
\end{flushleft}
\fi

\end{list}
\paragraph*{InsertParented}\hspace*{\fill}

\label{PasDoc_HierarchyTree.TStringCardinalTree-InsertParented}
\index{InsertParented}
\begin{list}{}{
\settowidth{\tmplength}{\textbf{Description}}
\setlength{\itemindent}{0cm}
\setlength{\listparindent}{0cm}
\setlength{\leftmargin}{\evensidemargin}
\addtolength{\leftmargin}{\tmplength}
\settowidth{\labelsep}{X}
\addtolength{\leftmargin}{\labelsep}
\setlength{\labelwidth}{\tmplength}
}
\item[\textbf{Declaration}\hfill]
\ifpdf
\begin{flushleft}
\fi
\begin{ttfamily}
public function InsertParented(const AParent: TPasItemNode; const AName: string): TPasItemNode; overload;\end{ttfamily}

\ifpdf
\end{flushleft}
\fi

\end{list}
\paragraph*{MoveChildLast}\hspace*{\fill}

\label{PasDoc_HierarchyTree.TStringCardinalTree-MoveChildLast}
\index{MoveChildLast}
\begin{list}{}{
\settowidth{\tmplength}{\textbf{Description}}
\setlength{\itemindent}{0cm}
\setlength{\listparindent}{0cm}
\setlength{\leftmargin}{\evensidemargin}
\addtolength{\leftmargin}{\tmplength}
\settowidth{\labelsep}{X}
\addtolength{\leftmargin}{\labelsep}
\setlength{\labelwidth}{\tmplength}
}
\item[\textbf{Declaration}\hfill]
\ifpdf
\begin{flushleft}
\fi
\begin{ttfamily}
public procedure MoveChildLast(const Child, Parent: TPasItemNode);\end{ttfamily}

\ifpdf
\end{flushleft}
\fi

\end{list}
\paragraph*{Level}\hspace*{\fill}

\label{PasDoc_HierarchyTree.TStringCardinalTree-Level}
\index{Level}
\begin{list}{}{
\settowidth{\tmplength}{\textbf{Description}}
\setlength{\itemindent}{0cm}
\setlength{\listparindent}{0cm}
\setlength{\leftmargin}{\evensidemargin}
\addtolength{\leftmargin}{\tmplength}
\settowidth{\labelsep}{X}
\addtolength{\leftmargin}{\labelsep}
\setlength{\labelwidth}{\tmplength}
}
\item[\textbf{Declaration}\hfill]
\ifpdf
\begin{flushleft}
\fi
\begin{ttfamily}
public function Level(const ANode: TPasItemNode): Integer;\end{ttfamily}

\ifpdf
\end{flushleft}
\fi

\end{list}
\paragraph*{NextItem}\hspace*{\fill}

\label{PasDoc_HierarchyTree.TStringCardinalTree-NextItem}
\index{NextItem}
\begin{list}{}{
\settowidth{\tmplength}{\textbf{Description}}
\setlength{\itemindent}{0cm}
\setlength{\listparindent}{0cm}
\setlength{\leftmargin}{\evensidemargin}
\addtolength{\leftmargin}{\tmplength}
\settowidth{\labelsep}{X}
\addtolength{\leftmargin}{\labelsep}
\setlength{\labelwidth}{\tmplength}
}
\item[\textbf{Declaration}\hfill]
\ifpdf
\begin{flushleft}
\fi
\begin{ttfamily}
public function NextItem(const ANode: TPasItemNode): TPasItemNode;\end{ttfamily}

\ifpdf
\end{flushleft}
\fi

\end{list}
\paragraph*{Sort}\hspace*{\fill}

\label{PasDoc_HierarchyTree.TStringCardinalTree-Sort}
\index{Sort}
\begin{list}{}{
\settowidth{\tmplength}{\textbf{Description}}
\setlength{\itemindent}{0cm}
\setlength{\listparindent}{0cm}
\setlength{\leftmargin}{\evensidemargin}
\addtolength{\leftmargin}{\tmplength}
\settowidth{\labelsep}{X}
\addtolength{\leftmargin}{\labelsep}
\setlength{\labelwidth}{\tmplength}
}
\item[\textbf{Declaration}\hfill]
\ifpdf
\begin{flushleft}
\fi
\begin{ttfamily}
public procedure Sort;\end{ttfamily}

\ifpdf
\end{flushleft}
\fi

\end{list}
\paragraph*{Create}\hspace*{\fill}

\label{PasDoc_HierarchyTree.TStringCardinalTree-Create}
\index{Create}
\begin{list}{}{
\settowidth{\tmplength}{\textbf{Description}}
\setlength{\itemindent}{0cm}
\setlength{\listparindent}{0cm}
\setlength{\leftmargin}{\evensidemargin}
\addtolength{\leftmargin}{\tmplength}
\settowidth{\labelsep}{X}
\addtolength{\leftmargin}{\labelsep}
\setlength{\labelwidth}{\tmplength}
}
\item[\textbf{Declaration}\hfill]
\ifpdf
\begin{flushleft}
\fi
\begin{ttfamily}
public constructor Create;\end{ttfamily}

\ifpdf
\end{flushleft}
\fi

\end{list}
\paragraph*{Destroy}\hspace*{\fill}

\label{PasDoc_HierarchyTree.TStringCardinalTree-Destroy}
\index{Destroy}
\begin{list}{}{
\settowidth{\tmplength}{\textbf{Description}}
\setlength{\itemindent}{0cm}
\setlength{\listparindent}{0cm}
\setlength{\leftmargin}{\evensidemargin}
\addtolength{\leftmargin}{\tmplength}
\settowidth{\labelsep}{X}
\addtolength{\leftmargin}{\labelsep}
\setlength{\labelwidth}{\tmplength}
}
\item[\textbf{Declaration}\hfill]
\ifpdf
\begin{flushleft}
\fi
\begin{ttfamily}
public destructor Destroy; override;\end{ttfamily}

\ifpdf
\end{flushleft}
\fi

\end{list}
\section{Functions and Procedures}
\ifpdf
\subsection*{\large{\textbf{NewStringCardinalTree}}\normalsize\hspace{1ex}\hrulefill}
\else
\subsection*{NewStringCardinalTree}
\fi
\label{PasDoc_HierarchyTree-NewStringCardinalTree}
\index{NewStringCardinalTree}
\begin{list}{}{
\settowidth{\tmplength}{\textbf{Description}}
\setlength{\itemindent}{0cm}
\setlength{\listparindent}{0cm}
\setlength{\leftmargin}{\evensidemargin}
\addtolength{\leftmargin}{\tmplength}
\settowidth{\labelsep}{X}
\addtolength{\leftmargin}{\labelsep}
\setlength{\labelwidth}{\tmplength}
}
\item[\textbf{Declaration}\hfill]
\ifpdf
\begin{flushleft}
\fi
\begin{ttfamily}
function NewStringCardinalTree: TStringCardinalTree;\end{ttfamily}

\ifpdf
\end{flushleft}
\fi

\end{list}
\section{Author}
\par
Johannes Berg {$<$}johannes@sipsolutions.de{$>$}

\chapter{Unit PasDoc{\_}Items}
\label{PasDoc_Items}
\index{PasDoc{\_}Items}
\section{Description}
defines all items that can appear within a Pascal unit's interface\hfill\vspace*{1ex}

        

For each item (type, variable, class etc.) that may appear in a Pascal source code file and can thus be taken into the documentation, this unit provides an object type which will store name, unit, description and more on this item.
\section{Uses}
\begin{itemize}
\item \begin{ttfamily}SysUtils\end{ttfamily}\item \begin{ttfamily}PasDoc{\_}Types\end{ttfamily}(\ref{PasDoc_Types})\item \begin{ttfamily}PasDoc{\_}StringVector\end{ttfamily}(\ref{PasDoc_StringVector})\item \begin{ttfamily}PasDoc{\_}ObjectVector\end{ttfamily}(\ref{PasDoc_ObjectVector})\item \begin{ttfamily}PasDoc{\_}Hashes\end{ttfamily}(\ref{PasDoc_Hashes})\item \begin{ttfamily}Classes\end{ttfamily}\item \begin{ttfamily}PasDoc{\_}TagManager\end{ttfamily}(\ref{PasDoc_TagManager})\item \begin{ttfamily}PasDoc{\_}Serialize\end{ttfamily}(\ref{PasDoc_Serialize})\item \begin{ttfamily}PasDoc{\_}SortSettings\end{ttfamily}(\ref{PasDoc_SortSettings})\item \begin{ttfamily}PasDoc{\_}StringPairVector\end{ttfamily}(\ref{PasDoc_StringPairVector})\item \begin{ttfamily}PasDoc{\_}Tokenizer\end{ttfamily}(\ref{PasDoc_Tokenizer})\end{itemize}
\section{Overview}
\begin{description}
\item[\texttt{\begin{ttfamily}TRawDescriptionInfo\end{ttfamily} Record}]Raw description, in other words: the contents of comment before given item.
\item[\texttt{\begin{ttfamily}TBaseItem\end{ttfamily} Class}]This is a basic item class, that is linkable, and has some \begin{ttfamily}RawDescription\end{ttfamily}(\ref{PasDoc_Items.TBaseItem-RawDescription}).
\item[\texttt{\begin{ttfamily}TPasItem\end{ttfamily} Class}]This is a \begin{ttfamily}TBaseItem\end{ttfamily}(\ref{PasDoc_Items.TBaseItem}) descendant that is always declared inside some Pascal source file.
\item[\texttt{\begin{ttfamily}TPasConstant\end{ttfamily} Class}]Pascal constant.
\item[\texttt{\begin{ttfamily}TPasFieldVariable\end{ttfamily} Class}]Pascal global variable or field or nested constant of CIO.
\item[\texttt{\begin{ttfamily}TPasType\end{ttfamily} Class}]Pascal type (but not a procedural type --- these are expressed as \begin{ttfamily}TPasMethod\end{ttfamily}(\ref{PasDoc_Items.TPasMethod}).)
\item[\texttt{\begin{ttfamily}TPasEnum\end{ttfamily} Class}]Enumerated type.
\item[\texttt{\begin{ttfamily}TPasMethod\end{ttfamily} Class}]This represents: \begin{enumerate}
\setcounter{enumi}{0} \setcounter{enumii}{0} \setcounter{enumiii}{0} \setcounter{enumiv}{0} 
\item global function/procedure,
\setcounter{enumi}{1} \setcounter{enumii}{1} \setcounter{enumiii}{1} \setcounter{enumiv}{1} 
\item method (function/procedure of a class/interface/object),
\setcounter{enumi}{2} \setcounter{enumii}{2} \setcounter{enumiii}{2} \setcounter{enumiv}{2} 
\item pointer type to one of the above (in this case Name is the type name).
\end{enumerate}
\item[\texttt{\begin{ttfamily}TPasProperty\end{ttfamily} Class}]
\item[\texttt{\begin{ttfamily}TPasCio\end{ttfamily} Class}]Extends \begin{ttfamily}TPasItem\end{ttfamily}(\ref{PasDoc_Items.TPasItem}) to store all items in a class / an object, e.g. fields.
\item[\texttt{\begin{ttfamily}EAnchorAlreadyExists\end{ttfamily} Class}]
\item[\texttt{\begin{ttfamily}TExternalItem\end{ttfamily} Class}]\begin{ttfamily}TExternalItem\end{ttfamily} extends \begin{ttfamily}TBaseItem\end{ttfamily}(\ref{PasDoc_Items.TBaseItem}) to store extra information about a project.
\item[\texttt{\begin{ttfamily}TExternalItemList\end{ttfamily} Class}]\begin{ttfamily}TExternalItemList\end{ttfamily} extends \begin{ttfamily}TObjectVector\end{ttfamily}(\ref{PasDoc_ObjectVector.TObjectVector}) to store non{-}nil instances of \begin{ttfamily}TExternalItem\end{ttfamily}(\ref{PasDoc_Items.TExternalItem})
\item[\texttt{\begin{ttfamily}TAnchorItem\end{ttfamily} Class}]
\item[\texttt{\begin{ttfamily}TPasUnit\end{ttfamily} Class}]extends \begin{ttfamily}TPasItem\end{ttfamily}(\ref{PasDoc_Items.TPasItem}) to store anything about a unit, its constants, types etc.; also provides methods for parsing a complete unit.
\item[\texttt{\begin{ttfamily}TBaseItems\end{ttfamily} Class}]Container class to store a list of \begin{ttfamily}TBaseItem\end{ttfamily}(\ref{PasDoc_Items.TBaseItem})s.
\item[\texttt{\begin{ttfamily}TPasItems\end{ttfamily} Class}]Container class to store a list of \begin{ttfamily}TPasItem\end{ttfamily}(\ref{PasDoc_Items.TPasItem})s.
\item[\texttt{\begin{ttfamily}TPasMethods\end{ttfamily} Class}]Collection of methods.
\item[\texttt{\begin{ttfamily}TPasProperties\end{ttfamily} Class}]Collection of properties.
\item[\texttt{\begin{ttfamily}TPasNestedCios\end{ttfamily} Class}]Collection of classes / records / interfaces.
\item[\texttt{\begin{ttfamily}TPasTypes\end{ttfamily} Class}]Collection of types.
\item[\texttt{\begin{ttfamily}TPasUnits\end{ttfamily} Class}]Collection of units.
\end{description}
\begin{description}
\item[\texttt{MethodTypeToString}]Returns lowercased keyword associated with given method type.
\item[\texttt{VisibilitiesToStr}]Returns VisibilityStr for each value in Visibilities, delimited by commas.
\item[\texttt{VisToStr}]
\end{description}
\section{Classes, Interfaces, Objects and Records}
\ifpdf
\subsection*{\large{\textbf{TRawDescriptionInfo Record}}\normalsize\hspace{1ex}\hrulefill}
\else
\subsection*{TRawDescriptionInfo Record}
\fi
\label{PasDoc_Items.TRawDescriptionInfo}
\index{TRawDescriptionInfo}
\subsubsection*{\large{\textbf{Description}}\normalsize\hspace{1ex}\hfill}
Raw description, in other words: the contents of comment before given item. Besides the content, this also specifies filename, begin and end positions of given comment.\subsubsection*{\large{\textbf{Fields}}\normalsize\hspace{1ex}\hfill}
\begin{list}{}{
\settowidth{\tmplength}{\textbf{BeginPosition}}
\setlength{\itemindent}{0cm}
\setlength{\listparindent}{0cm}
\setlength{\leftmargin}{\evensidemargin}
\addtolength{\leftmargin}{\tmplength}
\settowidth{\labelsep}{X}
\addtolength{\leftmargin}{\labelsep}
\setlength{\labelwidth}{\tmplength}
}
\label{PasDoc_Items.TRawDescriptionInfo-Content}
\index{Content}
\item[\textbf{Content}\hfill]
\ifpdf
\begin{flushleft}
\fi
\begin{ttfamily}
public Content: string;\end{ttfamily}

\ifpdf
\end{flushleft}
\fi


\par This is the actual content the comment.\label{PasDoc_Items.TRawDescriptionInfo-StreamName}
\index{StreamName}
\item[\textbf{StreamName}\hfill]
\ifpdf
\begin{flushleft}
\fi
\begin{ttfamily}
public StreamName: string;\end{ttfamily}

\ifpdf
\end{flushleft}
\fi


\par \begin{ttfamily}StreamName\end{ttfamily} is the name of the TStream from which this comment was read. Will be '' if no comment was found. It will be ' ' if the comment was somehow read from more than one stream.\label{PasDoc_Items.TRawDescriptionInfo-BeginPosition}
\index{BeginPosition}
\item[\textbf{BeginPosition}\hfill]
\ifpdf
\begin{flushleft}
\fi
\begin{ttfamily}
public BeginPosition: Int64;\end{ttfamily}

\ifpdf
\end{flushleft}
\fi


\par \begin{ttfamily}BeginPosition\end{ttfamily} is the position in the stream of the start of the comment.\label{PasDoc_Items.TRawDescriptionInfo-EndPosition}
\index{EndPosition}
\item[\textbf{EndPosition}\hfill]
\ifpdf
\begin{flushleft}
\fi
\begin{ttfamily}
public EndPosition: Int64;\end{ttfamily}

\ifpdf
\end{flushleft}
\fi


\par \begin{ttfamily}EndPosition\end{ttfamily} is the position in the stream of the character immediately after the end of the comment describing the item.\end{list}
\ifpdf
\subsection*{\large{\textbf{TBaseItem Class}}\normalsize\hspace{1ex}\hrulefill}
\else
\subsection*{TBaseItem Class}
\fi
\label{PasDoc_Items.TBaseItem}
\index{TBaseItem}
\subsubsection*{\large{\textbf{Hierarchy}}\normalsize\hspace{1ex}\hfill}
TBaseItem {$>$} \begin{ttfamily}TSerializable\end{ttfamily}(\ref{PasDoc_Serialize.TSerializable}) {$>$} 
TObject
\subsubsection*{\large{\textbf{Description}}\normalsize\hspace{1ex}\hfill}
This is a basic item class, that is linkable, and has some \begin{ttfamily}RawDescription\end{ttfamily}(\ref{PasDoc_Items.TBaseItem-RawDescription}).\subsubsection*{\large{\textbf{Properties}}\normalsize\hspace{1ex}\hfill}
\begin{list}{}{
\settowidth{\tmplength}{\textbf{DetailedDescription}}
\setlength{\itemindent}{0cm}
\setlength{\listparindent}{0cm}
\setlength{\leftmargin}{\evensidemargin}
\addtolength{\leftmargin}{\tmplength}
\settowidth{\labelsep}{X}
\addtolength{\leftmargin}{\labelsep}
\setlength{\labelwidth}{\tmplength}
}
\label{PasDoc_Items.TBaseItem-DetailedDescription}
\index{DetailedDescription}
\item[\textbf{DetailedDescription}\hfill]
\ifpdf
\begin{flushleft}
\fi
\begin{ttfamily}
public property DetailedDescription: string
      read FDetailedDescription write FDetailedDescription;\end{ttfamily}

\ifpdf
\end{flushleft}
\fi


\par Detailed description of this item.

In case of TPasItem, this is something more elaborate than \begin{ttfamily}TPasItem.AbstractDescription\end{ttfamily}(\ref{PasDoc_Items.TPasItem-AbstractDescription}).

This is already in the form suitable for final output, ready to be put inside final documentation.\label{PasDoc_Items.TBaseItem-RawDescription}
\index{RawDescription}
\item[\textbf{RawDescription}\hfill]
\ifpdf
\begin{flushleft}
\fi
\begin{ttfamily}
public property RawDescription: string
      read GetRawDescription write WriteRawDescription;\end{ttfamily}

\ifpdf
\end{flushleft}
\fi


\par This stores unexpanded version (as specified in user's comment in source code of parsed units) of description of this item.

Actually, this is just a shortcut to \begin{ttfamily}\begin{ttfamily}RawDescriptionInfo\end{ttfamily}(\ref{PasDoc_Items.TBaseItem-RawDescriptionInfo}).Content\end{ttfamily}\label{PasDoc_Items.TBaseItem-FullLink}
\index{FullLink}
\item[\textbf{FullLink}\hfill]
\ifpdf
\begin{flushleft}
\fi
\begin{ttfamily}
public property FullLink: string read FFullLink write FFullLink;\end{ttfamily}

\ifpdf
\end{flushleft}
\fi


\par a full link that should be enough to link this item from anywhere else\label{PasDoc_Items.TBaseItem-LastMod}
\index{LastMod}
\item[\textbf{LastMod}\hfill]
\ifpdf
\begin{flushleft}
\fi
\begin{ttfamily}
public property LastMod: string read FLastMod write FLastMod;\end{ttfamily}

\ifpdf
\end{flushleft}
\fi


\par Contains '' or string with date of last modification. This string is already in the form suitable for final output format (i.e. already processed by TDocGenerator.ConvertString).\label{PasDoc_Items.TBaseItem-Name}
\index{Name}
\item[\textbf{Name}\hfill]
\ifpdf
\begin{flushleft}
\fi
\begin{ttfamily}
public property Name: string read FName write FName;\end{ttfamily}

\ifpdf
\end{flushleft}
\fi


\par name of the item\label{PasDoc_Items.TBaseItem-Authors}
\index{Authors}
\item[\textbf{Authors}\hfill]
\ifpdf
\begin{flushleft}
\fi
\begin{ttfamily}
public property Authors: TStringVector read FAuthors write SetAuthors;\end{ttfamily}

\ifpdf
\end{flushleft}
\fi


\par list of strings, each representing one author of this item\label{PasDoc_Items.TBaseItem-Created}
\index{Created}
\item[\textbf{Created}\hfill]
\ifpdf
\begin{flushleft}
\fi
\begin{ttfamily}
public property Created: string read FCreated;\end{ttfamily}

\ifpdf
\end{flushleft}
\fi


\par Contains '' or string with date of creation. This string is already in the form suitable for final output format (i.e. already processed by TDocGenerator.ConvertString).\label{PasDoc_Items.TBaseItem-AutoLinkHereAllowed}
\index{AutoLinkHereAllowed}
\item[\textbf{AutoLinkHereAllowed}\hfill]
\ifpdf
\begin{flushleft}
\fi
\begin{ttfamily}
public property AutoLinkHereAllowed: boolean
      read FAutoLinkHereAllowed write FAutoLinkHereAllowed default true;\end{ttfamily}

\ifpdf
\end{flushleft}
\fi


\par Is auto{-}link mechanism allowed to create link to this item ? This may be set to \begin{ttfamily}False\end{ttfamily} by @noAutoLinkHere tag in item's description.\end{list}
\subsubsection*{\large{\textbf{Methods}}\normalsize\hspace{1ex}\hfill}
\paragraph*{Serialize}\hspace*{\fill}

\label{PasDoc_Items.TBaseItem-Serialize}
\index{Serialize}
\begin{list}{}{
\settowidth{\tmplength}{\textbf{Description}}
\setlength{\itemindent}{0cm}
\setlength{\listparindent}{0cm}
\setlength{\leftmargin}{\evensidemargin}
\addtolength{\leftmargin}{\tmplength}
\settowidth{\labelsep}{X}
\addtolength{\leftmargin}{\labelsep}
\setlength{\labelwidth}{\tmplength}
}
\item[\textbf{Declaration}\hfill]
\ifpdf
\begin{flushleft}
\fi
\begin{ttfamily}
protected procedure Serialize(const ADestination: TStream); override;\end{ttfamily}

\ifpdf
\end{flushleft}
\fi

\par
\item[\textbf{Description}]
Serialization of TPasItem need to store in stream only data that is generated by parser. That's because current approach treats "loading from cache" as equivalent to parsing a unit and stores to cache right after parsing a unit. So what is generated by parser must be written to cache.

That said,

\begin{enumerate}
\setcounter{enumi}{0} \setcounter{enumii}{0} \setcounter{enumiii}{0} \setcounter{enumiv}{0} 
\item  It will not break anything if you will accidentally store in cache something that is not generated by parser. That's because saving to cache will be done anyway right after doing parsing, so properties not initialized by parser will have their initial values anyway. You're just wasting memory for cache, and some cache saving/loading time.
\setcounter{enumi}{1} \setcounter{enumii}{1} \setcounter{enumiii}{1} \setcounter{enumiv}{1} 
\item  For now, in implementation of serialize/deserialize we try to add even things not generated by parser in a commented out code. This way if approach to cache will change some day, we will be able to use this code.
\end{enumerate}

\end{list}
\paragraph*{Deserialize}\hspace*{\fill}

\label{PasDoc_Items.TBaseItem-Deserialize}
\index{Deserialize}
\begin{list}{}{
\settowidth{\tmplength}{\textbf{Description}}
\setlength{\itemindent}{0cm}
\setlength{\listparindent}{0cm}
\setlength{\leftmargin}{\evensidemargin}
\addtolength{\leftmargin}{\tmplength}
\settowidth{\labelsep}{X}
\addtolength{\leftmargin}{\labelsep}
\setlength{\labelwidth}{\tmplength}
}
\item[\textbf{Declaration}\hfill]
\ifpdf
\begin{flushleft}
\fi
\begin{ttfamily}
protected procedure Deserialize(const ASource: TStream); override;\end{ttfamily}

\ifpdf
\end{flushleft}
\fi

\end{list}
\paragraph*{Create}\hspace*{\fill}

\label{PasDoc_Items.TBaseItem-Create}
\index{Create}
\begin{list}{}{
\settowidth{\tmplength}{\textbf{Description}}
\setlength{\itemindent}{0cm}
\setlength{\listparindent}{0cm}
\setlength{\leftmargin}{\evensidemargin}
\addtolength{\leftmargin}{\tmplength}
\settowidth{\labelsep}{X}
\addtolength{\leftmargin}{\labelsep}
\setlength{\labelwidth}{\tmplength}
}
\item[\textbf{Declaration}\hfill]
\ifpdf
\begin{flushleft}
\fi
\begin{ttfamily}
public constructor Create; override;\end{ttfamily}

\ifpdf
\end{flushleft}
\fi

\end{list}
\paragraph*{Destroy}\hspace*{\fill}

\label{PasDoc_Items.TBaseItem-Destroy}
\index{Destroy}
\begin{list}{}{
\settowidth{\tmplength}{\textbf{Description}}
\setlength{\itemindent}{0cm}
\setlength{\listparindent}{0cm}
\setlength{\leftmargin}{\evensidemargin}
\addtolength{\leftmargin}{\tmplength}
\settowidth{\labelsep}{X}
\addtolength{\leftmargin}{\labelsep}
\setlength{\labelwidth}{\tmplength}
}
\item[\textbf{Declaration}\hfill]
\ifpdf
\begin{flushleft}
\fi
\begin{ttfamily}
public destructor Destroy; override;\end{ttfamily}

\ifpdf
\end{flushleft}
\fi

\end{list}
\paragraph*{RegisterTags}\hspace*{\fill}

\label{PasDoc_Items.TBaseItem-RegisterTags}
\index{RegisterTags}
\begin{list}{}{
\settowidth{\tmplength}{\textbf{Description}}
\setlength{\itemindent}{0cm}
\setlength{\listparindent}{0cm}
\setlength{\leftmargin}{\evensidemargin}
\addtolength{\leftmargin}{\tmplength}
\settowidth{\labelsep}{X}
\addtolength{\leftmargin}{\labelsep}
\setlength{\labelwidth}{\tmplength}
}
\item[\textbf{Declaration}\hfill]
\ifpdf
\begin{flushleft}
\fi
\begin{ttfamily}
public procedure RegisterTags(TagManager: TTagManager); virtual;\end{ttfamily}

\ifpdf
\end{flushleft}
\fi

\par
\item[\textbf{Description}]
It registers \begin{ttfamily}TTag\end{ttfamily}(\ref{PasDoc_TagManager.TTag})s that init \begin{ttfamily}Authors\end{ttfamily}(\ref{PasDoc_Items.TBaseItem-Authors}), \begin{ttfamily}Created\end{ttfamily}(\ref{PasDoc_Items.TBaseItem-Created}), \begin{ttfamily}LastMod\end{ttfamily}(\ref{PasDoc_Items.TBaseItem-LastMod}) and remove relevant tags from description. You can override it to add more handlers.

\end{list}
\paragraph*{FindItem}\hspace*{\fill}

\label{PasDoc_Items.TBaseItem-FindItem}
\index{FindItem}
\begin{list}{}{
\settowidth{\tmplength}{\textbf{Description}}
\setlength{\itemindent}{0cm}
\setlength{\listparindent}{0cm}
\setlength{\leftmargin}{\evensidemargin}
\addtolength{\leftmargin}{\tmplength}
\settowidth{\labelsep}{X}
\addtolength{\leftmargin}{\labelsep}
\setlength{\labelwidth}{\tmplength}
}
\item[\textbf{Declaration}\hfill]
\ifpdf
\begin{flushleft}
\fi
\begin{ttfamily}
public function FindItem(const ItemName: string): TBaseItem; virtual;\end{ttfamily}

\ifpdf
\end{flushleft}
\fi

\par
\item[\textbf{Description}]
Search for an item called ItemName \textit{inside this Pascal item}. For units, it searches for items declared \textit{inside this unit} (like a procedure, or a class in this unit). For classes it searches for items declared \textit{within this class} (like a method or a property). For an enumerated type, it searches for members of this enumerated type.

All normal rules of ObjectPascal scope apply, which means that e.g. if this item is a unit, \begin{ttfamily}FindItem\end{ttfamily} searches for a class named ItemName but it \textit{doesn't} search for a method named ItemName inside some class of this unit. Just like in ObjectPascal the scope of identifiers declared within the class always stays within the class. Of course, in ObjectPascal you can qualify a method name with a class name, and you can also do such qualified links in pasdoc, but this is not handled by this routine (see \begin{ttfamily}FindName\end{ttfamily}(\ref{PasDoc_Items.TBaseItem-FindName}) instead).

Returns nil if not found.

Note that it never compares ItemName with Self.Name. You may want to check this yourself if you want.

Note that for TPasItem descendants, it always returns also some TPasItem descendant (so if you use this method with some TPasItem instance, you can safely cast result of this method to TPasItem).

Implementation in this class always returns nil. Override as necessary.

\end{list}
\paragraph*{FindItemMaybeInAncestors}\hspace*{\fill}

\label{PasDoc_Items.TBaseItem-FindItemMaybeInAncestors}
\index{FindItemMaybeInAncestors}
\begin{list}{}{
\settowidth{\tmplength}{\textbf{Description}}
\setlength{\itemindent}{0cm}
\setlength{\listparindent}{0cm}
\setlength{\leftmargin}{\evensidemargin}
\addtolength{\leftmargin}{\tmplength}
\settowidth{\labelsep}{X}
\addtolength{\leftmargin}{\labelsep}
\setlength{\labelwidth}{\tmplength}
}
\item[\textbf{Declaration}\hfill]
\ifpdf
\begin{flushleft}
\fi
\begin{ttfamily}
public function FindItemMaybeInAncestors(const ItemName: string): TBaseItem; virtual;\end{ttfamily}

\ifpdf
\end{flushleft}
\fi

\par
\item[\textbf{Description}]
This is just like \begin{ttfamily}FindItem\end{ttfamily}(\ref{PasDoc_Items.TBaseItem-FindItem}), but in case of classes or such it should also search within ancestors. In this class, the default implementation just calls FindItem.

\end{list}
\paragraph*{FindName}\hspace*{\fill}

\label{PasDoc_Items.TBaseItem-FindName}
\index{FindName}
\begin{list}{}{
\settowidth{\tmplength}{\textbf{Description}}
\setlength{\itemindent}{0cm}
\setlength{\listparindent}{0cm}
\setlength{\leftmargin}{\evensidemargin}
\addtolength{\leftmargin}{\tmplength}
\settowidth{\labelsep}{X}
\addtolength{\leftmargin}{\labelsep}
\setlength{\labelwidth}{\tmplength}
}
\item[\textbf{Declaration}\hfill]
\ifpdf
\begin{flushleft}
\fi
\begin{ttfamily}
public function FindName(const NameParts: TNameParts): TBaseItem; virtual;\end{ttfamily}

\ifpdf
\end{flushleft}
\fi

\par
\item[\textbf{Description}]
Do all you can to find link specified by NameParts.

While searching this tries to mimic ObjectPascal identifier scope as much as it can. It seaches within this item, but also within class enclosing this item, within ancestors of this class, within unit enclosing this item, then within units used by unit of this item.

\end{list}
\paragraph*{RawDescriptionInfo}\hspace*{\fill}

\label{PasDoc_Items.TBaseItem-RawDescriptionInfo}
\index{RawDescriptionInfo}
\begin{list}{}{
\settowidth{\tmplength}{\textbf{Description}}
\setlength{\itemindent}{0cm}
\setlength{\listparindent}{0cm}
\setlength{\leftmargin}{\evensidemargin}
\addtolength{\leftmargin}{\tmplength}
\settowidth{\labelsep}{X}
\addtolength{\leftmargin}{\labelsep}
\setlength{\labelwidth}{\tmplength}
}
\item[\textbf{Declaration}\hfill]
\ifpdf
\begin{flushleft}
\fi
\begin{ttfamily}
public function RawDescriptionInfo: PRawDescriptionInfo;\end{ttfamily}

\ifpdf
\end{flushleft}
\fi

\par
\item[\textbf{Description}]
Full info about \begin{ttfamily}RawDescription\end{ttfamily}(\ref{PasDoc_Items.TBaseItem-RawDescription}) of this item, including it's filename and position.

This is intended to be initialized by parser.

This returns \begin{ttfamily}PRawDescriptionInfo\end{ttfamily}(\ref{PasDoc_Items-PRawDescriptionInfo}) instead of just \begin{ttfamily}TRawDescriptionInfo\end{ttfamily}(\ref{PasDoc_Items.TRawDescriptionInfo}) to allow natural setting of properties of this record (otherwise \texttt{Item.RawDescriptionInfo.StreamName~:=~'foo';\\
} would not work as expected) .

\end{list}
\paragraph*{QualifiedName}\hspace*{\fill}

\label{PasDoc_Items.TBaseItem-QualifiedName}
\index{QualifiedName}
\begin{list}{}{
\settowidth{\tmplength}{\textbf{Description}}
\setlength{\itemindent}{0cm}
\setlength{\listparindent}{0cm}
\setlength{\leftmargin}{\evensidemargin}
\addtolength{\leftmargin}{\tmplength}
\settowidth{\labelsep}{X}
\addtolength{\leftmargin}{\labelsep}
\setlength{\labelwidth}{\tmplength}
}
\item[\textbf{Declaration}\hfill]
\ifpdf
\begin{flushleft}
\fi
\begin{ttfamily}
public function QualifiedName: String; virtual;\end{ttfamily}

\ifpdf
\end{flushleft}
\fi

\par
\item[\textbf{Description}]
Returns the qualified name of the item. This is intended to return a concise and not ambigous name. E.g. in case of TPasItem it is overridden to return Name qualified by class name and unit name.

In this class this simply returns Name.

\end{list}
\paragraph*{BasePath}\hspace*{\fill}

\label{PasDoc_Items.TBaseItem-BasePath}
\index{BasePath}
\begin{list}{}{
\settowidth{\tmplength}{\textbf{Description}}
\setlength{\itemindent}{0cm}
\setlength{\listparindent}{0cm}
\setlength{\leftmargin}{\evensidemargin}
\addtolength{\leftmargin}{\tmplength}
\settowidth{\labelsep}{X}
\addtolength{\leftmargin}{\labelsep}
\setlength{\labelwidth}{\tmplength}
}
\item[\textbf{Declaration}\hfill]
\ifpdf
\begin{flushleft}
\fi
\begin{ttfamily}
public function BasePath: string; virtual;\end{ttfamily}

\ifpdf
\end{flushleft}
\fi

\par
\item[\textbf{Description}]
The full (absolute) path used to resolve filenames in this item's descriptions. Must always end with PathDelim. In this class, this simply returns GetCurrentDir (with PathDelim added if needed).

\end{list}
\ifpdf
\subsection*{\large{\textbf{TPasItem Class}}\normalsize\hspace{1ex}\hrulefill}
\else
\subsection*{TPasItem Class}
\fi
\label{PasDoc_Items.TPasItem}
\index{TPasItem}
\subsubsection*{\large{\textbf{Hierarchy}}\normalsize\hspace{1ex}\hfill}
TPasItem {$>$} \begin{ttfamily}TBaseItem\end{ttfamily}(\ref{PasDoc_Items.TBaseItem}) {$>$} \begin{ttfamily}TSerializable\end{ttfamily}(\ref{PasDoc_Serialize.TSerializable}) {$>$} 
TObject
\subsubsection*{\large{\textbf{Description}}\normalsize\hspace{1ex}\hfill}
This is a \begin{ttfamily}TBaseItem\end{ttfamily}(\ref{PasDoc_Items.TBaseItem}) descendant that is always declared inside some Pascal source file.

Parser creates only items of this class (e.g. never some basic \begin{ttfamily}TBaseItem\end{ttfamily}(\ref{PasDoc_Items.TBaseItem}) instance). This class introduces properties and methods pointing to parent unit (\begin{ttfamily}MyUnit\end{ttfamily}(\ref{PasDoc_Items.TPasItem-MyUnit})) and parent class/interface/object/record (\begin{ttfamily}MyObject\end{ttfamily}(\ref{PasDoc_Items.TPasItem-MyObject})). Also many other things not needed at \begin{ttfamily}TBaseItem\end{ttfamily}(\ref{PasDoc_Items.TBaseItem}) level are introduced here: things related to handling @abstract tag, @seealso tag, used to sorting items inside (\begin{ttfamily}Sort\end{ttfamily}(\ref{PasDoc_Items.TPasItem-Sort})) and some more.\subsubsection*{\large{\textbf{Properties}}\normalsize\hspace{1ex}\hfill}
\begin{list}{}{
\settowidth{\tmplength}{\textbf{AbstractDescriptionWasAutomatic}}
\setlength{\itemindent}{0cm}
\setlength{\listparindent}{0cm}
\setlength{\leftmargin}{\evensidemargin}
\addtolength{\leftmargin}{\tmplength}
\settowidth{\labelsep}{X}
\addtolength{\leftmargin}{\labelsep}
\setlength{\labelwidth}{\tmplength}
}
\label{PasDoc_Items.TPasItem-AbstractDescription}
\index{AbstractDescription}
\item[\textbf{AbstractDescription}\hfill]
\ifpdf
\begin{flushleft}
\fi
\begin{ttfamily}
public property AbstractDescription: string
      read FAbstractDescription write FAbstractDescription;\end{ttfamily}

\ifpdf
\end{flushleft}
\fi


\par Abstract description of this item. This is intended to be short (e.g. one sentence) description of this object.

This will be inited from @abstract tag in RawDescription, or cutted out from first sentence in RawDescription if {-}{-}auto{-}abstract was used.

Note that this is already in the form suitable for final output, with tags expanded, chars converted etc.\label{PasDoc_Items.TPasItem-AbstractDescriptionWasAutomatic}
\index{AbstractDescriptionWasAutomatic}
\item[\textbf{AbstractDescriptionWasAutomatic}\hfill]
\ifpdf
\begin{flushleft}
\fi
\begin{ttfamily}
public property AbstractDescriptionWasAutomatic: boolean
      read FAbstractDescriptionWasAutomatic
      write FAbstractDescriptionWasAutomatic;\end{ttfamily}

\ifpdf
\end{flushleft}
\fi


\par TDocGenerator.ExpandDescriptions sets this property to true if AutoAbstract was used and AbstractDescription of this item was automatically deduced from the 1st sentence of RawDescription.

Otherwise (if @abstract was specified explicitly, or there was no @abstract and AutoAbstract was false) this is set to false.

This is a useful hint for generators: it tells them that when they are printing \textit{both} AbstractDescription and DetailedDescription of the item in one place (e.g. TTexDocGenerator.WriteItemLongDescription and TGenericHTMLDocGenerator.WriteItemLongDescription both do this) then they should \textit{not} put any additional space between AbstractDescription and DetailedDescription.

This way when user will specify description like

\texttt{\\\nopagebreak[3]
\textit{{\{}~First~sentence.~Second~sentence.~{\}}}\\\nopagebreak[3]
}\textbf{procedure}\texttt{~Foo;\\
}

and {-}{-}auto{-}abstract was on, then "First sentence." is the AbstractDescription, " Second sentence." is DetailedDescription, AbstractDescriptionWasAutomatic is true and and TGenericHTMLDocGenerator.WriteItemLongDescription can print them as "First sentence. Second sentence."

Without this property, TGenericHTMLDocGenerator.WriteItemLongDescription would not be able to say that this abstract was deduced automatically and would print additional paragraph break that was not present in desscription, i.e. "First sentence.{$<$}p{$>$} Second sentence."\label{PasDoc_Items.TPasItem-MyUnit}
\index{MyUnit}
\item[\textbf{MyUnit}\hfill]
\ifpdf
\begin{flushleft}
\fi
\begin{ttfamily}
public property MyUnit: TPasUnit read FMyUnit write FMyUnit;\end{ttfamily}

\ifpdf
\end{flushleft}
\fi


\par Unit of this item.\label{PasDoc_Items.TPasItem-MyObject}
\index{MyObject}
\item[\textbf{MyObject}\hfill]
\ifpdf
\begin{flushleft}
\fi
\begin{ttfamily}
public property MyObject: TPasCio read FMyObject write FMyObject;\end{ttfamily}

\ifpdf
\end{flushleft}
\fi


\par If this item is part of a class (or record, object., interface...), the corresponding class is stored here. \begin{ttfamily}Nil\end{ttfamily} otherwise.\label{PasDoc_Items.TPasItem-MyEnum}
\index{MyEnum}
\item[\textbf{MyEnum}\hfill]
\ifpdf
\begin{flushleft}
\fi
\begin{ttfamily}
public property MyEnum: TPasEnum read FMyEnum write FMyEnum;\end{ttfamily}

\ifpdf
\end{flushleft}
\fi


\par If this item is a member of an enumerated type, then the enclosing enumerated type is stored here. \begin{ttfamily}Nil\end{ttfamily} otherwise.\label{PasDoc_Items.TPasItem-Visibility}
\index{Visibility}
\item[\textbf{Visibility}\hfill]
\ifpdf
\begin{flushleft}
\fi
\begin{ttfamily}
public property Visibility: TVisibility read FVisibility write FVisibility;\end{ttfamily}

\ifpdf
\end{flushleft}
\fi


\par  \label{PasDoc_Items.TPasItem-HintDirectives}
\index{HintDirectives}
\item[\textbf{HintDirectives}\hfill]
\ifpdf
\begin{flushleft}
\fi
\begin{ttfamily}
public property HintDirectives: THintDirectives read FHintDirectives write FHintDirectives;\end{ttfamily}

\ifpdf
\end{flushleft}
\fi


\par Hint directives specify is this item deprecated, platform{-}specific, library{-}specific, or experimental.\label{PasDoc_Items.TPasItem-DeprecatedNote}
\index{DeprecatedNote}
\item[\textbf{DeprecatedNote}\hfill]
\ifpdf
\begin{flushleft}
\fi
\begin{ttfamily}
public property DeprecatedNote: string
      read FDeprecatedNote write FDeprecatedNote;\end{ttfamily}

\ifpdf
\end{flushleft}
\fi


\par Deprecation note, specified as a string after "deprecated" directive. Empty if none, always empty if \begin{ttfamily}HintDirectives\end{ttfamily}(\ref{PasDoc_Items.TPasItem-HintDirectives}) does not contain hdDeprecated.\label{PasDoc_Items.TPasItem-FullDeclaration}
\index{FullDeclaration}
\item[\textbf{FullDeclaration}\hfill]
\ifpdf
\begin{flushleft}
\fi
\begin{ttfamily}
public property FullDeclaration: string read FFullDeclaration write FFullDeclaration;\end{ttfamily}

\ifpdf
\end{flushleft}
\fi


\par Full declaration of the item. This is full parsed declaration of the given item.

Note that that this is not used for some descendants. Right now it's used only with \begin{itemize}
\item TPasConstant
\item TPasFieldVariable (includes type, default values, etc.)
\item TPasType
\item TPasMethod (includes parameter list, procedural directives, etc.)
\item TPasProperty (includes read/write and storage specifiers, etc.)
\item TPasEnum

But in this special case, '...' is used instead of listing individual members, e.g. 'TEnumName = (...)'. You can get list of Members using TPasEnum.Members. Eventual specifics of each member should be also specified somewhere inside Members items, e.g. \texttt{TMyEnum~=~(meOne,~meTwo~=~3);\\
} and \texttt{TMyEnum~=~(meOne,~meTwo);\\
} will both result in TPasEnum with equal FullDeclaration (just \begin{ttfamily}'TMyEnum = (...)'\end{ttfamily}) but this \begin{ttfamily}'= 3'\end{ttfamily} should be marked somewhere inside Members[1] properties.
\item TPasItem when it's a CIO's field.
\end{itemize}

The intention is that in the future all TPasItem descendants will always have approprtate FullDeclaration set. It all requires adjusting appropriate places in PasDoc{\_}Parser to generate appropriate FullDeclaration.\label{PasDoc_Items.TPasItem-SeeAlso}
\index{SeeAlso}
\item[\textbf{SeeAlso}\hfill]
\ifpdf
\begin{flushleft}
\fi
\begin{ttfamily}
public property SeeAlso: TStringPairVector read FSeeAlso;\end{ttfamily}

\ifpdf
\end{flushleft}
\fi


\par Items here are collected from @seealso tags.

Name of each item is the 1st part of @seealso parameter. Value is the 2nd part of @seealso parameter.\label{PasDoc_Items.TPasItem-Attributes}
\index{Attributes}
\item[\textbf{Attributes}\hfill]
\ifpdf
\begin{flushleft}
\fi
\begin{ttfamily}
public property Attributes: TStringPairVector read FAttributes;\end{ttfamily}

\ifpdf
\end{flushleft}
\fi


\par List of attributes defined for this item\label{PasDoc_Items.TPasItem-Params}
\index{Params}
\item[\textbf{Params}\hfill]
\ifpdf
\begin{flushleft}
\fi
\begin{ttfamily}
public property Params: TStringPairVector read FParams;\end{ttfamily}

\ifpdf
\end{flushleft}
\fi


\par Parameters of method or property.

Name of each item is the name of parameter (without any surrounding whitespace), Value of each item is users description for this item (in already{-}expanded form).

This is already in the form processed by \begin{ttfamily}TTagManager.Execute\end{ttfamily}(\ref{PasDoc_TagManager.TTagManager-Execute}), i.e. with links resolved, html characters escaped etc. So \textit{don't} convert them (e.g. before writing to the final docs) once again (by some ExpandDescription or ConvertString or anything like that).\label{PasDoc_Items.TPasItem-Raises}
\index{Raises}
\item[\textbf{Raises}\hfill]
\ifpdf
\begin{flushleft}
\fi
\begin{ttfamily}
public property Raises: TStringPairVector read FRaises;\end{ttfamily}

\ifpdf
\end{flushleft}
\fi


\par Exceptions raised by the method, or by property getter/setter.

Name of each item is the name of exception class (without any surrounding whitespace), Value of each item is users description for this item (in already{-}expanded form).

This is already in the form processed by \begin{ttfamily}TTagManager.Execute\end{ttfamily}(\ref{PasDoc_TagManager.TTagManager-Execute}), i.e. with links resolved, html characters escaped etc. So \textit{don't} convert them (e.g. before writing to the final docs) once again (by some ExpandDescription or ConvertString or anything like that).\end{list}
\subsubsection*{\large{\textbf{Methods}}\normalsize\hspace{1ex}\hfill}
\paragraph*{Serialize}\hspace*{\fill}

\label{PasDoc_Items.TPasItem-Serialize}
\index{Serialize}
\begin{list}{}{
\settowidth{\tmplength}{\textbf{Description}}
\setlength{\itemindent}{0cm}
\setlength{\listparindent}{0cm}
\setlength{\leftmargin}{\evensidemargin}
\addtolength{\leftmargin}{\tmplength}
\settowidth{\labelsep}{X}
\addtolength{\leftmargin}{\labelsep}
\setlength{\labelwidth}{\tmplength}
}
\item[\textbf{Declaration}\hfill]
\ifpdf
\begin{flushleft}
\fi
\begin{ttfamily}
protected procedure Serialize(const ADestination: TStream); override;\end{ttfamily}

\ifpdf
\end{flushleft}
\fi

\end{list}
\paragraph*{Deserialize}\hspace*{\fill}

\label{PasDoc_Items.TPasItem-Deserialize}
\index{Deserialize}
\begin{list}{}{
\settowidth{\tmplength}{\textbf{Description}}
\setlength{\itemindent}{0cm}
\setlength{\listparindent}{0cm}
\setlength{\leftmargin}{\evensidemargin}
\addtolength{\leftmargin}{\tmplength}
\settowidth{\labelsep}{X}
\addtolength{\leftmargin}{\labelsep}
\setlength{\labelwidth}{\tmplength}
}
\item[\textbf{Declaration}\hfill]
\ifpdf
\begin{flushleft}
\fi
\begin{ttfamily}
protected procedure Deserialize(const ASource: TStream); override;\end{ttfamily}

\ifpdf
\end{flushleft}
\fi

\end{list}
\paragraph*{FindNameWithinUnit}\hspace*{\fill}

\label{PasDoc_Items.TPasItem-FindNameWithinUnit}
\index{FindNameWithinUnit}
\begin{list}{}{
\settowidth{\tmplength}{\textbf{Description}}
\setlength{\itemindent}{0cm}
\setlength{\listparindent}{0cm}
\setlength{\leftmargin}{\evensidemargin}
\addtolength{\leftmargin}{\tmplength}
\settowidth{\labelsep}{X}
\addtolength{\leftmargin}{\labelsep}
\setlength{\labelwidth}{\tmplength}
}
\item[\textbf{Declaration}\hfill]
\ifpdf
\begin{flushleft}
\fi
\begin{ttfamily}
protected function FindNameWithinUnit(const NameParts: TNameParts): TBaseItem; virtual;\end{ttfamily}

\ifpdf
\end{flushleft}
\fi

\par
\item[\textbf{Description}]
This does the same thing as \begin{ttfamily}FindName\end{ttfamily}(\ref{PasDoc_Items.TPasItem-FindName}) but it \textit{doesn't} scan other units. If this item is a unit, it searches only inside this unit, else it searches only inside \begin{ttfamily}MyUnit\end{ttfamily}(\ref{PasDoc_Items.TPasItem-MyUnit}) unit.

Actually \begin{ttfamily}FindName\end{ttfamily}(\ref{PasDoc_Items.TPasItem-FindName}) uses this function.

\end{list}
\paragraph*{Create}\hspace*{\fill}

\label{PasDoc_Items.TPasItem-Create}
\index{Create}
\begin{list}{}{
\settowidth{\tmplength}{\textbf{Description}}
\setlength{\itemindent}{0cm}
\setlength{\listparindent}{0cm}
\setlength{\leftmargin}{\evensidemargin}
\addtolength{\leftmargin}{\tmplength}
\settowidth{\labelsep}{X}
\addtolength{\leftmargin}{\labelsep}
\setlength{\labelwidth}{\tmplength}
}
\item[\textbf{Declaration}\hfill]
\ifpdf
\begin{flushleft}
\fi
\begin{ttfamily}
public constructor Create; override;\end{ttfamily}

\ifpdf
\end{flushleft}
\fi

\end{list}
\paragraph*{Destroy}\hspace*{\fill}

\label{PasDoc_Items.TPasItem-Destroy}
\index{Destroy}
\begin{list}{}{
\settowidth{\tmplength}{\textbf{Description}}
\setlength{\itemindent}{0cm}
\setlength{\listparindent}{0cm}
\setlength{\leftmargin}{\evensidemargin}
\addtolength{\leftmargin}{\tmplength}
\settowidth{\labelsep}{X}
\addtolength{\leftmargin}{\labelsep}
\setlength{\labelwidth}{\tmplength}
}
\item[\textbf{Declaration}\hfill]
\ifpdf
\begin{flushleft}
\fi
\begin{ttfamily}
public destructor Destroy; override;\end{ttfamily}

\ifpdf
\end{flushleft}
\fi

\end{list}
\paragraph*{FindName}\hspace*{\fill}

\label{PasDoc_Items.TPasItem-FindName}
\index{FindName}
\begin{list}{}{
\settowidth{\tmplength}{\textbf{Description}}
\setlength{\itemindent}{0cm}
\setlength{\listparindent}{0cm}
\setlength{\leftmargin}{\evensidemargin}
\addtolength{\leftmargin}{\tmplength}
\settowidth{\labelsep}{X}
\addtolength{\leftmargin}{\labelsep}
\setlength{\labelwidth}{\tmplength}
}
\item[\textbf{Declaration}\hfill]
\ifpdf
\begin{flushleft}
\fi
\begin{ttfamily}
public function FindName(const NameParts: TNameParts): TBaseItem; override;\end{ttfamily}

\ifpdf
\end{flushleft}
\fi

\end{list}
\paragraph*{RegisterTags}\hspace*{\fill}

\label{PasDoc_Items.TPasItem-RegisterTags}
\index{RegisterTags}
\begin{list}{}{
\settowidth{\tmplength}{\textbf{Description}}
\setlength{\itemindent}{0cm}
\setlength{\listparindent}{0cm}
\setlength{\leftmargin}{\evensidemargin}
\addtolength{\leftmargin}{\tmplength}
\settowidth{\labelsep}{X}
\addtolength{\leftmargin}{\labelsep}
\setlength{\labelwidth}{\tmplength}
}
\item[\textbf{Declaration}\hfill]
\ifpdf
\begin{flushleft}
\fi
\begin{ttfamily}
public procedure RegisterTags(TagManager: TTagManager); override;\end{ttfamily}

\ifpdf
\end{flushleft}
\fi

\end{list}
\paragraph*{HasDescription}\hspace*{\fill}

\label{PasDoc_Items.TPasItem-HasDescription}
\index{HasDescription}
\begin{list}{}{
\settowidth{\tmplength}{\textbf{Description}}
\setlength{\itemindent}{0cm}
\setlength{\listparindent}{0cm}
\setlength{\leftmargin}{\evensidemargin}
\addtolength{\leftmargin}{\tmplength}
\settowidth{\labelsep}{X}
\addtolength{\leftmargin}{\labelsep}
\setlength{\labelwidth}{\tmplength}
}
\item[\textbf{Declaration}\hfill]
\ifpdf
\begin{flushleft}
\fi
\begin{ttfamily}
public function HasDescription: Boolean;\end{ttfamily}

\ifpdf
\end{flushleft}
\fi

\par
\item[\textbf{Description}]
Returns true if there is a DetailedDescription or AbstractDescription available.

\end{list}
\paragraph*{QualifiedName}\hspace*{\fill}

\label{PasDoc_Items.TPasItem-QualifiedName}
\index{QualifiedName}
\begin{list}{}{
\settowidth{\tmplength}{\textbf{Description}}
\setlength{\itemindent}{0cm}
\setlength{\listparindent}{0cm}
\setlength{\leftmargin}{\evensidemargin}
\addtolength{\leftmargin}{\tmplength}
\settowidth{\labelsep}{X}
\addtolength{\leftmargin}{\labelsep}
\setlength{\labelwidth}{\tmplength}
}
\item[\textbf{Declaration}\hfill]
\ifpdf
\begin{flushleft}
\fi
\begin{ttfamily}
public function QualifiedName: String; override;\end{ttfamily}

\ifpdf
\end{flushleft}
\fi

\end{list}
\paragraph*{UnitRelativeQualifiedName}\hspace*{\fill}

\label{PasDoc_Items.TPasItem-UnitRelativeQualifiedName}
\index{UnitRelativeQualifiedName}
\begin{list}{}{
\settowidth{\tmplength}{\textbf{Description}}
\setlength{\itemindent}{0cm}
\setlength{\listparindent}{0cm}
\setlength{\leftmargin}{\evensidemargin}
\addtolength{\leftmargin}{\tmplength}
\settowidth{\labelsep}{X}
\addtolength{\leftmargin}{\labelsep}
\setlength{\labelwidth}{\tmplength}
}
\item[\textbf{Declaration}\hfill]
\ifpdf
\begin{flushleft}
\fi
\begin{ttfamily}
public function UnitRelativeQualifiedName: string; virtual;\end{ttfamily}

\ifpdf
\end{flushleft}
\fi

\end{list}
\paragraph*{Sort}\hspace*{\fill}

\label{PasDoc_Items.TPasItem-Sort}
\index{Sort}
\begin{list}{}{
\settowidth{\tmplength}{\textbf{Description}}
\setlength{\itemindent}{0cm}
\setlength{\listparindent}{0cm}
\setlength{\leftmargin}{\evensidemargin}
\addtolength{\leftmargin}{\tmplength}
\settowidth{\labelsep}{X}
\addtolength{\leftmargin}{\labelsep}
\setlength{\labelwidth}{\tmplength}
}
\item[\textbf{Declaration}\hfill]
\ifpdf
\begin{flushleft}
\fi
\begin{ttfamily}
public procedure Sort(const SortSettings: TSortSettings); virtual;\end{ttfamily}

\ifpdf
\end{flushleft}
\fi

\par
\item[\textbf{Description}]
This recursively sorts all items inside this item, and all items inside these items, etc. E.g. in case of TPasUnit, this method sorts all variables, consts, CIOs etc. inside (honouring SortSettings), and also recursively calls Sort(SortSettings) for every CIO.

Note that this does not guarantee that absolutely everything inside will be really sorted. Some items may be deliberately left unsorted, e.g. Members of TPasEnum are never sorted (their declared order always matters, so we shouldn't sort them when displaying their documentation --- reader of such documentation would be seriously misleaded). Sorting of other things depends on SortSettings --- e.g. without ssMethods, CIOs methods will not be sorted.

So actually this method \textit{makes sure that all things that should be sorted are really sorted}.

\end{list}
\paragraph*{SetAttributes}\hspace*{\fill}

\label{PasDoc_Items.TPasItem-SetAttributes}
\index{SetAttributes}
\begin{list}{}{
\settowidth{\tmplength}{\textbf{Description}}
\setlength{\itemindent}{0cm}
\setlength{\listparindent}{0cm}
\setlength{\leftmargin}{\evensidemargin}
\addtolength{\leftmargin}{\tmplength}
\settowidth{\labelsep}{X}
\addtolength{\leftmargin}{\labelsep}
\setlength{\labelwidth}{\tmplength}
}
\item[\textbf{Declaration}\hfill]
\ifpdf
\begin{flushleft}
\fi
\begin{ttfamily}
public procedure SetAttributes(var Value: TStringPairVector);\end{ttfamily}

\ifpdf
\end{flushleft}
\fi

\end{list}
\paragraph*{BasePath}\hspace*{\fill}

\label{PasDoc_Items.TPasItem-BasePath}
\index{BasePath}
\begin{list}{}{
\settowidth{\tmplength}{\textbf{Description}}
\setlength{\itemindent}{0cm}
\setlength{\listparindent}{0cm}
\setlength{\leftmargin}{\evensidemargin}
\addtolength{\leftmargin}{\tmplength}
\settowidth{\labelsep}{X}
\addtolength{\leftmargin}{\labelsep}
\setlength{\labelwidth}{\tmplength}
}
\item[\textbf{Declaration}\hfill]
\ifpdf
\begin{flushleft}
\fi
\begin{ttfamily}
public function BasePath: string; override;\end{ttfamily}

\ifpdf
\end{flushleft}
\fi

\end{list}
\paragraph*{HasOptionalInfo}\hspace*{\fill}

\label{PasDoc_Items.TPasItem-HasOptionalInfo}
\index{HasOptionalInfo}
\begin{list}{}{
\settowidth{\tmplength}{\textbf{Description}}
\setlength{\itemindent}{0cm}
\setlength{\listparindent}{0cm}
\setlength{\leftmargin}{\evensidemargin}
\addtolength{\leftmargin}{\tmplength}
\settowidth{\labelsep}{X}
\addtolength{\leftmargin}{\labelsep}
\setlength{\labelwidth}{\tmplength}
}
\item[\textbf{Declaration}\hfill]
\ifpdf
\begin{flushleft}
\fi
\begin{ttfamily}
public function HasOptionalInfo: boolean; virtual;\end{ttfamily}

\ifpdf
\end{flushleft}
\fi

\par
\item[\textbf{Description}]
Is optional information (that may be empty for after parsing unit and expanding tags) specified. Currently this checks \begin{ttfamily}Params\end{ttfamily}(\ref{PasDoc_Items.TPasItem-Params}) and \begin{ttfamily}Raises\end{ttfamily}(\ref{PasDoc_Items.TPasItem-Raises}) and \begin{ttfamily}TPasMethod.Returns\end{ttfamily}(\ref{PasDoc_Items.TPasMethod-Returns}).

\end{list}
\ifpdf
\subsection*{\large{\textbf{TPasConstant Class}}\normalsize\hspace{1ex}\hrulefill}
\else
\subsection*{TPasConstant Class}
\fi
\label{PasDoc_Items.TPasConstant}
\index{TPasConstant}
\subsubsection*{\large{\textbf{Hierarchy}}\normalsize\hspace{1ex}\hfill}
TPasConstant {$>$} \begin{ttfamily}TPasItem\end{ttfamily}(\ref{PasDoc_Items.TPasItem}) {$>$} \begin{ttfamily}TBaseItem\end{ttfamily}(\ref{PasDoc_Items.TBaseItem}) {$>$} \begin{ttfamily}TSerializable\end{ttfamily}(\ref{PasDoc_Serialize.TSerializable}) {$>$} 
TObject
\subsubsection*{\large{\textbf{Description}}\normalsize\hspace{1ex}\hfill}
Pascal constant.\hfill\vspace*{1ex}



Precise definition of "constant" for pasdoc purposes is "a name associated with a value". Optionally, constant type may also be specified in declararion. Well, Pascal constant always has some type, but pasdoc is too weak to determine the implicit type of a constant, i.e. to unserstand that constand \begin{ttfamily}const A = 1\end{ttfamily} is of type Integer.\ifpdf
\subsection*{\large{\textbf{TPasFieldVariable Class}}\normalsize\hspace{1ex}\hrulefill}
\else
\subsection*{TPasFieldVariable Class}
\fi
\label{PasDoc_Items.TPasFieldVariable}
\index{TPasFieldVariable}
\subsubsection*{\large{\textbf{Hierarchy}}\normalsize\hspace{1ex}\hfill}
TPasFieldVariable {$>$} \begin{ttfamily}TPasItem\end{ttfamily}(\ref{PasDoc_Items.TPasItem}) {$>$} \begin{ttfamily}TBaseItem\end{ttfamily}(\ref{PasDoc_Items.TBaseItem}) {$>$} \begin{ttfamily}TSerializable\end{ttfamily}(\ref{PasDoc_Serialize.TSerializable}) {$>$} 
TObject
\subsubsection*{\large{\textbf{Description}}\normalsize\hspace{1ex}\hfill}
Pascal global variable or field or nested constant of CIO.\hfill\vspace*{1ex}



Precise definition is "a name with some type". And Optionally with some initial value, for global variables. It also holds a nested constant of extended classes and records. In the future we may introduce here some property like Type: TPasType.\subsubsection*{\large{\textbf{Properties}}\normalsize\hspace{1ex}\hfill}
\begin{list}{}{
\settowidth{\tmplength}{\textbf{IsConstant}}
\setlength{\itemindent}{0cm}
\setlength{\listparindent}{0cm}
\setlength{\leftmargin}{\evensidemargin}
\addtolength{\leftmargin}{\tmplength}
\settowidth{\labelsep}{X}
\addtolength{\leftmargin}{\labelsep}
\setlength{\labelwidth}{\tmplength}
}
\label{PasDoc_Items.TPasFieldVariable-IsConstant}
\index{IsConstant}
\item[\textbf{IsConstant}\hfill]
\ifpdf
\begin{flushleft}
\fi
\begin{ttfamily}
public property IsConstant: Boolean read FIsConstant write FIsConstant;\end{ttfamily}

\ifpdf
\end{flushleft}
\fi


\par Set if this is a nested constant field\end{list}
\subsubsection*{\large{\textbf{Methods}}\normalsize\hspace{1ex}\hfill}
\paragraph*{Serialize}\hspace*{\fill}

\label{PasDoc_Items.TPasFieldVariable-Serialize}
\index{Serialize}
\begin{list}{}{
\settowidth{\tmplength}{\textbf{Description}}
\setlength{\itemindent}{0cm}
\setlength{\listparindent}{0cm}
\setlength{\leftmargin}{\evensidemargin}
\addtolength{\leftmargin}{\tmplength}
\settowidth{\labelsep}{X}
\addtolength{\leftmargin}{\labelsep}
\setlength{\labelwidth}{\tmplength}
}
\item[\textbf{Declaration}\hfill]
\ifpdf
\begin{flushleft}
\fi
\begin{ttfamily}
protected procedure Serialize(const ADestination: TStream); override;\end{ttfamily}

\ifpdf
\end{flushleft}
\fi

\end{list}
\paragraph*{Deserialize}\hspace*{\fill}

\label{PasDoc_Items.TPasFieldVariable-Deserialize}
\index{Deserialize}
\begin{list}{}{
\settowidth{\tmplength}{\textbf{Description}}
\setlength{\itemindent}{0cm}
\setlength{\listparindent}{0cm}
\setlength{\leftmargin}{\evensidemargin}
\addtolength{\leftmargin}{\tmplength}
\settowidth{\labelsep}{X}
\addtolength{\leftmargin}{\labelsep}
\setlength{\labelwidth}{\tmplength}
}
\item[\textbf{Declaration}\hfill]
\ifpdf
\begin{flushleft}
\fi
\begin{ttfamily}
protected procedure Deserialize(const ASource: TStream); override;\end{ttfamily}

\ifpdf
\end{flushleft}
\fi

\end{list}
\ifpdf
\subsection*{\large{\textbf{TPasType Class}}\normalsize\hspace{1ex}\hrulefill}
\else
\subsection*{TPasType Class}
\fi
\label{PasDoc_Items.TPasType}
\index{TPasType}
\subsubsection*{\large{\textbf{Hierarchy}}\normalsize\hspace{1ex}\hfill}
TPasType {$>$} \begin{ttfamily}TPasItem\end{ttfamily}(\ref{PasDoc_Items.TPasItem}) {$>$} \begin{ttfamily}TBaseItem\end{ttfamily}(\ref{PasDoc_Items.TBaseItem}) {$>$} \begin{ttfamily}TSerializable\end{ttfamily}(\ref{PasDoc_Serialize.TSerializable}) {$>$} 
TObject
\subsubsection*{\large{\textbf{Description}}\normalsize\hspace{1ex}\hfill}
Pascal type (but not a procedural type --- these are expressed as \begin{ttfamily}TPasMethod\end{ttfamily}(\ref{PasDoc_Items.TPasMethod}).)\ifpdf
\subsection*{\large{\textbf{TPasEnum Class}}\normalsize\hspace{1ex}\hrulefill}
\else
\subsection*{TPasEnum Class}
\fi
\label{PasDoc_Items.TPasEnum}
\index{TPasEnum}
\subsubsection*{\large{\textbf{Hierarchy}}\normalsize\hspace{1ex}\hfill}
TPasEnum {$>$} \begin{ttfamily}TPasType\end{ttfamily}(\ref{PasDoc_Items.TPasType}) {$>$} \begin{ttfamily}TPasItem\end{ttfamily}(\ref{PasDoc_Items.TPasItem}) {$>$} \begin{ttfamily}TBaseItem\end{ttfamily}(\ref{PasDoc_Items.TBaseItem}) {$>$} \begin{ttfamily}TSerializable\end{ttfamily}(\ref{PasDoc_Serialize.TSerializable}) {$>$} 
TObject
\subsubsection*{\large{\textbf{Description}}\normalsize\hspace{1ex}\hfill}
Enumerated type.\subsubsection*{\large{\textbf{Properties}}\normalsize\hspace{1ex}\hfill}
\begin{list}{}{
\settowidth{\tmplength}{\textbf{Members}}
\setlength{\itemindent}{0cm}
\setlength{\listparindent}{0cm}
\setlength{\leftmargin}{\evensidemargin}
\addtolength{\leftmargin}{\tmplength}
\settowidth{\labelsep}{X}
\addtolength{\leftmargin}{\labelsep}
\setlength{\labelwidth}{\tmplength}
}
\label{PasDoc_Items.TPasEnum-Members}
\index{Members}
\item[\textbf{Members}\hfill]
\ifpdf
\begin{flushleft}
\fi
\begin{ttfamily}
public property Members: TPasItems read FMembers;\end{ttfamily}

\ifpdf
\end{flushleft}
\fi


\par  \end{list}
\subsubsection*{\large{\textbf{Fields}}\normalsize\hspace{1ex}\hfill}
\begin{list}{}{
\settowidth{\tmplength}{\textbf{FMembers}}
\setlength{\itemindent}{0cm}
\setlength{\listparindent}{0cm}
\setlength{\leftmargin}{\evensidemargin}
\addtolength{\leftmargin}{\tmplength}
\settowidth{\labelsep}{X}
\addtolength{\leftmargin}{\labelsep}
\setlength{\labelwidth}{\tmplength}
}
\label{PasDoc_Items.TPasEnum-FMembers}
\index{FMembers}
\item[\textbf{FMembers}\hfill]
\ifpdf
\begin{flushleft}
\fi
\begin{ttfamily}
protected FMembers: TPasItems;\end{ttfamily}

\ifpdf
\end{flushleft}
\fi


\par  \end{list}
\subsubsection*{\large{\textbf{Methods}}\normalsize\hspace{1ex}\hfill}
\paragraph*{Serialize}\hspace*{\fill}

\label{PasDoc_Items.TPasEnum-Serialize}
\index{Serialize}
\begin{list}{}{
\settowidth{\tmplength}{\textbf{Description}}
\setlength{\itemindent}{0cm}
\setlength{\listparindent}{0cm}
\setlength{\leftmargin}{\evensidemargin}
\addtolength{\leftmargin}{\tmplength}
\settowidth{\labelsep}{X}
\addtolength{\leftmargin}{\labelsep}
\setlength{\labelwidth}{\tmplength}
}
\item[\textbf{Declaration}\hfill]
\ifpdf
\begin{flushleft}
\fi
\begin{ttfamily}
protected procedure Serialize(const ADestination: TStream); override;\end{ttfamily}

\ifpdf
\end{flushleft}
\fi

\end{list}
\paragraph*{Deserialize}\hspace*{\fill}

\label{PasDoc_Items.TPasEnum-Deserialize}
\index{Deserialize}
\begin{list}{}{
\settowidth{\tmplength}{\textbf{Description}}
\setlength{\itemindent}{0cm}
\setlength{\listparindent}{0cm}
\setlength{\leftmargin}{\evensidemargin}
\addtolength{\leftmargin}{\tmplength}
\settowidth{\labelsep}{X}
\addtolength{\leftmargin}{\labelsep}
\setlength{\labelwidth}{\tmplength}
}
\item[\textbf{Declaration}\hfill]
\ifpdf
\begin{flushleft}
\fi
\begin{ttfamily}
protected procedure Deserialize(const ASource: TStream); override;\end{ttfamily}

\ifpdf
\end{flushleft}
\fi

\end{list}
\paragraph*{StoreValueTag}\hspace*{\fill}

\label{PasDoc_Items.TPasEnum-StoreValueTag}
\index{StoreValueTag}
\begin{list}{}{
\settowidth{\tmplength}{\textbf{Description}}
\setlength{\itemindent}{0cm}
\setlength{\listparindent}{0cm}
\setlength{\leftmargin}{\evensidemargin}
\addtolength{\leftmargin}{\tmplength}
\settowidth{\labelsep}{X}
\addtolength{\leftmargin}{\labelsep}
\setlength{\labelwidth}{\tmplength}
}
\item[\textbf{Declaration}\hfill]
\ifpdf
\begin{flushleft}
\fi
\begin{ttfamily}
protected procedure StoreValueTag(ThisTag: TTag; var ThisTagData: TObject; EnclosingTag: TTag; var EnclosingTagData: TObject; const TagParameter: string; var ReplaceStr: string);\end{ttfamily}

\ifpdf
\end{flushleft}
\fi

\end{list}
\paragraph*{RegisterTags}\hspace*{\fill}

\label{PasDoc_Items.TPasEnum-RegisterTags}
\index{RegisterTags}
\begin{list}{}{
\settowidth{\tmplength}{\textbf{Description}}
\setlength{\itemindent}{0cm}
\setlength{\listparindent}{0cm}
\setlength{\leftmargin}{\evensidemargin}
\addtolength{\leftmargin}{\tmplength}
\settowidth{\labelsep}{X}
\addtolength{\leftmargin}{\labelsep}
\setlength{\labelwidth}{\tmplength}
}
\item[\textbf{Declaration}\hfill]
\ifpdf
\begin{flushleft}
\fi
\begin{ttfamily}
public procedure RegisterTags(TagManager: TTagManager); override;\end{ttfamily}

\ifpdf
\end{flushleft}
\fi

\end{list}
\paragraph*{FindItem}\hspace*{\fill}

\label{PasDoc_Items.TPasEnum-FindItem}
\index{FindItem}
\begin{list}{}{
\settowidth{\tmplength}{\textbf{Description}}
\setlength{\itemindent}{0cm}
\setlength{\listparindent}{0cm}
\setlength{\leftmargin}{\evensidemargin}
\addtolength{\leftmargin}{\tmplength}
\settowidth{\labelsep}{X}
\addtolength{\leftmargin}{\labelsep}
\setlength{\labelwidth}{\tmplength}
}
\item[\textbf{Declaration}\hfill]
\ifpdf
\begin{flushleft}
\fi
\begin{ttfamily}
public function FindItem(const ItemName: string): TBaseItem; override;\end{ttfamily}

\ifpdf
\end{flushleft}
\fi

\par
\item[\textbf{Description}]
Searches for a member of this enumerated type.

\end{list}
\paragraph*{Destroy}\hspace*{\fill}

\label{PasDoc_Items.TPasEnum-Destroy}
\index{Destroy}
\begin{list}{}{
\settowidth{\tmplength}{\textbf{Description}}
\setlength{\itemindent}{0cm}
\setlength{\listparindent}{0cm}
\setlength{\leftmargin}{\evensidemargin}
\addtolength{\leftmargin}{\tmplength}
\settowidth{\labelsep}{X}
\addtolength{\leftmargin}{\labelsep}
\setlength{\labelwidth}{\tmplength}
}
\item[\textbf{Declaration}\hfill]
\ifpdf
\begin{flushleft}
\fi
\begin{ttfamily}
public destructor Destroy; override;\end{ttfamily}

\ifpdf
\end{flushleft}
\fi

\end{list}
\paragraph*{Create}\hspace*{\fill}

\label{PasDoc_Items.TPasEnum-Create}
\index{Create}
\begin{list}{}{
\settowidth{\tmplength}{\textbf{Description}}
\setlength{\itemindent}{0cm}
\setlength{\listparindent}{0cm}
\setlength{\leftmargin}{\evensidemargin}
\addtolength{\leftmargin}{\tmplength}
\settowidth{\labelsep}{X}
\addtolength{\leftmargin}{\labelsep}
\setlength{\labelwidth}{\tmplength}
}
\item[\textbf{Declaration}\hfill]
\ifpdf
\begin{flushleft}
\fi
\begin{ttfamily}
public constructor Create; override;\end{ttfamily}

\ifpdf
\end{flushleft}
\fi

\end{list}
\ifpdf
\subsection*{\large{\textbf{TPasMethod Class}}\normalsize\hspace{1ex}\hrulefill}
\else
\subsection*{TPasMethod Class}
\fi
\label{PasDoc_Items.TPasMethod}
\index{TPasMethod}
\subsubsection*{\large{\textbf{Hierarchy}}\normalsize\hspace{1ex}\hfill}
TPasMethod {$>$} \begin{ttfamily}TPasItem\end{ttfamily}(\ref{PasDoc_Items.TPasItem}) {$>$} \begin{ttfamily}TBaseItem\end{ttfamily}(\ref{PasDoc_Items.TBaseItem}) {$>$} \begin{ttfamily}TSerializable\end{ttfamily}(\ref{PasDoc_Serialize.TSerializable}) {$>$} 
TObject
\subsubsection*{\large{\textbf{Description}}\normalsize\hspace{1ex}\hfill}
This represents: \begin{enumerate}
\setcounter{enumi}{0} \setcounter{enumii}{0} \setcounter{enumiii}{0} \setcounter{enumiv}{0} 
\item global function/procedure,
\setcounter{enumi}{1} \setcounter{enumii}{1} \setcounter{enumiii}{1} \setcounter{enumiv}{1} 
\item method (function/procedure of a class/interface/object),
\setcounter{enumi}{2} \setcounter{enumii}{2} \setcounter{enumiii}{2} \setcounter{enumiv}{2} 
\item pointer type to one of the above (in this case Name is the type name).
\end{enumerate}\subsubsection*{\large{\textbf{Properties}}\normalsize\hspace{1ex}\hfill}
\begin{list}{}{
\settowidth{\tmplength}{\textbf{Directives}}
\setlength{\itemindent}{0cm}
\setlength{\listparindent}{0cm}
\setlength{\leftmargin}{\evensidemargin}
\addtolength{\leftmargin}{\tmplength}
\settowidth{\labelsep}{X}
\addtolength{\leftmargin}{\labelsep}
\setlength{\labelwidth}{\tmplength}
}
\label{PasDoc_Items.TPasMethod-What}
\index{What}
\item[\textbf{What}\hfill]
\ifpdf
\begin{flushleft}
\fi
\begin{ttfamily}
public property What: TMethodType read FWhat write FWhat;\end{ttfamily}

\ifpdf
\end{flushleft}
\fi


\par  \label{PasDoc_Items.TPasMethod-Returns}
\index{Returns}
\item[\textbf{Returns}\hfill]
\ifpdf
\begin{flushleft}
\fi
\begin{ttfamily}
public property Returns: string read FReturns;\end{ttfamily}

\ifpdf
\end{flushleft}
\fi


\par What does the method return.

This is already in the form processed by \begin{ttfamily}TTagManager.Execute\end{ttfamily}(\ref{PasDoc_TagManager.TTagManager-Execute}), i.e. with links resolved, html characters escaped etc. So \textit{don't} convert them (e.g. before writing to the final docs) once again (by some ExpandDescription or ConvertString or anything like that).\label{PasDoc_Items.TPasMethod-Directives}
\index{Directives}
\item[\textbf{Directives}\hfill]
\ifpdf
\begin{flushleft}
\fi
\begin{ttfamily}
public property Directives: TStandardDirectives read FDirectives write FDirectives;\end{ttfamily}

\ifpdf
\end{flushleft}
\fi


\par Set of method directive flags\end{list}
\subsubsection*{\large{\textbf{Fields}}\normalsize\hspace{1ex}\hfill}
\begin{list}{}{
\settowidth{\tmplength}{\textbf{FDirectives}}
\setlength{\itemindent}{0cm}
\setlength{\listparindent}{0cm}
\setlength{\leftmargin}{\evensidemargin}
\addtolength{\leftmargin}{\tmplength}
\settowidth{\labelsep}{X}
\addtolength{\leftmargin}{\labelsep}
\setlength{\labelwidth}{\tmplength}
}
\label{PasDoc_Items.TPasMethod-FReturns}
\index{FReturns}
\item[\textbf{FReturns}\hfill]
\ifpdf
\begin{flushleft}
\fi
\begin{ttfamily}
protected FReturns: string;\end{ttfamily}

\ifpdf
\end{flushleft}
\fi


\par  \label{PasDoc_Items.TPasMethod-FWhat}
\index{FWhat}
\item[\textbf{FWhat}\hfill]
\ifpdf
\begin{flushleft}
\fi
\begin{ttfamily}
protected FWhat: TMethodType;\end{ttfamily}

\ifpdf
\end{flushleft}
\fi


\par  \label{PasDoc_Items.TPasMethod-FDirectives}
\index{FDirectives}
\item[\textbf{FDirectives}\hfill]
\ifpdf
\begin{flushleft}
\fi
\begin{ttfamily}
protected FDirectives: TStandardDirectives;\end{ttfamily}

\ifpdf
\end{flushleft}
\fi


\par  \end{list}
\subsubsection*{\large{\textbf{Methods}}\normalsize\hspace{1ex}\hfill}
\paragraph*{Serialize}\hspace*{\fill}

\label{PasDoc_Items.TPasMethod-Serialize}
\index{Serialize}
\begin{list}{}{
\settowidth{\tmplength}{\textbf{Description}}
\setlength{\itemindent}{0cm}
\setlength{\listparindent}{0cm}
\setlength{\leftmargin}{\evensidemargin}
\addtolength{\leftmargin}{\tmplength}
\settowidth{\labelsep}{X}
\addtolength{\leftmargin}{\labelsep}
\setlength{\labelwidth}{\tmplength}
}
\item[\textbf{Declaration}\hfill]
\ifpdf
\begin{flushleft}
\fi
\begin{ttfamily}
protected procedure Serialize(const ADestination: TStream); override;\end{ttfamily}

\ifpdf
\end{flushleft}
\fi

\end{list}
\paragraph*{Deserialize}\hspace*{\fill}

\label{PasDoc_Items.TPasMethod-Deserialize}
\index{Deserialize}
\begin{list}{}{
\settowidth{\tmplength}{\textbf{Description}}
\setlength{\itemindent}{0cm}
\setlength{\listparindent}{0cm}
\setlength{\leftmargin}{\evensidemargin}
\addtolength{\leftmargin}{\tmplength}
\settowidth{\labelsep}{X}
\addtolength{\leftmargin}{\labelsep}
\setlength{\labelwidth}{\tmplength}
}
\item[\textbf{Declaration}\hfill]
\ifpdf
\begin{flushleft}
\fi
\begin{ttfamily}
protected procedure Deserialize(const ASource: TStream); override;\end{ttfamily}

\ifpdf
\end{flushleft}
\fi

\end{list}
\paragraph*{StoreReturnsTag}\hspace*{\fill}

\label{PasDoc_Items.TPasMethod-StoreReturnsTag}
\index{StoreReturnsTag}
\begin{list}{}{
\settowidth{\tmplength}{\textbf{Description}}
\setlength{\itemindent}{0cm}
\setlength{\listparindent}{0cm}
\setlength{\leftmargin}{\evensidemargin}
\addtolength{\leftmargin}{\tmplength}
\settowidth{\labelsep}{X}
\addtolength{\leftmargin}{\labelsep}
\setlength{\labelwidth}{\tmplength}
}
\item[\textbf{Declaration}\hfill]
\ifpdf
\begin{flushleft}
\fi
\begin{ttfamily}
protected procedure StoreReturnsTag(ThisTag: TTag; var ThisTagData: TObject; EnclosingTag: TTag; var EnclosingTagData: TObject; const TagParameter: string; var ReplaceStr: string);\end{ttfamily}

\ifpdf
\end{flushleft}
\fi

\end{list}
\paragraph*{Create}\hspace*{\fill}

\label{PasDoc_Items.TPasMethod-Create}
\index{Create}
\begin{list}{}{
\settowidth{\tmplength}{\textbf{Description}}
\setlength{\itemindent}{0cm}
\setlength{\listparindent}{0cm}
\setlength{\leftmargin}{\evensidemargin}
\addtolength{\leftmargin}{\tmplength}
\settowidth{\labelsep}{X}
\addtolength{\leftmargin}{\labelsep}
\setlength{\labelwidth}{\tmplength}
}
\item[\textbf{Declaration}\hfill]
\ifpdf
\begin{flushleft}
\fi
\begin{ttfamily}
public constructor Create; override;\end{ttfamily}

\ifpdf
\end{flushleft}
\fi

\end{list}
\paragraph*{Destroy}\hspace*{\fill}

\label{PasDoc_Items.TPasMethod-Destroy}
\index{Destroy}
\begin{list}{}{
\settowidth{\tmplength}{\textbf{Description}}
\setlength{\itemindent}{0cm}
\setlength{\listparindent}{0cm}
\setlength{\leftmargin}{\evensidemargin}
\addtolength{\leftmargin}{\tmplength}
\settowidth{\labelsep}{X}
\addtolength{\leftmargin}{\labelsep}
\setlength{\labelwidth}{\tmplength}
}
\item[\textbf{Declaration}\hfill]
\ifpdf
\begin{flushleft}
\fi
\begin{ttfamily}
public destructor Destroy; override;\end{ttfamily}

\ifpdf
\end{flushleft}
\fi

\end{list}
\paragraph*{RegisterTags}\hspace*{\fill}

\label{PasDoc_Items.TPasMethod-RegisterTags}
\index{RegisterTags}
\begin{list}{}{
\settowidth{\tmplength}{\textbf{Description}}
\setlength{\itemindent}{0cm}
\setlength{\listparindent}{0cm}
\setlength{\leftmargin}{\evensidemargin}
\addtolength{\leftmargin}{\tmplength}
\settowidth{\labelsep}{X}
\addtolength{\leftmargin}{\labelsep}
\setlength{\labelwidth}{\tmplength}
}
\item[\textbf{Declaration}\hfill]
\ifpdf
\begin{flushleft}
\fi
\begin{ttfamily}
public procedure RegisterTags(TagManager: TTagManager); override;\end{ttfamily}

\ifpdf
\end{flushleft}
\fi

\par
\item[\textbf{Description}]
In addition to inherited, this also registers \begin{ttfamily}TTag\end{ttfamily}(\ref{PasDoc_TagManager.TTag}) that inits \begin{ttfamily}Returns\end{ttfamily}(\ref{PasDoc_Items.TPasMethod-Returns}).

\end{list}
\paragraph*{HasOptionalInfo}\hspace*{\fill}

\label{PasDoc_Items.TPasMethod-HasOptionalInfo}
\index{HasOptionalInfo}
\begin{list}{}{
\settowidth{\tmplength}{\textbf{Description}}
\setlength{\itemindent}{0cm}
\setlength{\listparindent}{0cm}
\setlength{\leftmargin}{\evensidemargin}
\addtolength{\leftmargin}{\tmplength}
\settowidth{\labelsep}{X}
\addtolength{\leftmargin}{\labelsep}
\setlength{\labelwidth}{\tmplength}
}
\item[\textbf{Declaration}\hfill]
\ifpdf
\begin{flushleft}
\fi
\begin{ttfamily}
public function HasOptionalInfo: boolean; override;\end{ttfamily}

\ifpdf
\end{flushleft}
\fi

\end{list}
\ifpdf
\subsection*{\large{\textbf{TPasProperty Class}}\normalsize\hspace{1ex}\hrulefill}
\else
\subsection*{TPasProperty Class}
\fi
\label{PasDoc_Items.TPasProperty}
\index{TPasProperty}
\subsubsection*{\large{\textbf{Hierarchy}}\normalsize\hspace{1ex}\hfill}
TPasProperty {$>$} \begin{ttfamily}TPasItem\end{ttfamily}(\ref{PasDoc_Items.TPasItem}) {$>$} \begin{ttfamily}TBaseItem\end{ttfamily}(\ref{PasDoc_Items.TBaseItem}) {$>$} \begin{ttfamily}TSerializable\end{ttfamily}(\ref{PasDoc_Serialize.TSerializable}) {$>$} 
TObject
\subsubsection*{\large{\textbf{Description}}\normalsize\hspace{1ex}\hfill}
no description available, TPasItem description followsThis is a \begin{ttfamily}TBaseItem\end{ttfamily}(\ref{PasDoc_Items.TBaseItem}) descendant that is always declared inside some Pascal source file.

Parser creates only items of this class (e.g. never some basic \begin{ttfamily}TBaseItem\end{ttfamily}(\ref{PasDoc_Items.TBaseItem}) instance). This class introduces properties and methods pointing to parent unit (\begin{ttfamily}MyUnit\end{ttfamily}(\ref{PasDoc_Items.TPasItem-MyUnit})) and parent class/interface/object/record (\begin{ttfamily}MyObject\end{ttfamily}(\ref{PasDoc_Items.TPasItem-MyObject})). Also many other things not needed at \begin{ttfamily}TBaseItem\end{ttfamily}(\ref{PasDoc_Items.TBaseItem}) level are introduced here: things related to handling @abstract tag, @seealso tag, used to sorting items inside (\begin{ttfamily}Sort\end{ttfamily}(\ref{PasDoc_Items.TPasItem-Sort})) and some more.\subsubsection*{\large{\textbf{Properties}}\normalsize\hspace{1ex}\hfill}
\begin{list}{}{
\settowidth{\tmplength}{\textbf{IndexDecl}}
\setlength{\itemindent}{0cm}
\setlength{\listparindent}{0cm}
\setlength{\leftmargin}{\evensidemargin}
\addtolength{\leftmargin}{\tmplength}
\settowidth{\labelsep}{X}
\addtolength{\leftmargin}{\labelsep}
\setlength{\labelwidth}{\tmplength}
}
\label{PasDoc_Items.TPasProperty-IndexDecl}
\index{IndexDecl}
\item[\textbf{IndexDecl}\hfill]
\ifpdf
\begin{flushleft}
\fi
\begin{ttfamily}
public property IndexDecl: string read FIndexDecl write FIndexDecl;\end{ttfamily}

\ifpdf
\end{flushleft}
\fi


\par contains the optional index declaration, including brackets\label{PasDoc_Items.TPasProperty-Proptype}
\index{Proptype}
\item[\textbf{Proptype}\hfill]
\ifpdf
\begin{flushleft}
\fi
\begin{ttfamily}
public property Proptype: string read FPropType write FPropType;\end{ttfamily}

\ifpdf
\end{flushleft}
\fi


\par contains the type of the property\label{PasDoc_Items.TPasProperty-Reader}
\index{Reader}
\item[\textbf{Reader}\hfill]
\ifpdf
\begin{flushleft}
\fi
\begin{ttfamily}
public property Reader: string read FReader write FReader;\end{ttfamily}

\ifpdf
\end{flushleft}
\fi


\par read specifier\label{PasDoc_Items.TPasProperty-Writer}
\index{Writer}
\item[\textbf{Writer}\hfill]
\ifpdf
\begin{flushleft}
\fi
\begin{ttfamily}
public property Writer: string read FWriter write FWriter;\end{ttfamily}

\ifpdf
\end{flushleft}
\fi


\par write specifier\label{PasDoc_Items.TPasProperty-Default}
\index{Default}
\item[\textbf{Default}\hfill]
\ifpdf
\begin{flushleft}
\fi
\begin{ttfamily}
public property Default: Boolean read FDefault write FDefault;\end{ttfamily}

\ifpdf
\end{flushleft}
\fi


\par true if the property is the default property\label{PasDoc_Items.TPasProperty-DefaultID}
\index{DefaultID}
\item[\textbf{DefaultID}\hfill]
\ifpdf
\begin{flushleft}
\fi
\begin{ttfamily}
public property DefaultID: string read FDefaultID write FDefaultID;\end{ttfamily}

\ifpdf
\end{flushleft}
\fi


\par keeps default value specifier\label{PasDoc_Items.TPasProperty-NoDefault}
\index{NoDefault}
\item[\textbf{NoDefault}\hfill]
\ifpdf
\begin{flushleft}
\fi
\begin{ttfamily}
public property NoDefault: Boolean read FNoDefault write FNoDefault;\end{ttfamily}

\ifpdf
\end{flushleft}
\fi


\par true if Nodefault property\label{PasDoc_Items.TPasProperty-StoredId}
\index{StoredId}
\item[\textbf{StoredId}\hfill]
\ifpdf
\begin{flushleft}
\fi
\begin{ttfamily}
public property StoredId: string read FStoredID write FStoredID;\end{ttfamily}

\ifpdf
\end{flushleft}
\fi


\par keeps Stored specifier\end{list}
\subsubsection*{\large{\textbf{Fields}}\normalsize\hspace{1ex}\hfill}
\begin{list}{}{
\settowidth{\tmplength}{\textbf{FNoDefault}}
\setlength{\itemindent}{0cm}
\setlength{\listparindent}{0cm}
\setlength{\leftmargin}{\evensidemargin}
\addtolength{\leftmargin}{\tmplength}
\settowidth{\labelsep}{X}
\addtolength{\leftmargin}{\labelsep}
\setlength{\labelwidth}{\tmplength}
}
\label{PasDoc_Items.TPasProperty-FDefault}
\index{FDefault}
\item[\textbf{FDefault}\hfill]
\ifpdf
\begin{flushleft}
\fi
\begin{ttfamily}
protected FDefault: Boolean;\end{ttfamily}

\ifpdf
\end{flushleft}
\fi


\par  \label{PasDoc_Items.TPasProperty-FNoDefault}
\index{FNoDefault}
\item[\textbf{FNoDefault}\hfill]
\ifpdf
\begin{flushleft}
\fi
\begin{ttfamily}
protected FNoDefault: Boolean;\end{ttfamily}

\ifpdf
\end{flushleft}
\fi


\par  \label{PasDoc_Items.TPasProperty-FIndexDecl}
\index{FIndexDecl}
\item[\textbf{FIndexDecl}\hfill]
\ifpdf
\begin{flushleft}
\fi
\begin{ttfamily}
protected FIndexDecl: string;\end{ttfamily}

\ifpdf
\end{flushleft}
\fi


\par  \label{PasDoc_Items.TPasProperty-FStoredID}
\index{FStoredID}
\item[\textbf{FStoredID}\hfill]
\ifpdf
\begin{flushleft}
\fi
\begin{ttfamily}
protected FStoredID: string;\end{ttfamily}

\ifpdf
\end{flushleft}
\fi


\par  \label{PasDoc_Items.TPasProperty-FDefaultID}
\index{FDefaultID}
\item[\textbf{FDefaultID}\hfill]
\ifpdf
\begin{flushleft}
\fi
\begin{ttfamily}
protected FDefaultID: string;\end{ttfamily}

\ifpdf
\end{flushleft}
\fi


\par  \label{PasDoc_Items.TPasProperty-FWriter}
\index{FWriter}
\item[\textbf{FWriter}\hfill]
\ifpdf
\begin{flushleft}
\fi
\begin{ttfamily}
protected FWriter: string;\end{ttfamily}

\ifpdf
\end{flushleft}
\fi


\par  \label{PasDoc_Items.TPasProperty-FPropType}
\index{FPropType}
\item[\textbf{FPropType}\hfill]
\ifpdf
\begin{flushleft}
\fi
\begin{ttfamily}
protected FPropType: string;\end{ttfamily}

\ifpdf
\end{flushleft}
\fi


\par  \label{PasDoc_Items.TPasProperty-FReader}
\index{FReader}
\item[\textbf{FReader}\hfill]
\ifpdf
\begin{flushleft}
\fi
\begin{ttfamily}
protected FReader: string;\end{ttfamily}

\ifpdf
\end{flushleft}
\fi


\par  \end{list}
\subsubsection*{\large{\textbf{Methods}}\normalsize\hspace{1ex}\hfill}
\paragraph*{Serialize}\hspace*{\fill}

\label{PasDoc_Items.TPasProperty-Serialize}
\index{Serialize}
\begin{list}{}{
\settowidth{\tmplength}{\textbf{Description}}
\setlength{\itemindent}{0cm}
\setlength{\listparindent}{0cm}
\setlength{\leftmargin}{\evensidemargin}
\addtolength{\leftmargin}{\tmplength}
\settowidth{\labelsep}{X}
\addtolength{\leftmargin}{\labelsep}
\setlength{\labelwidth}{\tmplength}
}
\item[\textbf{Declaration}\hfill]
\ifpdf
\begin{flushleft}
\fi
\begin{ttfamily}
protected procedure Serialize(const ADestination: TStream); override;\end{ttfamily}

\ifpdf
\end{flushleft}
\fi

\end{list}
\paragraph*{Deserialize}\hspace*{\fill}

\label{PasDoc_Items.TPasProperty-Deserialize}
\index{Deserialize}
\begin{list}{}{
\settowidth{\tmplength}{\textbf{Description}}
\setlength{\itemindent}{0cm}
\setlength{\listparindent}{0cm}
\setlength{\leftmargin}{\evensidemargin}
\addtolength{\leftmargin}{\tmplength}
\settowidth{\labelsep}{X}
\addtolength{\leftmargin}{\labelsep}
\setlength{\labelwidth}{\tmplength}
}
\item[\textbf{Declaration}\hfill]
\ifpdf
\begin{flushleft}
\fi
\begin{ttfamily}
protected procedure Deserialize(const ASource: TStream); override;\end{ttfamily}

\ifpdf
\end{flushleft}
\fi

\end{list}
\ifpdf
\subsection*{\large{\textbf{TPasCio Class}}\normalsize\hspace{1ex}\hrulefill}
\else
\subsection*{TPasCio Class}
\fi
\label{PasDoc_Items.TPasCio}
\index{TPasCio}
\subsubsection*{\large{\textbf{Hierarchy}}\normalsize\hspace{1ex}\hfill}
TPasCio {$>$} \begin{ttfamily}TPasType\end{ttfamily}(\ref{PasDoc_Items.TPasType}) {$>$} \begin{ttfamily}TPasItem\end{ttfamily}(\ref{PasDoc_Items.TPasItem}) {$>$} \begin{ttfamily}TBaseItem\end{ttfamily}(\ref{PasDoc_Items.TBaseItem}) {$>$} \begin{ttfamily}TSerializable\end{ttfamily}(\ref{PasDoc_Serialize.TSerializable}) {$>$} 
TObject
\subsubsection*{\large{\textbf{Description}}\normalsize\hspace{1ex}\hfill}
Extends \begin{ttfamily}TPasItem\end{ttfamily}(\ref{PasDoc_Items.TPasItem}) to store all items in a class / an object, e.g. fields.\subsubsection*{\large{\textbf{Properties}}\normalsize\hspace{1ex}\hfill}
\begin{list}{}{
\settowidth{\tmplength}{\textbf{HelperTypeIdentifier}}
\setlength{\itemindent}{0cm}
\setlength{\listparindent}{0cm}
\setlength{\leftmargin}{\evensidemargin}
\addtolength{\leftmargin}{\tmplength}
\settowidth{\labelsep}{X}
\addtolength{\leftmargin}{\labelsep}
\setlength{\labelwidth}{\tmplength}
}
\label{PasDoc_Items.TPasCio-Ancestors}
\index{Ancestors}
\item[\textbf{Ancestors}\hfill]
\ifpdf
\begin{flushleft}
\fi
\begin{ttfamily}
public property Ancestors: TStringPairVector read FAncestors;\end{ttfamily}

\ifpdf
\end{flushleft}
\fi


\par Name of the ancestor (class, object, interface). Each item is a TStringPair, with \begin{itemize}
\item \begin{ttfamily}Name\end{ttfamily} is the name (single Pascal identifier) of this ancestor,
\item \begin{ttfamily}Value\end{ttfamily} is the full declaration of this ancestor. For example, in addition to Name, this may include "specialize" directive (for FPC generic specialization) at the beginning. And "{$<$}foo,bar{$>$}" section at the end (for FPC or Delphi generic specialization).
\item \begin{ttfamily}Data\end{ttfamily} is a TPasItem reference to this ancestor, or \begin{ttfamily}Nil\end{ttfamily} if not found. This is assigned only in TDocGenerator.BuildLinks.
\end{itemize}

Note that each ancestor is a TPasItem, \textit{not necessarily} TPasCio. Consider e.g. the case \texttt{\\\nopagebreak[3]
TMyStringList~=~Classes.TStringList;\\\nopagebreak[3]
TMyExtendedStringList~=~}\textbf{class}\texttt{(TMyStringList)\\\nopagebreak[3]
~~...\\\nopagebreak[3]
}\textbf{end}\texttt{;\\
} At least for now, such declaration will result in TPasType (not TPasCio!) with Name = 'TMyStringList', which means that ancestor of TMyExtendedStringList will be a TPasType instance.

Note that the PasDoc{\_}Parser already takes care of correctly setting Ancestors when user didn't specify any ancestor name at cio declaration. E.g. if this cio is a class, and user didn't specify ancestor name at class declaration, and this class name is not 'TObject' (in case pasdoc parses the RTL), the Ancestors[0] will be set to 'TObject'.\label{PasDoc_Items.TPasCio-Cios}
\index{Cios}
\item[\textbf{Cios}\hfill]
\ifpdf
\begin{flushleft}
\fi
\begin{ttfamily}
public property Cios: TPasNestedCios read FCios;\end{ttfamily}

\ifpdf
\end{flushleft}
\fi


\par Nested classes (and records, interfaces...).\label{PasDoc_Items.TPasCio-ClassDirective}
\index{ClassDirective}
\item[\textbf{ClassDirective}\hfill]
\ifpdf
\begin{flushleft}
\fi
\begin{ttfamily}
public property ClassDirective: TClassDirective read FClassDirective
      write FClassDirective;\end{ttfamily}

\ifpdf
\end{flushleft}
\fi


\par \begin{ttfamily}ClassDirective\end{ttfamily} is used to indicate whether a class is sealed or abstract.\label{PasDoc_Items.TPasCio-Fields}
\index{Fields}
\item[\textbf{Fields}\hfill]
\ifpdf
\begin{flushleft}
\fi
\begin{ttfamily}
public property Fields: TPasItems read FFields;\end{ttfamily}

\ifpdf
\end{flushleft}
\fi


\par list of all fields\label{PasDoc_Items.TPasCio-HelperTypeIdentifier}
\index{HelperTypeIdentifier}
\item[\textbf{HelperTypeIdentifier}\hfill]
\ifpdf
\begin{flushleft}
\fi
\begin{ttfamily}
public property HelperTypeIdentifier: string read  FHelperTypeIdentifier
                                          write FHelperTypeIdentifier;\end{ttfamily}

\ifpdf
\end{flushleft}
\fi


\par Class or record helper type identifier\label{PasDoc_Items.TPasCio-Methods}
\index{Methods}
\item[\textbf{Methods}\hfill]
\ifpdf
\begin{flushleft}
\fi
\begin{ttfamily}
public property Methods: TPasMethods read FMethods;\end{ttfamily}

\ifpdf
\end{flushleft}
\fi


\par list of all methods\label{PasDoc_Items.TPasCio-Properties}
\index{Properties}
\item[\textbf{Properties}\hfill]
\ifpdf
\begin{flushleft}
\fi
\begin{ttfamily}
public property Properties: TPasProperties read FProperties;\end{ttfamily}

\ifpdf
\end{flushleft}
\fi


\par list of properties\label{PasDoc_Items.TPasCio-MyType}
\index{MyType}
\item[\textbf{MyType}\hfill]
\ifpdf
\begin{flushleft}
\fi
\begin{ttfamily}
public property MyType: TCIOType read FMyType write FMyType;\end{ttfamily}

\ifpdf
\end{flushleft}
\fi


\par determines if this is a class, an interface or an object\label{PasDoc_Items.TPasCio-OutputFileName}
\index{OutputFileName}
\item[\textbf{OutputFileName}\hfill]
\ifpdf
\begin{flushleft}
\fi
\begin{ttfamily}
public property OutputFileName: string read FOutputFileName write FOutputFileName;\end{ttfamily}

\ifpdf
\end{flushleft}
\fi


\par name of documentation output file (if each class / object gets its own file, that's the case for HTML, but not for TeX)\label{PasDoc_Items.TPasCio-Types}
\index{Types}
\item[\textbf{Types}\hfill]
\ifpdf
\begin{flushleft}
\fi
\begin{ttfamily}
public property Types: TPasTypes read FTypes;\end{ttfamily}

\ifpdf
\end{flushleft}
\fi


\par Simple nested types (that don't fall into \begin{ttfamily}Cios\end{ttfamily}(\ref{PasDoc_Items.TPasCio-Cios})).\label{PasDoc_Items.TPasCio-NameWithGeneric}
\index{NameWithGeneric}
\item[\textbf{NameWithGeneric}\hfill]
\ifpdf
\begin{flushleft}
\fi
\begin{ttfamily}
public property NameWithGeneric: string read FNameWithGeneric write FNameWithGeneric;\end{ttfamily}

\ifpdf
\end{flushleft}
\fi


\par Name, with optional "generic" directive before (for FPC generics) and generic type identifiers list "{$<$}foo,bar{$>$}" after (for FPC and Delphi generics).\end{list}
\subsubsection*{\large{\textbf{Fields}}\normalsize\hspace{1ex}\hfill}
\begin{list}{}{
\settowidth{\tmplength}{\textbf{FHelperTypeIdentifier}}
\setlength{\itemindent}{0cm}
\setlength{\listparindent}{0cm}
\setlength{\leftmargin}{\evensidemargin}
\addtolength{\leftmargin}{\tmplength}
\settowidth{\labelsep}{X}
\addtolength{\leftmargin}{\labelsep}
\setlength{\labelwidth}{\tmplength}
}
\label{PasDoc_Items.TPasCio-FClassDirective}
\index{FClassDirective}
\item[\textbf{FClassDirective}\hfill]
\ifpdf
\begin{flushleft}
\fi
\begin{ttfamily}
protected FClassDirective: TClassDirective;\end{ttfamily}

\ifpdf
\end{flushleft}
\fi


\par  \label{PasDoc_Items.TPasCio-FFields}
\index{FFields}
\item[\textbf{FFields}\hfill]
\ifpdf
\begin{flushleft}
\fi
\begin{ttfamily}
protected FFields: TPasItems;\end{ttfamily}

\ifpdf
\end{flushleft}
\fi


\par  \label{PasDoc_Items.TPasCio-FMethods}
\index{FMethods}
\item[\textbf{FMethods}\hfill]
\ifpdf
\begin{flushleft}
\fi
\begin{ttfamily}
protected FMethods: TPasMethods;\end{ttfamily}

\ifpdf
\end{flushleft}
\fi


\par  \label{PasDoc_Items.TPasCio-FProperties}
\index{FProperties}
\item[\textbf{FProperties}\hfill]
\ifpdf
\begin{flushleft}
\fi
\begin{ttfamily}
protected FProperties: TPasProperties;\end{ttfamily}

\ifpdf
\end{flushleft}
\fi


\par  \label{PasDoc_Items.TPasCio-FAncestors}
\index{FAncestors}
\item[\textbf{FAncestors}\hfill]
\ifpdf
\begin{flushleft}
\fi
\begin{ttfamily}
protected FAncestors: TStringPairVector;\end{ttfamily}

\ifpdf
\end{flushleft}
\fi


\par  \label{PasDoc_Items.TPasCio-FOutputFileName}
\index{FOutputFileName}
\item[\textbf{FOutputFileName}\hfill]
\ifpdf
\begin{flushleft}
\fi
\begin{ttfamily}
protected FOutputFileName: string;\end{ttfamily}

\ifpdf
\end{flushleft}
\fi


\par  \label{PasDoc_Items.TPasCio-FMyType}
\index{FMyType}
\item[\textbf{FMyType}\hfill]
\ifpdf
\begin{flushleft}
\fi
\begin{ttfamily}
protected FMyType: TCIOType;\end{ttfamily}

\ifpdf
\end{flushleft}
\fi


\par  \label{PasDoc_Items.TPasCio-FHelperTypeIdentifier}
\index{FHelperTypeIdentifier}
\item[\textbf{FHelperTypeIdentifier}\hfill]
\ifpdf
\begin{flushleft}
\fi
\begin{ttfamily}
protected FHelperTypeIdentifier: string;\end{ttfamily}

\ifpdf
\end{flushleft}
\fi


\par  \label{PasDoc_Items.TPasCio-FCios}
\index{FCios}
\item[\textbf{FCios}\hfill]
\ifpdf
\begin{flushleft}
\fi
\begin{ttfamily}
protected FCios: TPasNestedCios;\end{ttfamily}

\ifpdf
\end{flushleft}
\fi


\par  \label{PasDoc_Items.TPasCio-FTypes}
\index{FTypes}
\item[\textbf{FTypes}\hfill]
\ifpdf
\begin{flushleft}
\fi
\begin{ttfamily}
protected FTypes: TPasTypes;\end{ttfamily}

\ifpdf
\end{flushleft}
\fi


\par  \label{PasDoc_Items.TPasCio-FNameWithGeneric}
\index{FNameWithGeneric}
\item[\textbf{FNameWithGeneric}\hfill]
\ifpdf
\begin{flushleft}
\fi
\begin{ttfamily}
protected FNameWithGeneric: string;\end{ttfamily}

\ifpdf
\end{flushleft}
\fi


\par  \end{list}
\subsubsection*{\large{\textbf{Methods}}\normalsize\hspace{1ex}\hfill}
\paragraph*{Serialize}\hspace*{\fill}

\label{PasDoc_Items.TPasCio-Serialize}
\index{Serialize}
\begin{list}{}{
\settowidth{\tmplength}{\textbf{Description}}
\setlength{\itemindent}{0cm}
\setlength{\listparindent}{0cm}
\setlength{\leftmargin}{\evensidemargin}
\addtolength{\leftmargin}{\tmplength}
\settowidth{\labelsep}{X}
\addtolength{\leftmargin}{\labelsep}
\setlength{\labelwidth}{\tmplength}
}
\item[\textbf{Declaration}\hfill]
\ifpdf
\begin{flushleft}
\fi
\begin{ttfamily}
protected procedure Serialize(const ADestination: TStream); override;\end{ttfamily}

\ifpdf
\end{flushleft}
\fi

\end{list}
\paragraph*{Deserialize}\hspace*{\fill}

\label{PasDoc_Items.TPasCio-Deserialize}
\index{Deserialize}
\begin{list}{}{
\settowidth{\tmplength}{\textbf{Description}}
\setlength{\itemindent}{0cm}
\setlength{\listparindent}{0cm}
\setlength{\leftmargin}{\evensidemargin}
\addtolength{\leftmargin}{\tmplength}
\settowidth{\labelsep}{X}
\addtolength{\leftmargin}{\labelsep}
\setlength{\labelwidth}{\tmplength}
}
\item[\textbf{Declaration}\hfill]
\ifpdf
\begin{flushleft}
\fi
\begin{ttfamily}
protected procedure Deserialize(const ASource: TStream); override;\end{ttfamily}

\ifpdf
\end{flushleft}
\fi

\end{list}
\paragraph*{StoreMemberTag}\hspace*{\fill}

\label{PasDoc_Items.TPasCio-StoreMemberTag}
\index{StoreMemberTag}
\begin{list}{}{
\settowidth{\tmplength}{\textbf{Description}}
\setlength{\itemindent}{0cm}
\setlength{\listparindent}{0cm}
\setlength{\leftmargin}{\evensidemargin}
\addtolength{\leftmargin}{\tmplength}
\settowidth{\labelsep}{X}
\addtolength{\leftmargin}{\labelsep}
\setlength{\labelwidth}{\tmplength}
}
\item[\textbf{Declaration}\hfill]
\ifpdf
\begin{flushleft}
\fi
\begin{ttfamily}
protected procedure StoreMemberTag(ThisTag: TTag; var ThisTagData: TObject; EnclosingTag: TTag; var EnclosingTagData: TObject; const TagParameter: string; var ReplaceStr: string);\end{ttfamily}

\ifpdf
\end{flushleft}
\fi

\end{list}
\paragraph*{Create}\hspace*{\fill}

\label{PasDoc_Items.TPasCio-Create}
\index{Create}
\begin{list}{}{
\settowidth{\tmplength}{\textbf{Description}}
\setlength{\itemindent}{0cm}
\setlength{\listparindent}{0cm}
\setlength{\leftmargin}{\evensidemargin}
\addtolength{\leftmargin}{\tmplength}
\settowidth{\labelsep}{X}
\addtolength{\leftmargin}{\labelsep}
\setlength{\labelwidth}{\tmplength}
}
\item[\textbf{Declaration}\hfill]
\ifpdf
\begin{flushleft}
\fi
\begin{ttfamily}
public constructor Create; override;\end{ttfamily}

\ifpdf
\end{flushleft}
\fi

\end{list}
\paragraph*{Destroy}\hspace*{\fill}

\label{PasDoc_Items.TPasCio-Destroy}
\index{Destroy}
\begin{list}{}{
\settowidth{\tmplength}{\textbf{Description}}
\setlength{\itemindent}{0cm}
\setlength{\listparindent}{0cm}
\setlength{\leftmargin}{\evensidemargin}
\addtolength{\leftmargin}{\tmplength}
\settowidth{\labelsep}{X}
\addtolength{\leftmargin}{\labelsep}
\setlength{\labelwidth}{\tmplength}
}
\item[\textbf{Declaration}\hfill]
\ifpdf
\begin{flushleft}
\fi
\begin{ttfamily}
public destructor Destroy; override;\end{ttfamily}

\ifpdf
\end{flushleft}
\fi

\end{list}
\paragraph*{FindItem}\hspace*{\fill}

\label{PasDoc_Items.TPasCio-FindItem}
\index{FindItem}
\begin{list}{}{
\settowidth{\tmplength}{\textbf{Description}}
\setlength{\itemindent}{0cm}
\setlength{\listparindent}{0cm}
\setlength{\leftmargin}{\evensidemargin}
\addtolength{\leftmargin}{\tmplength}
\settowidth{\labelsep}{X}
\addtolength{\leftmargin}{\labelsep}
\setlength{\labelwidth}{\tmplength}
}
\item[\textbf{Declaration}\hfill]
\ifpdf
\begin{flushleft}
\fi
\begin{ttfamily}
public function FindItem(const ItemName: string): TBaseItem; override;\end{ttfamily}

\ifpdf
\end{flushleft}
\fi

\par
\item[\textbf{Description}]
If this class (or interface or object) contains a field, method or property with the name of ItemName, the corresponding item pointer is returned.

\end{list}
\paragraph*{FindItemMaybeInAncestors}\hspace*{\fill}

\label{PasDoc_Items.TPasCio-FindItemMaybeInAncestors}
\index{FindItemMaybeInAncestors}
\begin{list}{}{
\settowidth{\tmplength}{\textbf{Description}}
\setlength{\itemindent}{0cm}
\setlength{\listparindent}{0cm}
\setlength{\leftmargin}{\evensidemargin}
\addtolength{\leftmargin}{\tmplength}
\settowidth{\labelsep}{X}
\addtolength{\leftmargin}{\labelsep}
\setlength{\labelwidth}{\tmplength}
}
\item[\textbf{Declaration}\hfill]
\ifpdf
\begin{flushleft}
\fi
\begin{ttfamily}
public function FindItemMaybeInAncestors(const ItemName: string): TBaseItem; override;\end{ttfamily}

\ifpdf
\end{flushleft}
\fi

\end{list}
\paragraph*{FindItemInAncestors}\hspace*{\fill}

\label{PasDoc_Items.TPasCio-FindItemInAncestors}
\index{FindItemInAncestors}
\begin{list}{}{
\settowidth{\tmplength}{\textbf{Description}}
\setlength{\itemindent}{0cm}
\setlength{\listparindent}{0cm}
\setlength{\leftmargin}{\evensidemargin}
\addtolength{\leftmargin}{\tmplength}
\settowidth{\labelsep}{X}
\addtolength{\leftmargin}{\labelsep}
\setlength{\labelwidth}{\tmplength}
}
\item[\textbf{Declaration}\hfill]
\ifpdf
\begin{flushleft}
\fi
\begin{ttfamily}
public function FindItemInAncestors(const ItemName: string): TPasItem;\end{ttfamily}

\ifpdf
\end{flushleft}
\fi

\par
\item[\textbf{Description}]
This searches for item (field, method or property) defined in ancestor of this cio. I.e. searches within the FirstAncestor, then within FirstAncestor.FirstAncestor, and so on. Returns nil if not found.

\end{list}
\paragraph*{Sort}\hspace*{\fill}

\label{PasDoc_Items.TPasCio-Sort}
\index{Sort}
\begin{list}{}{
\settowidth{\tmplength}{\textbf{Description}}
\setlength{\itemindent}{0cm}
\setlength{\listparindent}{0cm}
\setlength{\leftmargin}{\evensidemargin}
\addtolength{\leftmargin}{\tmplength}
\settowidth{\labelsep}{X}
\addtolength{\leftmargin}{\labelsep}
\setlength{\labelwidth}{\tmplength}
}
\item[\textbf{Declaration}\hfill]
\ifpdf
\begin{flushleft}
\fi
\begin{ttfamily}
public procedure Sort(const SortSettings: TSortSettings); override;\end{ttfamily}

\ifpdf
\end{flushleft}
\fi

\end{list}
\paragraph*{RegisterTags}\hspace*{\fill}

\label{PasDoc_Items.TPasCio-RegisterTags}
\index{RegisterTags}
\begin{list}{}{
\settowidth{\tmplength}{\textbf{Description}}
\setlength{\itemindent}{0cm}
\setlength{\listparindent}{0cm}
\setlength{\leftmargin}{\evensidemargin}
\addtolength{\leftmargin}{\tmplength}
\settowidth{\labelsep}{X}
\addtolength{\leftmargin}{\labelsep}
\setlength{\labelwidth}{\tmplength}
}
\item[\textbf{Declaration}\hfill]
\ifpdf
\begin{flushleft}
\fi
\begin{ttfamily}
public procedure RegisterTags(TagManager: TTagManager); override;\end{ttfamily}

\ifpdf
\end{flushleft}
\fi

\end{list}
\paragraph*{FirstAncestor}\hspace*{\fill}

\label{PasDoc_Items.TPasCio-FirstAncestor}
\index{FirstAncestor}
\begin{list}{}{
\settowidth{\tmplength}{\textbf{Description}}
\setlength{\itemindent}{0cm}
\setlength{\listparindent}{0cm}
\setlength{\leftmargin}{\evensidemargin}
\addtolength{\leftmargin}{\tmplength}
\settowidth{\labelsep}{X}
\addtolength{\leftmargin}{\labelsep}
\setlength{\labelwidth}{\tmplength}
}
\item[\textbf{Declaration}\hfill]
\ifpdf
\begin{flushleft}
\fi
\begin{ttfamily}
public function FirstAncestor: TPasItem;\end{ttfamily}

\ifpdf
\end{flushleft}
\fi

\par
\item[\textbf{Description}]
This returns Ancestors[0].Data, i.e. instance of the first ancestor of this Cio (or nil if it couldn't be found), or nil if Ancestors.Count = 0.

\end{list}
\paragraph*{FirstAncestorName}\hspace*{\fill}

\label{PasDoc_Items.TPasCio-FirstAncestorName}
\index{FirstAncestorName}
\begin{list}{}{
\settowidth{\tmplength}{\textbf{Description}}
\setlength{\itemindent}{0cm}
\setlength{\listparindent}{0cm}
\setlength{\leftmargin}{\evensidemargin}
\addtolength{\leftmargin}{\tmplength}
\settowidth{\labelsep}{X}
\addtolength{\leftmargin}{\labelsep}
\setlength{\labelwidth}{\tmplength}
}
\item[\textbf{Declaration}\hfill]
\ifpdf
\begin{flushleft}
\fi
\begin{ttfamily}
public function FirstAncestorName: string;\end{ttfamily}

\ifpdf
\end{flushleft}
\fi

\par
\item[\textbf{Description}]
This returns the name of first ancestor of this Cio.

If Ancestor.Count {$>$} 0 then it simply returns Ancestors[0], i.e. the name of the first ancestor as was specified at class declaration, else it returns ''.

So this method is \textit{roughly} something like \begin{ttfamily}FirstAncestor.Name\end{ttfamily}, but with a few notable differences:

\begin{itemize}
\item  FirstAncestor is nil if the ancestor was not found in items parsed by pasdoc. But this method will still return in this case name of ancestor.
\item \begin{ttfamily}FirstAncestor.Name\end{ttfamily} is the name of ancestor as specified at declaration of an ancestor. But this method is the name of ancestor as specified at declaration of this cio --- with the same letter case, with optional unit specifier.
\end{itemize}

If this function returns '', then you can be sure that FirstAncestor returns nil. The other way around is not necessarily true --- FirstAncestor may be nil, but still this function may return something {$<$}{$>$} ''.

\end{list}
\paragraph*{ShowVisibility}\hspace*{\fill}

\label{PasDoc_Items.TPasCio-ShowVisibility}
\index{ShowVisibility}
\begin{list}{}{
\settowidth{\tmplength}{\textbf{Description}}
\setlength{\itemindent}{0cm}
\setlength{\listparindent}{0cm}
\setlength{\leftmargin}{\evensidemargin}
\addtolength{\leftmargin}{\tmplength}
\settowidth{\labelsep}{X}
\addtolength{\leftmargin}{\labelsep}
\setlength{\labelwidth}{\tmplength}
}
\item[\textbf{Declaration}\hfill]
\ifpdf
\begin{flushleft}
\fi
\begin{ttfamily}
public function ShowVisibility: boolean;\end{ttfamily}

\ifpdf
\end{flushleft}
\fi

\par
\item[\textbf{Description}]
Is Visibility of items (Fields, Methods, Properties) important ?

\end{list}
\ifpdf
\subsection*{\large{\textbf{EAnchorAlreadyExists Class}}\normalsize\hspace{1ex}\hrulefill}
\else
\subsection*{EAnchorAlreadyExists Class}
\fi
\label{PasDoc_Items.EAnchorAlreadyExists}
\index{EAnchorAlreadyExists}
\subsubsection*{\large{\textbf{Hierarchy}}\normalsize\hspace{1ex}\hfill}
EAnchorAlreadyExists {$>$} Exception
%%%%Description
\ifpdf
\subsection*{\large{\textbf{TExternalItem Class}}\normalsize\hspace{1ex}\hrulefill}
\else
\subsection*{TExternalItem Class}
\fi
\label{PasDoc_Items.TExternalItem}
\index{TExternalItem}
\subsubsection*{\large{\textbf{Hierarchy}}\normalsize\hspace{1ex}\hfill}
TExternalItem {$>$} \begin{ttfamily}TBaseItem\end{ttfamily}(\ref{PasDoc_Items.TBaseItem}) {$>$} \begin{ttfamily}TSerializable\end{ttfamily}(\ref{PasDoc_Serialize.TSerializable}) {$>$} 
TObject
\subsubsection*{\large{\textbf{Description}}\normalsize\hspace{1ex}\hfill}
\begin{ttfamily}TExternalItem\end{ttfamily} extends \begin{ttfamily}TBaseItem\end{ttfamily}(\ref{PasDoc_Items.TBaseItem}) to store extra information about a project. \begin{ttfamily}TExternalItem\end{ttfamily} is used to hold an introduction and conclusion to the project.\subsubsection*{\large{\textbf{Properties}}\normalsize\hspace{1ex}\hfill}
\begin{list}{}{
\settowidth{\tmplength}{\textbf{OutputFileName}}
\setlength{\itemindent}{0cm}
\setlength{\listparindent}{0cm}
\setlength{\leftmargin}{\evensidemargin}
\addtolength{\leftmargin}{\tmplength}
\settowidth{\labelsep}{X}
\addtolength{\leftmargin}{\labelsep}
\setlength{\labelwidth}{\tmplength}
}
\label{PasDoc_Items.TExternalItem-OutputFileName}
\index{OutputFileName}
\item[\textbf{OutputFileName}\hfill]
\ifpdf
\begin{flushleft}
\fi
\begin{ttfamily}
public property OutputFileName: string read FOutputFileName write SetOutputFileName;\end{ttfamily}

\ifpdf
\end{flushleft}
\fi


\par name of documentation output file\label{PasDoc_Items.TExternalItem-ShortTitle}
\index{ShortTitle}
\item[\textbf{ShortTitle}\hfill]
\ifpdf
\begin{flushleft}
\fi
\begin{ttfamily}
public property ShortTitle: string read FShortTitle write FShortTitle;\end{ttfamily}

\ifpdf
\end{flushleft}
\fi


\par  \label{PasDoc_Items.TExternalItem-SourceFileName}
\index{SourceFileName}
\item[\textbf{SourceFileName}\hfill]
\ifpdf
\begin{flushleft}
\fi
\begin{ttfamily}
public property SourceFileName: string read FSourceFilename write FSourceFilename;\end{ttfamily}

\ifpdf
\end{flushleft}
\fi


\par  \label{PasDoc_Items.TExternalItem-Title}
\index{Title}
\item[\textbf{Title}\hfill]
\ifpdf
\begin{flushleft}
\fi
\begin{ttfamily}
public property Title: string read FTitle write FTitle;\end{ttfamily}

\ifpdf
\end{flushleft}
\fi


\par  \label{PasDoc_Items.TExternalItem-Anchors}
\index{Anchors}
\item[\textbf{Anchors}\hfill]
\ifpdf
\begin{flushleft}
\fi
\begin{ttfamily}
public property Anchors: TBaseItems read FAnchors;\end{ttfamily}

\ifpdf
\end{flushleft}
\fi


\par \begin{ttfamily}Anchors\end{ttfamily} holds a list of \begin{ttfamily}TAnchorItem\end{ttfamily}(\ref{PasDoc_Items.TAnchorItem})s that represent anchors and sections within the \begin{ttfamily}TExternalItem\end{ttfamily}. The \begin{ttfamily}TAnchorItem\end{ttfamily}(\ref{PasDoc_Items.TAnchorItem})s have no content so, they should not be indexed separately.\end{list}
\subsubsection*{\large{\textbf{Methods}}\normalsize\hspace{1ex}\hfill}
\paragraph*{HandleTitleTag}\hspace*{\fill}

\label{PasDoc_Items.TExternalItem-HandleTitleTag}
\index{HandleTitleTag}
\begin{list}{}{
\settowidth{\tmplength}{\textbf{Description}}
\setlength{\itemindent}{0cm}
\setlength{\listparindent}{0cm}
\setlength{\leftmargin}{\evensidemargin}
\addtolength{\leftmargin}{\tmplength}
\settowidth{\labelsep}{X}
\addtolength{\leftmargin}{\labelsep}
\setlength{\labelwidth}{\tmplength}
}
\item[\textbf{Declaration}\hfill]
\ifpdf
\begin{flushleft}
\fi
\begin{ttfamily}
protected procedure HandleTitleTag(ThisTag: TTag; var ThisTagData: TObject; EnclosingTag: TTag; var EnclosingTagData: TObject; const TagParameter: string; var ReplaceStr: string);\end{ttfamily}

\ifpdf
\end{flushleft}
\fi

\end{list}
\paragraph*{HandleShortTitleTag}\hspace*{\fill}

\label{PasDoc_Items.TExternalItem-HandleShortTitleTag}
\index{HandleShortTitleTag}
\begin{list}{}{
\settowidth{\tmplength}{\textbf{Description}}
\setlength{\itemindent}{0cm}
\setlength{\listparindent}{0cm}
\setlength{\leftmargin}{\evensidemargin}
\addtolength{\leftmargin}{\tmplength}
\settowidth{\labelsep}{X}
\addtolength{\leftmargin}{\labelsep}
\setlength{\labelwidth}{\tmplength}
}
\item[\textbf{Declaration}\hfill]
\ifpdf
\begin{flushleft}
\fi
\begin{ttfamily}
protected procedure HandleShortTitleTag(ThisTag: TTag; var ThisTagData: TObject; EnclosingTag: TTag; var EnclosingTagData: TObject; const TagParameter: string; var ReplaceStr: string);\end{ttfamily}

\ifpdf
\end{flushleft}
\fi

\end{list}
\paragraph*{Create}\hspace*{\fill}

\label{PasDoc_Items.TExternalItem-Create}
\index{Create}
\begin{list}{}{
\settowidth{\tmplength}{\textbf{Description}}
\setlength{\itemindent}{0cm}
\setlength{\listparindent}{0cm}
\setlength{\leftmargin}{\evensidemargin}
\addtolength{\leftmargin}{\tmplength}
\settowidth{\labelsep}{X}
\addtolength{\leftmargin}{\labelsep}
\setlength{\labelwidth}{\tmplength}
}
\item[\textbf{Declaration}\hfill]
\ifpdf
\begin{flushleft}
\fi
\begin{ttfamily}
public Constructor Create; override;\end{ttfamily}

\ifpdf
\end{flushleft}
\fi

\end{list}
\paragraph*{Destroy}\hspace*{\fill}

\label{PasDoc_Items.TExternalItem-Destroy}
\index{Destroy}
\begin{list}{}{
\settowidth{\tmplength}{\textbf{Description}}
\setlength{\itemindent}{0cm}
\setlength{\listparindent}{0cm}
\setlength{\leftmargin}{\evensidemargin}
\addtolength{\leftmargin}{\tmplength}
\settowidth{\labelsep}{X}
\addtolength{\leftmargin}{\labelsep}
\setlength{\labelwidth}{\tmplength}
}
\item[\textbf{Declaration}\hfill]
\ifpdf
\begin{flushleft}
\fi
\begin{ttfamily}
public destructor Destroy; override;\end{ttfamily}

\ifpdf
\end{flushleft}
\fi

\end{list}
\paragraph*{RegisterTags}\hspace*{\fill}

\label{PasDoc_Items.TExternalItem-RegisterTags}
\index{RegisterTags}
\begin{list}{}{
\settowidth{\tmplength}{\textbf{Description}}
\setlength{\itemindent}{0cm}
\setlength{\listparindent}{0cm}
\setlength{\leftmargin}{\evensidemargin}
\addtolength{\leftmargin}{\tmplength}
\settowidth{\labelsep}{X}
\addtolength{\leftmargin}{\labelsep}
\setlength{\labelwidth}{\tmplength}
}
\item[\textbf{Declaration}\hfill]
\ifpdf
\begin{flushleft}
\fi
\begin{ttfamily}
public procedure RegisterTags(TagManager: TTagManager); override;\end{ttfamily}

\ifpdf
\end{flushleft}
\fi

\end{list}
\paragraph*{FindItem}\hspace*{\fill}

\label{PasDoc_Items.TExternalItem-FindItem}
\index{FindItem}
\begin{list}{}{
\settowidth{\tmplength}{\textbf{Description}}
\setlength{\itemindent}{0cm}
\setlength{\listparindent}{0cm}
\setlength{\leftmargin}{\evensidemargin}
\addtolength{\leftmargin}{\tmplength}
\settowidth{\labelsep}{X}
\addtolength{\leftmargin}{\labelsep}
\setlength{\labelwidth}{\tmplength}
}
\item[\textbf{Declaration}\hfill]
\ifpdf
\begin{flushleft}
\fi
\begin{ttfamily}
public function FindItem(const ItemName: string): TBaseItem; override;\end{ttfamily}

\ifpdf
\end{flushleft}
\fi

\end{list}
\paragraph*{AddAnchor}\hspace*{\fill}

\label{PasDoc_Items.TExternalItem-AddAnchor}
\index{AddAnchor}
\begin{list}{}{
\settowidth{\tmplength}{\textbf{Description}}
\setlength{\itemindent}{0cm}
\setlength{\listparindent}{0cm}
\setlength{\leftmargin}{\evensidemargin}
\addtolength{\leftmargin}{\tmplength}
\settowidth{\labelsep}{X}
\addtolength{\leftmargin}{\labelsep}
\setlength{\labelwidth}{\tmplength}
}
\item[\textbf{Declaration}\hfill]
\ifpdf
\begin{flushleft}
\fi
\begin{ttfamily}
public procedure AddAnchor(const AnchorItem: TAnchorItem); overload;\end{ttfamily}

\ifpdf
\end{flushleft}
\fi

\end{list}
\paragraph*{AddAnchor}\hspace*{\fill}

\label{PasDoc_Items.TExternalItem-AddAnchor}
\index{AddAnchor}
\begin{list}{}{
\settowidth{\tmplength}{\textbf{Description}}
\setlength{\itemindent}{0cm}
\setlength{\listparindent}{0cm}
\setlength{\leftmargin}{\evensidemargin}
\addtolength{\leftmargin}{\tmplength}
\settowidth{\labelsep}{X}
\addtolength{\leftmargin}{\labelsep}
\setlength{\labelwidth}{\tmplength}
}
\item[\textbf{Declaration}\hfill]
\ifpdf
\begin{flushleft}
\fi
\begin{ttfamily}
public function AddAnchor(const AnchorName: string): TAnchorItem; overload;\end{ttfamily}

\ifpdf
\end{flushleft}
\fi

\par
\item[\textbf{Description}]
If item with Name (case ignored) already exists, this raises exception EAnchorAlreadyExists. Otherwise it adds TAnchorItem with given name to Anchors. It also returns created TAnchorItem.

\end{list}
\paragraph*{BasePath}\hspace*{\fill}

\label{PasDoc_Items.TExternalItem-BasePath}
\index{BasePath}
\begin{list}{}{
\settowidth{\tmplength}{\textbf{Description}}
\setlength{\itemindent}{0cm}
\setlength{\listparindent}{0cm}
\setlength{\leftmargin}{\evensidemargin}
\addtolength{\leftmargin}{\tmplength}
\settowidth{\labelsep}{X}
\addtolength{\leftmargin}{\labelsep}
\setlength{\labelwidth}{\tmplength}
}
\item[\textbf{Declaration}\hfill]
\ifpdf
\begin{flushleft}
\fi
\begin{ttfamily}
public function BasePath: string; override;\end{ttfamily}

\ifpdf
\end{flushleft}
\fi

\end{list}
\ifpdf
\subsection*{\large{\textbf{TExternalItemList Class}}\normalsize\hspace{1ex}\hrulefill}
\else
\subsection*{TExternalItemList Class}
\fi
\label{PasDoc_Items.TExternalItemList}
\index{TExternalItemList}
\subsubsection*{\large{\textbf{Hierarchy}}\normalsize\hspace{1ex}\hfill}
TExternalItemList {$>$} \begin{ttfamily}TObjectVector\end{ttfamily}(\ref{PasDoc_ObjectVector.TObjectVector}) {$>$} 
TObjectList
\subsubsection*{\large{\textbf{Description}}\normalsize\hspace{1ex}\hfill}
\begin{ttfamily}TExternalItemList\end{ttfamily} extends \begin{ttfamily}TObjectVector\end{ttfamily}(\ref{PasDoc_ObjectVector.TObjectVector}) to store non{-}nil instances of \begin{ttfamily}TExternalItem\end{ttfamily}(\ref{PasDoc_Items.TExternalItem})\subsubsection*{\large{\textbf{Methods}}\normalsize\hspace{1ex}\hfill}
\paragraph*{Get}\hspace*{\fill}

\label{PasDoc_Items.TExternalItemList-Get}
\index{Get}
\begin{list}{}{
\settowidth{\tmplength}{\textbf{Description}}
\setlength{\itemindent}{0cm}
\setlength{\listparindent}{0cm}
\setlength{\leftmargin}{\evensidemargin}
\addtolength{\leftmargin}{\tmplength}
\settowidth{\labelsep}{X}
\addtolength{\leftmargin}{\labelsep}
\setlength{\labelwidth}{\tmplength}
}
\item[\textbf{Declaration}\hfill]
\ifpdf
\begin{flushleft}
\fi
\begin{ttfamily}
public function Get(Index: Integer): TExternalItem;\end{ttfamily}

\ifpdf
\end{flushleft}
\fi

\end{list}
\ifpdf
\subsection*{\large{\textbf{TAnchorItem Class}}\normalsize\hspace{1ex}\hrulefill}
\else
\subsection*{TAnchorItem Class}
\fi
\label{PasDoc_Items.TAnchorItem}
\index{TAnchorItem}
\subsubsection*{\large{\textbf{Hierarchy}}\normalsize\hspace{1ex}\hfill}
TAnchorItem {$>$} \begin{ttfamily}TBaseItem\end{ttfamily}(\ref{PasDoc_Items.TBaseItem}) {$>$} \begin{ttfamily}TSerializable\end{ttfamily}(\ref{PasDoc_Serialize.TSerializable}) {$>$} 
TObject
\subsubsection*{\large{\textbf{Description}}\normalsize\hspace{1ex}\hfill}
no description available, TBaseItem description followsThis is a basic item class, that is linkable, and has some \begin{ttfamily}RawDescription\end{ttfamily}(\ref{PasDoc_Items.TBaseItem-RawDescription}).\subsubsection*{\large{\textbf{Properties}}\normalsize\hspace{1ex}\hfill}
\begin{list}{}{
\settowidth{\tmplength}{\textbf{SectionCaption}}
\setlength{\itemindent}{0cm}
\setlength{\listparindent}{0cm}
\setlength{\leftmargin}{\evensidemargin}
\addtolength{\leftmargin}{\tmplength}
\settowidth{\labelsep}{X}
\addtolength{\leftmargin}{\labelsep}
\setlength{\labelwidth}{\tmplength}
}
\label{PasDoc_Items.TAnchorItem-ExternalItem}
\index{ExternalItem}
\item[\textbf{ExternalItem}\hfill]
\ifpdf
\begin{flushleft}
\fi
\begin{ttfamily}
public property ExternalItem: TExternalItem read FExternalItem write FExternalItem;\end{ttfamily}

\ifpdf
\end{flushleft}
\fi


\par  \label{PasDoc_Items.TAnchorItem-SectionLevel}
\index{SectionLevel}
\item[\textbf{SectionLevel}\hfill]
\ifpdf
\begin{flushleft}
\fi
\begin{ttfamily}
public property SectionLevel: Integer
      read FSectionLevel write FSectionLevel default 0;\end{ttfamily}

\ifpdf
\end{flushleft}
\fi


\par If this is an anchor for a section, this tells section level (as was specified in the @section tag). Otherwise this is 0.\label{PasDoc_Items.TAnchorItem-SectionCaption}
\index{SectionCaption}
\item[\textbf{SectionCaption}\hfill]
\ifpdf
\begin{flushleft}
\fi
\begin{ttfamily}
public property SectionCaption: string
      read FSectionCaption write FSectionCaption;\end{ttfamily}

\ifpdf
\end{flushleft}
\fi


\par If this is an anchor for a section, this tells section caption (as was specified in the @section tag).\end{list}
\ifpdf
\subsection*{\large{\textbf{TPasUnit Class}}\normalsize\hspace{1ex}\hrulefill}
\else
\subsection*{TPasUnit Class}
\fi
\label{PasDoc_Items.TPasUnit}
\index{TPasUnit}
\subsubsection*{\large{\textbf{Hierarchy}}\normalsize\hspace{1ex}\hfill}
TPasUnit {$>$} \begin{ttfamily}TPasItem\end{ttfamily}(\ref{PasDoc_Items.TPasItem}) {$>$} \begin{ttfamily}TBaseItem\end{ttfamily}(\ref{PasDoc_Items.TBaseItem}) {$>$} \begin{ttfamily}TSerializable\end{ttfamily}(\ref{PasDoc_Serialize.TSerializable}) {$>$} 
TObject
\subsubsection*{\large{\textbf{Description}}\normalsize\hspace{1ex}\hfill}
extends \begin{ttfamily}TPasItem\end{ttfamily}(\ref{PasDoc_Items.TPasItem}) to store anything about a unit, its constants, types etc.; also provides methods for parsing a complete unit.

Note: Remember to always set \begin{ttfamily}CacheDateTime\end{ttfamily}(\ref{PasDoc_Items.TPasUnit-CacheDateTime}) after deserializing this unit.\subsubsection*{\large{\textbf{Properties}}\normalsize\hspace{1ex}\hfill}
\begin{list}{}{
\settowidth{\tmplength}{\textbf{SourceFileDateTime}}
\setlength{\itemindent}{0cm}
\setlength{\listparindent}{0cm}
\setlength{\leftmargin}{\evensidemargin}
\addtolength{\leftmargin}{\tmplength}
\settowidth{\labelsep}{X}
\addtolength{\leftmargin}{\labelsep}
\setlength{\labelwidth}{\tmplength}
}
\label{PasDoc_Items.TPasUnit-CIOs}
\index{CIOs}
\item[\textbf{CIOs}\hfill]
\ifpdf
\begin{flushleft}
\fi
\begin{ttfamily}
public property CIOs: TPasItems read FCIOs;\end{ttfamily}

\ifpdf
\end{flushleft}
\fi


\par list of classes, interfaces, objects, and records defined in this unit\label{PasDoc_Items.TPasUnit-Constants}
\index{Constants}
\item[\textbf{Constants}\hfill]
\ifpdf
\begin{flushleft}
\fi
\begin{ttfamily}
public property Constants: TPasItems read FConstants;\end{ttfamily}

\ifpdf
\end{flushleft}
\fi


\par list of constants defined in this unit\label{PasDoc_Items.TPasUnit-FuncsProcs}
\index{FuncsProcs}
\item[\textbf{FuncsProcs}\hfill]
\ifpdf
\begin{flushleft}
\fi
\begin{ttfamily}
public property FuncsProcs: TPasMethods read FFuncsProcs;\end{ttfamily}

\ifpdf
\end{flushleft}
\fi


\par list of functions and procedures defined in this unit\label{PasDoc_Items.TPasUnit-UsesUnits}
\index{UsesUnits}
\item[\textbf{UsesUnits}\hfill]
\ifpdf
\begin{flushleft}
\fi
\begin{ttfamily}
public property UsesUnits: TStringVector read FUsesUnits;\end{ttfamily}

\ifpdf
\end{flushleft}
\fi


\par The names of all units mentioned in a uses clause in the interface section of this unit.

This is never nil.

After \begin{ttfamily}TDocGenerator.BuildLinks\end{ttfamily}(\ref{PasDoc_Gen.TDocGenerator-BuildLinks}), for every i: UsesUnits.Objects[i] will point to TPasUnit object with Name = UsesUnits[i] (or nil, if pasdoc's didn't parse such unit). In other words, you will be able to use UsesUnits.Objects[i] to obtain given unit's instance, as parsed by pasdoc.\label{PasDoc_Items.TPasUnit-Types}
\index{Types}
\item[\textbf{Types}\hfill]
\ifpdf
\begin{flushleft}
\fi
\begin{ttfamily}
public property Types: TPasTypes read FTypes;\end{ttfamily}

\ifpdf
\end{flushleft}
\fi


\par list of types defined in this unit\label{PasDoc_Items.TPasUnit-Variables}
\index{Variables}
\item[\textbf{Variables}\hfill]
\ifpdf
\begin{flushleft}
\fi
\begin{ttfamily}
public property Variables: TPasItems read FVariables;\end{ttfamily}

\ifpdf
\end{flushleft}
\fi


\par list of variables defined in this unit\label{PasDoc_Items.TPasUnit-OutputFileName}
\index{OutputFileName}
\item[\textbf{OutputFileName}\hfill]
\ifpdf
\begin{flushleft}
\fi
\begin{ttfamily}
public property OutputFileName: string read FOutputFileName write FOutputFileName;\end{ttfamily}

\ifpdf
\end{flushleft}
\fi


\par name of documentation output file THIS SHOULD NOT BE HERE!\label{PasDoc_Items.TPasUnit-SourceFileName}
\index{SourceFileName}
\item[\textbf{SourceFileName}\hfill]
\ifpdf
\begin{flushleft}
\fi
\begin{ttfamily}
public property SourceFileName: string read FSourceFilename write FSourceFilename;\end{ttfamily}

\ifpdf
\end{flushleft}
\fi


\par  \label{PasDoc_Items.TPasUnit-SourceFileDateTime}
\index{SourceFileDateTime}
\item[\textbf{SourceFileDateTime}\hfill]
\ifpdf
\begin{flushleft}
\fi
\begin{ttfamily}
public property SourceFileDateTime: TDateTime
      read FSourceFileDateTime write FSourceFileDateTime;\end{ttfamily}

\ifpdf
\end{flushleft}
\fi


\par  \label{PasDoc_Items.TPasUnit-CacheDateTime}
\index{CacheDateTime}
\item[\textbf{CacheDateTime}\hfill]
\ifpdf
\begin{flushleft}
\fi
\begin{ttfamily}
public property CacheDateTime: TDateTime
      read FCacheDateTime write FCacheDateTime;\end{ttfamily}

\ifpdf
\end{flushleft}
\fi


\par If WasDeserialized then this specifies the datetime of a cache data of this unit, i.e. when cache data was generated. If cache was obtained from a file then this is just the cache file modification date/time.

If not WasDeserialized then this property has undefined value -- don't use it.\label{PasDoc_Items.TPasUnit-IsUnit}
\index{IsUnit}
\item[\textbf{IsUnit}\hfill]
\ifpdf
\begin{flushleft}
\fi
\begin{ttfamily}
public property IsUnit: boolean read FIsUnit write FIsUnit;\end{ttfamily}

\ifpdf
\end{flushleft}
\fi


\par If \begin{ttfamily}False\end{ttfamily}, then this is a program or library file, not a regular unit (though it's treated by pasdoc almost like a unit, so we use TPasUnit class for this).\label{PasDoc_Items.TPasUnit-IsProgram}
\index{IsProgram}
\item[\textbf{IsProgram}\hfill]
\ifpdf
\begin{flushleft}
\fi
\begin{ttfamily}
public property IsProgram: boolean read FIsProgram write FIsProgram;\end{ttfamily}

\ifpdf
\end{flushleft}
\fi


\par  \end{list}
\subsubsection*{\large{\textbf{Fields}}\normalsize\hspace{1ex}\hfill}
\begin{list}{}{
\settowidth{\tmplength}{\textbf{FSourceFileDateTime}}
\setlength{\itemindent}{0cm}
\setlength{\listparindent}{0cm}
\setlength{\leftmargin}{\evensidemargin}
\addtolength{\leftmargin}{\tmplength}
\settowidth{\labelsep}{X}
\addtolength{\leftmargin}{\labelsep}
\setlength{\labelwidth}{\tmplength}
}
\label{PasDoc_Items.TPasUnit-FTypes}
\index{FTypes}
\item[\textbf{FTypes}\hfill]
\ifpdf
\begin{flushleft}
\fi
\begin{ttfamily}
protected FTypes: TPasTypes;\end{ttfamily}

\ifpdf
\end{flushleft}
\fi


\par  \label{PasDoc_Items.TPasUnit-FVariables}
\index{FVariables}
\item[\textbf{FVariables}\hfill]
\ifpdf
\begin{flushleft}
\fi
\begin{ttfamily}
protected FVariables: TPasItems;\end{ttfamily}

\ifpdf
\end{flushleft}
\fi


\par  \label{PasDoc_Items.TPasUnit-FCIOs}
\index{FCIOs}
\item[\textbf{FCIOs}\hfill]
\ifpdf
\begin{flushleft}
\fi
\begin{ttfamily}
protected FCIOs: TPasItems;\end{ttfamily}

\ifpdf
\end{flushleft}
\fi


\par  \label{PasDoc_Items.TPasUnit-FConstants}
\index{FConstants}
\item[\textbf{FConstants}\hfill]
\ifpdf
\begin{flushleft}
\fi
\begin{ttfamily}
protected FConstants: TPasItems;\end{ttfamily}

\ifpdf
\end{flushleft}
\fi


\par  \label{PasDoc_Items.TPasUnit-FFuncsProcs}
\index{FFuncsProcs}
\item[\textbf{FFuncsProcs}\hfill]
\ifpdf
\begin{flushleft}
\fi
\begin{ttfamily}
protected FFuncsProcs: TPasMethods;\end{ttfamily}

\ifpdf
\end{flushleft}
\fi


\par  \label{PasDoc_Items.TPasUnit-FUsesUnits}
\index{FUsesUnits}
\item[\textbf{FUsesUnits}\hfill]
\ifpdf
\begin{flushleft}
\fi
\begin{ttfamily}
protected FUsesUnits: TStringVector;\end{ttfamily}

\ifpdf
\end{flushleft}
\fi


\par  \label{PasDoc_Items.TPasUnit-FSourceFilename}
\index{FSourceFilename}
\item[\textbf{FSourceFilename}\hfill]
\ifpdf
\begin{flushleft}
\fi
\begin{ttfamily}
protected FSourceFilename: string;\end{ttfamily}

\ifpdf
\end{flushleft}
\fi


\par  \label{PasDoc_Items.TPasUnit-FOutputFileName}
\index{FOutputFileName}
\item[\textbf{FOutputFileName}\hfill]
\ifpdf
\begin{flushleft}
\fi
\begin{ttfamily}
protected FOutputFileName: string;\end{ttfamily}

\ifpdf
\end{flushleft}
\fi


\par  \label{PasDoc_Items.TPasUnit-FCacheDateTime}
\index{FCacheDateTime}
\item[\textbf{FCacheDateTime}\hfill]
\ifpdf
\begin{flushleft}
\fi
\begin{ttfamily}
protected FCacheDateTime: TDateTime;\end{ttfamily}

\ifpdf
\end{flushleft}
\fi


\par  \label{PasDoc_Items.TPasUnit-FSourceFileDateTime}
\index{FSourceFileDateTime}
\item[\textbf{FSourceFileDateTime}\hfill]
\ifpdf
\begin{flushleft}
\fi
\begin{ttfamily}
protected FSourceFileDateTime: TDateTime;\end{ttfamily}

\ifpdf
\end{flushleft}
\fi


\par  \label{PasDoc_Items.TPasUnit-FIsUnit}
\index{FIsUnit}
\item[\textbf{FIsUnit}\hfill]
\ifpdf
\begin{flushleft}
\fi
\begin{ttfamily}
protected FIsUnit: boolean;\end{ttfamily}

\ifpdf
\end{flushleft}
\fi


\par  \label{PasDoc_Items.TPasUnit-FIsProgram}
\index{FIsProgram}
\item[\textbf{FIsProgram}\hfill]
\ifpdf
\begin{flushleft}
\fi
\begin{ttfamily}
protected FIsProgram: boolean;\end{ttfamily}

\ifpdf
\end{flushleft}
\fi


\par  \end{list}
\subsubsection*{\large{\textbf{Methods}}\normalsize\hspace{1ex}\hfill}
\paragraph*{Serialize}\hspace*{\fill}

\label{PasDoc_Items.TPasUnit-Serialize}
\index{Serialize}
\begin{list}{}{
\settowidth{\tmplength}{\textbf{Description}}
\setlength{\itemindent}{0cm}
\setlength{\listparindent}{0cm}
\setlength{\leftmargin}{\evensidemargin}
\addtolength{\leftmargin}{\tmplength}
\settowidth{\labelsep}{X}
\addtolength{\leftmargin}{\labelsep}
\setlength{\labelwidth}{\tmplength}
}
\item[\textbf{Declaration}\hfill]
\ifpdf
\begin{flushleft}
\fi
\begin{ttfamily}
protected procedure Serialize(const ADestination: TStream); override;\end{ttfamily}

\ifpdf
\end{flushleft}
\fi

\end{list}
\paragraph*{Deserialize}\hspace*{\fill}

\label{PasDoc_Items.TPasUnit-Deserialize}
\index{Deserialize}
\begin{list}{}{
\settowidth{\tmplength}{\textbf{Description}}
\setlength{\itemindent}{0cm}
\setlength{\listparindent}{0cm}
\setlength{\leftmargin}{\evensidemargin}
\addtolength{\leftmargin}{\tmplength}
\settowidth{\labelsep}{X}
\addtolength{\leftmargin}{\labelsep}
\setlength{\labelwidth}{\tmplength}
}
\item[\textbf{Declaration}\hfill]
\ifpdf
\begin{flushleft}
\fi
\begin{ttfamily}
protected procedure Deserialize(const ASource: TStream); override;\end{ttfamily}

\ifpdf
\end{flushleft}
\fi

\end{list}
\paragraph*{Create}\hspace*{\fill}

\label{PasDoc_Items.TPasUnit-Create}
\index{Create}
\begin{list}{}{
\settowidth{\tmplength}{\textbf{Description}}
\setlength{\itemindent}{0cm}
\setlength{\listparindent}{0cm}
\setlength{\leftmargin}{\evensidemargin}
\addtolength{\leftmargin}{\tmplength}
\settowidth{\labelsep}{X}
\addtolength{\leftmargin}{\labelsep}
\setlength{\labelwidth}{\tmplength}
}
\item[\textbf{Declaration}\hfill]
\ifpdf
\begin{flushleft}
\fi
\begin{ttfamily}
public constructor Create; override;\end{ttfamily}

\ifpdf
\end{flushleft}
\fi

\end{list}
\paragraph*{Destroy}\hspace*{\fill}

\label{PasDoc_Items.TPasUnit-Destroy}
\index{Destroy}
\begin{list}{}{
\settowidth{\tmplength}{\textbf{Description}}
\setlength{\itemindent}{0cm}
\setlength{\listparindent}{0cm}
\setlength{\leftmargin}{\evensidemargin}
\addtolength{\leftmargin}{\tmplength}
\settowidth{\labelsep}{X}
\addtolength{\leftmargin}{\labelsep}
\setlength{\labelwidth}{\tmplength}
}
\item[\textbf{Declaration}\hfill]
\ifpdf
\begin{flushleft}
\fi
\begin{ttfamily}
public destructor Destroy; override;\end{ttfamily}

\ifpdf
\end{flushleft}
\fi

\end{list}
\paragraph*{AddCIO}\hspace*{\fill}

\label{PasDoc_Items.TPasUnit-AddCIO}
\index{AddCIO}
\begin{list}{}{
\settowidth{\tmplength}{\textbf{Description}}
\setlength{\itemindent}{0cm}
\setlength{\listparindent}{0cm}
\setlength{\leftmargin}{\evensidemargin}
\addtolength{\leftmargin}{\tmplength}
\settowidth{\labelsep}{X}
\addtolength{\leftmargin}{\labelsep}
\setlength{\labelwidth}{\tmplength}
}
\item[\textbf{Declaration}\hfill]
\ifpdf
\begin{flushleft}
\fi
\begin{ttfamily}
public procedure AddCIO(const i: TPasCio);\end{ttfamily}

\ifpdf
\end{flushleft}
\fi

\end{list}
\paragraph*{AddConstant}\hspace*{\fill}

\label{PasDoc_Items.TPasUnit-AddConstant}
\index{AddConstant}
\begin{list}{}{
\settowidth{\tmplength}{\textbf{Description}}
\setlength{\itemindent}{0cm}
\setlength{\listparindent}{0cm}
\setlength{\leftmargin}{\evensidemargin}
\addtolength{\leftmargin}{\tmplength}
\settowidth{\labelsep}{X}
\addtolength{\leftmargin}{\labelsep}
\setlength{\labelwidth}{\tmplength}
}
\item[\textbf{Declaration}\hfill]
\ifpdf
\begin{flushleft}
\fi
\begin{ttfamily}
public procedure AddConstant(const i: TPasItem);\end{ttfamily}

\ifpdf
\end{flushleft}
\fi

\end{list}
\paragraph*{AddType}\hspace*{\fill}

\label{PasDoc_Items.TPasUnit-AddType}
\index{AddType}
\begin{list}{}{
\settowidth{\tmplength}{\textbf{Description}}
\setlength{\itemindent}{0cm}
\setlength{\listparindent}{0cm}
\setlength{\leftmargin}{\evensidemargin}
\addtolength{\leftmargin}{\tmplength}
\settowidth{\labelsep}{X}
\addtolength{\leftmargin}{\labelsep}
\setlength{\labelwidth}{\tmplength}
}
\item[\textbf{Declaration}\hfill]
\ifpdf
\begin{flushleft}
\fi
\begin{ttfamily}
public procedure AddType(const i: TPasItem);\end{ttfamily}

\ifpdf
\end{flushleft}
\fi

\end{list}
\paragraph*{AddVariable}\hspace*{\fill}

\label{PasDoc_Items.TPasUnit-AddVariable}
\index{AddVariable}
\begin{list}{}{
\settowidth{\tmplength}{\textbf{Description}}
\setlength{\itemindent}{0cm}
\setlength{\listparindent}{0cm}
\setlength{\leftmargin}{\evensidemargin}
\addtolength{\leftmargin}{\tmplength}
\settowidth{\labelsep}{X}
\addtolength{\leftmargin}{\labelsep}
\setlength{\labelwidth}{\tmplength}
}
\item[\textbf{Declaration}\hfill]
\ifpdf
\begin{flushleft}
\fi
\begin{ttfamily}
public procedure AddVariable(const i: TPasItem);\end{ttfamily}

\ifpdf
\end{flushleft}
\fi

\end{list}
\paragraph*{FindInsideSomeClass}\hspace*{\fill}

\label{PasDoc_Items.TPasUnit-FindInsideSomeClass}
\index{FindInsideSomeClass}
\begin{list}{}{
\settowidth{\tmplength}{\textbf{Description}}
\setlength{\itemindent}{0cm}
\setlength{\listparindent}{0cm}
\setlength{\leftmargin}{\evensidemargin}
\addtolength{\leftmargin}{\tmplength}
\settowidth{\labelsep}{X}
\addtolength{\leftmargin}{\labelsep}
\setlength{\labelwidth}{\tmplength}
}
\item[\textbf{Declaration}\hfill]
\ifpdf
\begin{flushleft}
\fi
\begin{ttfamily}
public function FindInsideSomeClass(const AClassName, ItemInsideClass: string): TPasItem;\end{ttfamily}

\ifpdf
\end{flushleft}
\fi

\end{list}
\paragraph*{FindInsideSomeEnum}\hspace*{\fill}

\label{PasDoc_Items.TPasUnit-FindInsideSomeEnum}
\index{FindInsideSomeEnum}
\begin{list}{}{
\settowidth{\tmplength}{\textbf{Description}}
\setlength{\itemindent}{0cm}
\setlength{\listparindent}{0cm}
\setlength{\leftmargin}{\evensidemargin}
\addtolength{\leftmargin}{\tmplength}
\settowidth{\labelsep}{X}
\addtolength{\leftmargin}{\labelsep}
\setlength{\labelwidth}{\tmplength}
}
\item[\textbf{Declaration}\hfill]
\ifpdf
\begin{flushleft}
\fi
\begin{ttfamily}
public function FindInsideSomeEnum(const EnumName, EnumMember: string): TPasItem;\end{ttfamily}

\ifpdf
\end{flushleft}
\fi

\end{list}
\paragraph*{FindItem}\hspace*{\fill}

\label{PasDoc_Items.TPasUnit-FindItem}
\index{FindItem}
\begin{list}{}{
\settowidth{\tmplength}{\textbf{Description}}
\setlength{\itemindent}{0cm}
\setlength{\listparindent}{0cm}
\setlength{\leftmargin}{\evensidemargin}
\addtolength{\leftmargin}{\tmplength}
\settowidth{\labelsep}{X}
\addtolength{\leftmargin}{\labelsep}
\setlength{\labelwidth}{\tmplength}
}
\item[\textbf{Declaration}\hfill]
\ifpdf
\begin{flushleft}
\fi
\begin{ttfamily}
public function FindItem(const ItemName: string): TBaseItem; override;\end{ttfamily}

\ifpdf
\end{flushleft}
\fi

\end{list}
\paragraph*{Sort}\hspace*{\fill}

\label{PasDoc_Items.TPasUnit-Sort}
\index{Sort}
\begin{list}{}{
\settowidth{\tmplength}{\textbf{Description}}
\setlength{\itemindent}{0cm}
\setlength{\listparindent}{0cm}
\setlength{\leftmargin}{\evensidemargin}
\addtolength{\leftmargin}{\tmplength}
\settowidth{\labelsep}{X}
\addtolength{\leftmargin}{\labelsep}
\setlength{\labelwidth}{\tmplength}
}
\item[\textbf{Declaration}\hfill]
\ifpdf
\begin{flushleft}
\fi
\begin{ttfamily}
public procedure Sort(const SortSettings: TSortSettings); override;\end{ttfamily}

\ifpdf
\end{flushleft}
\fi

\end{list}
\paragraph*{FileNewerThanCache}\hspace*{\fill}

\label{PasDoc_Items.TPasUnit-FileNewerThanCache}
\index{FileNewerThanCache}
\begin{list}{}{
\settowidth{\tmplength}{\textbf{Description}}
\setlength{\itemindent}{0cm}
\setlength{\listparindent}{0cm}
\setlength{\leftmargin}{\evensidemargin}
\addtolength{\leftmargin}{\tmplength}
\settowidth{\labelsep}{X}
\addtolength{\leftmargin}{\labelsep}
\setlength{\labelwidth}{\tmplength}
}
\item[\textbf{Declaration}\hfill]
\ifpdf
\begin{flushleft}
\fi
\begin{ttfamily}
public function FileNewerThanCache(const FileName: string): boolean;\end{ttfamily}

\ifpdf
\end{flushleft}
\fi

\par
\item[\textbf{Description}]
Returns if unit WasDeserialized, and file FileName exists, and file FileName is newer than CacheDateTime.

So if FileName contains some info generated from information of this unit, then we can somehow assume that FileName still contains valid information and we don't have to write it once again.

Sure, we're not really 100{\%} sure that FileName still contains valid information, but that's how current approach to cache works.

\end{list}
\paragraph*{BasePath}\hspace*{\fill}

\label{PasDoc_Items.TPasUnit-BasePath}
\index{BasePath}
\begin{list}{}{
\settowidth{\tmplength}{\textbf{Description}}
\setlength{\itemindent}{0cm}
\setlength{\listparindent}{0cm}
\setlength{\leftmargin}{\evensidemargin}
\addtolength{\leftmargin}{\tmplength}
\settowidth{\labelsep}{X}
\addtolength{\leftmargin}{\labelsep}
\setlength{\labelwidth}{\tmplength}
}
\item[\textbf{Declaration}\hfill]
\ifpdf
\begin{flushleft}
\fi
\begin{ttfamily}
public function BasePath: string; override;\end{ttfamily}

\ifpdf
\end{flushleft}
\fi

\end{list}
\ifpdf
\subsection*{\large{\textbf{TBaseItems Class}}\normalsize\hspace{1ex}\hrulefill}
\else
\subsection*{TBaseItems Class}
\fi
\label{PasDoc_Items.TBaseItems}
\index{TBaseItems}
\subsubsection*{\large{\textbf{Hierarchy}}\normalsize\hspace{1ex}\hfill}
TBaseItems {$>$} \begin{ttfamily}TObjectVector\end{ttfamily}(\ref{PasDoc_ObjectVector.TObjectVector}) {$>$} 
TObjectList
\subsubsection*{\large{\textbf{Description}}\normalsize\hspace{1ex}\hfill}
Container class to store a list of \begin{ttfamily}TBaseItem\end{ttfamily}(\ref{PasDoc_Items.TBaseItem})s.\subsubsection*{\large{\textbf{Methods}}\normalsize\hspace{1ex}\hfill}
\paragraph*{Create}\hspace*{\fill}

\label{PasDoc_Items.TBaseItems-Create}
\index{Create}
\begin{list}{}{
\settowidth{\tmplength}{\textbf{Description}}
\setlength{\itemindent}{0cm}
\setlength{\listparindent}{0cm}
\setlength{\leftmargin}{\evensidemargin}
\addtolength{\leftmargin}{\tmplength}
\settowidth{\labelsep}{X}
\addtolength{\leftmargin}{\labelsep}
\setlength{\labelwidth}{\tmplength}
}
\item[\textbf{Declaration}\hfill]
\ifpdf
\begin{flushleft}
\fi
\begin{ttfamily}
public constructor Create(const AOwnsObject: Boolean); override;\end{ttfamily}

\ifpdf
\end{flushleft}
\fi

\end{list}
\paragraph*{Destroy}\hspace*{\fill}

\label{PasDoc_Items.TBaseItems-Destroy}
\index{Destroy}
\begin{list}{}{
\settowidth{\tmplength}{\textbf{Description}}
\setlength{\itemindent}{0cm}
\setlength{\listparindent}{0cm}
\setlength{\leftmargin}{\evensidemargin}
\addtolength{\leftmargin}{\tmplength}
\settowidth{\labelsep}{X}
\addtolength{\leftmargin}{\labelsep}
\setlength{\labelwidth}{\tmplength}
}
\item[\textbf{Declaration}\hfill]
\ifpdf
\begin{flushleft}
\fi
\begin{ttfamily}
public destructor Destroy; override;\end{ttfamily}

\ifpdf
\end{flushleft}
\fi

\end{list}
\paragraph*{FindListItem}\hspace*{\fill}

\label{PasDoc_Items.TBaseItems-FindListItem}
\index{FindListItem}
\begin{list}{}{
\settowidth{\tmplength}{\textbf{Description}}
\setlength{\itemindent}{0cm}
\setlength{\listparindent}{0cm}
\setlength{\leftmargin}{\evensidemargin}
\addtolength{\leftmargin}{\tmplength}
\settowidth{\labelsep}{X}
\addtolength{\leftmargin}{\labelsep}
\setlength{\labelwidth}{\tmplength}
}
\item[\textbf{Declaration}\hfill]
\ifpdf
\begin{flushleft}
\fi
\begin{ttfamily}
public function FindListItem(const AName: string): TBaseItem;\end{ttfamily}

\ifpdf
\end{flushleft}
\fi

\par
\item[\textbf{Description}]
Find a given item name on a list. In the base class (TBaseItems), this simply searches the items (not recursively).

In some cases, it may look within the items (recursively), when the identifiers inside the item are in same namespace as the items themselves. Example: it will look also inside enumerated types members, because (when "scoped enums" are off) the enumerated members are in the same namespace as the enumerated type name.

Returns \begin{ttfamily}Nil\end{ttfamily} if nothing can be found.

\end{list}
\paragraph*{InsertItems}\hspace*{\fill}

\label{PasDoc_Items.TBaseItems-InsertItems}
\index{InsertItems}
\begin{list}{}{
\settowidth{\tmplength}{\textbf{Description}}
\setlength{\itemindent}{0cm}
\setlength{\listparindent}{0cm}
\setlength{\leftmargin}{\evensidemargin}
\addtolength{\leftmargin}{\tmplength}
\settowidth{\labelsep}{X}
\addtolength{\leftmargin}{\labelsep}
\setlength{\labelwidth}{\tmplength}
}
\item[\textbf{Declaration}\hfill]
\ifpdf
\begin{flushleft}
\fi
\begin{ttfamily}
public procedure InsertItems(const c: TBaseItems);\end{ttfamily}

\ifpdf
\end{flushleft}
\fi

\par
\item[\textbf{Description}]
Inserts all items of C into this collection. Disposes C and sets it to nil.

\end{list}
\paragraph*{Add}\hspace*{\fill}

\label{PasDoc_Items.TBaseItems-Add}
\index{Add}
\begin{list}{}{
\settowidth{\tmplength}{\textbf{Description}}
\setlength{\itemindent}{0cm}
\setlength{\listparindent}{0cm}
\setlength{\leftmargin}{\evensidemargin}
\addtolength{\leftmargin}{\tmplength}
\settowidth{\labelsep}{X}
\addtolength{\leftmargin}{\labelsep}
\setlength{\labelwidth}{\tmplength}
}
\item[\textbf{Declaration}\hfill]
\ifpdf
\begin{flushleft}
\fi
\begin{ttfamily}
public procedure Add(const AObject: TBaseItem);\end{ttfamily}

\ifpdf
\end{flushleft}
\fi

\par
\item[\textbf{Description}]
During Add, AObject is associated with AObject.Name using hash table, so remember to set AObject.Name \textit{before} calling Add(AObject).

\end{list}
\paragraph*{ClearAndAdd}\hspace*{\fill}

\label{PasDoc_Items.TBaseItems-ClearAndAdd}
\index{ClearAndAdd}
\begin{list}{}{
\settowidth{\tmplength}{\textbf{Description}}
\setlength{\itemindent}{0cm}
\setlength{\listparindent}{0cm}
\setlength{\leftmargin}{\evensidemargin}
\addtolength{\leftmargin}{\tmplength}
\settowidth{\labelsep}{X}
\addtolength{\leftmargin}{\labelsep}
\setlength{\labelwidth}{\tmplength}
}
\item[\textbf{Declaration}\hfill]
\ifpdf
\begin{flushleft}
\fi
\begin{ttfamily}
public procedure ClearAndAdd(const AObject: TBaseItem);\end{ttfamily}

\ifpdf
\end{flushleft}
\fi

\par
\item[\textbf{Description}]
This is a shortcut for doing \begin{ttfamily}Clear\end{ttfamily}(\ref{PasDoc_Items.TBaseItems-Clear}) and then \begin{ttfamily}Add(AObject)\end{ttfamily}(\ref{PasDoc_Items.TBaseItems-Add}). Useful when you want the list to contain exactly the one given AObject.

\end{list}
\paragraph*{Delete}\hspace*{\fill}

\label{PasDoc_Items.TBaseItems-Delete}
\index{Delete}
\begin{list}{}{
\settowidth{\tmplength}{\textbf{Description}}
\setlength{\itemindent}{0cm}
\setlength{\listparindent}{0cm}
\setlength{\leftmargin}{\evensidemargin}
\addtolength{\leftmargin}{\tmplength}
\settowidth{\labelsep}{X}
\addtolength{\leftmargin}{\labelsep}
\setlength{\labelwidth}{\tmplength}
}
\item[\textbf{Declaration}\hfill]
\ifpdf
\begin{flushleft}
\fi
\begin{ttfamily}
public procedure Delete(const AIndex: Integer);\end{ttfamily}

\ifpdf
\end{flushleft}
\fi

\end{list}
\paragraph*{Clear}\hspace*{\fill}

\label{PasDoc_Items.TBaseItems-Clear}
\index{Clear}
\begin{list}{}{
\settowidth{\tmplength}{\textbf{Description}}
\setlength{\itemindent}{0cm}
\setlength{\listparindent}{0cm}
\setlength{\leftmargin}{\evensidemargin}
\addtolength{\leftmargin}{\tmplength}
\settowidth{\labelsep}{X}
\addtolength{\leftmargin}{\labelsep}
\setlength{\labelwidth}{\tmplength}
}
\item[\textbf{Declaration}\hfill]
\ifpdf
\begin{flushleft}
\fi
\begin{ttfamily}
public procedure Clear; override;\end{ttfamily}

\ifpdf
\end{flushleft}
\fi

\end{list}
\ifpdf
\subsection*{\large{\textbf{TPasItems Class}}\normalsize\hspace{1ex}\hrulefill}
\else
\subsection*{TPasItems Class}
\fi
\label{PasDoc_Items.TPasItems}
\index{TPasItems}
\subsubsection*{\large{\textbf{Hierarchy}}\normalsize\hspace{1ex}\hfill}
TPasItems {$>$} \begin{ttfamily}TBaseItems\end{ttfamily}(\ref{PasDoc_Items.TBaseItems}) {$>$} \begin{ttfamily}TObjectVector\end{ttfamily}(\ref{PasDoc_ObjectVector.TObjectVector}) {$>$} 
TObjectList
\subsubsection*{\large{\textbf{Description}}\normalsize\hspace{1ex}\hfill}
Container class to store a list of \begin{ttfamily}TPasItem\end{ttfamily}(\ref{PasDoc_Items.TPasItem})s.\subsubsection*{\large{\textbf{Properties}}\normalsize\hspace{1ex}\hfill}
\begin{list}{}{
\settowidth{\tmplength}{\textbf{PasItemAt}}
\setlength{\itemindent}{0cm}
\setlength{\listparindent}{0cm}
\setlength{\leftmargin}{\evensidemargin}
\addtolength{\leftmargin}{\tmplength}
\settowidth{\labelsep}{X}
\addtolength{\leftmargin}{\labelsep}
\setlength{\labelwidth}{\tmplength}
}
\label{PasDoc_Items.TPasItems-PasItemAt}
\index{PasItemAt}
\item[\textbf{PasItemAt}\hfill]
\ifpdf
\begin{flushleft}
\fi
\begin{ttfamily}
public property PasItemAt[constAIndex:Integer]: TPasItem read GetPasItemAt
      write SetPasItemAt;\end{ttfamily}

\ifpdf
\end{flushleft}
\fi


\par  \end{list}
\subsubsection*{\large{\textbf{Methods}}\normalsize\hspace{1ex}\hfill}
\paragraph*{FindListItem}\hspace*{\fill}

\label{PasDoc_Items.TPasItems-FindListItem}
\index{FindListItem}
\begin{list}{}{
\settowidth{\tmplength}{\textbf{Description}}
\setlength{\itemindent}{0cm}
\setlength{\listparindent}{0cm}
\setlength{\leftmargin}{\evensidemargin}
\addtolength{\leftmargin}{\tmplength}
\settowidth{\labelsep}{X}
\addtolength{\leftmargin}{\labelsep}
\setlength{\labelwidth}{\tmplength}
}
\item[\textbf{Declaration}\hfill]
\ifpdf
\begin{flushleft}
\fi
\begin{ttfamily}
public function FindListItem(const AName: string): TPasItem;\end{ttfamily}

\ifpdf
\end{flushleft}
\fi

\par
\item[\textbf{Description}]
A comfortable routine that just calls inherited and casts result to TPasItem, since every item on this list must be always TPasItem.

\end{list}
\paragraph*{CopyItems}\hspace*{\fill}

\label{PasDoc_Items.TPasItems-CopyItems}
\index{CopyItems}
\begin{list}{}{
\settowidth{\tmplength}{\textbf{Description}}
\setlength{\itemindent}{0cm}
\setlength{\listparindent}{0cm}
\setlength{\leftmargin}{\evensidemargin}
\addtolength{\leftmargin}{\tmplength}
\settowidth{\labelsep}{X}
\addtolength{\leftmargin}{\labelsep}
\setlength{\labelwidth}{\tmplength}
}
\item[\textbf{Declaration}\hfill]
\ifpdf
\begin{flushleft}
\fi
\begin{ttfamily}
public procedure CopyItems(const c: TPasItems);\end{ttfamily}

\ifpdf
\end{flushleft}
\fi

\par
\item[\textbf{Description}]
Copies all Items from c to this object, not changing c at all.

\end{list}
\paragraph*{CountCIO}\hspace*{\fill}

\label{PasDoc_Items.TPasItems-CountCIO}
\index{CountCIO}
\begin{list}{}{
\settowidth{\tmplength}{\textbf{Description}}
\setlength{\itemindent}{0cm}
\setlength{\listparindent}{0cm}
\setlength{\leftmargin}{\evensidemargin}
\addtolength{\leftmargin}{\tmplength}
\settowidth{\labelsep}{X}
\addtolength{\leftmargin}{\labelsep}
\setlength{\labelwidth}{\tmplength}
}
\item[\textbf{Declaration}\hfill]
\ifpdf
\begin{flushleft}
\fi
\begin{ttfamily}
public procedure CountCIO(var c, i, o: Integer);\end{ttfamily}

\ifpdf
\end{flushleft}
\fi

\par
\item[\textbf{Description}]
Counts classes, interfaces and objects within this collection.

\end{list}
\paragraph*{RemovePrivateItems}\hspace*{\fill}

\label{PasDoc_Items.TPasItems-RemovePrivateItems}
\index{RemovePrivateItems}
\begin{list}{}{
\settowidth{\tmplength}{\textbf{Description}}
\setlength{\itemindent}{0cm}
\setlength{\listparindent}{0cm}
\setlength{\leftmargin}{\evensidemargin}
\addtolength{\leftmargin}{\tmplength}
\settowidth{\labelsep}{X}
\addtolength{\leftmargin}{\labelsep}
\setlength{\labelwidth}{\tmplength}
}
\item[\textbf{Declaration}\hfill]
\ifpdf
\begin{flushleft}
\fi
\begin{ttfamily}
public procedure RemovePrivateItems;\end{ttfamily}

\ifpdf
\end{flushleft}
\fi

\par
\item[\textbf{Description}]
Checks each element's Visibility field and removes all elements with a value of viPrivate.

\end{list}
\paragraph*{SortDeep}\hspace*{\fill}

\label{PasDoc_Items.TPasItems-SortDeep}
\index{SortDeep}
\begin{list}{}{
\settowidth{\tmplength}{\textbf{Description}}
\setlength{\itemindent}{0cm}
\setlength{\listparindent}{0cm}
\setlength{\leftmargin}{\evensidemargin}
\addtolength{\leftmargin}{\tmplength}
\settowidth{\labelsep}{X}
\addtolength{\leftmargin}{\labelsep}
\setlength{\labelwidth}{\tmplength}
}
\item[\textbf{Declaration}\hfill]
\ifpdf
\begin{flushleft}
\fi
\begin{ttfamily}
public procedure SortDeep(const SortSettings: TSortSettings);\end{ttfamily}

\ifpdf
\end{flushleft}
\fi

\par
\item[\textbf{Description}]
This sorts all items on this list by their name, and also calls \begin{ttfamily}Sort(SortSettings)\end{ttfamily}(\ref{PasDoc_Items.TPasItem-Sort}) for each of these items. This way it sorts recursively everything in this list.

This is equivalent to doing both \begin{ttfamily}SortShallow\end{ttfamily}(\ref{PasDoc_Items.TPasItems-SortShallow}) and \begin{ttfamily}SortOnlyInsideItems\end{ttfamily}(\ref{PasDoc_Items.TPasItems-SortOnlyInsideItems}).

\end{list}
\paragraph*{SortOnlyInsideItems}\hspace*{\fill}

\label{PasDoc_Items.TPasItems-SortOnlyInsideItems}
\index{SortOnlyInsideItems}
\begin{list}{}{
\settowidth{\tmplength}{\textbf{Description}}
\setlength{\itemindent}{0cm}
\setlength{\listparindent}{0cm}
\setlength{\leftmargin}{\evensidemargin}
\addtolength{\leftmargin}{\tmplength}
\settowidth{\labelsep}{X}
\addtolength{\leftmargin}{\labelsep}
\setlength{\labelwidth}{\tmplength}
}
\item[\textbf{Declaration}\hfill]
\ifpdf
\begin{flushleft}
\fi
\begin{ttfamily}
public procedure SortOnlyInsideItems(const SortSettings: TSortSettings);\end{ttfamily}

\ifpdf
\end{flushleft}
\fi

\par
\item[\textbf{Description}]
This calls \begin{ttfamily}Sort(SortSettings)\end{ttfamily}(\ref{PasDoc_Items.TPasItem-Sort}) for each of items on the list. It does \textit{not} sort the items on this list.

\end{list}
\paragraph*{SortShallow}\hspace*{\fill}

\label{PasDoc_Items.TPasItems-SortShallow}
\index{SortShallow}
\begin{list}{}{
\settowidth{\tmplength}{\textbf{Description}}
\setlength{\itemindent}{0cm}
\setlength{\listparindent}{0cm}
\setlength{\leftmargin}{\evensidemargin}
\addtolength{\leftmargin}{\tmplength}
\settowidth{\labelsep}{X}
\addtolength{\leftmargin}{\labelsep}
\setlength{\labelwidth}{\tmplength}
}
\item[\textbf{Declaration}\hfill]
\ifpdf
\begin{flushleft}
\fi
\begin{ttfamily}
public procedure SortShallow;\end{ttfamily}

\ifpdf
\end{flushleft}
\fi

\par
\item[\textbf{Description}]
This sorts all items on this list by their name. Unlike \begin{ttfamily}SortDeep\end{ttfamily}(\ref{PasDoc_Items.TPasItems-SortDeep}), it does \textit{not} call \begin{ttfamily}Sort\end{ttfamily}(\ref{PasDoc_Items.TPasItem-Sort}) for each of these items. So "items inside items" (e.g. class methods, if this list contains TPasCio objects) remain unsorted.

\end{list}
\paragraph*{SetFullDeclaration}\hspace*{\fill}

\label{PasDoc_Items.TPasItems-SetFullDeclaration}
\index{SetFullDeclaration}
\begin{list}{}{
\settowidth{\tmplength}{\textbf{Description}}
\setlength{\itemindent}{0cm}
\setlength{\listparindent}{0cm}
\setlength{\leftmargin}{\evensidemargin}
\addtolength{\leftmargin}{\tmplength}
\settowidth{\labelsep}{X}
\addtolength{\leftmargin}{\labelsep}
\setlength{\labelwidth}{\tmplength}
}
\item[\textbf{Declaration}\hfill]
\ifpdf
\begin{flushleft}
\fi
\begin{ttfamily}
public procedure SetFullDeclaration(PrefixName: boolean; const Suffix: string);\end{ttfamily}

\ifpdf
\end{flushleft}
\fi

\par
\item[\textbf{Description}]
Sets FullDeclaration of every item to \begin{enumerate}
\setcounter{enumi}{0} \setcounter{enumii}{0} \setcounter{enumiii}{0} \setcounter{enumiv}{0} 
\item Name of this item (only if PrefixName)
\setcounter{enumi}{1} \setcounter{enumii}{1} \setcounter{enumiii}{1} \setcounter{enumiv}{1} 
\item + Suffix.
\end{enumerate} Very useful if you have a couple of items that share a common declaration in source file, e.g. variables or fields declared like \texttt{\\\nopagebreak[3]
A,~B:~Integer;\\
}

\end{list}
\ifpdf
\subsection*{\large{\textbf{TPasMethods Class}}\normalsize\hspace{1ex}\hrulefill}
\else
\subsection*{TPasMethods Class}
\fi
\label{PasDoc_Items.TPasMethods}
\index{TPasMethods}
\subsubsection*{\large{\textbf{Hierarchy}}\normalsize\hspace{1ex}\hfill}
TPasMethods {$>$} \begin{ttfamily}TPasItems\end{ttfamily}(\ref{PasDoc_Items.TPasItems}) {$>$} \begin{ttfamily}TBaseItems\end{ttfamily}(\ref{PasDoc_Items.TBaseItems}) {$>$} \begin{ttfamily}TObjectVector\end{ttfamily}(\ref{PasDoc_ObjectVector.TObjectVector}) {$>$} 
TObjectList
\subsubsection*{\large{\textbf{Description}}\normalsize\hspace{1ex}\hfill}
Collection of methods.\subsubsection*{\large{\textbf{Methods}}\normalsize\hspace{1ex}\hfill}
\paragraph*{FindListItem}\hspace*{\fill}

\label{PasDoc_Items.TPasMethods-FindListItem}
\index{FindListItem}
\begin{list}{}{
\settowidth{\tmplength}{\textbf{Description}}
\setlength{\itemindent}{0cm}
\setlength{\listparindent}{0cm}
\setlength{\leftmargin}{\evensidemargin}
\addtolength{\leftmargin}{\tmplength}
\settowidth{\labelsep}{X}
\addtolength{\leftmargin}{\labelsep}
\setlength{\labelwidth}{\tmplength}
}
\item[\textbf{Declaration}\hfill]
\ifpdf
\begin{flushleft}
\fi
\begin{ttfamily}
public function FindListItem(const AName: string; Index: Integer): TPasMethod; overload;\end{ttfamily}

\ifpdf
\end{flushleft}
\fi

\par
\item[\textbf{Description}]
Find an Index{-}th item with given name on a list. Index is 0{-}based. There could be multiple items sharing the same name (overloads) while method of base class returns only the one most recently added item.

Returns \begin{ttfamily}Nil\end{ttfamily} if nothing can be found.

\end{list}
\ifpdf
\subsection*{\large{\textbf{TPasProperties Class}}\normalsize\hspace{1ex}\hrulefill}
\else
\subsection*{TPasProperties Class}
\fi
\label{PasDoc_Items.TPasProperties}
\index{TPasProperties}
\subsubsection*{\large{\textbf{Hierarchy}}\normalsize\hspace{1ex}\hfill}
TPasProperties {$>$} \begin{ttfamily}TPasItems\end{ttfamily}(\ref{PasDoc_Items.TPasItems}) {$>$} \begin{ttfamily}TBaseItems\end{ttfamily}(\ref{PasDoc_Items.TBaseItems}) {$>$} \begin{ttfamily}TObjectVector\end{ttfamily}(\ref{PasDoc_ObjectVector.TObjectVector}) {$>$} 
TObjectList
\subsubsection*{\large{\textbf{Description}}\normalsize\hspace{1ex}\hfill}
Collection of properties.\ifpdf
\subsection*{\large{\textbf{TPasNestedCios Class}}\normalsize\hspace{1ex}\hrulefill}
\else
\subsection*{TPasNestedCios Class}
\fi
\label{PasDoc_Items.TPasNestedCios}
\index{TPasNestedCios}
\subsubsection*{\large{\textbf{Hierarchy}}\normalsize\hspace{1ex}\hfill}
TPasNestedCios {$>$} \begin{ttfamily}TPasItems\end{ttfamily}(\ref{PasDoc_Items.TPasItems}) {$>$} \begin{ttfamily}TBaseItems\end{ttfamily}(\ref{PasDoc_Items.TBaseItems}) {$>$} \begin{ttfamily}TObjectVector\end{ttfamily}(\ref{PasDoc_ObjectVector.TObjectVector}) {$>$} 
TObjectList
\subsubsection*{\large{\textbf{Description}}\normalsize\hspace{1ex}\hfill}
Collection of classes / records / interfaces.\subsubsection*{\large{\textbf{Methods}}\normalsize\hspace{1ex}\hfill}
\paragraph*{Create}\hspace*{\fill}

\label{PasDoc_Items.TPasNestedCios-Create}
\index{Create}
\begin{list}{}{
\settowidth{\tmplength}{\textbf{Description}}
\setlength{\itemindent}{0cm}
\setlength{\listparindent}{0cm}
\setlength{\leftmargin}{\evensidemargin}
\addtolength{\leftmargin}{\tmplength}
\settowidth{\labelsep}{X}
\addtolength{\leftmargin}{\labelsep}
\setlength{\labelwidth}{\tmplength}
}
\item[\textbf{Declaration}\hfill]
\ifpdf
\begin{flushleft}
\fi
\begin{ttfamily}
public constructor Create; reintroduce;\end{ttfamily}

\ifpdf
\end{flushleft}
\fi

\end{list}
\ifpdf
\subsection*{\large{\textbf{TPasTypes Class}}\normalsize\hspace{1ex}\hrulefill}
\else
\subsection*{TPasTypes Class}
\fi
\label{PasDoc_Items.TPasTypes}
\index{TPasTypes}
\subsubsection*{\large{\textbf{Hierarchy}}\normalsize\hspace{1ex}\hfill}
TPasTypes {$>$} \begin{ttfamily}TPasItems\end{ttfamily}(\ref{PasDoc_Items.TPasItems}) {$>$} \begin{ttfamily}TBaseItems\end{ttfamily}(\ref{PasDoc_Items.TBaseItems}) {$>$} \begin{ttfamily}TObjectVector\end{ttfamily}(\ref{PasDoc_ObjectVector.TObjectVector}) {$>$} 
TObjectList
\subsubsection*{\large{\textbf{Description}}\normalsize\hspace{1ex}\hfill}
Collection of types.\subsubsection*{\large{\textbf{Methods}}\normalsize\hspace{1ex}\hfill}
\paragraph*{FindListItem}\hspace*{\fill}

\label{PasDoc_Items.TPasTypes-FindListItem}
\index{FindListItem}
\begin{list}{}{
\settowidth{\tmplength}{\textbf{Description}}
\setlength{\itemindent}{0cm}
\setlength{\listparindent}{0cm}
\setlength{\leftmargin}{\evensidemargin}
\addtolength{\leftmargin}{\tmplength}
\settowidth{\labelsep}{X}
\addtolength{\leftmargin}{\labelsep}
\setlength{\labelwidth}{\tmplength}
}
\item[\textbf{Declaration}\hfill]
\ifpdf
\begin{flushleft}
\fi
\begin{ttfamily}
public function FindListItem(const AName: string): TPasItem;\end{ttfamily}

\ifpdf
\end{flushleft}
\fi

\end{list}
\ifpdf
\subsection*{\large{\textbf{TPasUnits Class}}\normalsize\hspace{1ex}\hrulefill}
\else
\subsection*{TPasUnits Class}
\fi
\label{PasDoc_Items.TPasUnits}
\index{TPasUnits}
\subsubsection*{\large{\textbf{Hierarchy}}\normalsize\hspace{1ex}\hfill}
TPasUnits {$>$} \begin{ttfamily}TPasItems\end{ttfamily}(\ref{PasDoc_Items.TPasItems}) {$>$} \begin{ttfamily}TBaseItems\end{ttfamily}(\ref{PasDoc_Items.TBaseItems}) {$>$} \begin{ttfamily}TObjectVector\end{ttfamily}(\ref{PasDoc_ObjectVector.TObjectVector}) {$>$} 
TObjectList
\subsubsection*{\large{\textbf{Description}}\normalsize\hspace{1ex}\hfill}
Collection of units.\subsubsection*{\large{\textbf{Properties}}\normalsize\hspace{1ex}\hfill}
\begin{list}{}{
\settowidth{\tmplength}{\textbf{UnitAt}}
\setlength{\itemindent}{0cm}
\setlength{\listparindent}{0cm}
\setlength{\leftmargin}{\evensidemargin}
\addtolength{\leftmargin}{\tmplength}
\settowidth{\labelsep}{X}
\addtolength{\leftmargin}{\labelsep}
\setlength{\labelwidth}{\tmplength}
}
\label{PasDoc_Items.TPasUnits-UnitAt}
\index{UnitAt}
\item[\textbf{UnitAt}\hfill]
\ifpdf
\begin{flushleft}
\fi
\begin{ttfamily}
public property UnitAt[constAIndex:Integer]: TPasUnit
      read GetUnitAt
      write SetUnitAt;\end{ttfamily}

\ifpdf
\end{flushleft}
\fi


\par  \end{list}
\subsubsection*{\large{\textbf{Methods}}\normalsize\hspace{1ex}\hfill}
\paragraph*{ExistsUnit}\hspace*{\fill}

\label{PasDoc_Items.TPasUnits-ExistsUnit}
\index{ExistsUnit}
\begin{list}{}{
\settowidth{\tmplength}{\textbf{Description}}
\setlength{\itemindent}{0cm}
\setlength{\listparindent}{0cm}
\setlength{\leftmargin}{\evensidemargin}
\addtolength{\leftmargin}{\tmplength}
\settowidth{\labelsep}{X}
\addtolength{\leftmargin}{\labelsep}
\setlength{\labelwidth}{\tmplength}
}
\item[\textbf{Declaration}\hfill]
\ifpdf
\begin{flushleft}
\fi
\begin{ttfamily}
public function ExistsUnit(const AUnit: TPasUnit): Boolean;\end{ttfamily}

\ifpdf
\end{flushleft}
\fi

\end{list}
\section{Functions and Procedures}
\ifpdf
\subsection*{\large{\textbf{MethodTypeToString}}\normalsize\hspace{1ex}\hrulefill}
\else
\subsection*{MethodTypeToString}
\fi
\label{PasDoc_Items-MethodTypeToString}
\index{MethodTypeToString}
\begin{list}{}{
\settowidth{\tmplength}{\textbf{Description}}
\setlength{\itemindent}{0cm}
\setlength{\listparindent}{0cm}
\setlength{\leftmargin}{\evensidemargin}
\addtolength{\leftmargin}{\tmplength}
\settowidth{\labelsep}{X}
\addtolength{\leftmargin}{\labelsep}
\setlength{\labelwidth}{\tmplength}
}
\item[\textbf{Declaration}\hfill]
\ifpdf
\begin{flushleft}
\fi
\begin{ttfamily}
function MethodTypeToString(const MethodType: TMethodType): string;\end{ttfamily}

\ifpdf
\end{flushleft}
\fi

\par
\item[\textbf{Description}]
Returns lowercased keyword associated with given method type.

\end{list}
\ifpdf
\subsection*{\large{\textbf{VisibilitiesToStr}}\normalsize\hspace{1ex}\hrulefill}
\else
\subsection*{VisibilitiesToStr}
\fi
\label{PasDoc_Items-VisibilitiesToStr}
\index{VisibilitiesToStr}
\begin{list}{}{
\settowidth{\tmplength}{\textbf{Description}}
\setlength{\itemindent}{0cm}
\setlength{\listparindent}{0cm}
\setlength{\leftmargin}{\evensidemargin}
\addtolength{\leftmargin}{\tmplength}
\settowidth{\labelsep}{X}
\addtolength{\leftmargin}{\labelsep}
\setlength{\labelwidth}{\tmplength}
}
\item[\textbf{Declaration}\hfill]
\ifpdf
\begin{flushleft}
\fi
\begin{ttfamily}
function VisibilitiesToStr(const Visibilities: TVisibilities): string;\end{ttfamily}

\ifpdf
\end{flushleft}
\fi

\par
\item[\textbf{Description}]
Returns VisibilityStr for each value in Visibilities, delimited by commas.

\end{list}
\ifpdf
\subsection*{\large{\textbf{VisToStr}}\normalsize\hspace{1ex}\hrulefill}
\else
\subsection*{VisToStr}
\fi
\label{PasDoc_Items-VisToStr}
\index{VisToStr}
\begin{list}{}{
\settowidth{\tmplength}{\textbf{Description}}
\setlength{\itemindent}{0cm}
\setlength{\listparindent}{0cm}
\setlength{\leftmargin}{\evensidemargin}
\addtolength{\leftmargin}{\tmplength}
\settowidth{\labelsep}{X}
\addtolength{\leftmargin}{\labelsep}
\setlength{\labelwidth}{\tmplength}
}
\item[\textbf{Declaration}\hfill]
\ifpdf
\begin{flushleft}
\fi
\begin{ttfamily}
function VisToStr(const Vis: TVisibility): string;\end{ttfamily}

\ifpdf
\end{flushleft}
\fi

\end{list}
\section{Types}
\ifpdf
\subsection*{\large{\textbf{TVisibility}}\normalsize\hspace{1ex}\hrulefill}
\else
\subsection*{TVisibility}
\fi
\label{PasDoc_Items-TVisibility}
\index{TVisibility}
\begin{list}{}{
\settowidth{\tmplength}{\textbf{Description}}
\setlength{\itemindent}{0cm}
\setlength{\listparindent}{0cm}
\setlength{\leftmargin}{\evensidemargin}
\addtolength{\leftmargin}{\tmplength}
\settowidth{\labelsep}{X}
\addtolength{\leftmargin}{\labelsep}
\setlength{\labelwidth}{\tmplength}
}
\item[\textbf{Declaration}\hfill]
\ifpdf
\begin{flushleft}
\fi
\begin{ttfamily}
TVisibility = (...);\end{ttfamily}

\ifpdf
\end{flushleft}
\fi

\par
\item[\textbf{Description}]
Visibility of a field/method.\item[\textbf{Values}]
\begin{description}
\item[\texttt{viPublished}] \label{PasDoc_Items-viPublished}
\index{}
indicates field or method is published
\item[\texttt{viPublic}] \label{PasDoc_Items-viPublic}
\index{}
indicates field or method is public
\item[\texttt{viProtected}] \label{PasDoc_Items-viProtected}
\index{}
indicates field or method is protected
\item[\texttt{viStrictProtected}] \label{PasDoc_Items-viStrictProtected}
\index{}
indicates field or method is strict protected
\item[\texttt{viPrivate}] \label{PasDoc_Items-viPrivate}
\index{}
indicates field or method is private
\item[\texttt{viStrictPrivate}] \label{PasDoc_Items-viStrictPrivate}
\index{}
indicates field or method is strict private
\item[\texttt{viAutomated}] \label{PasDoc_Items-viAutomated}
\index{}
indicates field or method is automated
\item[\texttt{viImplicit}] \label{PasDoc_Items-viImplicit}
\index{}
implicit visibility, marks the implicit members if user used {-}{-}implicit{-}visibility=implicit command{-}line option.
\end{description}


\end{list}
\ifpdf
\subsection*{\large{\textbf{TVisibilities}}\normalsize\hspace{1ex}\hrulefill}
\else
\subsection*{TVisibilities}
\fi
\label{PasDoc_Items-TVisibilities}
\index{TVisibilities}
\begin{list}{}{
\settowidth{\tmplength}{\textbf{Description}}
\setlength{\itemindent}{0cm}
\setlength{\listparindent}{0cm}
\setlength{\leftmargin}{\evensidemargin}
\addtolength{\leftmargin}{\tmplength}
\settowidth{\labelsep}{X}
\addtolength{\leftmargin}{\labelsep}
\setlength{\labelwidth}{\tmplength}
}
\item[\textbf{Declaration}\hfill]
\ifpdf
\begin{flushleft}
\fi
\begin{ttfamily}
TVisibilities = set of TVisibility;\end{ttfamily}

\ifpdf
\end{flushleft}
\fi

\end{list}
\ifpdf
\subsection*{\large{\textbf{TInfoMergeType}}\normalsize\hspace{1ex}\hrulefill}
\else
\subsection*{TInfoMergeType}
\fi
\label{PasDoc_Items-TInfoMergeType}
\index{TInfoMergeType}
\begin{list}{}{
\settowidth{\tmplength}{\textbf{Description}}
\setlength{\itemindent}{0cm}
\setlength{\listparindent}{0cm}
\setlength{\leftmargin}{\evensidemargin}
\addtolength{\leftmargin}{\tmplength}
\settowidth{\labelsep}{X}
\addtolength{\leftmargin}{\labelsep}
\setlength{\labelwidth}{\tmplength}
}
\item[\textbf{Declaration}\hfill]
\ifpdf
\begin{flushleft}
\fi
\begin{ttfamily}
TInfoMergeType = (...);\end{ttfamily}

\ifpdf
\end{flushleft}
\fi

\par
\item[\textbf{Description}]
Type of merging intf section and impl section metadata of an item\item[\textbf{Values}]
\begin{description}
\item[\texttt{imtNone}] \label{PasDoc_Items-imtNone}
\index{}
impl section is not scanned {-} default behavior
\item[\texttt{imtPreferIntf}] \label{PasDoc_Items-imtPreferIntf}
\index{}
data is taken from intf, if it's empty {-} from impl
\item[\texttt{imtJoin}] \label{PasDoc_Items-imtJoin}
\index{}
data is concatenated
\item[\texttt{imtPreferImpl}] \label{PasDoc_Items-imtPreferImpl}
\index{}
data is taken from impl, if it's empty {-} from intf
\end{description}


\end{list}
\ifpdf
\subsection*{\large{\textbf{PRawDescriptionInfo}}\normalsize\hspace{1ex}\hrulefill}
\else
\subsection*{PRawDescriptionInfo}
\fi
\label{PasDoc_Items-PRawDescriptionInfo}
\index{PRawDescriptionInfo}
\begin{list}{}{
\settowidth{\tmplength}{\textbf{Description}}
\setlength{\itemindent}{0cm}
\setlength{\listparindent}{0cm}
\setlength{\leftmargin}{\evensidemargin}
\addtolength{\leftmargin}{\tmplength}
\settowidth{\labelsep}{X}
\addtolength{\leftmargin}{\labelsep}
\setlength{\labelwidth}{\tmplength}
}
\item[\textbf{Declaration}\hfill]
\ifpdf
\begin{flushleft}
\fi
\begin{ttfamily}
PRawDescriptionInfo = {\^{}}TRawDescriptionInfo;\end{ttfamily}

\ifpdf
\end{flushleft}
\fi

\end{list}
\ifpdf
\subsection*{\large{\textbf{THintDirective}}\normalsize\hspace{1ex}\hrulefill}
\else
\subsection*{THintDirective}
\fi
\label{PasDoc_Items-THintDirective}
\index{THintDirective}
\begin{list}{}{
\settowidth{\tmplength}{\textbf{Description}}
\setlength{\itemindent}{0cm}
\setlength{\listparindent}{0cm}
\setlength{\leftmargin}{\evensidemargin}
\addtolength{\leftmargin}{\tmplength}
\settowidth{\labelsep}{X}
\addtolength{\leftmargin}{\labelsep}
\setlength{\labelwidth}{\tmplength}
}
\item[\textbf{Declaration}\hfill]
\ifpdf
\begin{flushleft}
\fi
\begin{ttfamily}
THintDirective = (...);\end{ttfamily}

\ifpdf
\end{flushleft}
\fi

\par
\item[\textbf{Description}]
 \item[\textbf{Values}]
\begin{description}
\item[\texttt{hdDeprecated}] \label{PasDoc_Items-hdDeprecated}
\index{}
 
\item[\texttt{hdPlatform}] \label{PasDoc_Items-hdPlatform}
\index{}
 
\item[\texttt{hdLibrary}] \label{PasDoc_Items-hdLibrary}
\index{}
 
\item[\texttt{hdExperimental}] \label{PasDoc_Items-hdExperimental}
\index{}
 
\end{description}


\end{list}
\ifpdf
\subsection*{\large{\textbf{THintDirectives}}\normalsize\hspace{1ex}\hrulefill}
\else
\subsection*{THintDirectives}
\fi
\label{PasDoc_Items-THintDirectives}
\index{THintDirectives}
\begin{list}{}{
\settowidth{\tmplength}{\textbf{Description}}
\setlength{\itemindent}{0cm}
\setlength{\listparindent}{0cm}
\setlength{\leftmargin}{\evensidemargin}
\addtolength{\leftmargin}{\tmplength}
\settowidth{\labelsep}{X}
\addtolength{\leftmargin}{\labelsep}
\setlength{\labelwidth}{\tmplength}
}
\item[\textbf{Declaration}\hfill]
\ifpdf
\begin{flushleft}
\fi
\begin{ttfamily}
THintDirectives = set of THintDirective;\end{ttfamily}

\ifpdf
\end{flushleft}
\fi

\end{list}
\ifpdf
\subsection*{\large{\textbf{TMethodType}}\normalsize\hspace{1ex}\hrulefill}
\else
\subsection*{TMethodType}
\fi
\label{PasDoc_Items-TMethodType}
\index{TMethodType}
\begin{list}{}{
\settowidth{\tmplength}{\textbf{Description}}
\setlength{\itemindent}{0cm}
\setlength{\listparindent}{0cm}
\setlength{\leftmargin}{\evensidemargin}
\addtolength{\leftmargin}{\tmplength}
\settowidth{\labelsep}{X}
\addtolength{\leftmargin}{\labelsep}
\setlength{\labelwidth}{\tmplength}
}
\item[\textbf{Declaration}\hfill]
\ifpdf
\begin{flushleft}
\fi
\begin{ttfamily}
TMethodType = (...);\end{ttfamily}

\ifpdf
\end{flushleft}
\fi

\par
\item[\textbf{Description}]
Methodtype for \begin{ttfamily}TPasMethod\end{ttfamily}(\ref{PasDoc_Items.TPasMethod})\item[\textbf{Values}]
\begin{description}
\item[\texttt{METHOD{\_}CONSTRUCTOR}] \label{PasDoc_Items-METHOD_CONSTRUCTOR}
\index{}
 
\item[\texttt{METHOD{\_}DESTRUCTOR}] \label{PasDoc_Items-METHOD_DESTRUCTOR}
\index{}
 
\item[\texttt{METHOD{\_}FUNCTION}] \label{PasDoc_Items-METHOD_FUNCTION}
\index{}
 
\item[\texttt{METHOD{\_}PROCEDURE}] \label{PasDoc_Items-METHOD_PROCEDURE}
\index{}
 
\item[\texttt{METHOD{\_}OPERATOR}] \label{PasDoc_Items-METHOD_OPERATOR}
\index{}
 
\end{description}


\end{list}
\ifpdf
\subsection*{\large{\textbf{TCIOType}}\normalsize\hspace{1ex}\hrulefill}
\else
\subsection*{TCIOType}
\fi
\label{PasDoc_Items-TCIOType}
\index{TCIOType}
\begin{list}{}{
\settowidth{\tmplength}{\textbf{Description}}
\setlength{\itemindent}{0cm}
\setlength{\listparindent}{0cm}
\setlength{\leftmargin}{\evensidemargin}
\addtolength{\leftmargin}{\tmplength}
\settowidth{\labelsep}{X}
\addtolength{\leftmargin}{\labelsep}
\setlength{\labelwidth}{\tmplength}
}
\item[\textbf{Declaration}\hfill]
\ifpdf
\begin{flushleft}
\fi
\begin{ttfamily}
TCIOType = (...);\end{ttfamily}

\ifpdf
\end{flushleft}
\fi

\par
\item[\textbf{Description}]
enumeration type to determine type of \begin{ttfamily}TPasCio\end{ttfamily}(\ref{PasDoc_Items.TPasCio}) item\item[\textbf{Values}]
\begin{description}
\item[\texttt{CIO{\_}CLASS}] \label{PasDoc_Items-CIO_CLASS}
\index{}
 
\item[\texttt{CIO{\_}PACKEDCLASS}] \label{PasDoc_Items-CIO_PACKEDCLASS}
\index{}
 
\item[\texttt{CIO{\_}DISPINTERFACE}] \label{PasDoc_Items-CIO_DISPINTERFACE}
\index{}
 
\item[\texttt{CIO{\_}INTERFACE}] \label{PasDoc_Items-CIO_INTERFACE}
\index{}
 
\item[\texttt{CIO{\_}OBJECT}] \label{PasDoc_Items-CIO_OBJECT}
\index{}
 
\item[\texttt{CIO{\_}PACKEDOBJECT}] \label{PasDoc_Items-CIO_PACKEDOBJECT}
\index{}
 
\item[\texttt{CIO{\_}RECORD}] \label{PasDoc_Items-CIO_RECORD}
\index{}
 
\item[\texttt{CIO{\_}PACKEDRECORD}] \label{PasDoc_Items-CIO_PACKEDRECORD}
\index{}
 
\end{description}


\end{list}
\ifpdf
\subsection*{\large{\textbf{TClassDirective}}\normalsize\hspace{1ex}\hrulefill}
\else
\subsection*{TClassDirective}
\fi
\label{PasDoc_Items-TClassDirective}
\index{TClassDirective}
\begin{list}{}{
\settowidth{\tmplength}{\textbf{Description}}
\setlength{\itemindent}{0cm}
\setlength{\listparindent}{0cm}
\setlength{\leftmargin}{\evensidemargin}
\addtolength{\leftmargin}{\tmplength}
\settowidth{\labelsep}{X}
\addtolength{\leftmargin}{\labelsep}
\setlength{\labelwidth}{\tmplength}
}
\item[\textbf{Declaration}\hfill]
\ifpdf
\begin{flushleft}
\fi
\begin{ttfamily}
TClassDirective = (...);\end{ttfamily}

\ifpdf
\end{flushleft}
\fi

\par
\item[\textbf{Description}]
 \item[\textbf{Values}]
\begin{description}
\item[\texttt{CT{\_}NONE}] \label{PasDoc_Items-CT_NONE}
\index{}
 
\item[\texttt{CT{\_}ABSTRACT}] \label{PasDoc_Items-CT_ABSTRACT}
\index{}
 
\item[\texttt{CT{\_}SEALED}] \label{PasDoc_Items-CT_SEALED}
\index{}
 
\item[\texttt{CT{\_}HELPER}] \label{PasDoc_Items-CT_HELPER}
\index{}
 
\end{description}


\end{list}
\section{Constants}
\ifpdf
\subsection*{\large{\textbf{VisibilityStr}}\normalsize\hspace{1ex}\hrulefill}
\else
\subsection*{VisibilityStr}
\fi
\label{PasDoc_Items-VisibilityStr}
\index{VisibilityStr}
\begin{list}{}{
\settowidth{\tmplength}{\textbf{Description}}
\setlength{\itemindent}{0cm}
\setlength{\listparindent}{0cm}
\setlength{\leftmargin}{\evensidemargin}
\addtolength{\leftmargin}{\tmplength}
\settowidth{\labelsep}{X}
\addtolength{\leftmargin}{\labelsep}
\setlength{\labelwidth}{\tmplength}
}
\item[\textbf{Declaration}\hfill]
\ifpdf
\begin{flushleft}
\fi
\begin{ttfamily}
VisibilityStr: array[TVisibility] of string[16] =
  (
   'published',
   'public',
   'protected',
   'strict protected',
   'private',
   'strict private',
   'automated',
   'implicit'
  );\end{ttfamily}

\ifpdf
\end{flushleft}
\fi

\end{list}
\ifpdf
\subsection*{\large{\textbf{AllVisibilities}}\normalsize\hspace{1ex}\hrulefill}
\else
\subsection*{AllVisibilities}
\fi
\label{PasDoc_Items-AllVisibilities}
\index{AllVisibilities}
\begin{list}{}{
\settowidth{\tmplength}{\textbf{Description}}
\setlength{\itemindent}{0cm}
\setlength{\listparindent}{0cm}
\setlength{\leftmargin}{\evensidemargin}
\addtolength{\leftmargin}{\tmplength}
\settowidth{\labelsep}{X}
\addtolength{\leftmargin}{\labelsep}
\setlength{\labelwidth}{\tmplength}
}
\item[\textbf{Declaration}\hfill]
\ifpdf
\begin{flushleft}
\fi
\begin{ttfamily}
AllVisibilities: TVisibilities = [Low(TVisibility) .. High(TVisibility)];\end{ttfamily}

\ifpdf
\end{flushleft}
\fi

\end{list}
\ifpdf
\subsection*{\large{\textbf{DefaultVisibilities}}\normalsize\hspace{1ex}\hrulefill}
\else
\subsection*{DefaultVisibilities}
\fi
\label{PasDoc_Items-DefaultVisibilities}
\index{DefaultVisibilities}
\begin{list}{}{
\settowidth{\tmplength}{\textbf{Description}}
\setlength{\itemindent}{0cm}
\setlength{\listparindent}{0cm}
\setlength{\leftmargin}{\evensidemargin}
\addtolength{\leftmargin}{\tmplength}
\settowidth{\labelsep}{X}
\addtolength{\leftmargin}{\labelsep}
\setlength{\labelwidth}{\tmplength}
}
\item[\textbf{Declaration}\hfill]
\ifpdf
\begin{flushleft}
\fi
\begin{ttfamily}
DefaultVisibilities: TVisibilities =
    [viProtected, viPublic, viPublished, viAutomated];\end{ttfamily}

\ifpdf
\end{flushleft}
\fi

\end{list}
\ifpdf
\subsection*{\large{\textbf{InfoMergeTypeStr}}\normalsize\hspace{1ex}\hrulefill}
\else
\subsection*{InfoMergeTypeStr}
\fi
\label{PasDoc_Items-InfoMergeTypeStr}
\index{InfoMergeTypeStr}
\begin{list}{}{
\settowidth{\tmplength}{\textbf{Description}}
\setlength{\itemindent}{0cm}
\setlength{\listparindent}{0cm}
\setlength{\leftmargin}{\evensidemargin}
\addtolength{\leftmargin}{\tmplength}
\settowidth{\labelsep}{X}
\addtolength{\leftmargin}{\labelsep}
\setlength{\labelwidth}{\tmplength}
}
\item[\textbf{Declaration}\hfill]
\ifpdf
\begin{flushleft}
\fi
\begin{ttfamily}
InfoMergeTypeStr: array[TInfoMergeType] of string = (
    'none',
    'prefer-interface',
    'join',
    'prefer-implementation'
  );\end{ttfamily}

\ifpdf
\end{flushleft}
\fi

\end{list}
\ifpdf
\subsection*{\large{\textbf{CIORecordType}}\normalsize\hspace{1ex}\hrulefill}
\else
\subsection*{CIORecordType}
\fi
\label{PasDoc_Items-CIORecordType}
\index{CIORecordType}
\begin{list}{}{
\settowidth{\tmplength}{\textbf{Description}}
\setlength{\itemindent}{0cm}
\setlength{\listparindent}{0cm}
\setlength{\leftmargin}{\evensidemargin}
\addtolength{\leftmargin}{\tmplength}
\settowidth{\labelsep}{X}
\addtolength{\leftmargin}{\labelsep}
\setlength{\labelwidth}{\tmplength}
}
\item[\textbf{Declaration}\hfill]
\ifpdf
\begin{flushleft}
\fi
\begin{ttfamily}
CIORecordType = [CIO{\_}RECORD, CIO{\_}PACKEDRECORD];\end{ttfamily}

\ifpdf
\end{flushleft}
\fi

\end{list}
\ifpdf
\subsection*{\large{\textbf{CIONonHierarchy}}\normalsize\hspace{1ex}\hrulefill}
\else
\subsection*{CIONonHierarchy}
\fi
\label{PasDoc_Items-CIONonHierarchy}
\index{CIONonHierarchy}
\begin{list}{}{
\settowidth{\tmplength}{\textbf{Description}}
\setlength{\itemindent}{0cm}
\setlength{\listparindent}{0cm}
\setlength{\leftmargin}{\evensidemargin}
\addtolength{\leftmargin}{\tmplength}
\settowidth{\labelsep}{X}
\addtolength{\leftmargin}{\labelsep}
\setlength{\labelwidth}{\tmplength}
}
\item[\textbf{Declaration}\hfill]
\ifpdf
\begin{flushleft}
\fi
\begin{ttfamily}
CIONonHierarchy = CIORecordType;\end{ttfamily}

\ifpdf
\end{flushleft}
\fi

\end{list}
\ifpdf
\subsection*{\large{\textbf{EmptyRawDescriptionInfo}}\normalsize\hspace{1ex}\hrulefill}
\else
\subsection*{EmptyRawDescriptionInfo}
\fi
\label{PasDoc_Items-EmptyRawDescriptionInfo}
\index{EmptyRawDescriptionInfo}
\begin{list}{}{
\settowidth{\tmplength}{\textbf{Description}}
\setlength{\itemindent}{0cm}
\setlength{\listparindent}{0cm}
\setlength{\leftmargin}{\evensidemargin}
\addtolength{\leftmargin}{\tmplength}
\settowidth{\labelsep}{X}
\addtolength{\leftmargin}{\labelsep}
\setlength{\labelwidth}{\tmplength}
}
\item[\textbf{Declaration}\hfill]
\ifpdf
\begin{flushleft}
\fi
\begin{ttfamily}
EmptyRawDescriptionInfo: TRawDescriptionInfo =
  ( Content: ''; StreamName: ''; BeginPosition: -1; EndPosition: -1; );\end{ttfamily}

\ifpdf
\end{flushleft}
\fi

\end{list}
\section{Authors}
\par
Johannes Berg {$<$}johannes@sipsolutions.de{$>$}

\par
Ralf Junker (delphi@zeitungsjunge.de)

\par
Marco Schmidt (marcoschmidt@geocities.com)

\par
Michalis Kamburelis

\par
Richard B. Winston {$<$}rbwinst@usgs.gov{$>$}

\par
Damien Honeyford

\par
Arno Garrels {$<$}first name.name@nospamgmx.de{$>$}

\section{Created}
\par
11 Mar 1999


\chapter{Unit PasDoc{\_}Languages}
\label{PasDoc_Languages}
\index{PasDoc{\_}Languages}
\section{Description}
PasDoc language definitions and translations.\hfill\vspace*{1ex}

                            
\section{Overview}
\begin{description}
\item[\texttt{\begin{ttfamily}TLanguageRecord\end{ttfamily} Record}]
\item[\texttt{\begin{ttfamily}TPasDocLanguages\end{ttfamily} Class}]Language class to hold all translated strings
\end{description}
\begin{description}
\item[\texttt{LanguageFromIndex}]Full language name
\item[\texttt{LanguageFromID}]
\item[\texttt{SyntaxFromIndex}]Language abbreviation
\item[\texttt{SyntaxFromID}]
\item[\texttt{IDfromLanguage}]Search for language by short or long name
\item[\texttt{Translation}]Manual translation of id into lang
\item[\texttt{LanguageFromStr}]Find a language with Syntax = S (case ignored).
\item[\texttt{LanguageDescriptor}]access LANGUAGE{\_}ARRAY
\item[\texttt{LanguageCode}]Language code, using an official standardardized language names, suitable for Aspell or HTML.
\end{description}
\section{Classes, Interfaces, Objects and Records}
\ifpdf
\subsection*{\large{\textbf{TLanguageRecord Record}}\normalsize\hspace{1ex}\hrulefill}
\else
\subsection*{TLanguageRecord Record}
\fi
\label{PasDoc_Languages.TLanguageRecord}
\index{TLanguageRecord}
%%%%Description
\subsubsection*{\large{\textbf{Fields}}\normalsize\hspace{1ex}\hfill}
\begin{list}{}{
\settowidth{\tmplength}{\textbf{AspellLanguage}}
\setlength{\itemindent}{0cm}
\setlength{\listparindent}{0cm}
\setlength{\leftmargin}{\evensidemargin}
\addtolength{\leftmargin}{\tmplength}
\settowidth{\labelsep}{X}
\addtolength{\leftmargin}{\labelsep}
\setlength{\labelwidth}{\tmplength}
}
\label{PasDoc_Languages.TLanguageRecord-Table}
\index{Table}
\item[\textbf{Table}\hfill]
\ifpdf
\begin{flushleft}
\fi
\begin{ttfamily}
public Table: PTransTable;\end{ttfamily}

\ifpdf
\end{flushleft}
\fi


\par  \label{PasDoc_Languages.TLanguageRecord-Name}
\index{Name}
\item[\textbf{Name}\hfill]
\ifpdf
\begin{flushleft}
\fi
\begin{ttfamily}
public Name: string;\end{ttfamily}

\ifpdf
\end{flushleft}
\fi


\par  \label{PasDoc_Languages.TLanguageRecord-Syntax}
\index{Syntax}
\item[\textbf{Syntax}\hfill]
\ifpdf
\begin{flushleft}
\fi
\begin{ttfamily}
public Syntax: string;\end{ttfamily}

\ifpdf
\end{flushleft}
\fi


\par  \label{PasDoc_Languages.TLanguageRecord-CharSet}
\index{CharSet}
\item[\textbf{CharSet}\hfill]
\ifpdf
\begin{flushleft}
\fi
\begin{ttfamily}
public CharSet: string;\end{ttfamily}

\ifpdf
\end{flushleft}
\fi


\par  \label{PasDoc_Languages.TLanguageRecord-AspellLanguage}
\index{AspellLanguage}
\item[\textbf{AspellLanguage}\hfill]
\ifpdf
\begin{flushleft}
\fi
\begin{ttfamily}
public AspellLanguage: string;\end{ttfamily}

\ifpdf
\end{flushleft}
\fi


\par Name of this language as used by Aspell, see \href{http://aspell.net/man-html/Supported.html}{http://aspell.net/man-html/Supported.html} .

Set this to empty string if it's the same as our Syntax up to a dot. So a Syntax = 'pl' or Syntax = 'pl.iso{-}8859{-}2' already indicates AspellLanguage = 'pl'.

TODO: In the future, it would be nice if all language names used by PasDoc and Aspell matched. Aspell language naming follows the standard \href{http://en.wikipedia.org/wiki/ISO_639-1}{http://en.wikipedia.org/wiki/ISO{\_}639-1} as far as I see, and we should probably follow it too (currently, we deviate for some languages).

So in the future, we'll probably replace Syntax and AspellLanguage by LanguageCode and CharsetCode. LanguageCode = code (suitable for both PasDoc and Aspell command{-}line; the thing currently up to a dot in Syntax), CharsetCode = the short representation of CharSet (the thing currently after a dot in Syntax).\end{list}
\ifpdf
\subsection*{\large{\textbf{TPasDocLanguages Class}}\normalsize\hspace{1ex}\hrulefill}
\else
\subsection*{TPasDocLanguages Class}
\fi
\label{PasDoc_Languages.TPasDocLanguages}
\index{TPasDocLanguages}
\subsubsection*{\large{\textbf{Hierarchy}}\normalsize\hspace{1ex}\hfill}
TPasDocLanguages {$>$} TObject
\subsubsection*{\large{\textbf{Description}}\normalsize\hspace{1ex}\hfill}
Language class to hold all translated strings\subsubsection*{\large{\textbf{Properties}}\normalsize\hspace{1ex}\hfill}
\begin{list}{}{
\settowidth{\tmplength}{\textbf{Translation}}
\setlength{\itemindent}{0cm}
\setlength{\listparindent}{0cm}
\setlength{\leftmargin}{\evensidemargin}
\addtolength{\leftmargin}{\tmplength}
\settowidth{\labelsep}{X}
\addtolength{\leftmargin}{\labelsep}
\setlength{\labelwidth}{\tmplength}
}
\label{PasDoc_Languages.TPasDocLanguages-CharSet}
\index{CharSet}
\item[\textbf{CharSet}\hfill]
\ifpdf
\begin{flushleft}
\fi
\begin{ttfamily}
public property CharSet: string read FCharSet;\end{ttfamily}

\ifpdf
\end{flushleft}
\fi


\par Charset for current language\label{PasDoc_Languages.TPasDocLanguages-Translation}
\index{Translation}
\item[\textbf{Translation}\hfill]
\ifpdf
\begin{flushleft}
\fi
\begin{ttfamily}
public property Translation[ATranslationID:TTranslationID]: string read GetTranslation;\end{ttfamily}

\ifpdf
\end{flushleft}
\fi


\par  \label{PasDoc_Languages.TPasDocLanguages-Language}
\index{Language}
\item[\textbf{Language}\hfill]
\ifpdf
\begin{flushleft}
\fi
\begin{ttfamily}
public property Language: TLanguageID read FLanguage write SetLanguage
      default DEFAULT{\_}LANGUAGE;\end{ttfamily}

\ifpdf
\end{flushleft}
\fi


\par  \end{list}
\subsubsection*{\large{\textbf{Fields}}\normalsize\hspace{1ex}\hfill}
\begin{list}{}{
\settowidth{\tmplength}{\textbf{FCharSet}}
\setlength{\itemindent}{0cm}
\setlength{\listparindent}{0cm}
\setlength{\leftmargin}{\evensidemargin}
\addtolength{\leftmargin}{\tmplength}
\settowidth{\labelsep}{X}
\addtolength{\leftmargin}{\labelsep}
\setlength{\labelwidth}{\tmplength}
}
\label{PasDoc_Languages.TPasDocLanguages-FCharSet}
\index{FCharSet}
\item[\textbf{FCharSet}\hfill]
\ifpdf
\begin{flushleft}
\fi
\begin{ttfamily}
protected FCharSet: string;\end{ttfamily}

\ifpdf
\end{flushleft}
\fi


\par  \end{list}
\subsubsection*{\large{\textbf{Methods}}\normalsize\hspace{1ex}\hfill}
\paragraph*{GetTranslation}\hspace*{\fill}

\label{PasDoc_Languages.TPasDocLanguages-GetTranslation}
\index{GetTranslation}
\begin{list}{}{
\settowidth{\tmplength}{\textbf{Description}}
\setlength{\itemindent}{0cm}
\setlength{\listparindent}{0cm}
\setlength{\leftmargin}{\evensidemargin}
\addtolength{\leftmargin}{\tmplength}
\settowidth{\labelsep}{X}
\addtolength{\leftmargin}{\labelsep}
\setlength{\labelwidth}{\tmplength}
}
\item[\textbf{Declaration}\hfill]
\ifpdf
\begin{flushleft}
\fi
\begin{ttfamily}
protected function GetTranslation(ATranslationID: TTranslationID): string;\end{ttfamily}

\ifpdf
\end{flushleft}
\fi

\par
\item[\textbf{Description}]
gets a translation token

\end{list}
\paragraph*{Create}\hspace*{\fill}

\label{PasDoc_Languages.TPasDocLanguages-Create}
\index{Create}
\begin{list}{}{
\settowidth{\tmplength}{\textbf{Description}}
\setlength{\itemindent}{0cm}
\setlength{\listparindent}{0cm}
\setlength{\leftmargin}{\evensidemargin}
\addtolength{\leftmargin}{\tmplength}
\settowidth{\labelsep}{X}
\addtolength{\leftmargin}{\labelsep}
\setlength{\labelwidth}{\tmplength}
}
\item[\textbf{Declaration}\hfill]
\ifpdf
\begin{flushleft}
\fi
\begin{ttfamily}
public constructor Create;\end{ttfamily}

\ifpdf
\end{flushleft}
\fi

\end{list}
\section{Functions and Procedures}
\ifpdf
\subsection*{\large{\textbf{LanguageFromIndex}}\normalsize\hspace{1ex}\hrulefill}
\else
\subsection*{LanguageFromIndex}
\fi
\label{PasDoc_Languages-LanguageFromIndex}
\index{LanguageFromIndex}
\begin{list}{}{
\settowidth{\tmplength}{\textbf{Description}}
\setlength{\itemindent}{0cm}
\setlength{\listparindent}{0cm}
\setlength{\leftmargin}{\evensidemargin}
\addtolength{\leftmargin}{\tmplength}
\settowidth{\labelsep}{X}
\addtolength{\leftmargin}{\labelsep}
\setlength{\labelwidth}{\tmplength}
}
\item[\textbf{Declaration}\hfill]
\ifpdf
\begin{flushleft}
\fi
\begin{ttfamily}
function LanguageFromIndex(i: integer): string;\end{ttfamily}

\ifpdf
\end{flushleft}
\fi

\par
\item[\textbf{Description}]
Full language name

\end{list}
\ifpdf
\subsection*{\large{\textbf{LanguageFromID}}\normalsize\hspace{1ex}\hrulefill}
\else
\subsection*{LanguageFromID}
\fi
\label{PasDoc_Languages-LanguageFromID}
\index{LanguageFromID}
\begin{list}{}{
\settowidth{\tmplength}{\textbf{Description}}
\setlength{\itemindent}{0cm}
\setlength{\listparindent}{0cm}
\setlength{\leftmargin}{\evensidemargin}
\addtolength{\leftmargin}{\tmplength}
\settowidth{\labelsep}{X}
\addtolength{\leftmargin}{\labelsep}
\setlength{\labelwidth}{\tmplength}
}
\item[\textbf{Declaration}\hfill]
\ifpdf
\begin{flushleft}
\fi
\begin{ttfamily}
function LanguageFromID(i: TLanguageID): string;\end{ttfamily}

\ifpdf
\end{flushleft}
\fi

\end{list}
\ifpdf
\subsection*{\large{\textbf{SyntaxFromIndex}}\normalsize\hspace{1ex}\hrulefill}
\else
\subsection*{SyntaxFromIndex}
\fi
\label{PasDoc_Languages-SyntaxFromIndex}
\index{SyntaxFromIndex}
\begin{list}{}{
\settowidth{\tmplength}{\textbf{Description}}
\setlength{\itemindent}{0cm}
\setlength{\listparindent}{0cm}
\setlength{\leftmargin}{\evensidemargin}
\addtolength{\leftmargin}{\tmplength}
\settowidth{\labelsep}{X}
\addtolength{\leftmargin}{\labelsep}
\setlength{\labelwidth}{\tmplength}
}
\item[\textbf{Declaration}\hfill]
\ifpdf
\begin{flushleft}
\fi
\begin{ttfamily}
function SyntaxFromIndex(i: integer): string;\end{ttfamily}

\ifpdf
\end{flushleft}
\fi

\par
\item[\textbf{Description}]
Language abbreviation

\end{list}
\ifpdf
\subsection*{\large{\textbf{SyntaxFromID}}\normalsize\hspace{1ex}\hrulefill}
\else
\subsection*{SyntaxFromID}
\fi
\label{PasDoc_Languages-SyntaxFromID}
\index{SyntaxFromID}
\begin{list}{}{
\settowidth{\tmplength}{\textbf{Description}}
\setlength{\itemindent}{0cm}
\setlength{\listparindent}{0cm}
\setlength{\leftmargin}{\evensidemargin}
\addtolength{\leftmargin}{\tmplength}
\settowidth{\labelsep}{X}
\addtolength{\leftmargin}{\labelsep}
\setlength{\labelwidth}{\tmplength}
}
\item[\textbf{Declaration}\hfill]
\ifpdf
\begin{flushleft}
\fi
\begin{ttfamily}
function SyntaxFromID(i: TLanguageID): string;\end{ttfamily}

\ifpdf
\end{flushleft}
\fi

\end{list}
\ifpdf
\subsection*{\large{\textbf{IDfromLanguage}}\normalsize\hspace{1ex}\hrulefill}
\else
\subsection*{IDfromLanguage}
\fi
\label{PasDoc_Languages-IDfromLanguage}
\index{IDfromLanguage}
\begin{list}{}{
\settowidth{\tmplength}{\textbf{Description}}
\setlength{\itemindent}{0cm}
\setlength{\listparindent}{0cm}
\setlength{\leftmargin}{\evensidemargin}
\addtolength{\leftmargin}{\tmplength}
\settowidth{\labelsep}{X}
\addtolength{\leftmargin}{\labelsep}
\setlength{\labelwidth}{\tmplength}
}
\item[\textbf{Declaration}\hfill]
\ifpdf
\begin{flushleft}
\fi
\begin{ttfamily}
function IDfromLanguage(const s: string): TLanguageID;\end{ttfamily}

\ifpdf
\end{flushleft}
\fi

\par
\item[\textbf{Description}]
Search for language by short or long name

\end{list}
\ifpdf
\subsection*{\large{\textbf{Translation}}\normalsize\hspace{1ex}\hrulefill}
\else
\subsection*{Translation}
\fi
\label{PasDoc_Languages-Translation}
\index{Translation}
\begin{list}{}{
\settowidth{\tmplength}{\textbf{Description}}
\setlength{\itemindent}{0cm}
\setlength{\listparindent}{0cm}
\setlength{\leftmargin}{\evensidemargin}
\addtolength{\leftmargin}{\tmplength}
\settowidth{\labelsep}{X}
\addtolength{\leftmargin}{\labelsep}
\setlength{\labelwidth}{\tmplength}
}
\item[\textbf{Declaration}\hfill]
\ifpdf
\begin{flushleft}
\fi
\begin{ttfamily}
function Translation(id: TTranslationID; lang: TLanguageID): string;\end{ttfamily}

\ifpdf
\end{flushleft}
\fi

\par
\item[\textbf{Description}]
Manual translation of id into lang

\end{list}
\ifpdf
\subsection*{\large{\textbf{LanguageFromStr}}\normalsize\hspace{1ex}\hrulefill}
\else
\subsection*{LanguageFromStr}
\fi
\label{PasDoc_Languages-LanguageFromStr}
\index{LanguageFromStr}
\begin{list}{}{
\settowidth{\tmplength}{\textbf{Description}}
\setlength{\itemindent}{0cm}
\setlength{\listparindent}{0cm}
\setlength{\leftmargin}{\evensidemargin}
\addtolength{\leftmargin}{\tmplength}
\settowidth{\labelsep}{X}
\addtolength{\leftmargin}{\labelsep}
\setlength{\labelwidth}{\tmplength}
}
\item[\textbf{Declaration}\hfill]
\ifpdf
\begin{flushleft}
\fi
\begin{ttfamily}
function LanguageFromStr(S: string; out LanguageId: TLanguageID): boolean;\end{ttfamily}

\ifpdf
\end{flushleft}
\fi

\par
\item[\textbf{Description}]
Find a language with Syntax = S (case ignored). Returns \begin{ttfamily}True\end{ttfamily} and sets LanguageId if found, otherwise returns \begin{ttfamily}False\end{ttfamily}.

\end{list}
\ifpdf
\subsection*{\large{\textbf{LanguageDescriptor}}\normalsize\hspace{1ex}\hrulefill}
\else
\subsection*{LanguageDescriptor}
\fi
\label{PasDoc_Languages-LanguageDescriptor}
\index{LanguageDescriptor}
\begin{list}{}{
\settowidth{\tmplength}{\textbf{Description}}
\setlength{\itemindent}{0cm}
\setlength{\listparindent}{0cm}
\setlength{\leftmargin}{\evensidemargin}
\addtolength{\leftmargin}{\tmplength}
\settowidth{\labelsep}{X}
\addtolength{\leftmargin}{\labelsep}
\setlength{\labelwidth}{\tmplength}
}
\item[\textbf{Declaration}\hfill]
\ifpdf
\begin{flushleft}
\fi
\begin{ttfamily}
function LanguageDescriptor(id: TLanguageID): PLanguageRecord;\end{ttfamily}

\ifpdf
\end{flushleft}
\fi

\par
\item[\textbf{Description}]
access LANGUAGE{\_}ARRAY

\end{list}
\ifpdf
\subsection*{\large{\textbf{LanguageCode}}\normalsize\hspace{1ex}\hrulefill}
\else
\subsection*{LanguageCode}
\fi
\label{PasDoc_Languages-LanguageCode}
\index{LanguageCode}
\begin{list}{}{
\settowidth{\tmplength}{\textbf{Description}}
\setlength{\itemindent}{0cm}
\setlength{\listparindent}{0cm}
\setlength{\leftmargin}{\evensidemargin}
\addtolength{\leftmargin}{\tmplength}
\settowidth{\labelsep}{X}
\addtolength{\leftmargin}{\labelsep}
\setlength{\labelwidth}{\tmplength}
}
\item[\textbf{Declaration}\hfill]
\ifpdf
\begin{flushleft}
\fi
\begin{ttfamily}
function LanguageCode(const Language: TLanguageID): string;\end{ttfamily}

\ifpdf
\end{flushleft}
\fi

\par
\item[\textbf{Description}]
Language code, using an official standardardized language names, suitable for Aspell or HTML.

\end{list}
\section{Types}
\ifpdf
\subsection*{\large{\textbf{TLanguageID}}\normalsize\hspace{1ex}\hrulefill}
\else
\subsection*{TLanguageID}
\fi
\label{PasDoc_Languages-TLanguageID}
\index{TLanguageID}
\begin{list}{}{
\settowidth{\tmplength}{\textbf{Description}}
\setlength{\itemindent}{0cm}
\setlength{\listparindent}{0cm}
\setlength{\leftmargin}{\evensidemargin}
\addtolength{\leftmargin}{\tmplength}
\settowidth{\labelsep}{X}
\addtolength{\leftmargin}{\labelsep}
\setlength{\labelwidth}{\tmplength}
}
\item[\textbf{Declaration}\hfill]
\ifpdf
\begin{flushleft}
\fi
\begin{ttfamily}
TLanguageID = (...);\end{ttfamily}

\ifpdf
\end{flushleft}
\fi

\par
\item[\textbf{Description}]
An enumeration type of all supported languages\item[\textbf{Values}]
\begin{description}
\item[\texttt{lgBosnian}] \label{PasDoc_Languages-lgBosnian}
\index{}
 
\item[\texttt{lgBrazilian{\_}1252}] \label{PasDoc_Languages-lgBrazilian_1252}
\index{}
 
\item[\texttt{lgBrazilian{\_}utf8}] \label{PasDoc_Languages-lgBrazilian_utf8}
\index{}
 
\item[\texttt{lgBulgarian}] \label{PasDoc_Languages-lgBulgarian}
\index{}
 
\item[\texttt{lgCatalan}] \label{PasDoc_Languages-lgCatalan}
\index{}
 
\item[\texttt{lgChinese{\_}gb2312}] \label{PasDoc_Languages-lgChinese_gb2312}
\index{}
 
\item[\texttt{lgCroatian}] \label{PasDoc_Languages-lgCroatian}
\index{}
 
\item[\texttt{lgDanish}] \label{PasDoc_Languages-lgDanish}
\index{}
 
\item[\texttt{lgDutch}] \label{PasDoc_Languages-lgDutch}
\index{}
 
\item[\texttt{lgEnglish}] \label{PasDoc_Languages-lgEnglish}
\index{}
 
\item[\texttt{lgFrench{\_}ISO{\_}8859{\_}15}] \label{PasDoc_Languages-lgFrench_ISO_8859_15}
\index{}
 
\item[\texttt{lgFrench{\_}UTF{\_}8}] \label{PasDoc_Languages-lgFrench_UTF_8}
\index{}
 
\item[\texttt{lgGerman{\_}ISO{\_}8859{\_}15}] \label{PasDoc_Languages-lgGerman_ISO_8859_15}
\index{}
 
\item[\texttt{lgGerman{\_}UTF{\_}8}] \label{PasDoc_Languages-lgGerman_UTF_8}
\index{}
 
\item[\texttt{lgIndonesian}] \label{PasDoc_Languages-lgIndonesian}
\index{}
 
\item[\texttt{lgItalian}] \label{PasDoc_Languages-lgItalian}
\index{}
 
\item[\texttt{lgJavanese}] \label{PasDoc_Languages-lgJavanese}
\index{}
 
\item[\texttt{lgPolish{\_}CP1250}] \label{PasDoc_Languages-lgPolish_CP1250}
\index{}
 
\item[\texttt{lgPolish{\_}ISO{\_}8859{\_}2}] \label{PasDoc_Languages-lgPolish_ISO_8859_2}
\index{}
 
\item[\texttt{lgRussian{\_}1251}] \label{PasDoc_Languages-lgRussian_1251}
\index{}
 
\item[\texttt{lgRussian{\_}utf8}] \label{PasDoc_Languages-lgRussian_utf8}
\index{}
 
\item[\texttt{lgRussian{\_}866}] \label{PasDoc_Languages-lgRussian_866}
\index{}
 
\item[\texttt{lgRussian{\_}koi8}] \label{PasDoc_Languages-lgRussian_koi8}
\index{}
 
\item[\texttt{lgSlovak}] \label{PasDoc_Languages-lgSlovak}
\index{}
 
\item[\texttt{lgSpanish}] \label{PasDoc_Languages-lgSpanish}
\index{}
 
\item[\texttt{lgSwedish}] \label{PasDoc_Languages-lgSwedish}
\index{}
 
\item[\texttt{lgHungarian{\_}1250}] \label{PasDoc_Languages-lgHungarian_1250}
\index{}
 
\item[\texttt{lgCzech{\_}CP1250}] \label{PasDoc_Languages-lgCzech_CP1250}
\index{}
 
\item[\texttt{lgCzech{\_}ISO{\_}8859{\_}2}] \label{PasDoc_Languages-lgCzech_ISO_8859_2}
\index{}
 
\end{description}


\end{list}
\ifpdf
\subsection*{\large{\textbf{TTranslationID}}\normalsize\hspace{1ex}\hrulefill}
\else
\subsection*{TTranslationID}
\fi
\label{PasDoc_Languages-TTranslationID}
\index{TTranslationID}
\begin{list}{}{
\settowidth{\tmplength}{\textbf{Description}}
\setlength{\itemindent}{0cm}
\setlength{\listparindent}{0cm}
\setlength{\leftmargin}{\evensidemargin}
\addtolength{\leftmargin}{\tmplength}
\settowidth{\labelsep}{X}
\addtolength{\leftmargin}{\labelsep}
\setlength{\labelwidth}{\tmplength}
}
\item[\textbf{Declaration}\hfill]
\ifpdf
\begin{flushleft}
\fi
\begin{ttfamily}
TTranslationID = (...);\end{ttfamily}

\ifpdf
\end{flushleft}
\fi

\par
\item[\textbf{Description}]
An enumeration type of all static output texts. Warning: count and order changed!\item[\textbf{Values}]
\begin{description}
\item[\texttt{trNoTrans}] \label{PasDoc_Languages-trNoTrans}
\index{}
no translation ID assigned, so far
\item[\texttt{trLanguage}] \label{PasDoc_Languages-trLanguage}
\index{}
the language name (English, ASCII), e.g. for file names.
\item[\texttt{trUnits}] \label{PasDoc_Languages-trUnits}
\index{}
map
\item[\texttt{trClassHierarchy}] \label{PasDoc_Languages-trClassHierarchy}
\index{}
 
\item[\texttt{trCio}] \label{PasDoc_Languages-trCio}
\index{}
 
\item[\texttt{trNestedCR}] \label{PasDoc_Languages-trNestedCR}
\index{}
 
\item[\texttt{trNestedTypes}] \label{PasDoc_Languages-trNestedTypes}
\index{}
 
\item[\texttt{trIdentifiers}] \label{PasDoc_Languages-trIdentifiers}
\index{}
 
\item[\texttt{trGvUses}] \label{PasDoc_Languages-trGvUses}
\index{}
 
\item[\texttt{trGvClasses}] \label{PasDoc_Languages-trGvClasses}
\index{}
 
\item[\texttt{trClasses}] \label{PasDoc_Languages-trClasses}
\index{}
tables and members
\item[\texttt{trClass}] \label{PasDoc_Languages-trClass}
\index{}
 
\item[\texttt{trDispInterface}] \label{PasDoc_Languages-trDispInterface}
\index{}
 
\item[\texttt{trInterface}] \label{PasDoc_Languages-trInterface}
\index{}
 
\item[\texttt{trObjects}] \label{PasDoc_Languages-trObjects}
\index{}
 
\item[\texttt{trObject}] \label{PasDoc_Languages-trObject}
\index{}
 
\item[\texttt{trRecord}] \label{PasDoc_Languages-trRecord}
\index{}
 
\item[\texttt{trPacked}] \label{PasDoc_Languages-trPacked}
\index{}
 
\item[\texttt{trHierarchy}] \label{PasDoc_Languages-trHierarchy}
\index{}
 
\item[\texttt{trFields}] \label{PasDoc_Languages-trFields}
\index{}
 
\item[\texttt{trMethods}] \label{PasDoc_Languages-trMethods}
\index{}
 
\item[\texttt{trProperties}] \label{PasDoc_Languages-trProperties}
\index{}
 
\item[\texttt{trLibrary}] \label{PasDoc_Languages-trLibrary}
\index{}
 
\item[\texttt{trPackage}] \label{PasDoc_Languages-trPackage}
\index{}
 
\item[\texttt{trProgram}] \label{PasDoc_Languages-trProgram}
\index{}
 
\item[\texttt{trUnit}] \label{PasDoc_Languages-trUnit}
\index{}
 
\item[\texttt{trUses}] \label{PasDoc_Languages-trUses}
\index{}
 
\item[\texttt{trConstants}] \label{PasDoc_Languages-trConstants}
\index{}
 
\item[\texttt{trFunctionsAndProcedures}] \label{PasDoc_Languages-trFunctionsAndProcedures}
\index{}
 
\item[\texttt{trTypes}] \label{PasDoc_Languages-trTypes}
\index{}
 
\item[\texttt{trType}] \label{PasDoc_Languages-trType}
\index{}
 
\item[\texttt{trVariables}] \label{PasDoc_Languages-trVariables}
\index{}
 
\item[\texttt{trAuthors}] \label{PasDoc_Languages-trAuthors}
\index{}
 
\item[\texttt{trAuthor}] \label{PasDoc_Languages-trAuthor}
\index{}
 
\item[\texttt{trCreated}] \label{PasDoc_Languages-trCreated}
\index{}
 
\item[\texttt{trLastModified}] \label{PasDoc_Languages-trLastModified}
\index{}
 
\item[\texttt{trSubroutine}] \label{PasDoc_Languages-trSubroutine}
\index{}
 
\item[\texttt{trParameters}] \label{PasDoc_Languages-trParameters}
\index{}
 
\item[\texttt{trReturns}] \label{PasDoc_Languages-trReturns}
\index{}
 
\item[\texttt{trExceptionsRaised}] \label{PasDoc_Languages-trExceptionsRaised}
\index{}
 
\item[\texttt{trExceptions}] \label{PasDoc_Languages-trExceptions}
\index{}
 
\item[\texttt{trException}] \label{PasDoc_Languages-trException}
\index{}
 
\item[\texttt{trEnum}] \label{PasDoc_Languages-trEnum}
\index{}
 
\item[\texttt{trVisibility}] \label{PasDoc_Languages-trVisibility}
\index{}
visibilities
\item[\texttt{trPrivate}] \label{PasDoc_Languages-trPrivate}
\index{}
 
\item[\texttt{trStrictPrivate}] \label{PasDoc_Languages-trStrictPrivate}
\index{}
 
\item[\texttt{trProtected}] \label{PasDoc_Languages-trProtected}
\index{}
 
\item[\texttt{trStrictProtected}] \label{PasDoc_Languages-trStrictProtected}
\index{}
 
\item[\texttt{trPublic}] \label{PasDoc_Languages-trPublic}
\index{}
 
\item[\texttt{trPublished}] \label{PasDoc_Languages-trPublished}
\index{}
 
\item[\texttt{trAutomated}] \label{PasDoc_Languages-trAutomated}
\index{}
 
\item[\texttt{trImplicit}] \label{PasDoc_Languages-trImplicit}
\index{}
 
\item[\texttt{trDeprecated}] \label{PasDoc_Languages-trDeprecated}
\index{}
hints
\item[\texttt{trPlatformSpecific}] \label{PasDoc_Languages-trPlatformSpecific}
\index{}
 
\item[\texttt{trLibrarySpecific}] \label{PasDoc_Languages-trLibrarySpecific}
\index{}
 
\item[\texttt{trExperimental}] \label{PasDoc_Languages-trExperimental}
\index{}
 
\item[\texttt{trOverview}] \label{PasDoc_Languages-trOverview}
\index{}
headings
\item[\texttt{trIntroduction}] \label{PasDoc_Languages-trIntroduction}
\index{}
 
\item[\texttt{trConclusion}] \label{PasDoc_Languages-trConclusion}
\index{}
 
\item[\texttt{trAdditionalFile}] \label{PasDoc_Languages-trAdditionalFile}
\index{}
 
\item[\texttt{trEnclosingClass}] \label{PasDoc_Languages-trEnclosingClass}
\index{}
 
\item[\texttt{trHeadlineCio}] \label{PasDoc_Languages-trHeadlineCio}
\index{}
 
\item[\texttt{trHeadlineConstants}] \label{PasDoc_Languages-trHeadlineConstants}
\index{}
 
\item[\texttt{trHeadlineFunctionsAndProcedures}] \label{PasDoc_Languages-trHeadlineFunctionsAndProcedures}
\index{}
 
\item[\texttt{trHeadlineIdentifiers}] \label{PasDoc_Languages-trHeadlineIdentifiers}
\index{}
 
\item[\texttt{trHeadlineTypes}] \label{PasDoc_Languages-trHeadlineTypes}
\index{}
 
\item[\texttt{trHeadlineUnits}] \label{PasDoc_Languages-trHeadlineUnits}
\index{}
 
\item[\texttt{trHeadlineVariables}] \label{PasDoc_Languages-trHeadlineVariables}
\index{}
 
\item[\texttt{trSummaryCio}] \label{PasDoc_Languages-trSummaryCio}
\index{}
 
\item[\texttt{trDeclaration}] \label{PasDoc_Languages-trDeclaration}
\index{}
column headings
\item[\texttt{trDescription}] \label{PasDoc_Languages-trDescription}
\index{}
as column OR section heading!
\item[\texttt{trDescriptions}] \label{PasDoc_Languages-trDescriptions}
\index{}
section heading for detailed descriptions
\item[\texttt{trName}] \label{PasDoc_Languages-trName}
\index{}
 
\item[\texttt{trValues}] \label{PasDoc_Languages-trValues}
\index{}
 
\item[\texttt{trWarningTag}] \label{PasDoc_Languages-trWarningTag}
\index{}
tags with inbuilt heading
\item[\texttt{trNoteTag}] \label{PasDoc_Languages-trNoteTag}
\index{}
 
\item[\texttt{trNone}] \label{PasDoc_Languages-trNone}
\index{}
empty tables
\item[\texttt{trNoCIOs}] \label{PasDoc_Languages-trNoCIOs}
\index{}
 
\item[\texttt{trNoCIOsForHierarchy}] \label{PasDoc_Languages-trNoCIOsForHierarchy}
\index{}
 
\item[\texttt{trNoTypes}] \label{PasDoc_Languages-trNoTypes}
\index{}
 
\item[\texttt{trNoVariables}] \label{PasDoc_Languages-trNoVariables}
\index{}
 
\item[\texttt{trNoConstants}] \label{PasDoc_Languages-trNoConstants}
\index{}
 
\item[\texttt{trNoFunctions}] \label{PasDoc_Languages-trNoFunctions}
\index{}
 
\item[\texttt{trNoIdentifiers}] \label{PasDoc_Languages-trNoIdentifiers}
\index{}
 
\item[\texttt{trHelp}] \label{PasDoc_Languages-trHelp}
\index{}
misc
\item[\texttt{trLegend}] \label{PasDoc_Languages-trLegend}
\index{}
 
\item[\texttt{trMarker}] \label{PasDoc_Languages-trMarker}
\index{}
 
\item[\texttt{trWarningOverwrite}] \label{PasDoc_Languages-trWarningOverwrite}
\index{}
 
\item[\texttt{trWarning}] \label{PasDoc_Languages-trWarning}
\index{}
 
\item[\texttt{trGeneratedBy}] \label{PasDoc_Languages-trGeneratedBy}
\index{}
 
\item[\texttt{trGeneratedOn}] \label{PasDoc_Languages-trGeneratedOn}
\index{}
 
\item[\texttt{trOnDateTime}] \label{PasDoc_Languages-trOnDateTime}
\index{}
 
\item[\texttt{trSearch}] \label{PasDoc_Languages-trSearch}
\index{}
 
\item[\texttt{trSeeAlso}] \label{PasDoc_Languages-trSeeAlso}
\index{}
 
\item[\texttt{trNested}] \label{PasDoc_Languages-trNested}
\index{}
 
\item[\texttt{trAttributes}] \label{PasDoc_Languages-trAttributes}
\index{}
add more here
\item[\texttt{trDummy}] \label{PasDoc_Languages-trDummy}
\index{}
 
\end{description}


\end{list}
\ifpdf
\subsection*{\large{\textbf{RTransTable}}\normalsize\hspace{1ex}\hrulefill}
\else
\subsection*{RTransTable}
\fi
\label{PasDoc_Languages-RTransTable}
\index{RTransTable}
\begin{list}{}{
\settowidth{\tmplength}{\textbf{Description}}
\setlength{\itemindent}{0cm}
\setlength{\listparindent}{0cm}
\setlength{\leftmargin}{\evensidemargin}
\addtolength{\leftmargin}{\tmplength}
\settowidth{\labelsep}{X}
\addtolength{\leftmargin}{\labelsep}
\setlength{\labelwidth}{\tmplength}
}
\item[\textbf{Declaration}\hfill]
\ifpdf
\begin{flushleft}
\fi
\begin{ttfamily}
RTransTable = array[TTranslationID] of string;\end{ttfamily}

\ifpdf
\end{flushleft}
\fi

\par
\item[\textbf{Description}]
array holding the translated strings, or empty for default (English) text.

\end{list}
\ifpdf
\subsection*{\large{\textbf{PTransTable}}\normalsize\hspace{1ex}\hrulefill}
\else
\subsection*{PTransTable}
\fi
\label{PasDoc_Languages-PTransTable}
\index{PTransTable}
\begin{list}{}{
\settowidth{\tmplength}{\textbf{Description}}
\setlength{\itemindent}{0cm}
\setlength{\listparindent}{0cm}
\setlength{\leftmargin}{\evensidemargin}
\addtolength{\leftmargin}{\tmplength}
\settowidth{\labelsep}{X}
\addtolength{\leftmargin}{\labelsep}
\setlength{\labelwidth}{\tmplength}
}
\item[\textbf{Declaration}\hfill]
\ifpdf
\begin{flushleft}
\fi
\begin{ttfamily}
PTransTable = {\^{}}RTransTable;\end{ttfamily}

\ifpdf
\end{flushleft}
\fi

\end{list}
\ifpdf
\subsection*{\large{\textbf{PLanguageRecord}}\normalsize\hspace{1ex}\hrulefill}
\else
\subsection*{PLanguageRecord}
\fi
\label{PasDoc_Languages-PLanguageRecord}
\index{PLanguageRecord}
\begin{list}{}{
\settowidth{\tmplength}{\textbf{Description}}
\setlength{\itemindent}{0cm}
\setlength{\listparindent}{0cm}
\setlength{\leftmargin}{\evensidemargin}
\addtolength{\leftmargin}{\tmplength}
\settowidth{\labelsep}{X}
\addtolength{\leftmargin}{\labelsep}
\setlength{\labelwidth}{\tmplength}
}
\item[\textbf{Declaration}\hfill]
\ifpdf
\begin{flushleft}
\fi
\begin{ttfamily}
PLanguageRecord = {\^{}}TLanguageRecord;\end{ttfamily}

\ifpdf
\end{flushleft}
\fi

\par
\item[\textbf{Description}]
language descriptor

\end{list}
\section{Constants}
\ifpdf
\subsection*{\large{\textbf{DEFAULT{\_}LANGUAGE}}\normalsize\hspace{1ex}\hrulefill}
\else
\subsection*{DEFAULT{\_}LANGUAGE}
\fi
\label{PasDoc_Languages-DEFAULT_LANGUAGE}
\index{DEFAULT{\_}LANGUAGE}
\begin{list}{}{
\settowidth{\tmplength}{\textbf{Description}}
\setlength{\itemindent}{0cm}
\setlength{\listparindent}{0cm}
\setlength{\leftmargin}{\evensidemargin}
\addtolength{\leftmargin}{\tmplength}
\settowidth{\labelsep}{X}
\addtolength{\leftmargin}{\labelsep}
\setlength{\labelwidth}{\tmplength}
}
\item[\textbf{Declaration}\hfill]
\ifpdf
\begin{flushleft}
\fi
\begin{ttfamily}
DEFAULT{\_}LANGUAGE = lgEnglish;\end{ttfamily}

\ifpdf
\end{flushleft}
\fi

\end{list}
\ifpdf
\subsection*{\large{\textbf{lgDefault}}\normalsize\hspace{1ex}\hrulefill}
\else
\subsection*{lgDefault}
\fi
\label{PasDoc_Languages-lgDefault}
\index{lgDefault}
\begin{list}{}{
\settowidth{\tmplength}{\textbf{Description}}
\setlength{\itemindent}{0cm}
\setlength{\listparindent}{0cm}
\setlength{\leftmargin}{\evensidemargin}
\addtolength{\leftmargin}{\tmplength}
\settowidth{\labelsep}{X}
\addtolength{\leftmargin}{\labelsep}
\setlength{\labelwidth}{\tmplength}
}
\item[\textbf{Declaration}\hfill]
\ifpdf
\begin{flushleft}
\fi
\begin{ttfamily}
lgDefault = lgEnglish;\end{ttfamily}

\ifpdf
\end{flushleft}
\fi

\end{list}
\section{Authors}
\par
Johannes Berg {$<$}johannes AT sipsolutions.de{$>$}

\par
Ralf Junker {$<$}delphi AT zeitungsjunge.de{$>$}

\par
Andrew Andreev {$<$}andrew AT alteragate.net{$>$} (Bulgarian translation)

\par
Alexander Lisnevsky {$<$}alisnevsky AT yandex.ru{$>$} (Russian translation)

\par
Hendy Irawan {$<$}ceefour AT gauldong.net{$>$} (Indonesian and Javanese translation)

\par
Ivan Montes Velencoso (Catalan and Spanish translations)

\par
Javi (Spanish translation)

\par
Jean Dit Bailleul (Frensh translation)

\par
Marc Weustinks (Dutch translation)

\par
Martin Hansen {$<$}mh AT geus.dk{$>$} (Danish translation)

\par
Michele Bersini {$<$}michele.bersini AT smartit.it{$>$} (Italian translation)

\par
Peter Simkovic {$<$}simkovic{\_}jr AT manal.sk{$>$} (Slovak translation)

\par
Peter Th{\_}rnqvist {$<$}pt AT timemetrics.se{$>$} (Swedish translation)

\par
Rodrigo Urubatan Ferreira Jardim {$<$}rodrigo AT netscape.net{$>$} (Brasilian translation)

\par
Alexandre da Silva {$<$}simpsomboy AT gmail.com{$>$} (Brasilian translation - Update)

\par
Alexsander da Rosa {$<$}alex AT rednaxel.com{$>$} (Brasilian translation - UTF8)

\par
Vitaly Kovalenko {$<$}v{\_}l{\_}kovalenko AT alsy.by{$>$} (Russian translation)

\par
Grzegorz Skoczylas {$<$}gskoczylas AT rekord.pl{$>$} (corrected Polish translation)

\par
Jonas Gergo {$<$}jonas.gergo AT ch...{$>$} (Hungarian translation)

\par
Michalis Kamburelis

\par
Ascanio Pressato (Some Italian translation)

\par
JBarbero Quiter (updated Spanish translation)

\par
Liu Chuanjun {$<$}1000copy AT gmail.com{$>$} (Chinese gb2312 translation)

\par
Liu Da {$<$}xmacmail AT gmail.com{$>$} (Chinese gb2312 translation)

\par
DoDi

\par
Rene Mihula {$<$}rene.mihula@gmail.com{$>$} (Czech translation)

\par
Yann Merignac (French translation)

\par
Arno Garrels {$<$}first name.name@nospamgmx.de{$>$}

\chapter{Unit PasDoc{\_}Main}
\label{PasDoc_Main}
\index{PasDoc{\_}Main}
\section{Description}
Provides the Main procedure.
\section{Overview}
\begin{description}
\item[\texttt{Main}]This is the main procedure of PasDoc, it does everything.
\end{description}
\section{Functions and Procedures}
\ifpdf
\subsection*{\large{\textbf{Main}}\normalsize\hspace{1ex}\hrulefill}
\else
\subsection*{Main}
\fi
\label{PasDoc_Main-Main}
\index{Main}
\begin{list}{}{
\settowidth{\tmplength}{\textbf{Description}}
\setlength{\itemindent}{0cm}
\setlength{\listparindent}{0cm}
\setlength{\leftmargin}{\evensidemargin}
\addtolength{\leftmargin}{\tmplength}
\settowidth{\labelsep}{X}
\addtolength{\leftmargin}{\labelsep}
\setlength{\labelwidth}{\tmplength}
}
\item[\textbf{Declaration}\hfill]
\ifpdf
\begin{flushleft}
\fi
\begin{ttfamily}
procedure Main;\end{ttfamily}

\ifpdf
\end{flushleft}
\fi

\par
\item[\textbf{Description}]
This is the main procedure of PasDoc, it does everything.

\end{list}
\chapter{Unit PasDoc{\_}ObjectVector}
\label{PasDoc_ObjectVector}
\index{PasDoc{\_}ObjectVector}
\section{Description}
  a simple object vector
\section{Uses}
\begin{itemize}
\item \begin{ttfamily}Contnrs\end{ttfamily}\item \begin{ttfamily}Classes\end{ttfamily}\end{itemize}
\section{Overview}
\begin{description}
\item[\texttt{\begin{ttfamily}TObjectVector\end{ttfamily} Class}]
\end{description}
\begin{description}
\item[\texttt{ObjectVectorIsNilOrEmpty}]
\end{description}
\section{Classes, Interfaces, Objects and Records}
\ifpdf
\subsection*{\large{\textbf{TObjectVector Class}}\normalsize\hspace{1ex}\hrulefill}
\else
\subsection*{TObjectVector Class}
\fi
\label{PasDoc_ObjectVector.TObjectVector}
\index{TObjectVector}
\subsubsection*{\large{\textbf{Hierarchy}}\normalsize\hspace{1ex}\hfill}
TObjectVector {$>$} TObjectList
%%%%Description
\subsubsection*{\large{\textbf{Methods}}\normalsize\hspace{1ex}\hfill}
\paragraph*{Create}\hspace*{\fill}

\label{PasDoc_ObjectVector.TObjectVector-Create}
\index{Create}
\begin{list}{}{
\settowidth{\tmplength}{\textbf{Description}}
\setlength{\itemindent}{0cm}
\setlength{\listparindent}{0cm}
\setlength{\leftmargin}{\evensidemargin}
\addtolength{\leftmargin}{\tmplength}
\settowidth{\labelsep}{X}
\addtolength{\leftmargin}{\labelsep}
\setlength{\labelwidth}{\tmplength}
}
\item[\textbf{Declaration}\hfill]
\ifpdf
\begin{flushleft}
\fi
\begin{ttfamily}
public constructor Create(const AOwnsObject: boolean); virtual;\end{ttfamily}

\ifpdf
\end{flushleft}
\fi

\par
\item[\textbf{Description}]
This is only to make constructor virtual, while original TObjectList has a static constructor.

\end{list}
\section{Functions and Procedures}
\ifpdf
\subsection*{\large{\textbf{ObjectVectorIsNilOrEmpty}}\normalsize\hspace{1ex}\hrulefill}
\else
\subsection*{ObjectVectorIsNilOrEmpty}
\fi
\label{PasDoc_ObjectVector-ObjectVectorIsNilOrEmpty}
\index{ObjectVectorIsNilOrEmpty}
\begin{list}{}{
\settowidth{\tmplength}{\textbf{Description}}
\setlength{\itemindent}{0cm}
\setlength{\listparindent}{0cm}
\setlength{\leftmargin}{\evensidemargin}
\addtolength{\leftmargin}{\tmplength}
\settowidth{\labelsep}{X}
\addtolength{\leftmargin}{\labelsep}
\setlength{\labelwidth}{\tmplength}
}
\item[\textbf{Declaration}\hfill]
\ifpdf
\begin{flushleft}
\fi
\begin{ttfamily}
function ObjectVectorIsNilOrEmpty(const AOV: TObjectVector): boolean;\end{ttfamily}

\ifpdf
\end{flushleft}
\fi

\end{list}
\section{Authors}
\par
Johannes Berg {$<$}johannes@sipsolutions.de{$>$}

\par
Michalis Kamburelis

\chapter{Unit PasDoc{\_}OptionParser}
\label{PasDoc_OptionParser}
\index{PasDoc{\_}OptionParser}
\section{Description}
The \begin{ttfamily}PasDoc{\_}OptionParser\end{ttfamily} unit --- easing command line parsing\hfill\vspace*{1ex}

 

To use this unit, create an object of \begin{ttfamily}TOptionParser\end{ttfamily}(\ref{PasDoc_OptionParser.TOptionParser}) and add options to it, each option descends from \begin{ttfamily}TOption\end{ttfamily}(\ref{PasDoc_OptionParser.TOption}). Then, call your object's \begin{ttfamily}TOptionParser.ParseOptions\end{ttfamily}(\ref{PasDoc_OptionParser.TOptionParser-ParseOptions}) method and options are parsed. After parsing, examine your option objects.
\section{Uses}
\begin{itemize}
\item \begin{ttfamily}Classes\end{ttfamily}\end{itemize}
\section{Overview}
\begin{description}
\item[\texttt{\begin{ttfamily}TOption\end{ttfamily} Class}]abstract base class for options
\item[\texttt{\begin{ttfamily}TBoolOption\end{ttfamily} Class}]simple boolean option
\item[\texttt{\begin{ttfamily}TValueOption\end{ttfamily} Class}]base class for all options that values
\item[\texttt{\begin{ttfamily}TIntegerOption\end{ttfamily} Class}]Integer option
\item[\texttt{\begin{ttfamily}TStringOption\end{ttfamily} Class}]String option
\item[\texttt{\begin{ttfamily}TStringOptionList\end{ttfamily} Class}]stringlist option
\item[\texttt{\begin{ttfamily}TPathListOption\end{ttfamily} Class}]pathlist option
\item[\texttt{\begin{ttfamily}TSetOption\end{ttfamily} Class}]useful for making a choice of things
\item[\texttt{\begin{ttfamily}TOptionParser\end{ttfamily} Class}]OptionParser --- instantiate one of these for commandline parsing
\end{description}
\section{Classes, Interfaces, Objects and Records}
\ifpdf
\subsection*{\large{\textbf{TOption Class}}\normalsize\hspace{1ex}\hrulefill}
\else
\subsection*{TOption Class}
\fi
\label{PasDoc_OptionParser.TOption}
\index{TOption}
\subsubsection*{\large{\textbf{Hierarchy}}\normalsize\hspace{1ex}\hfill}
TOption {$>$} TObject
\subsubsection*{\large{\textbf{Description}}\normalsize\hspace{1ex}\hfill}
abstract base class for options\hfill\vspace*{1ex}

 This class implements all the basic functionality and provides abstract methods for the \begin{ttfamily}TOptionParser\end{ttfamily}(\ref{PasDoc_OptionParser.TOptionParser}) class to call, which are overridden by descendants. It also provides function to write the explanation.\subsubsection*{\large{\textbf{Properties}}\normalsize\hspace{1ex}\hfill}
\begin{list}{}{
\settowidth{\tmplength}{\textbf{ShortCaseSensitive}}
\setlength{\itemindent}{0cm}
\setlength{\listparindent}{0cm}
\setlength{\leftmargin}{\evensidemargin}
\addtolength{\leftmargin}{\tmplength}
\settowidth{\labelsep}{X}
\addtolength{\leftmargin}{\labelsep}
\setlength{\labelwidth}{\tmplength}
}
\label{PasDoc_OptionParser.TOption-ShortForm}
\index{ShortForm}
\item[\textbf{ShortForm}\hfill]
\ifpdf
\begin{flushleft}
\fi
\begin{ttfamily}
public property ShortForm: char read FShort write FShort;\end{ttfamily}

\ifpdf
\end{flushleft}
\fi


\par Short form of the option --- single character --- if {\#}0 then not used\label{PasDoc_OptionParser.TOption-LongForm}
\index{LongForm}
\item[\textbf{LongForm}\hfill]
\ifpdf
\begin{flushleft}
\fi
\begin{ttfamily}
public property LongForm: string read FLong write FLong;\end{ttfamily}

\ifpdf
\end{flushleft}
\fi


\par long form of the option --- string --- if empty, then not used\label{PasDoc_OptionParser.TOption-ShortCaseSensitive}
\index{ShortCaseSensitive}
\item[\textbf{ShortCaseSensitive}\hfill]
\ifpdf
\begin{flushleft}
\fi
\begin{ttfamily}
public property ShortCaseSensitive: boolean read FShortSens write FShortSens;\end{ttfamily}

\ifpdf
\end{flushleft}
\fi


\par specified whether the short form should be case sensitive or not\label{PasDoc_OptionParser.TOption-LongCaseSensitive}
\index{LongCaseSensitive}
\item[\textbf{LongCaseSensitive}\hfill]
\ifpdf
\begin{flushleft}
\fi
\begin{ttfamily}
public property LongCaseSensitive: boolean read FLongSens write FLongSens;\end{ttfamily}

\ifpdf
\end{flushleft}
\fi


\par specifies whether the long form should be case sensitive or not\label{PasDoc_OptionParser.TOption-WasSpecified}
\index{WasSpecified}
\item[\textbf{WasSpecified}\hfill]
\ifpdf
\begin{flushleft}
\fi
\begin{ttfamily}
public property WasSpecified: boolean read FWasSpecified;\end{ttfamily}

\ifpdf
\end{flushleft}
\fi


\par signifies if the option was specified at least once\label{PasDoc_OptionParser.TOption-Explanation}
\index{Explanation}
\item[\textbf{Explanation}\hfill]
\ifpdf
\begin{flushleft}
\fi
\begin{ttfamily}
public property Explanation: string read FExplanation write FExplanation;\end{ttfamily}

\ifpdf
\end{flushleft}
\fi


\par explanation for the option, see also \begin{ttfamily}WriteExplanation\end{ttfamily}(\ref{PasDoc_OptionParser.TOption-WriteExplanation})\end{list}
\subsubsection*{\large{\textbf{Fields}}\normalsize\hspace{1ex}\hfill}
\begin{list}{}{
\settowidth{\tmplength}{\textbf{FWasSpecified}}
\setlength{\itemindent}{0cm}
\setlength{\listparindent}{0cm}
\setlength{\leftmargin}{\evensidemargin}
\addtolength{\leftmargin}{\tmplength}
\settowidth{\labelsep}{X}
\addtolength{\leftmargin}{\labelsep}
\setlength{\labelwidth}{\tmplength}
}
\label{PasDoc_OptionParser.TOption-FShort}
\index{FShort}
\item[\textbf{FShort}\hfill]
\ifpdf
\begin{flushleft}
\fi
\begin{ttfamily}
protected FShort: char;\end{ttfamily}

\ifpdf
\end{flushleft}
\fi


\par  \label{PasDoc_OptionParser.TOption-FLong}
\index{FLong}
\item[\textbf{FLong}\hfill]
\ifpdf
\begin{flushleft}
\fi
\begin{ttfamily}
protected FLong: string;\end{ttfamily}

\ifpdf
\end{flushleft}
\fi


\par  \label{PasDoc_OptionParser.TOption-FShortSens}
\index{FShortSens}
\item[\textbf{FShortSens}\hfill]
\ifpdf
\begin{flushleft}
\fi
\begin{ttfamily}
protected FShortSens: boolean;\end{ttfamily}

\ifpdf
\end{flushleft}
\fi


\par  \label{PasDoc_OptionParser.TOption-FLongSens}
\index{FLongSens}
\item[\textbf{FLongSens}\hfill]
\ifpdf
\begin{flushleft}
\fi
\begin{ttfamily}
protected FLongSens: boolean;\end{ttfamily}

\ifpdf
\end{flushleft}
\fi


\par  \label{PasDoc_OptionParser.TOption-FExplanation}
\index{FExplanation}
\item[\textbf{FExplanation}\hfill]
\ifpdf
\begin{flushleft}
\fi
\begin{ttfamily}
protected FExplanation: string;\end{ttfamily}

\ifpdf
\end{flushleft}
\fi


\par  \label{PasDoc_OptionParser.TOption-FWasSpecified}
\index{FWasSpecified}
\item[\textbf{FWasSpecified}\hfill]
\ifpdf
\begin{flushleft}
\fi
\begin{ttfamily}
protected FWasSpecified: boolean;\end{ttfamily}

\ifpdf
\end{flushleft}
\fi


\par  \label{PasDoc_OptionParser.TOption-FParser}
\index{FParser}
\item[\textbf{FParser}\hfill]
\ifpdf
\begin{flushleft}
\fi
\begin{ttfamily}
protected FParser: TOptionParser;\end{ttfamily}

\ifpdf
\end{flushleft}
\fi


\par  \end{list}
\subsubsection*{\large{\textbf{Methods}}\normalsize\hspace{1ex}\hfill}
\paragraph*{ParseOption}\hspace*{\fill}

\label{PasDoc_OptionParser.TOption-ParseOption}
\index{ParseOption}
\begin{list}{}{
\settowidth{\tmplength}{\textbf{Description}}
\setlength{\itemindent}{0cm}
\setlength{\listparindent}{0cm}
\setlength{\leftmargin}{\evensidemargin}
\addtolength{\leftmargin}{\tmplength}
\settowidth{\labelsep}{X}
\addtolength{\leftmargin}{\labelsep}
\setlength{\labelwidth}{\tmplength}
}
\item[\textbf{Declaration}\hfill]
\ifpdf
\begin{flushleft}
\fi
\begin{ttfamily}
protected function ParseOption(const AWords: TStrings): boolean; virtual; abstract;\end{ttfamily}

\ifpdf
\end{flushleft}
\fi

\end{list}
\paragraph*{Create}\hspace*{\fill}

\label{PasDoc_OptionParser.TOption-Create}
\index{Create}
\begin{list}{}{
\settowidth{\tmplength}{\textbf{Description}}
\setlength{\itemindent}{0cm}
\setlength{\listparindent}{0cm}
\setlength{\leftmargin}{\evensidemargin}
\addtolength{\leftmargin}{\tmplength}
\settowidth{\labelsep}{X}
\addtolength{\leftmargin}{\labelsep}
\setlength{\labelwidth}{\tmplength}
}
\item[\textbf{Declaration}\hfill]
\ifpdf
\begin{flushleft}
\fi
\begin{ttfamily}
public constructor Create(const AShort:char; const ALong: string);\end{ttfamily}

\ifpdf
\end{flushleft}
\fi

\par
\item[\textbf{Description}]
Create a new Option. Set AShort to {\#}0 in order to have no short option. Technically you can set ALong to '' to have no long option, but in practive *every* option should have long form. Don't override this in descendants (this always simply calls CreateEx). Override only CreateEx.

\end{list}
\paragraph*{CreateEx}\hspace*{\fill}

\label{PasDoc_OptionParser.TOption-CreateEx}
\index{CreateEx}
\begin{list}{}{
\settowidth{\tmplength}{\textbf{Description}}
\setlength{\itemindent}{0cm}
\setlength{\listparindent}{0cm}
\setlength{\leftmargin}{\evensidemargin}
\addtolength{\leftmargin}{\tmplength}
\settowidth{\labelsep}{X}
\addtolength{\leftmargin}{\labelsep}
\setlength{\labelwidth}{\tmplength}
}
\item[\textbf{Declaration}\hfill]
\ifpdf
\begin{flushleft}
\fi
\begin{ttfamily}
public constructor CreateEx(const AShort:char; const ALong: string; const AShortCaseSensitive, ALongCaseSensitive: boolean); virtual;\end{ttfamily}

\ifpdf
\end{flushleft}
\fi

\end{list}
\paragraph*{GetOptionWidth}\hspace*{\fill}

\label{PasDoc_OptionParser.TOption-GetOptionWidth}
\index{GetOptionWidth}
\begin{list}{}{
\settowidth{\tmplength}{\textbf{Description}}
\setlength{\itemindent}{0cm}
\setlength{\listparindent}{0cm}
\setlength{\leftmargin}{\evensidemargin}
\addtolength{\leftmargin}{\tmplength}
\settowidth{\labelsep}{X}
\addtolength{\leftmargin}{\labelsep}
\setlength{\labelwidth}{\tmplength}
}
\item[\textbf{Declaration}\hfill]
\ifpdf
\begin{flushleft}
\fi
\begin{ttfamily}
public function GetOptionWidth: Integer;\end{ttfamily}

\ifpdf
\end{flushleft}
\fi

\par
\item[\textbf{Description}]
returns the width of the string "{-}s, {-}{-}long{-}option" where s is the short option. Removes non{-}existant options (longoption = '' or shortoption = {\#}0)

\end{list}
\paragraph*{WriteExplanation}\hspace*{\fill}

\label{PasDoc_OptionParser.TOption-WriteExplanation}
\index{WriteExplanation}
\begin{list}{}{
\settowidth{\tmplength}{\textbf{Description}}
\setlength{\itemindent}{0cm}
\setlength{\listparindent}{0cm}
\setlength{\leftmargin}{\evensidemargin}
\addtolength{\leftmargin}{\tmplength}
\settowidth{\labelsep}{X}
\addtolength{\leftmargin}{\labelsep}
\setlength{\labelwidth}{\tmplength}
}
\item[\textbf{Declaration}\hfill]
\ifpdf
\begin{flushleft}
\fi
\begin{ttfamily}
public procedure WriteExplanation(const AOptWidth: Integer);\end{ttfamily}

\ifpdf
\end{flushleft}
\fi

\par
\item[\textbf{Description}]
writes the wrapped explanation including option format, AOptWidth determines how much it is indented {\&} wrapped

\end{list}
\ifpdf
\subsection*{\large{\textbf{TBoolOption Class}}\normalsize\hspace{1ex}\hrulefill}
\else
\subsection*{TBoolOption Class}
\fi
\label{PasDoc_OptionParser.TBoolOption}
\index{TBoolOption}
\subsubsection*{\large{\textbf{Hierarchy}}\normalsize\hspace{1ex}\hfill}
TBoolOption {$>$} \begin{ttfamily}TOption\end{ttfamily}(\ref{PasDoc_OptionParser.TOption}) {$>$} 
TObject
\subsubsection*{\large{\textbf{Description}}\normalsize\hspace{1ex}\hfill}
simple boolean option\hfill\vspace*{1ex}

 turned off when not specified, turned on when specified. Cannot handle {-}{-}option=false et al.\subsubsection*{\large{\textbf{Properties}}\normalsize\hspace{1ex}\hfill}
\begin{list}{}{
\settowidth{\tmplength}{\textbf{TurnedOn}}
\setlength{\itemindent}{0cm}
\setlength{\listparindent}{0cm}
\setlength{\leftmargin}{\evensidemargin}
\addtolength{\leftmargin}{\tmplength}
\settowidth{\labelsep}{X}
\addtolength{\leftmargin}{\labelsep}
\setlength{\labelwidth}{\tmplength}
}
\label{PasDoc_OptionParser.TBoolOption-TurnedOn}
\index{TurnedOn}
\item[\textbf{TurnedOn}\hfill]
\ifpdf
\begin{flushleft}
\fi
\begin{ttfamily}
public property TurnedOn: boolean read FWasSpecified;\end{ttfamily}

\ifpdf
\end{flushleft}
\fi


\par  \end{list}
\subsubsection*{\large{\textbf{Methods}}\normalsize\hspace{1ex}\hfill}
\paragraph*{ParseOption}\hspace*{\fill}

\label{PasDoc_OptionParser.TBoolOption-ParseOption}
\index{ParseOption}
\begin{list}{}{
\settowidth{\tmplength}{\textbf{Description}}
\setlength{\itemindent}{0cm}
\setlength{\listparindent}{0cm}
\setlength{\leftmargin}{\evensidemargin}
\addtolength{\leftmargin}{\tmplength}
\settowidth{\labelsep}{X}
\addtolength{\leftmargin}{\labelsep}
\setlength{\labelwidth}{\tmplength}
}
\item[\textbf{Declaration}\hfill]
\ifpdf
\begin{flushleft}
\fi
\begin{ttfamily}
protected function ParseOption(const AWords: TStrings): boolean; override;\end{ttfamily}

\ifpdf
\end{flushleft}
\fi

\end{list}
\ifpdf
\subsection*{\large{\textbf{TValueOption Class}}\normalsize\hspace{1ex}\hrulefill}
\else
\subsection*{TValueOption Class}
\fi
\label{PasDoc_OptionParser.TValueOption}
\index{TValueOption}
\subsubsection*{\large{\textbf{Hierarchy}}\normalsize\hspace{1ex}\hfill}
TValueOption {$>$} \begin{ttfamily}TOption\end{ttfamily}(\ref{PasDoc_OptionParser.TOption}) {$>$} 
TObject
\subsubsection*{\large{\textbf{Description}}\normalsize\hspace{1ex}\hfill}
base class for all options that values\hfill\vspace*{1ex}

 base class for all options that take one or more values of the form {-}{-}option=value or {-}{-}option value etc\subsubsection*{\large{\textbf{Methods}}\normalsize\hspace{1ex}\hfill}
\paragraph*{CheckValue}\hspace*{\fill}

\label{PasDoc_OptionParser.TValueOption-CheckValue}
\index{CheckValue}
\begin{list}{}{
\settowidth{\tmplength}{\textbf{Description}}
\setlength{\itemindent}{0cm}
\setlength{\listparindent}{0cm}
\setlength{\leftmargin}{\evensidemargin}
\addtolength{\leftmargin}{\tmplength}
\settowidth{\labelsep}{X}
\addtolength{\leftmargin}{\labelsep}
\setlength{\labelwidth}{\tmplength}
}
\item[\textbf{Declaration}\hfill]
\ifpdf
\begin{flushleft}
\fi
\begin{ttfamily}
protected function CheckValue(const AString: String): boolean; virtual; abstract;\end{ttfamily}

\ifpdf
\end{flushleft}
\fi

\end{list}
\paragraph*{ParseOption}\hspace*{\fill}

\label{PasDoc_OptionParser.TValueOption-ParseOption}
\index{ParseOption}
\begin{list}{}{
\settowidth{\tmplength}{\textbf{Description}}
\setlength{\itemindent}{0cm}
\setlength{\listparindent}{0cm}
\setlength{\leftmargin}{\evensidemargin}
\addtolength{\leftmargin}{\tmplength}
\settowidth{\labelsep}{X}
\addtolength{\leftmargin}{\labelsep}
\setlength{\labelwidth}{\tmplength}
}
\item[\textbf{Declaration}\hfill]
\ifpdf
\begin{flushleft}
\fi
\begin{ttfamily}
protected function ParseOption(const AWords: TStrings): boolean; override;\end{ttfamily}

\ifpdf
\end{flushleft}
\fi

\end{list}
\ifpdf
\subsection*{\large{\textbf{TIntegerOption Class}}\normalsize\hspace{1ex}\hrulefill}
\else
\subsection*{TIntegerOption Class}
\fi
\label{PasDoc_OptionParser.TIntegerOption}
\index{TIntegerOption}
\subsubsection*{\large{\textbf{Hierarchy}}\normalsize\hspace{1ex}\hfill}
TIntegerOption {$>$} \begin{ttfamily}TValueOption\end{ttfamily}(\ref{PasDoc_OptionParser.TValueOption}) {$>$} \begin{ttfamily}TOption\end{ttfamily}(\ref{PasDoc_OptionParser.TOption}) {$>$} 
TObject
\subsubsection*{\large{\textbf{Description}}\normalsize\hspace{1ex}\hfill}
Integer option\hfill\vspace*{1ex}

 accepts only integers\subsubsection*{\large{\textbf{Properties}}\normalsize\hspace{1ex}\hfill}
\begin{list}{}{
\settowidth{\tmplength}{\textbf{Value}}
\setlength{\itemindent}{0cm}
\setlength{\listparindent}{0cm}
\setlength{\leftmargin}{\evensidemargin}
\addtolength{\leftmargin}{\tmplength}
\settowidth{\labelsep}{X}
\addtolength{\leftmargin}{\labelsep}
\setlength{\labelwidth}{\tmplength}
}
\label{PasDoc_OptionParser.TIntegerOption-Value}
\index{Value}
\item[\textbf{Value}\hfill]
\ifpdf
\begin{flushleft}
\fi
\begin{ttfamily}
public property Value: Integer read FValue write FValue;\end{ttfamily}

\ifpdf
\end{flushleft}
\fi


\par  \end{list}
\subsubsection*{\large{\textbf{Fields}}\normalsize\hspace{1ex}\hfill}
\begin{list}{}{
\settowidth{\tmplength}{\textbf{FValue}}
\setlength{\itemindent}{0cm}
\setlength{\listparindent}{0cm}
\setlength{\leftmargin}{\evensidemargin}
\addtolength{\leftmargin}{\tmplength}
\settowidth{\labelsep}{X}
\addtolength{\leftmargin}{\labelsep}
\setlength{\labelwidth}{\tmplength}
}
\label{PasDoc_OptionParser.TIntegerOption-FValue}
\index{FValue}
\item[\textbf{FValue}\hfill]
\ifpdf
\begin{flushleft}
\fi
\begin{ttfamily}
protected FValue: Integer;\end{ttfamily}

\ifpdf
\end{flushleft}
\fi


\par  \end{list}
\subsubsection*{\large{\textbf{Methods}}\normalsize\hspace{1ex}\hfill}
\paragraph*{CheckValue}\hspace*{\fill}

\label{PasDoc_OptionParser.TIntegerOption-CheckValue}
\index{CheckValue}
\begin{list}{}{
\settowidth{\tmplength}{\textbf{Description}}
\setlength{\itemindent}{0cm}
\setlength{\listparindent}{0cm}
\setlength{\leftmargin}{\evensidemargin}
\addtolength{\leftmargin}{\tmplength}
\settowidth{\labelsep}{X}
\addtolength{\leftmargin}{\labelsep}
\setlength{\labelwidth}{\tmplength}
}
\item[\textbf{Declaration}\hfill]
\ifpdf
\begin{flushleft}
\fi
\begin{ttfamily}
protected function CheckValue(const AString: String): boolean; override;\end{ttfamily}

\ifpdf
\end{flushleft}
\fi

\end{list}
\ifpdf
\subsection*{\large{\textbf{TStringOption Class}}\normalsize\hspace{1ex}\hrulefill}
\else
\subsection*{TStringOption Class}
\fi
\label{PasDoc_OptionParser.TStringOption}
\index{TStringOption}
\subsubsection*{\large{\textbf{Hierarchy}}\normalsize\hspace{1ex}\hfill}
TStringOption {$>$} \begin{ttfamily}TValueOption\end{ttfamily}(\ref{PasDoc_OptionParser.TValueOption}) {$>$} \begin{ttfamily}TOption\end{ttfamily}(\ref{PasDoc_OptionParser.TOption}) {$>$} 
TObject
\subsubsection*{\large{\textbf{Description}}\normalsize\hspace{1ex}\hfill}
String option\hfill\vspace*{1ex}

 accepts a single string\subsubsection*{\large{\textbf{Properties}}\normalsize\hspace{1ex}\hfill}
\begin{list}{}{
\settowidth{\tmplength}{\textbf{Value}}
\setlength{\itemindent}{0cm}
\setlength{\listparindent}{0cm}
\setlength{\leftmargin}{\evensidemargin}
\addtolength{\leftmargin}{\tmplength}
\settowidth{\labelsep}{X}
\addtolength{\leftmargin}{\labelsep}
\setlength{\labelwidth}{\tmplength}
}
\label{PasDoc_OptionParser.TStringOption-Value}
\index{Value}
\item[\textbf{Value}\hfill]
\ifpdf
\begin{flushleft}
\fi
\begin{ttfamily}
public property Value: String read FValue write FValue;\end{ttfamily}

\ifpdf
\end{flushleft}
\fi


\par  \end{list}
\subsubsection*{\large{\textbf{Fields}}\normalsize\hspace{1ex}\hfill}
\begin{list}{}{
\settowidth{\tmplength}{\textbf{FValue}}
\setlength{\itemindent}{0cm}
\setlength{\listparindent}{0cm}
\setlength{\leftmargin}{\evensidemargin}
\addtolength{\leftmargin}{\tmplength}
\settowidth{\labelsep}{X}
\addtolength{\leftmargin}{\labelsep}
\setlength{\labelwidth}{\tmplength}
}
\label{PasDoc_OptionParser.TStringOption-FValue}
\index{FValue}
\item[\textbf{FValue}\hfill]
\ifpdf
\begin{flushleft}
\fi
\begin{ttfamily}
protected FValue: String;\end{ttfamily}

\ifpdf
\end{flushleft}
\fi


\par  \end{list}
\subsubsection*{\large{\textbf{Methods}}\normalsize\hspace{1ex}\hfill}
\paragraph*{CheckValue}\hspace*{\fill}

\label{PasDoc_OptionParser.TStringOption-CheckValue}
\index{CheckValue}
\begin{list}{}{
\settowidth{\tmplength}{\textbf{Description}}
\setlength{\itemindent}{0cm}
\setlength{\listparindent}{0cm}
\setlength{\leftmargin}{\evensidemargin}
\addtolength{\leftmargin}{\tmplength}
\settowidth{\labelsep}{X}
\addtolength{\leftmargin}{\labelsep}
\setlength{\labelwidth}{\tmplength}
}
\item[\textbf{Declaration}\hfill]
\ifpdf
\begin{flushleft}
\fi
\begin{ttfamily}
protected function CheckValue(const AString: String): boolean; override;\end{ttfamily}

\ifpdf
\end{flushleft}
\fi

\end{list}
\ifpdf
\subsection*{\large{\textbf{TStringOptionList Class}}\normalsize\hspace{1ex}\hrulefill}
\else
\subsection*{TStringOptionList Class}
\fi
\label{PasDoc_OptionParser.TStringOptionList}
\index{TStringOptionList}
\subsubsection*{\large{\textbf{Hierarchy}}\normalsize\hspace{1ex}\hfill}
TStringOptionList {$>$} \begin{ttfamily}TValueOption\end{ttfamily}(\ref{PasDoc_OptionParser.TValueOption}) {$>$} \begin{ttfamily}TOption\end{ttfamily}(\ref{PasDoc_OptionParser.TOption}) {$>$} 
TObject
\subsubsection*{\large{\textbf{Description}}\normalsize\hspace{1ex}\hfill}
stringlist option\hfill\vspace*{1ex}

 accepts multiple strings and collates them even if the option itself is specified more than one time\subsubsection*{\large{\textbf{Properties}}\normalsize\hspace{1ex}\hfill}
\begin{list}{}{
\settowidth{\tmplength}{\textbf{Values}}
\setlength{\itemindent}{0cm}
\setlength{\listparindent}{0cm}
\setlength{\leftmargin}{\evensidemargin}
\addtolength{\leftmargin}{\tmplength}
\settowidth{\labelsep}{X}
\addtolength{\leftmargin}{\labelsep}
\setlength{\labelwidth}{\tmplength}
}
\label{PasDoc_OptionParser.TStringOptionList-Values}
\index{Values}
\item[\textbf{Values}\hfill]
\ifpdf
\begin{flushleft}
\fi
\begin{ttfamily}
public property Values: TStringList read FValues;\end{ttfamily}

\ifpdf
\end{flushleft}
\fi


\par  \end{list}
\subsubsection*{\large{\textbf{Fields}}\normalsize\hspace{1ex}\hfill}
\begin{list}{}{
\settowidth{\tmplength}{\textbf{FValues}}
\setlength{\itemindent}{0cm}
\setlength{\listparindent}{0cm}
\setlength{\leftmargin}{\evensidemargin}
\addtolength{\leftmargin}{\tmplength}
\settowidth{\labelsep}{X}
\addtolength{\leftmargin}{\labelsep}
\setlength{\labelwidth}{\tmplength}
}
\label{PasDoc_OptionParser.TStringOptionList-FValues}
\index{FValues}
\item[\textbf{FValues}\hfill]
\ifpdf
\begin{flushleft}
\fi
\begin{ttfamily}
protected FValues: TStringList;\end{ttfamily}

\ifpdf
\end{flushleft}
\fi


\par  \end{list}
\subsubsection*{\large{\textbf{Methods}}\normalsize\hspace{1ex}\hfill}
\paragraph*{CheckValue}\hspace*{\fill}

\label{PasDoc_OptionParser.TStringOptionList-CheckValue}
\index{CheckValue}
\begin{list}{}{
\settowidth{\tmplength}{\textbf{Description}}
\setlength{\itemindent}{0cm}
\setlength{\listparindent}{0cm}
\setlength{\leftmargin}{\evensidemargin}
\addtolength{\leftmargin}{\tmplength}
\settowidth{\labelsep}{X}
\addtolength{\leftmargin}{\labelsep}
\setlength{\labelwidth}{\tmplength}
}
\item[\textbf{Declaration}\hfill]
\ifpdf
\begin{flushleft}
\fi
\begin{ttfamily}
protected function CheckValue(const AString: String): Boolean; override;\end{ttfamily}

\ifpdf
\end{flushleft}
\fi

\end{list}
\paragraph*{CreateEx}\hspace*{\fill}

\label{PasDoc_OptionParser.TStringOptionList-CreateEx}
\index{CreateEx}
\begin{list}{}{
\settowidth{\tmplength}{\textbf{Description}}
\setlength{\itemindent}{0cm}
\setlength{\listparindent}{0cm}
\setlength{\leftmargin}{\evensidemargin}
\addtolength{\leftmargin}{\tmplength}
\settowidth{\labelsep}{X}
\addtolength{\leftmargin}{\labelsep}
\setlength{\labelwidth}{\tmplength}
}
\item[\textbf{Declaration}\hfill]
\ifpdf
\begin{flushleft}
\fi
\begin{ttfamily}
public constructor CreateEx(const AShort: Char; const ALong: String; const AShortCaseSensitive, ALongCaseSensitive: Boolean); override;\end{ttfamily}

\ifpdf
\end{flushleft}
\fi

\end{list}
\paragraph*{Destroy}\hspace*{\fill}

\label{PasDoc_OptionParser.TStringOptionList-Destroy}
\index{Destroy}
\begin{list}{}{
\settowidth{\tmplength}{\textbf{Description}}
\setlength{\itemindent}{0cm}
\setlength{\listparindent}{0cm}
\setlength{\leftmargin}{\evensidemargin}
\addtolength{\leftmargin}{\tmplength}
\settowidth{\labelsep}{X}
\addtolength{\leftmargin}{\labelsep}
\setlength{\labelwidth}{\tmplength}
}
\item[\textbf{Declaration}\hfill]
\ifpdf
\begin{flushleft}
\fi
\begin{ttfamily}
public destructor Destroy; override;\end{ttfamily}

\ifpdf
\end{flushleft}
\fi

\end{list}
\ifpdf
\subsection*{\large{\textbf{TPathListOption Class}}\normalsize\hspace{1ex}\hrulefill}
\else
\subsection*{TPathListOption Class}
\fi
\label{PasDoc_OptionParser.TPathListOption}
\index{TPathListOption}
\subsubsection*{\large{\textbf{Hierarchy}}\normalsize\hspace{1ex}\hfill}
TPathListOption {$>$} \begin{ttfamily}TStringOptionList\end{ttfamily}(\ref{PasDoc_OptionParser.TStringOptionList}) {$>$} \begin{ttfamily}TValueOption\end{ttfamily}(\ref{PasDoc_OptionParser.TValueOption}) {$>$} \begin{ttfamily}TOption\end{ttfamily}(\ref{PasDoc_OptionParser.TOption}) {$>$} 
TObject
\subsubsection*{\large{\textbf{Description}}\normalsize\hspace{1ex}\hfill}
pathlist option\hfill\vspace*{1ex}

 accepts multiple strings paths and collates them even if the option itself is specified more than one time. Paths in a single option can be separated by the DirectorySeparator\subsubsection*{\large{\textbf{Methods}}\normalsize\hspace{1ex}\hfill}
\paragraph*{CheckValue}\hspace*{\fill}

\label{PasDoc_OptionParser.TPathListOption-CheckValue}
\index{CheckValue}
\begin{list}{}{
\settowidth{\tmplength}{\textbf{Description}}
\setlength{\itemindent}{0cm}
\setlength{\listparindent}{0cm}
\setlength{\leftmargin}{\evensidemargin}
\addtolength{\leftmargin}{\tmplength}
\settowidth{\labelsep}{X}
\addtolength{\leftmargin}{\labelsep}
\setlength{\labelwidth}{\tmplength}
}
\item[\textbf{Declaration}\hfill]
\ifpdf
\begin{flushleft}
\fi
\begin{ttfamily}
public function CheckValue(const AString: String): Boolean; override;\end{ttfamily}

\ifpdf
\end{flushleft}
\fi

\end{list}
\ifpdf
\subsection*{\large{\textbf{TSetOption Class}}\normalsize\hspace{1ex}\hrulefill}
\else
\subsection*{TSetOption Class}
\fi
\label{PasDoc_OptionParser.TSetOption}
\index{TSetOption}
\subsubsection*{\large{\textbf{Hierarchy}}\normalsize\hspace{1ex}\hfill}
TSetOption {$>$} \begin{ttfamily}TValueOption\end{ttfamily}(\ref{PasDoc_OptionParser.TValueOption}) {$>$} \begin{ttfamily}TOption\end{ttfamily}(\ref{PasDoc_OptionParser.TOption}) {$>$} 
TObject
\subsubsection*{\large{\textbf{Description}}\normalsize\hspace{1ex}\hfill}
useful for making a choice of things\hfill\vspace*{1ex}

 Values must not have a + or {-} sign as the last character as that can be used to add/remove items from the default set, specifying items without +/{-} at the end clears the default and uses only specified items\subsubsection*{\large{\textbf{Properties}}\normalsize\hspace{1ex}\hfill}
\begin{list}{}{
\settowidth{\tmplength}{\textbf{PossibleValues}}
\setlength{\itemindent}{0cm}
\setlength{\listparindent}{0cm}
\setlength{\leftmargin}{\evensidemargin}
\addtolength{\leftmargin}{\tmplength}
\settowidth{\labelsep}{X}
\addtolength{\leftmargin}{\labelsep}
\setlength{\labelwidth}{\tmplength}
}
\label{PasDoc_OptionParser.TSetOption-PossibleValues}
\index{PossibleValues}
\item[\textbf{PossibleValues}\hfill]
\ifpdf
\begin{flushleft}
\fi
\begin{ttfamily}
public property PossibleValues: string read GetPossibleValues write SetPossibleValues;\end{ttfamily}

\ifpdf
\end{flushleft}
\fi


\par  \label{PasDoc_OptionParser.TSetOption-Values}
\index{Values}
\item[\textbf{Values}\hfill]
\ifpdf
\begin{flushleft}
\fi
\begin{ttfamily}
public property Values: string read GetValues write SetValues;\end{ttfamily}

\ifpdf
\end{flushleft}
\fi


\par  \end{list}
\subsubsection*{\large{\textbf{Fields}}\normalsize\hspace{1ex}\hfill}
\begin{list}{}{
\settowidth{\tmplength}{\textbf{FPossibleValues}}
\setlength{\itemindent}{0cm}
\setlength{\listparindent}{0cm}
\setlength{\leftmargin}{\evensidemargin}
\addtolength{\leftmargin}{\tmplength}
\settowidth{\labelsep}{X}
\addtolength{\leftmargin}{\labelsep}
\setlength{\labelwidth}{\tmplength}
}
\label{PasDoc_OptionParser.TSetOption-FPossibleValues}
\index{FPossibleValues}
\item[\textbf{FPossibleValues}\hfill]
\ifpdf
\begin{flushleft}
\fi
\begin{ttfamily}
protected FPossibleValues: TStringList;\end{ttfamily}

\ifpdf
\end{flushleft}
\fi


\par  \label{PasDoc_OptionParser.TSetOption-FValues}
\index{FValues}
\item[\textbf{FValues}\hfill]
\ifpdf
\begin{flushleft}
\fi
\begin{ttfamily}
protected FValues: TStringList;\end{ttfamily}

\ifpdf
\end{flushleft}
\fi


\par  \end{list}
\subsubsection*{\large{\textbf{Methods}}\normalsize\hspace{1ex}\hfill}
\paragraph*{GetPossibleValues}\hspace*{\fill}

\label{PasDoc_OptionParser.TSetOption-GetPossibleValues}
\index{GetPossibleValues}
\begin{list}{}{
\settowidth{\tmplength}{\textbf{Description}}
\setlength{\itemindent}{0cm}
\setlength{\listparindent}{0cm}
\setlength{\leftmargin}{\evensidemargin}
\addtolength{\leftmargin}{\tmplength}
\settowidth{\labelsep}{X}
\addtolength{\leftmargin}{\labelsep}
\setlength{\labelwidth}{\tmplength}
}
\item[\textbf{Declaration}\hfill]
\ifpdf
\begin{flushleft}
\fi
\begin{ttfamily}
protected function GetPossibleValues: string;\end{ttfamily}

\ifpdf
\end{flushleft}
\fi

\end{list}
\paragraph*{SetPossibleValues}\hspace*{\fill}

\label{PasDoc_OptionParser.TSetOption-SetPossibleValues}
\index{SetPossibleValues}
\begin{list}{}{
\settowidth{\tmplength}{\textbf{Description}}
\setlength{\itemindent}{0cm}
\setlength{\listparindent}{0cm}
\setlength{\leftmargin}{\evensidemargin}
\addtolength{\leftmargin}{\tmplength}
\settowidth{\labelsep}{X}
\addtolength{\leftmargin}{\labelsep}
\setlength{\labelwidth}{\tmplength}
}
\item[\textbf{Declaration}\hfill]
\ifpdf
\begin{flushleft}
\fi
\begin{ttfamily}
protected procedure SetPossibleValues(const Value: string);\end{ttfamily}

\ifpdf
\end{flushleft}
\fi

\end{list}
\paragraph*{CheckValue}\hspace*{\fill}

\label{PasDoc_OptionParser.TSetOption-CheckValue}
\index{CheckValue}
\begin{list}{}{
\settowidth{\tmplength}{\textbf{Description}}
\setlength{\itemindent}{0cm}
\setlength{\listparindent}{0cm}
\setlength{\leftmargin}{\evensidemargin}
\addtolength{\leftmargin}{\tmplength}
\settowidth{\labelsep}{X}
\addtolength{\leftmargin}{\labelsep}
\setlength{\labelwidth}{\tmplength}
}
\item[\textbf{Declaration}\hfill]
\ifpdf
\begin{flushleft}
\fi
\begin{ttfamily}
protected function CheckValue(const AString: String): Boolean; override;\end{ttfamily}

\ifpdf
\end{flushleft}
\fi

\end{list}
\paragraph*{GetValues}\hspace*{\fill}

\label{PasDoc_OptionParser.TSetOption-GetValues}
\index{GetValues}
\begin{list}{}{
\settowidth{\tmplength}{\textbf{Description}}
\setlength{\itemindent}{0cm}
\setlength{\listparindent}{0cm}
\setlength{\leftmargin}{\evensidemargin}
\addtolength{\leftmargin}{\tmplength}
\settowidth{\labelsep}{X}
\addtolength{\leftmargin}{\labelsep}
\setlength{\labelwidth}{\tmplength}
}
\item[\textbf{Declaration}\hfill]
\ifpdf
\begin{flushleft}
\fi
\begin{ttfamily}
protected function GetValues: string;\end{ttfamily}

\ifpdf
\end{flushleft}
\fi

\end{list}
\paragraph*{SetValues}\hspace*{\fill}

\label{PasDoc_OptionParser.TSetOption-SetValues}
\index{SetValues}
\begin{list}{}{
\settowidth{\tmplength}{\textbf{Description}}
\setlength{\itemindent}{0cm}
\setlength{\listparindent}{0cm}
\setlength{\leftmargin}{\evensidemargin}
\addtolength{\leftmargin}{\tmplength}
\settowidth{\labelsep}{X}
\addtolength{\leftmargin}{\labelsep}
\setlength{\labelwidth}{\tmplength}
}
\item[\textbf{Declaration}\hfill]
\ifpdf
\begin{flushleft}
\fi
\begin{ttfamily}
protected procedure SetValues(const Value: string);\end{ttfamily}

\ifpdf
\end{flushleft}
\fi

\end{list}
\paragraph*{CreateEx}\hspace*{\fill}

\label{PasDoc_OptionParser.TSetOption-CreateEx}
\index{CreateEx}
\begin{list}{}{
\settowidth{\tmplength}{\textbf{Description}}
\setlength{\itemindent}{0cm}
\setlength{\listparindent}{0cm}
\setlength{\leftmargin}{\evensidemargin}
\addtolength{\leftmargin}{\tmplength}
\settowidth{\labelsep}{X}
\addtolength{\leftmargin}{\labelsep}
\setlength{\labelwidth}{\tmplength}
}
\item[\textbf{Declaration}\hfill]
\ifpdf
\begin{flushleft}
\fi
\begin{ttfamily}
public constructor CreateEx(const AShort: Char; const ALong: String; const AShortCaseSensitive, ALongCaseSensitive: Boolean); override;\end{ttfamily}

\ifpdf
\end{flushleft}
\fi

\end{list}
\paragraph*{Destroy}\hspace*{\fill}

\label{PasDoc_OptionParser.TSetOption-Destroy}
\index{Destroy}
\begin{list}{}{
\settowidth{\tmplength}{\textbf{Description}}
\setlength{\itemindent}{0cm}
\setlength{\listparindent}{0cm}
\setlength{\leftmargin}{\evensidemargin}
\addtolength{\leftmargin}{\tmplength}
\settowidth{\labelsep}{X}
\addtolength{\leftmargin}{\labelsep}
\setlength{\labelwidth}{\tmplength}
}
\item[\textbf{Declaration}\hfill]
\ifpdf
\begin{flushleft}
\fi
\begin{ttfamily}
public destructor Destroy; override;\end{ttfamily}

\ifpdf
\end{flushleft}
\fi

\end{list}
\paragraph*{HasValue}\hspace*{\fill}

\label{PasDoc_OptionParser.TSetOption-HasValue}
\index{HasValue}
\begin{list}{}{
\settowidth{\tmplength}{\textbf{Description}}
\setlength{\itemindent}{0cm}
\setlength{\listparindent}{0cm}
\setlength{\leftmargin}{\evensidemargin}
\addtolength{\leftmargin}{\tmplength}
\settowidth{\labelsep}{X}
\addtolength{\leftmargin}{\labelsep}
\setlength{\labelwidth}{\tmplength}
}
\item[\textbf{Declaration}\hfill]
\ifpdf
\begin{flushleft}
\fi
\begin{ttfamily}
public function HasValue(const AValue: string): boolean;\end{ttfamily}

\ifpdf
\end{flushleft}
\fi

\end{list}
\ifpdf
\subsection*{\large{\textbf{TOptionParser Class}}\normalsize\hspace{1ex}\hrulefill}
\else
\subsection*{TOptionParser Class}
\fi
\label{PasDoc_OptionParser.TOptionParser}
\index{TOptionParser}
\subsubsection*{\large{\textbf{Hierarchy}}\normalsize\hspace{1ex}\hfill}
TOptionParser {$>$} TObject
\subsubsection*{\large{\textbf{Description}}\normalsize\hspace{1ex}\hfill}
OptionParser --- instantiate one of these for commandline parsing\hfill\vspace*{1ex}

 This class is the main parsing class, although a lot of parsing is handled by \begin{ttfamily}TOption\end{ttfamily}(\ref{PasDoc_OptionParser.TOption}) and its descendants instead.\subsubsection*{\large{\textbf{Properties}}\normalsize\hspace{1ex}\hfill}
\begin{list}{}{
\settowidth{\tmplength}{\textbf{IncludeFileOptionName}}
\setlength{\itemindent}{0cm}
\setlength{\listparindent}{0cm}
\setlength{\leftmargin}{\evensidemargin}
\addtolength{\leftmargin}{\tmplength}
\settowidth{\labelsep}{X}
\addtolength{\leftmargin}{\labelsep}
\setlength{\labelwidth}{\tmplength}
}
\label{PasDoc_OptionParser.TOptionParser-LeftList}
\index{LeftList}
\item[\textbf{LeftList}\hfill]
\ifpdf
\begin{flushleft}
\fi
\begin{ttfamily}
public property LeftList: TStringList read FLeftList;\end{ttfamily}

\ifpdf
\end{flushleft}
\fi


\par This StringList contains all the items from the command line that could not be parsed. Includes options that didn't accept their value and non{-}options like filenames specified on the command line\label{PasDoc_OptionParser.TOptionParser-OptionsCount}
\index{OptionsCount}
\item[\textbf{OptionsCount}\hfill]
\ifpdf
\begin{flushleft}
\fi
\begin{ttfamily}
public property OptionsCount: Integer read GetOptionsCount;\end{ttfamily}

\ifpdf
\end{flushleft}
\fi


\par The number of option objects that were added to this parser\label{PasDoc_OptionParser.TOptionParser-Options}
\index{Options}
\item[\textbf{Options}\hfill]
\ifpdf
\begin{flushleft}
\fi
\begin{ttfamily}
public property Options[constAIndex:Integer]: TOption read GetOption;\end{ttfamily}

\ifpdf
\end{flushleft}
\fi


\par retrieve an option by index --- you can use this and \begin{ttfamily}OptionsCount\end{ttfamily}(\ref{PasDoc_OptionParser.TOptionParser-OptionsCount}) to iterate through the options that this parser owns\label{PasDoc_OptionParser.TOptionParser-ByName}
\index{ByName}
\item[\textbf{ByName}\hfill]
\ifpdf
\begin{flushleft}
\fi
\begin{ttfamily}
public property ByName[constAName:string]: TOption read GetOptionByLongName;\end{ttfamily}

\ifpdf
\end{flushleft}
\fi


\par retrieve an option by its long form. Case sensitivity of the options is taken into account!\label{PasDoc_OptionParser.TOptionParser-ByShortName}
\index{ByShortName}
\item[\textbf{ByShortName}\hfill]
\ifpdf
\begin{flushleft}
\fi
\begin{ttfamily}
public property ByShortName[constAName:char]: TOption read GetOptionByShortname;\end{ttfamily}

\ifpdf
\end{flushleft}
\fi


\par retrieve an option by its short form. Case sensitivity of the options is taken into account!\label{PasDoc_OptionParser.TOptionParser-ShortOptionStart}
\index{ShortOptionStart}
\item[\textbf{ShortOptionStart}\hfill]
\ifpdf
\begin{flushleft}
\fi
\begin{ttfamily}
public property ShortOptionStart: Char read FShortOptionChar write FShortOptionChar default DefShortOptionChar;\end{ttfamily}

\ifpdf
\end{flushleft}
\fi


\par introductory character to be used for short options\label{PasDoc_OptionParser.TOptionParser-LongOptionStart}
\index{LongOptionStart}
\item[\textbf{LongOptionStart}\hfill]
\ifpdf
\begin{flushleft}
\fi
\begin{ttfamily}
public property LongOptionStart: String read FLongOptionString write FLongOptionString;\end{ttfamily}

\ifpdf
\end{flushleft}
\fi


\par introductory string to be used for long options\label{PasDoc_OptionParser.TOptionParser-IncludeFileOptionName}
\index{IncludeFileOptionName}
\item[\textbf{IncludeFileOptionName}\hfill]
\ifpdf
\begin{flushleft}
\fi
\begin{ttfamily}
public property IncludeFileOptionName: string read FIncludeFileOptionName write FIncludeFileOptionName;\end{ttfamily}

\ifpdf
\end{flushleft}
\fi


\par name of an option to include config file\label{PasDoc_OptionParser.TOptionParser-IncludeFileOptionExpl}
\index{IncludeFileOptionExpl}
\item[\textbf{IncludeFileOptionExpl}\hfill]
\ifpdf
\begin{flushleft}
\fi
\begin{ttfamily}
public property IncludeFileOptionExpl: string read FIncludeFileOptionExpl write FIncludeFileOptionExpl;\end{ttfamily}

\ifpdf
\end{flushleft}
\fi


\par explanation of an option to include config file\end{list}
\subsubsection*{\large{\textbf{Fields}}\normalsize\hspace{1ex}\hfill}
\begin{list}{}{
\settowidth{\tmplength}{\textbf{FIncludeFileOptionName}}
\setlength{\itemindent}{0cm}
\setlength{\listparindent}{0cm}
\setlength{\leftmargin}{\evensidemargin}
\addtolength{\leftmargin}{\tmplength}
\settowidth{\labelsep}{X}
\addtolength{\leftmargin}{\labelsep}
\setlength{\labelwidth}{\tmplength}
}
\label{PasDoc_OptionParser.TOptionParser-FParams}
\index{FParams}
\item[\textbf{FParams}\hfill]
\ifpdf
\begin{flushleft}
\fi
\begin{ttfamily}
protected FParams: TStringList;\end{ttfamily}

\ifpdf
\end{flushleft}
\fi


\par  \label{PasDoc_OptionParser.TOptionParser-FOptions}
\index{FOptions}
\item[\textbf{FOptions}\hfill]
\ifpdf
\begin{flushleft}
\fi
\begin{ttfamily}
protected FOptions: TList;\end{ttfamily}

\ifpdf
\end{flushleft}
\fi


\par  \label{PasDoc_OptionParser.TOptionParser-FLeftList}
\index{FLeftList}
\item[\textbf{FLeftList}\hfill]
\ifpdf
\begin{flushleft}
\fi
\begin{ttfamily}
protected FLeftList: TStringList;\end{ttfamily}

\ifpdf
\end{flushleft}
\fi


\par  \label{PasDoc_OptionParser.TOptionParser-FShortOptionChar}
\index{FShortOptionChar}
\item[\textbf{FShortOptionChar}\hfill]
\ifpdf
\begin{flushleft}
\fi
\begin{ttfamily}
protected FShortOptionChar: Char;\end{ttfamily}

\ifpdf
\end{flushleft}
\fi


\par  \label{PasDoc_OptionParser.TOptionParser-FLongOptionString}
\index{FLongOptionString}
\item[\textbf{FLongOptionString}\hfill]
\ifpdf
\begin{flushleft}
\fi
\begin{ttfamily}
protected FLongOptionString: string;\end{ttfamily}

\ifpdf
\end{flushleft}
\fi


\par  \label{PasDoc_OptionParser.TOptionParser-FIncludeFileOptionName}
\index{FIncludeFileOptionName}
\item[\textbf{FIncludeFileOptionName}\hfill]
\ifpdf
\begin{flushleft}
\fi
\begin{ttfamily}
protected FIncludeFileOptionName: string;\end{ttfamily}

\ifpdf
\end{flushleft}
\fi


\par  \label{PasDoc_OptionParser.TOptionParser-FIncludeFileOptionExpl}
\index{FIncludeFileOptionExpl}
\item[\textbf{FIncludeFileOptionExpl}\hfill]
\ifpdf
\begin{flushleft}
\fi
\begin{ttfamily}
protected FIncludeFileOptionExpl: string;\end{ttfamily}

\ifpdf
\end{flushleft}
\fi


\par  \end{list}
\subsubsection*{\large{\textbf{Methods}}\normalsize\hspace{1ex}\hfill}
\paragraph*{GetOption}\hspace*{\fill}

\label{PasDoc_OptionParser.TOptionParser-GetOption}
\index{GetOption}
\begin{list}{}{
\settowidth{\tmplength}{\textbf{Description}}
\setlength{\itemindent}{0cm}
\setlength{\listparindent}{0cm}
\setlength{\leftmargin}{\evensidemargin}
\addtolength{\leftmargin}{\tmplength}
\settowidth{\labelsep}{X}
\addtolength{\leftmargin}{\labelsep}
\setlength{\labelwidth}{\tmplength}
}
\item[\textbf{Declaration}\hfill]
\ifpdf
\begin{flushleft}
\fi
\begin{ttfamily}
protected function GetOption(const AIndex: Integer): TOption;\end{ttfamily}

\ifpdf
\end{flushleft}
\fi

\end{list}
\paragraph*{GetOptionsCount}\hspace*{\fill}

\label{PasDoc_OptionParser.TOptionParser-GetOptionsCount}
\index{GetOptionsCount}
\begin{list}{}{
\settowidth{\tmplength}{\textbf{Description}}
\setlength{\itemindent}{0cm}
\setlength{\listparindent}{0cm}
\setlength{\leftmargin}{\evensidemargin}
\addtolength{\leftmargin}{\tmplength}
\settowidth{\labelsep}{X}
\addtolength{\leftmargin}{\labelsep}
\setlength{\labelwidth}{\tmplength}
}
\item[\textbf{Declaration}\hfill]
\ifpdf
\begin{flushleft}
\fi
\begin{ttfamily}
protected function GetOptionsCount: Integer;\end{ttfamily}

\ifpdf
\end{flushleft}
\fi

\end{list}
\paragraph*{GetOptionByLongName}\hspace*{\fill}

\label{PasDoc_OptionParser.TOptionParser-GetOptionByLongName}
\index{GetOptionByLongName}
\begin{list}{}{
\settowidth{\tmplength}{\textbf{Description}}
\setlength{\itemindent}{0cm}
\setlength{\listparindent}{0cm}
\setlength{\leftmargin}{\evensidemargin}
\addtolength{\leftmargin}{\tmplength}
\settowidth{\labelsep}{X}
\addtolength{\leftmargin}{\labelsep}
\setlength{\labelwidth}{\tmplength}
}
\item[\textbf{Declaration}\hfill]
\ifpdf
\begin{flushleft}
\fi
\begin{ttfamily}
protected function GetOptionByLongName(const AName: string): TOption;\end{ttfamily}

\ifpdf
\end{flushleft}
\fi

\end{list}
\paragraph*{GetOptionByShortname}\hspace*{\fill}

\label{PasDoc_OptionParser.TOptionParser-GetOptionByShortname}
\index{GetOptionByShortname}
\begin{list}{}{
\settowidth{\tmplength}{\textbf{Description}}
\setlength{\itemindent}{0cm}
\setlength{\listparindent}{0cm}
\setlength{\leftmargin}{\evensidemargin}
\addtolength{\leftmargin}{\tmplength}
\settowidth{\labelsep}{X}
\addtolength{\leftmargin}{\labelsep}
\setlength{\labelwidth}{\tmplength}
}
\item[\textbf{Declaration}\hfill]
\ifpdf
\begin{flushleft}
\fi
\begin{ttfamily}
protected function GetOptionByShortname(const AName: char): TOption;\end{ttfamily}

\ifpdf
\end{flushleft}
\fi

\end{list}
\paragraph*{Create}\hspace*{\fill}

\label{PasDoc_OptionParser.TOptionParser-Create}
\index{Create}
\begin{list}{}{
\settowidth{\tmplength}{\textbf{Description}}
\setlength{\itemindent}{0cm}
\setlength{\listparindent}{0cm}
\setlength{\leftmargin}{\evensidemargin}
\addtolength{\leftmargin}{\tmplength}
\settowidth{\labelsep}{X}
\addtolength{\leftmargin}{\labelsep}
\setlength{\labelwidth}{\tmplength}
}
\item[\textbf{Declaration}\hfill]
\ifpdf
\begin{flushleft}
\fi
\begin{ttfamily}
public constructor Create; virtual;\end{ttfamily}

\ifpdf
\end{flushleft}
\fi

\par
\item[\textbf{Description}]
Create without any options --- this will parse the current command line

\end{list}
\paragraph*{CreateParams}\hspace*{\fill}

\label{PasDoc_OptionParser.TOptionParser-CreateParams}
\index{CreateParams}
\begin{list}{}{
\settowidth{\tmplength}{\textbf{Description}}
\setlength{\itemindent}{0cm}
\setlength{\listparindent}{0cm}
\setlength{\leftmargin}{\evensidemargin}
\addtolength{\leftmargin}{\tmplength}
\settowidth{\labelsep}{X}
\addtolength{\leftmargin}{\labelsep}
\setlength{\labelwidth}{\tmplength}
}
\item[\textbf{Declaration}\hfill]
\ifpdf
\begin{flushleft}
\fi
\begin{ttfamily}
public constructor CreateParams(const AParams: TStrings); virtual;\end{ttfamily}

\ifpdf
\end{flushleft}
\fi

\par
\item[\textbf{Description}]
Create with parameters to be used instead of command line

\end{list}
\paragraph*{Destroy}\hspace*{\fill}

\label{PasDoc_OptionParser.TOptionParser-Destroy}
\index{Destroy}
\begin{list}{}{
\settowidth{\tmplength}{\textbf{Description}}
\setlength{\itemindent}{0cm}
\setlength{\listparindent}{0cm}
\setlength{\leftmargin}{\evensidemargin}
\addtolength{\leftmargin}{\tmplength}
\settowidth{\labelsep}{X}
\addtolength{\leftmargin}{\labelsep}
\setlength{\labelwidth}{\tmplength}
}
\item[\textbf{Declaration}\hfill]
\ifpdf
\begin{flushleft}
\fi
\begin{ttfamily}
public destructor Destroy; override;\end{ttfamily}

\ifpdf
\end{flushleft}
\fi

\par
\item[\textbf{Description}]
destroy the option parser object and all associated \begin{ttfamily}TOption\end{ttfamily}(\ref{PasDoc_OptionParser.TOption}) objects

\end{list}
\paragraph*{AddOption}\hspace*{\fill}

\label{PasDoc_OptionParser.TOptionParser-AddOption}
\index{AddOption}
\begin{list}{}{
\settowidth{\tmplength}{\textbf{Description}}
\setlength{\itemindent}{0cm}
\setlength{\listparindent}{0cm}
\setlength{\leftmargin}{\evensidemargin}
\addtolength{\leftmargin}{\tmplength}
\settowidth{\labelsep}{X}
\addtolength{\leftmargin}{\labelsep}
\setlength{\labelwidth}{\tmplength}
}
\item[\textbf{Declaration}\hfill]
\ifpdf
\begin{flushleft}
\fi
\begin{ttfamily}
public function AddOption(const AOption: TOption): TOption;\end{ttfamily}

\ifpdf
\end{flushleft}
\fi

\par
\item[\textbf{Description}]
Add a \begin{ttfamily}TOption\end{ttfamily}(\ref{PasDoc_OptionParser.TOption}) descendant to be included in parsing the command line

\end{list}
\paragraph*{ParseOptions}\hspace*{\fill}

\label{PasDoc_OptionParser.TOptionParser-ParseOptions}
\index{ParseOptions}
\begin{list}{}{
\settowidth{\tmplength}{\textbf{Description}}
\setlength{\itemindent}{0cm}
\setlength{\listparindent}{0cm}
\setlength{\leftmargin}{\evensidemargin}
\addtolength{\leftmargin}{\tmplength}
\settowidth{\labelsep}{X}
\addtolength{\leftmargin}{\labelsep}
\setlength{\labelwidth}{\tmplength}
}
\item[\textbf{Declaration}\hfill]
\ifpdf
\begin{flushleft}
\fi
\begin{ttfamily}
public procedure ParseOptions;\end{ttfamily}

\ifpdf
\end{flushleft}
\fi

\par
\item[\textbf{Description}]
Parse the specified command line, see also \begin{ttfamily}Create\end{ttfamily}(\ref{PasDoc_OptionParser.TOptionParser-Create})

\end{list}
\paragraph*{WriteExplanations}\hspace*{\fill}

\label{PasDoc_OptionParser.TOptionParser-WriteExplanations}
\index{WriteExplanations}
\begin{list}{}{
\settowidth{\tmplength}{\textbf{Description}}
\setlength{\itemindent}{0cm}
\setlength{\listparindent}{0cm}
\setlength{\leftmargin}{\evensidemargin}
\addtolength{\leftmargin}{\tmplength}
\settowidth{\labelsep}{X}
\addtolength{\leftmargin}{\labelsep}
\setlength{\labelwidth}{\tmplength}
}
\item[\textbf{Declaration}\hfill]
\ifpdf
\begin{flushleft}
\fi
\begin{ttfamily}
public procedure WriteExplanations;\end{ttfamily}

\ifpdf
\end{flushleft}
\fi

\par
\item[\textbf{Description}]
output explanations for all options to stdout, will nicely format the output and wrap explanations

\end{list}
\section{Constants}
\ifpdf
\subsection*{\large{\textbf{DefShortOptionChar}}\normalsize\hspace{1ex}\hrulefill}
\else
\subsection*{DefShortOptionChar}
\fi
\label{PasDoc_OptionParser-DefShortOptionChar}
\index{DefShortOptionChar}
\begin{list}{}{
\settowidth{\tmplength}{\textbf{Description}}
\setlength{\itemindent}{0cm}
\setlength{\listparindent}{0cm}
\setlength{\leftmargin}{\evensidemargin}
\addtolength{\leftmargin}{\tmplength}
\settowidth{\labelsep}{X}
\addtolength{\leftmargin}{\labelsep}
\setlength{\labelwidth}{\tmplength}
}
\item[\textbf{Declaration}\hfill]
\ifpdf
\begin{flushleft}
\fi
\begin{ttfamily}
DefShortOptionChar = '-';\end{ttfamily}

\ifpdf
\end{flushleft}
\fi

\par
\item[\textbf{Description}]
default short option character used

\end{list}
\ifpdf
\subsection*{\large{\textbf{DefLongOptionString}}\normalsize\hspace{1ex}\hrulefill}
\else
\subsection*{DefLongOptionString}
\fi
\label{PasDoc_OptionParser-DefLongOptionString}
\index{DefLongOptionString}
\begin{list}{}{
\settowidth{\tmplength}{\textbf{Description}}
\setlength{\itemindent}{0cm}
\setlength{\listparindent}{0cm}
\setlength{\leftmargin}{\evensidemargin}
\addtolength{\leftmargin}{\tmplength}
\settowidth{\labelsep}{X}
\addtolength{\leftmargin}{\labelsep}
\setlength{\labelwidth}{\tmplength}
}
\item[\textbf{Declaration}\hfill]
\ifpdf
\begin{flushleft}
\fi
\begin{ttfamily}
DefLongOptionString = '--';\end{ttfamily}

\ifpdf
\end{flushleft}
\fi

\par
\item[\textbf{Description}]
default long option string used

\end{list}
\ifpdf
\subsection*{\large{\textbf{OptionFileChar}}\normalsize\hspace{1ex}\hrulefill}
\else
\subsection*{OptionFileChar}
\fi
\label{PasDoc_OptionParser-OptionFileChar}
\index{OptionFileChar}
\begin{list}{}{
\settowidth{\tmplength}{\textbf{Description}}
\setlength{\itemindent}{0cm}
\setlength{\listparindent}{0cm}
\setlength{\leftmargin}{\evensidemargin}
\addtolength{\leftmargin}{\tmplength}
\settowidth{\labelsep}{X}
\addtolength{\leftmargin}{\labelsep}
\setlength{\labelwidth}{\tmplength}
}
\item[\textbf{Declaration}\hfill]
\ifpdf
\begin{flushleft}
\fi
\begin{ttfamily}
OptionFileChar = '@';\end{ttfamily}

\ifpdf
\end{flushleft}
\fi

\par
\item[\textbf{Description}]
Marks "include config file" option

\end{list}
\ifpdf
\subsection*{\large{\textbf{CfgMacroCfgPath}}\normalsize\hspace{1ex}\hrulefill}
\else
\subsection*{CfgMacroCfgPath}
\fi
\label{PasDoc_OptionParser-CfgMacroCfgPath}
\index{CfgMacroCfgPath}
\begin{list}{}{
\settowidth{\tmplength}{\textbf{Description}}
\setlength{\itemindent}{0cm}
\setlength{\listparindent}{0cm}
\setlength{\leftmargin}{\evensidemargin}
\addtolength{\leftmargin}{\tmplength}
\settowidth{\labelsep}{X}
\addtolength{\leftmargin}{\labelsep}
\setlength{\labelwidth}{\tmplength}
}
\item[\textbf{Declaration}\hfill]
\ifpdf
\begin{flushleft}
\fi
\begin{ttfamily}
CfgMacroCfgPath = '{\$}CFG{\_}PATH';\end{ttfamily}

\ifpdf
\end{flushleft}
\fi

\par
\item[\textbf{Description}]
Special substitution that, if found inside a config file, will be replaced with actual path of the file

\end{list}
\ifpdf
\subsection*{\large{\textbf{OptionIndent}}\normalsize\hspace{1ex}\hrulefill}
\else
\subsection*{OptionIndent}
\fi
\label{PasDoc_OptionParser-OptionIndent}
\index{OptionIndent}
\begin{list}{}{
\settowidth{\tmplength}{\textbf{Description}}
\setlength{\itemindent}{0cm}
\setlength{\listparindent}{0cm}
\setlength{\leftmargin}{\evensidemargin}
\addtolength{\leftmargin}{\tmplength}
\settowidth{\labelsep}{X}
\addtolength{\leftmargin}{\labelsep}
\setlength{\labelwidth}{\tmplength}
}
\item[\textbf{Declaration}\hfill]
\ifpdf
\begin{flushleft}
\fi
\begin{ttfamily}
OptionIndent = '  ';\end{ttfamily}

\ifpdf
\end{flushleft}
\fi

\par
\item[\textbf{Description}]
Indentation of option's name from the start of console line

\end{list}
\ifpdf
\subsection*{\large{\textbf{OptionSep}}\normalsize\hspace{1ex}\hrulefill}
\else
\subsection*{OptionSep}
\fi
\label{PasDoc_OptionParser-OptionSep}
\index{OptionSep}
\begin{list}{}{
\settowidth{\tmplength}{\textbf{Description}}
\setlength{\itemindent}{0cm}
\setlength{\listparindent}{0cm}
\setlength{\leftmargin}{\evensidemargin}
\addtolength{\leftmargin}{\tmplength}
\settowidth{\labelsep}{X}
\addtolength{\leftmargin}{\labelsep}
\setlength{\labelwidth}{\tmplength}
}
\item[\textbf{Declaration}\hfill]
\ifpdf
\begin{flushleft}
\fi
\begin{ttfamily}
OptionSep = '  ';\end{ttfamily}

\ifpdf
\end{flushleft}
\fi

\par
\item[\textbf{Description}]
Separator between option's name and explanation

\end{list}
\ifpdf
\subsection*{\large{\textbf{ConsoleWidth}}\normalsize\hspace{1ex}\hrulefill}
\else
\subsection*{ConsoleWidth}
\fi
\label{PasDoc_OptionParser-ConsoleWidth}
\index{ConsoleWidth}
\begin{list}{}{
\settowidth{\tmplength}{\textbf{Description}}
\setlength{\itemindent}{0cm}
\setlength{\listparindent}{0cm}
\setlength{\leftmargin}{\evensidemargin}
\addtolength{\leftmargin}{\tmplength}
\settowidth{\labelsep}{X}
\addtolength{\leftmargin}{\labelsep}
\setlength{\labelwidth}{\tmplength}
}
\item[\textbf{Declaration}\hfill]
\ifpdf
\begin{flushleft}
\fi
\begin{ttfamily}
ConsoleWidth = 80;\end{ttfamily}

\ifpdf
\end{flushleft}
\fi

\par
\item[\textbf{Description}]
Width of console

\end{list}
\section{Author}
\par
Johannes Berg {$<$}johannes@sipsolutions.de{$>$}

\chapter{Unit PasDoc{\_}Parser}
\label{PasDoc_Parser}
\index{PasDoc{\_}Parser}
\section{Description}
Parse ObjectPascal code.\hfill\vspace*{1ex}

     

Contains the \begin{ttfamily}TParser\end{ttfamily}(\ref{PasDoc_Parser.TParser}) object, which can parse an ObjectPascal code, and put the collected information into the TPasUnit instance.
\section{Uses}
\begin{itemize}
\item \begin{ttfamily}SysUtils\end{ttfamily}\item \begin{ttfamily}Classes\end{ttfamily}\item \begin{ttfamily}Contnrs\end{ttfamily}\item \begin{ttfamily}StrUtils\end{ttfamily}\item \begin{ttfamily}PasDoc{\_}Types\end{ttfamily}(\ref{PasDoc_Types})\item \begin{ttfamily}PasDoc{\_}Items\end{ttfamily}(\ref{PasDoc_Items})\item \begin{ttfamily}PasDoc{\_}Scanner\end{ttfamily}(\ref{PasDoc_Scanner})\item \begin{ttfamily}PasDoc{\_}Tokenizer\end{ttfamily}(\ref{PasDoc_Tokenizer})\item \begin{ttfamily}PasDoc{\_}StringPairVector\end{ttfamily}(\ref{PasDoc_StringPairVector})\item \begin{ttfamily}PasDoc{\_}StringVector\end{ttfamily}(\ref{PasDoc_StringVector})\end{itemize}
\section{Overview}
\begin{description}
\item[\texttt{\begin{ttfamily}EInternalParserError\end{ttfamily} Class}]Raised when an impossible situation (indicating bug in pasdoc) occurs.
\item[\texttt{\begin{ttfamily}TPasCioHelper\end{ttfamily} Class}]\begin{ttfamily}TPasCioHelper\end{ttfamily} stores a CIO reference and current state.
\item[\texttt{\begin{ttfamily}TPasCioHelperStack\end{ttfamily} Class}]A stack of \begin{ttfamily}TPasCioHelper\end{ttfamily}(\ref{PasDoc_Parser.TPasCioHelper}) objects currently used to parse nested classes and records
\item[\texttt{\begin{ttfamily}TRawDescriptionInfoList\end{ttfamily} Class}]\begin{ttfamily}TRawDescriptionInfoList\end{ttfamily} stores a series of \begin{ttfamily}TRawDescriptionInfos\end{ttfamily}(\ref{PasDoc_Items.TRawDescriptionInfo}).
\item[\texttt{\begin{ttfamily}TParser\end{ttfamily} Class}]Parser class that will process a complete unit file and all of its include files, regarding directives.
\end{description}
\section{Classes, Interfaces, Objects and Records}
\ifpdf
\subsection*{\large{\textbf{EInternalParserError Class}}\normalsize\hspace{1ex}\hrulefill}
\else
\subsection*{EInternalParserError Class}
\fi
\label{PasDoc_Parser.EInternalParserError}
\index{EInternalParserError}
\subsubsection*{\large{\textbf{Hierarchy}}\normalsize\hspace{1ex}\hfill}
EInternalParserError {$>$} Exception
\subsubsection*{\large{\textbf{Description}}\normalsize\hspace{1ex}\hfill}
Raised when an impossible situation (indicating bug in pasdoc) occurs.\ifpdf
\subsection*{\large{\textbf{TPasCioHelper Class}}\normalsize\hspace{1ex}\hrulefill}
\else
\subsection*{TPasCioHelper Class}
\fi
\label{PasDoc_Parser.TPasCioHelper}
\index{TPasCioHelper}
\subsubsection*{\large{\textbf{Hierarchy}}\normalsize\hspace{1ex}\hfill}
TPasCioHelper {$>$} TObject
\subsubsection*{\large{\textbf{Description}}\normalsize\hspace{1ex}\hfill}
\begin{ttfamily}TPasCioHelper\end{ttfamily} stores a CIO reference and current state.\subsubsection*{\large{\textbf{Properties}}\normalsize\hspace{1ex}\hfill}
\begin{list}{}{
\settowidth{\tmplength}{\textbf{CurVisibility}}
\setlength{\itemindent}{0cm}
\setlength{\listparindent}{0cm}
\setlength{\leftmargin}{\evensidemargin}
\addtolength{\leftmargin}{\tmplength}
\settowidth{\labelsep}{X}
\addtolength{\leftmargin}{\labelsep}
\setlength{\labelwidth}{\tmplength}
}
\label{PasDoc_Parser.TPasCioHelper-Cio}
\index{Cio}
\item[\textbf{Cio}\hfill]
\ifpdf
\begin{flushleft}
\fi
\begin{ttfamily}
public property Cio: TPasCio read FCio write FCio;\end{ttfamily}

\ifpdf
\end{flushleft}
\fi


\par  \label{PasDoc_Parser.TPasCioHelper-CurVisibility}
\index{CurVisibility}
\item[\textbf{CurVisibility}\hfill]
\ifpdf
\begin{flushleft}
\fi
\begin{ttfamily}
public property CurVisibility: TVisibility read FCurVisibility write FCurVisibility;\end{ttfamily}

\ifpdf
\end{flushleft}
\fi


\par  \label{PasDoc_Parser.TPasCioHelper-Mode}
\index{Mode}
\item[\textbf{Mode}\hfill]
\ifpdf
\begin{flushleft}
\fi
\begin{ttfamily}
public property Mode: TItemParseMode read FMode write FMode;\end{ttfamily}

\ifpdf
\end{flushleft}
\fi


\par  \label{PasDoc_Parser.TPasCioHelper-SkipCioDecl}
\index{SkipCioDecl}
\item[\textbf{SkipCioDecl}\hfill]
\ifpdf
\begin{flushleft}
\fi
\begin{ttfamily}
public property SkipCioDecl: Boolean read FSkipCioDecl write FSkipCioDecl;\end{ttfamily}

\ifpdf
\end{flushleft}
\fi


\par  \end{list}
\subsubsection*{\large{\textbf{Methods}}\normalsize\hspace{1ex}\hfill}
\paragraph*{FreeAll}\hspace*{\fill}

\label{PasDoc_Parser.TPasCioHelper-FreeAll}
\index{FreeAll}
\begin{list}{}{
\settowidth{\tmplength}{\textbf{Description}}
\setlength{\itemindent}{0cm}
\setlength{\listparindent}{0cm}
\setlength{\leftmargin}{\evensidemargin}
\addtolength{\leftmargin}{\tmplength}
\settowidth{\labelsep}{X}
\addtolength{\leftmargin}{\labelsep}
\setlength{\labelwidth}{\tmplength}
}
\item[\textbf{Declaration}\hfill]
\ifpdf
\begin{flushleft}
\fi
\begin{ttfamily}
public procedure FreeAll;\end{ttfamily}

\ifpdf
\end{flushleft}
\fi

\par
\item[\textbf{Description}]
Frees included objects and calls its own destructor. Objects are not owned by default.

\end{list}
\ifpdf
\subsection*{\large{\textbf{TPasCioHelperStack Class}}\normalsize\hspace{1ex}\hrulefill}
\else
\subsection*{TPasCioHelperStack Class}
\fi
\label{PasDoc_Parser.TPasCioHelperStack}
\index{TPasCioHelperStack}
\subsubsection*{\large{\textbf{Hierarchy}}\normalsize\hspace{1ex}\hfill}
TPasCioHelperStack {$>$} TObjectStack
\subsubsection*{\large{\textbf{Description}}\normalsize\hspace{1ex}\hfill}
A stack of \begin{ttfamily}TPasCioHelper\end{ttfamily}(\ref{PasDoc_Parser.TPasCioHelper}) objects currently used to parse nested classes and records\subsubsection*{\large{\textbf{Methods}}\normalsize\hspace{1ex}\hfill}
\paragraph*{Clear}\hspace*{\fill}

\label{PasDoc_Parser.TPasCioHelperStack-Clear}
\index{Clear}
\begin{list}{}{
\settowidth{\tmplength}{\textbf{Description}}
\setlength{\itemindent}{0cm}
\setlength{\listparindent}{0cm}
\setlength{\leftmargin}{\evensidemargin}
\addtolength{\leftmargin}{\tmplength}
\settowidth{\labelsep}{X}
\addtolength{\leftmargin}{\labelsep}
\setlength{\labelwidth}{\tmplength}
}
\item[\textbf{Declaration}\hfill]
\ifpdf
\begin{flushleft}
\fi
\begin{ttfamily}
public procedure Clear;\end{ttfamily}

\ifpdf
\end{flushleft}
\fi

\par
\item[\textbf{Description}]
Frees all items including their CIOs and clears the stack

\end{list}
\paragraph*{Push}\hspace*{\fill}

\label{PasDoc_Parser.TPasCioHelperStack-Push}
\index{Push}
\begin{list}{}{
\settowidth{\tmplength}{\textbf{Description}}
\setlength{\itemindent}{0cm}
\setlength{\listparindent}{0cm}
\setlength{\leftmargin}{\evensidemargin}
\addtolength{\leftmargin}{\tmplength}
\settowidth{\labelsep}{X}
\addtolength{\leftmargin}{\labelsep}
\setlength{\labelwidth}{\tmplength}
}
\item[\textbf{Declaration}\hfill]
\ifpdf
\begin{flushleft}
\fi
\begin{ttfamily}
public function Push(AHelper: TPasCioHelper): TPasCioHelper; inline;\end{ttfamily}

\ifpdf
\end{flushleft}
\fi

\end{list}
\paragraph*{Pop}\hspace*{\fill}

\label{PasDoc_Parser.TPasCioHelperStack-Pop}
\index{Pop}
\begin{list}{}{
\settowidth{\tmplength}{\textbf{Description}}
\setlength{\itemindent}{0cm}
\setlength{\listparindent}{0cm}
\setlength{\leftmargin}{\evensidemargin}
\addtolength{\leftmargin}{\tmplength}
\settowidth{\labelsep}{X}
\addtolength{\leftmargin}{\labelsep}
\setlength{\labelwidth}{\tmplength}
}
\item[\textbf{Declaration}\hfill]
\ifpdf
\begin{flushleft}
\fi
\begin{ttfamily}
public function Pop: TPasCioHelper; inline;\end{ttfamily}

\ifpdf
\end{flushleft}
\fi

\end{list}
\paragraph*{Peek}\hspace*{\fill}

\label{PasDoc_Parser.TPasCioHelperStack-Peek}
\index{Peek}
\begin{list}{}{
\settowidth{\tmplength}{\textbf{Description}}
\setlength{\itemindent}{0cm}
\setlength{\listparindent}{0cm}
\setlength{\leftmargin}{\evensidemargin}
\addtolength{\leftmargin}{\tmplength}
\settowidth{\labelsep}{X}
\addtolength{\leftmargin}{\labelsep}
\setlength{\labelwidth}{\tmplength}
}
\item[\textbf{Declaration}\hfill]
\ifpdf
\begin{flushleft}
\fi
\begin{ttfamily}
public function Peek: TPasCioHelper; inline;\end{ttfamily}

\ifpdf
\end{flushleft}
\fi

\end{list}
\ifpdf
\subsection*{\large{\textbf{TRawDescriptionInfoList Class}}\normalsize\hspace{1ex}\hrulefill}
\else
\subsection*{TRawDescriptionInfoList Class}
\fi
\label{PasDoc_Parser.TRawDescriptionInfoList}
\index{TRawDescriptionInfoList}
\subsubsection*{\large{\textbf{Hierarchy}}\normalsize\hspace{1ex}\hfill}
TRawDescriptionInfoList {$>$} TObject
\subsubsection*{\large{\textbf{Description}}\normalsize\hspace{1ex}\hfill}
\begin{ttfamily}TRawDescriptionInfoList\end{ttfamily} stores a series of \begin{ttfamily}TRawDescriptionInfos\end{ttfamily}(\ref{PasDoc_Items.TRawDescriptionInfo}). It is modelled after TStringList but has only the minimum number of methods required for use in PasDoc.\subsubsection*{\large{\textbf{Properties}}\normalsize\hspace{1ex}\hfill}
\begin{list}{}{
\settowidth{\tmplength}{\textbf{Count}}
\setlength{\itemindent}{0cm}
\setlength{\listparindent}{0cm}
\setlength{\leftmargin}{\evensidemargin}
\addtolength{\leftmargin}{\tmplength}
\settowidth{\labelsep}{X}
\addtolength{\leftmargin}{\labelsep}
\setlength{\labelwidth}{\tmplength}
}
\label{PasDoc_Parser.TRawDescriptionInfoList-Count}
\index{Count}
\item[\textbf{Count}\hfill]
\ifpdf
\begin{flushleft}
\fi
\begin{ttfamily}
public property Count: integer read FCount;\end{ttfamily}

\ifpdf
\end{flushleft}
\fi


\par \begin{ttfamily}Count\end{ttfamily} is the number of \begin{ttfamily}TRawDescriptionInfos\end{ttfamily}(\ref{PasDoc_Items.TRawDescriptionInfo}) in \begin{ttfamily}TRawDescriptionInfoList\end{ttfamily}.\label{PasDoc_Parser.TRawDescriptionInfoList-Items}
\index{Items}
\item[\textbf{Items}\hfill]
\ifpdf
\begin{flushleft}
\fi
\begin{ttfamily}
public property Items[Index:integer]: TRawDescriptionInfo read GetItems;\end{ttfamily}

\ifpdf
\end{flushleft}
\fi


\par \begin{ttfamily}Items\end{ttfamily} provides read access to the \begin{ttfamily}TRawDescriptionInfos\end{ttfamily}(\ref{PasDoc_Items.TRawDescriptionInfo}) in \begin{ttfamily}TRawDescriptionInfoList\end{ttfamily}.\end{list}
\subsubsection*{\large{\textbf{Methods}}\normalsize\hspace{1ex}\hfill}
\paragraph*{Append}\hspace*{\fill}

\label{PasDoc_Parser.TRawDescriptionInfoList-Append}
\index{Append}
\begin{list}{}{
\settowidth{\tmplength}{\textbf{Description}}
\setlength{\itemindent}{0cm}
\setlength{\listparindent}{0cm}
\setlength{\leftmargin}{\evensidemargin}
\addtolength{\leftmargin}{\tmplength}
\settowidth{\labelsep}{X}
\addtolength{\leftmargin}{\labelsep}
\setlength{\labelwidth}{\tmplength}
}
\item[\textbf{Declaration}\hfill]
\ifpdf
\begin{flushleft}
\fi
\begin{ttfamily}
public function Append(Comment: TRawDescriptionInfo): integer;\end{ttfamily}

\ifpdf
\end{flushleft}
\fi

\par
\item[\textbf{Description}]
\begin{ttfamily}Append\end{ttfamily} adds a new \begin{ttfamily}TRawDescriptionInfo\end{ttfamily}(\ref{PasDoc_Items.TRawDescriptionInfo}) to \begin{ttfamily}TRawDescriptionInfoList\end{ttfamily}.

\end{list}
\paragraph*{Create}\hspace*{\fill}

\label{PasDoc_Parser.TRawDescriptionInfoList-Create}
\index{Create}
\begin{list}{}{
\settowidth{\tmplength}{\textbf{Description}}
\setlength{\itemindent}{0cm}
\setlength{\listparindent}{0cm}
\setlength{\leftmargin}{\evensidemargin}
\addtolength{\leftmargin}{\tmplength}
\settowidth{\labelsep}{X}
\addtolength{\leftmargin}{\labelsep}
\setlength{\labelwidth}{\tmplength}
}
\item[\textbf{Declaration}\hfill]
\ifpdf
\begin{flushleft}
\fi
\begin{ttfamily}
public Constructor Create;\end{ttfamily}

\ifpdf
\end{flushleft}
\fi

\end{list}
\ifpdf
\subsection*{\large{\textbf{TParser Class}}\normalsize\hspace{1ex}\hrulefill}
\else
\subsection*{TParser Class}
\fi
\label{PasDoc_Parser.TParser}
\index{TParser}
\subsubsection*{\large{\textbf{Hierarchy}}\normalsize\hspace{1ex}\hfill}
TParser {$>$} TObject
\subsubsection*{\large{\textbf{Description}}\normalsize\hspace{1ex}\hfill}
Parser class that will process a complete unit file and all of its include files, regarding directives. When creating this object constructor \begin{ttfamily}Create\end{ttfamily}(\ref{PasDoc_Parser.TParser-Create}) takes as an argument an input stream and a list of directives. Parsing work is done by calling \begin{ttfamily}ParseUnitOrProgram\end{ttfamily}(\ref{PasDoc_Parser.TParser-ParseUnitOrProgram}) method. If no errors appear, should return a \begin{ttfamily}TPasUnit\end{ttfamily}(\ref{PasDoc_Items.TPasUnit}) object with all information on the unit. Else exception is raised.

Things that parser inits in items it returns:

\begin{itemize}
\item Of every TPasItem : Name, RawDescription, Visibility, HintDirectives, DeprecatedNote, FullDeclararation (note: for now not all items get sensible FullDeclararation, but the intention is to improve this over time; see \begin{ttfamily}TPasItem.FullDeclaration\end{ttfamily}(\ref{PasDoc_Items.TPasItem-FullDeclaration}) to know where FullDeclararation is available now).

Note to IsDeprecated: parser inits it basing on hint directive "deprecated" presence in source file; it doesn't handle the fact that @deprecated tag may be specified inside RawDescription.

Note to RawDescription: parser inits them from user's comments that preceded given item in source file. It doesn't handle the fact that @member and @value tags may also assign RawDescription for some item.
\item Of TPasCio: Ancestors, Fields, Methods, Properties, MyType.
\item Of TPasEnum: Members, FullDeclararation.
\item Of TPasMethod: What.
\item Of TPasVarConst: FullDeclaration.
\item Of TPasProperty: IndexDecl, FullDeclaration. PropType (only if was specified in property declaration). It was intended that parser will also set Default, NoDefault, StoredId, DefaultId, Reader, Writer attributes, but it's still not implemented.
\item Of TPasUnit; UsesUnits, Types, Variables, CIOs, Constants, FuncsProcs.
\end{itemize}

It doesn't init other values. E.g. AbstractDescription or DetailedDescription of TPasItem should be inited while expanding this item's tags. E.g. SourceFileDateTime and SourceFileName of TPasUnit must be set by other means.\subsubsection*{\large{\textbf{Properties}}\normalsize\hspace{1ex}\hfill}
\begin{list}{}{
\settowidth{\tmplength}{\textbf{ImplicitVisibility}}
\setlength{\itemindent}{0cm}
\setlength{\listparindent}{0cm}
\setlength{\leftmargin}{\evensidemargin}
\addtolength{\leftmargin}{\tmplength}
\settowidth{\labelsep}{X}
\addtolength{\leftmargin}{\labelsep}
\setlength{\labelwidth}{\tmplength}
}
\label{PasDoc_Parser.TParser-OnMessage}
\index{OnMessage}
\item[\textbf{OnMessage}\hfill]
\ifpdf
\begin{flushleft}
\fi
\begin{ttfamily}
public property OnMessage: TPasDocMessageEvent read FOnMessage write FOnMessage;\end{ttfamily}

\ifpdf
\end{flushleft}
\fi


\par  \label{PasDoc_Parser.TParser-CommentMarkers}
\index{CommentMarkers}
\item[\textbf{CommentMarkers}\hfill]
\ifpdf
\begin{flushleft}
\fi
\begin{ttfamily}
public property CommentMarkers: TStringList read FCommentMarkers write SetCommentMarkers;\end{ttfamily}

\ifpdf
\end{flushleft}
\fi


\par  \label{PasDoc_Parser.TParser-MarkersOptional}
\index{MarkersOptional}
\item[\textbf{MarkersOptional}\hfill]
\ifpdf
\begin{flushleft}
\fi
\begin{ttfamily}
public property MarkersOptional: boolean read fMarkersOptional write fMarkersOptional;\end{ttfamily}

\ifpdf
\end{flushleft}
\fi


\par  \label{PasDoc_Parser.TParser-IgnoreLeading}
\index{IgnoreLeading}
\item[\textbf{IgnoreLeading}\hfill]
\ifpdf
\begin{flushleft}
\fi
\begin{ttfamily}
public property IgnoreLeading: string read FIgnoreLeading write FIgnoreLeading;\end{ttfamily}

\ifpdf
\end{flushleft}
\fi


\par  \label{PasDoc_Parser.TParser-IgnoreMarkers}
\index{IgnoreMarkers}
\item[\textbf{IgnoreMarkers}\hfill]
\ifpdf
\begin{flushleft}
\fi
\begin{ttfamily}
public property IgnoreMarkers: TStringList read FIgnoreMarkers write SetIgnoreMarkers;\end{ttfamily}

\ifpdf
\end{flushleft}
\fi


\par  \label{PasDoc_Parser.TParser-ShowVisibilities}
\index{ShowVisibilities}
\item[\textbf{ShowVisibilities}\hfill]
\ifpdf
\begin{flushleft}
\fi
\begin{ttfamily}
public property ShowVisibilities: TVisibilities
      read FShowVisibilities write FShowVisibilities;\end{ttfamily}

\ifpdf
\end{flushleft}
\fi


\par  \label{PasDoc_Parser.TParser-ImplicitVisibility}
\index{ImplicitVisibility}
\item[\textbf{ImplicitVisibility}\hfill]
\ifpdf
\begin{flushleft}
\fi
\begin{ttfamily}
public property ImplicitVisibility: TImplicitVisibility
      read FImplicitVisibility write FImplicitVisibility;\end{ttfamily}

\ifpdf
\end{flushleft}
\fi


\par See command{-}line option {-}{-}implicit{-}visibility documentation at [\href{https://github.com/pasdoc/pasdoc/wiki/ImplicitVisibilityOption}{https://github.com/pasdoc/pasdoc/wiki/ImplicitVisibilityOption}]\label{PasDoc_Parser.TParser-AutoBackComments}
\index{AutoBackComments}
\item[\textbf{AutoBackComments}\hfill]
\ifpdf
\begin{flushleft}
\fi
\begin{ttfamily}
public property AutoBackComments: boolean read FAutoBackComments write FAutoBackComments;\end{ttfamily}

\ifpdf
\end{flushleft}
\fi


\par See command{-}line option {-}{-}auto{-}back{-}comments documentation at [\href{https://github.com/pasdoc/pasdoc/wiki/AutoBackComments}{https://github.com/pasdoc/pasdoc/wiki/AutoBackComments}]\label{PasDoc_Parser.TParser-InfoMergeType}
\index{InfoMergeType}
\item[\textbf{InfoMergeType}\hfill]
\ifpdf
\begin{flushleft}
\fi
\begin{ttfamily}
public property InfoMergeType: TInfoMergeType read FInfoMergeType write FInfoMergeType;\end{ttfamily}

\ifpdf
\end{flushleft}
\fi


\par TODO comment\end{list}
\subsubsection*{\large{\textbf{Methods}}\normalsize\hspace{1ex}\hfill}
\paragraph*{Create}\hspace*{\fill}

\label{PasDoc_Parser.TParser-Create}
\index{Create}
\begin{list}{}{
\settowidth{\tmplength}{\textbf{Description}}
\setlength{\itemindent}{0cm}
\setlength{\listparindent}{0cm}
\setlength{\leftmargin}{\evensidemargin}
\addtolength{\leftmargin}{\tmplength}
\settowidth{\labelsep}{X}
\addtolength{\leftmargin}{\labelsep}
\setlength{\labelwidth}{\tmplength}
}
\item[\textbf{Declaration}\hfill]
\ifpdf
\begin{flushleft}
\fi
\begin{ttfamily}
public constructor Create( const InputStream: TStream; const Directives: TStringVector; const IncludeFilePaths: TStringVector; const OnMessageEvent: TPasDocMessageEvent; const VerbosityLevel: Cardinal; const AStreamName, AStreamPath: string; const AHandleMacros: boolean);\end{ttfamily}

\ifpdf
\end{flushleft}
\fi

\par
\item[\textbf{Description}]
Create a parser, initialize the scanner with input stream S. All strings in SD are defined compiler directives.

\end{list}
\paragraph*{Destroy}\hspace*{\fill}

\label{PasDoc_Parser.TParser-Destroy}
\index{Destroy}
\begin{list}{}{
\settowidth{\tmplength}{\textbf{Description}}
\setlength{\itemindent}{0cm}
\setlength{\listparindent}{0cm}
\setlength{\leftmargin}{\evensidemargin}
\addtolength{\leftmargin}{\tmplength}
\settowidth{\labelsep}{X}
\addtolength{\leftmargin}{\labelsep}
\setlength{\labelwidth}{\tmplength}
}
\item[\textbf{Declaration}\hfill]
\ifpdf
\begin{flushleft}
\fi
\begin{ttfamily}
public destructor Destroy; override;\end{ttfamily}

\ifpdf
\end{flushleft}
\fi

\par
\item[\textbf{Description}]
Release all dynamically allocated memory.

\end{list}
\paragraph*{ParseUnitOrProgram}\hspace*{\fill}

\label{PasDoc_Parser.TParser-ParseUnitOrProgram}
\index{ParseUnitOrProgram}
\begin{list}{}{
\settowidth{\tmplength}{\textbf{Description}}
\setlength{\itemindent}{0cm}
\setlength{\listparindent}{0cm}
\setlength{\leftmargin}{\evensidemargin}
\addtolength{\leftmargin}{\tmplength}
\settowidth{\labelsep}{X}
\addtolength{\leftmargin}{\labelsep}
\setlength{\labelwidth}{\tmplength}
}
\item[\textbf{Declaration}\hfill]
\ifpdf
\begin{flushleft}
\fi
\begin{ttfamily}
public procedure ParseUnitOrProgram(var U: TPasUnit);\end{ttfamily}

\ifpdf
\end{flushleft}
\fi

\par
\item[\textbf{Description}]
This does the real parsing work, creating U unit and parsing InputStream and filling all U properties.

\end{list}
\section{Types}
\ifpdf
\subsection*{\large{\textbf{TItemParseMode}}\normalsize\hspace{1ex}\hrulefill}
\else
\subsection*{TItemParseMode}
\fi
\label{PasDoc_Parser-TItemParseMode}
\index{TItemParseMode}
\begin{list}{}{
\settowidth{\tmplength}{\textbf{Description}}
\setlength{\itemindent}{0cm}
\setlength{\listparindent}{0cm}
\setlength{\leftmargin}{\evensidemargin}
\addtolength{\leftmargin}{\tmplength}
\settowidth{\labelsep}{X}
\addtolength{\leftmargin}{\labelsep}
\setlength{\labelwidth}{\tmplength}
}
\item[\textbf{Declaration}\hfill]
\ifpdf
\begin{flushleft}
\fi
\begin{ttfamily}
TItemParseMode = (...);\end{ttfamily}

\ifpdf
\end{flushleft}
\fi

\par
\item[\textbf{Description}]
 \item[\textbf{Values}]
\begin{description}
\item[\texttt{pmUndefined}] \label{PasDoc_Parser-pmUndefined}
\index{}
 
\item[\texttt{pmConst}] \label{PasDoc_Parser-pmConst}
\index{}
 
\item[\texttt{pmVar}] \label{PasDoc_Parser-pmVar}
\index{}
 
\item[\texttt{pmType}] \label{PasDoc_Parser-pmType}
\index{}
 
\end{description}


\end{list}
\ifpdf
\subsection*{\large{\textbf{TOwnerItemType}}\normalsize\hspace{1ex}\hrulefill}
\else
\subsection*{TOwnerItemType}
\fi
\label{PasDoc_Parser-TOwnerItemType}
\index{TOwnerItemType}
\begin{list}{}{
\settowidth{\tmplength}{\textbf{Description}}
\setlength{\itemindent}{0cm}
\setlength{\listparindent}{0cm}
\setlength{\leftmargin}{\evensidemargin}
\addtolength{\leftmargin}{\tmplength}
\settowidth{\labelsep}{X}
\addtolength{\leftmargin}{\labelsep}
\setlength{\labelwidth}{\tmplength}
}
\item[\textbf{Declaration}\hfill]
\ifpdf
\begin{flushleft}
\fi
\begin{ttfamily}
TOwnerItemType = (...);\end{ttfamily}

\ifpdf
\end{flushleft}
\fi

\par
\item[\textbf{Description}]
 \item[\textbf{Values}]
\begin{description}
\item[\texttt{otUnit}] \label{PasDoc_Parser-otUnit}
\index{}
 
\item[\texttt{otCio}] \label{PasDoc_Parser-otCio}
\index{}
 
\end{description}


\end{list}
\section{Authors}
\par
Ralf Junker (delphi@zeitungsjunge.de)

\par
Marco Schmidt (marcoschmidt@geocities.com)

\par
Johannes Berg {$<$}johannes@sipsolutions.de{$>$}

\par
Michalis Kamburelis

\par
Arno Garrels {$<$}first name.name@nospamgmx.de{$>$}

\chapter{Unit PasDoc{\_}ProcessLineTalk}
\label{PasDoc_ProcessLineTalk}
\index{PasDoc{\_}ProcessLineTalk}
\section{Description}
Talking with another process through pipes.\hfill\vspace*{1ex}

  
\section{Uses}
\begin{itemize}
\item \begin{ttfamily}SysUtils\end{ttfamily}\item \begin{ttfamily}Classes\end{ttfamily}\end{itemize}
\section{Overview}
\begin{description}
\item[\texttt{\begin{ttfamily}TTextReader\end{ttfamily} Class}]TTextReader reads given Stream line by line.
\item[\texttt{\begin{ttfamily}TProcessLineTalk\end{ttfamily} Class}]This is a subclass of TProcess that allows to easy "talk" with executed process by pipes (read process stdout/stderr, write to process stdin) on a line{-}by{-}line basis.
\end{description}
\section{Classes, Interfaces, Objects and Records}
\ifpdf
\subsection*{\large{\textbf{TTextReader Class}}\normalsize\hspace{1ex}\hrulefill}
\else
\subsection*{TTextReader Class}
\fi
\label{PasDoc_ProcessLineTalk.TTextReader}
\index{TTextReader}
\subsubsection*{\large{\textbf{Hierarchy}}\normalsize\hspace{1ex}\hfill}
TTextReader {$>$} TObject
\subsubsection*{\large{\textbf{Description}}\normalsize\hspace{1ex}\hfill}
TTextReader reads given Stream line by line. Lines may be terminated in Stream with {\#}13, {\#}10, {\#}13+{\#}10 or {\#}10+{\#}13. This way I can treat any TStream quite like standard Pascal text files: I have simple Readln method.

After calling Readln or Eof you should STOP directly using underlying Stream (but you CAN use Stream right after creating TTextReader.Create(Stream) and before any Readln or Eof operations on this TTextReader).

Original version of this class comes from Michalis Kamburelis code library, see [\href{http://www.camelot.homedns.org/~michalis/}{http://www.camelot.homedns.org/~michalis/}], unit base/KambiClassUtils.pas.\subsubsection*{\large{\textbf{Methods}}\normalsize\hspace{1ex}\hfill}
\paragraph*{CreateFromFileStream}\hspace*{\fill}

\label{PasDoc_ProcessLineTalk.TTextReader-CreateFromFileStream}
\index{CreateFromFileStream}
\begin{list}{}{
\settowidth{\tmplength}{\textbf{Description}}
\setlength{\itemindent}{0cm}
\setlength{\listparindent}{0cm}
\setlength{\leftmargin}{\evensidemargin}
\addtolength{\leftmargin}{\tmplength}
\settowidth{\labelsep}{X}
\addtolength{\leftmargin}{\labelsep}
\setlength{\labelwidth}{\tmplength}
}
\item[\textbf{Declaration}\hfill]
\ifpdf
\begin{flushleft}
\fi
\begin{ttfamily}
public constructor CreateFromFileStream(const FileName: string);\end{ttfamily}

\ifpdf
\end{flushleft}
\fi

\par
\item[\textbf{Description}]
This is a comfortable constructor, equivalent to TTextReader.Create(TFileStream.Create(FileName, fmOpenRead or fmShareDenyWrite), true)

\end{list}
\paragraph*{Create}\hspace*{\fill}

\label{PasDoc_ProcessLineTalk.TTextReader-Create}
\index{Create}
\begin{list}{}{
\settowidth{\tmplength}{\textbf{Description}}
\setlength{\itemindent}{0cm}
\setlength{\listparindent}{0cm}
\setlength{\leftmargin}{\evensidemargin}
\addtolength{\leftmargin}{\tmplength}
\settowidth{\labelsep}{X}
\addtolength{\leftmargin}{\labelsep}
\setlength{\labelwidth}{\tmplength}
}
\item[\textbf{Declaration}\hfill]
\ifpdf
\begin{flushleft}
\fi
\begin{ttfamily}
public constructor Create(AStream: TStream; AOwnsStream: boolean);\end{ttfamily}

\ifpdf
\end{flushleft}
\fi

\par
\item[\textbf{Description}]
If AOwnsStream then in Destroy we will free Stream object.

\end{list}
\paragraph*{Destroy}\hspace*{\fill}

\label{PasDoc_ProcessLineTalk.TTextReader-Destroy}
\index{Destroy}
\begin{list}{}{
\settowidth{\tmplength}{\textbf{Description}}
\setlength{\itemindent}{0cm}
\setlength{\listparindent}{0cm}
\setlength{\leftmargin}{\evensidemargin}
\addtolength{\leftmargin}{\tmplength}
\settowidth{\labelsep}{X}
\addtolength{\leftmargin}{\labelsep}
\setlength{\labelwidth}{\tmplength}
}
\item[\textbf{Declaration}\hfill]
\ifpdf
\begin{flushleft}
\fi
\begin{ttfamily}
public destructor Destroy; override;\end{ttfamily}

\ifpdf
\end{flushleft}
\fi

\end{list}
\paragraph*{Readln}\hspace*{\fill}

\label{PasDoc_ProcessLineTalk.TTextReader-Readln}
\index{Readln}
\begin{list}{}{
\settowidth{\tmplength}{\textbf{Description}}
\setlength{\itemindent}{0cm}
\setlength{\listparindent}{0cm}
\setlength{\leftmargin}{\evensidemargin}
\addtolength{\leftmargin}{\tmplength}
\settowidth{\labelsep}{X}
\addtolength{\leftmargin}{\labelsep}
\setlength{\labelwidth}{\tmplength}
}
\item[\textbf{Declaration}\hfill]
\ifpdf
\begin{flushleft}
\fi
\begin{ttfamily}
public function Readln: string;\end{ttfamily}

\ifpdf
\end{flushleft}
\fi

\par
\item[\textbf{Description}]
Reads next line from Stream. Returned string does not contain any end{-}of{-}line characters.

\end{list}
\paragraph*{Eof}\hspace*{\fill}

\label{PasDoc_ProcessLineTalk.TTextReader-Eof}
\index{Eof}
\begin{list}{}{
\settowidth{\tmplength}{\textbf{Description}}
\setlength{\itemindent}{0cm}
\setlength{\listparindent}{0cm}
\setlength{\leftmargin}{\evensidemargin}
\addtolength{\leftmargin}{\tmplength}
\settowidth{\labelsep}{X}
\addtolength{\leftmargin}{\labelsep}
\setlength{\labelwidth}{\tmplength}
}
\item[\textbf{Declaration}\hfill]
\ifpdf
\begin{flushleft}
\fi
\begin{ttfamily}
public function Eof: boolean;\end{ttfamily}

\ifpdf
\end{flushleft}
\fi

\end{list}
\ifpdf
\subsection*{\large{\textbf{TProcessLineTalk Class}}\normalsize\hspace{1ex}\hrulefill}
\else
\subsection*{TProcessLineTalk Class}
\fi
\label{PasDoc_ProcessLineTalk.TProcessLineTalk}
\index{TProcessLineTalk}
\subsubsection*{\large{\textbf{Hierarchy}}\normalsize\hspace{1ex}\hfill}
TProcessLineTalk {$>$} TComponent
\subsubsection*{\large{\textbf{Description}}\normalsize\hspace{1ex}\hfill}
This is a subclass of TProcess that allows to easy "talk" with executed process by pipes (read process stdout/stderr, write to process stdin) on a line{-}by{-}line basis.

If symbol HAS{\_}PROCESS is not defined, this defines a junky implementation of TProcessLineTalk class that can't do anything and raises exception when you try to execute a process.\subsubsection*{\large{\textbf{Properties}}\normalsize\hspace{1ex}\hfill}
\begin{list}{}{
\settowidth{\tmplength}{\textbf{CommandLine}}
\setlength{\itemindent}{0cm}
\setlength{\listparindent}{0cm}
\setlength{\leftmargin}{\evensidemargin}
\addtolength{\leftmargin}{\tmplength}
\settowidth{\labelsep}{X}
\addtolength{\leftmargin}{\labelsep}
\setlength{\labelwidth}{\tmplength}
}
\label{PasDoc_ProcessLineTalk.TProcessLineTalk-CommandLine}
\index{CommandLine}
\item[\textbf{CommandLine}\hfill]
\ifpdf
\begin{flushleft}
\fi
\begin{ttfamily}
published property CommandLine: string read FCommandLine write FCommandLine;\end{ttfamily}

\ifpdf
\end{flushleft}
\fi


\par  \label{PasDoc_ProcessLineTalk.TProcessLineTalk-Executable}
\index{Executable}
\item[\textbf{Executable}\hfill]
\ifpdf
\begin{flushleft}
\fi
\begin{ttfamily}
published property Executable: string read FExecutable write FExecutable;\end{ttfamily}

\ifpdf
\end{flushleft}
\fi


\par  \label{PasDoc_ProcessLineTalk.TProcessLineTalk-Parameters}
\index{Parameters}
\item[\textbf{Parameters}\hfill]
\ifpdf
\begin{flushleft}
\fi
\begin{ttfamily}
published property Parameters: TStrings read FParameters;\end{ttfamily}

\ifpdf
\end{flushleft}
\fi


\par  \end{list}
\subsubsection*{\large{\textbf{Methods}}\normalsize\hspace{1ex}\hfill}
\paragraph*{Execute}\hspace*{\fill}

\label{PasDoc_ProcessLineTalk.TProcessLineTalk-Execute}
\index{Execute}
\begin{list}{}{
\settowidth{\tmplength}{\textbf{Description}}
\setlength{\itemindent}{0cm}
\setlength{\listparindent}{0cm}
\setlength{\leftmargin}{\evensidemargin}
\addtolength{\leftmargin}{\tmplength}
\settowidth{\labelsep}{X}
\addtolength{\leftmargin}{\labelsep}
\setlength{\labelwidth}{\tmplength}
}
\item[\textbf{Declaration}\hfill]
\ifpdf
\begin{flushleft}
\fi
\begin{ttfamily}
public procedure Execute;\end{ttfamily}

\ifpdf
\end{flushleft}
\fi

\end{list}
\paragraph*{WriteLine}\hspace*{\fill}

\label{PasDoc_ProcessLineTalk.TProcessLineTalk-WriteLine}
\index{WriteLine}
\begin{list}{}{
\settowidth{\tmplength}{\textbf{Description}}
\setlength{\itemindent}{0cm}
\setlength{\listparindent}{0cm}
\setlength{\leftmargin}{\evensidemargin}
\addtolength{\leftmargin}{\tmplength}
\settowidth{\labelsep}{X}
\addtolength{\leftmargin}{\labelsep}
\setlength{\labelwidth}{\tmplength}
}
\item[\textbf{Declaration}\hfill]
\ifpdf
\begin{flushleft}
\fi
\begin{ttfamily}
public procedure WriteLine(const S: string);\end{ttfamily}

\ifpdf
\end{flushleft}
\fi

\end{list}
\paragraph*{ReadLine}\hspace*{\fill}

\label{PasDoc_ProcessLineTalk.TProcessLineTalk-ReadLine}
\index{ReadLine}
\begin{list}{}{
\settowidth{\tmplength}{\textbf{Description}}
\setlength{\itemindent}{0cm}
\setlength{\listparindent}{0cm}
\setlength{\leftmargin}{\evensidemargin}
\addtolength{\leftmargin}{\tmplength}
\settowidth{\labelsep}{X}
\addtolength{\leftmargin}{\labelsep}
\setlength{\labelwidth}{\tmplength}
}
\item[\textbf{Declaration}\hfill]
\ifpdf
\begin{flushleft}
\fi
\begin{ttfamily}
public function ReadLine: string;\end{ttfamily}

\ifpdf
\end{flushleft}
\fi

\end{list}
\paragraph*{Create}\hspace*{\fill}

\label{PasDoc_ProcessLineTalk.TProcessLineTalk-Create}
\index{Create}
\begin{list}{}{
\settowidth{\tmplength}{\textbf{Description}}
\setlength{\itemindent}{0cm}
\setlength{\listparindent}{0cm}
\setlength{\leftmargin}{\evensidemargin}
\addtolength{\leftmargin}{\tmplength}
\settowidth{\labelsep}{X}
\addtolength{\leftmargin}{\labelsep}
\setlength{\labelwidth}{\tmplength}
}
\item[\textbf{Declaration}\hfill]
\ifpdf
\begin{flushleft}
\fi
\begin{ttfamily}
public constructor Create(AOwner: TComponent); override;\end{ttfamily}

\ifpdf
\end{flushleft}
\fi

\end{list}
\paragraph*{Destroy}\hspace*{\fill}

\label{PasDoc_ProcessLineTalk.TProcessLineTalk-Destroy}
\index{Destroy}
\begin{list}{}{
\settowidth{\tmplength}{\textbf{Description}}
\setlength{\itemindent}{0cm}
\setlength{\listparindent}{0cm}
\setlength{\leftmargin}{\evensidemargin}
\addtolength{\leftmargin}{\tmplength}
\settowidth{\labelsep}{X}
\addtolength{\leftmargin}{\labelsep}
\setlength{\labelwidth}{\tmplength}
}
\item[\textbf{Declaration}\hfill]
\ifpdf
\begin{flushleft}
\fi
\begin{ttfamily}
public destructor Destroy; override;\end{ttfamily}

\ifpdf
\end{flushleft}
\fi

\end{list}
\section{Authors}
\par
Michalis Kamburelis

\par
Arno Garrels {$<$}first name.name@nospamgmx.de{$>$}

\chapter{Unit PasDoc{\_}Reg}
\label{PasDoc_Reg}
\index{PasDoc{\_}Reg}
\section{Description}
Registers the PasDoc components into the IDE. \hfill\vspace*{1ex}

   

TODO: We have some properties in TPasDoc and generators components that should be registered with filename editors.
\section{Overview}
\begin{description}
\item[\texttt{Register}]Registers the PasDoc components into the IDE.
\end{description}
\section{Functions and Procedures}
\ifpdf
\subsection*{\large{\textbf{Register}}\normalsize\hspace{1ex}\hrulefill}
\else
\subsection*{Register}
\fi
\label{PasDoc_Reg-Register}
\index{Register}
\begin{list}{}{
\settowidth{\tmplength}{\textbf{Description}}
\setlength{\itemindent}{0cm}
\setlength{\listparindent}{0cm}
\setlength{\leftmargin}{\evensidemargin}
\addtolength{\leftmargin}{\tmplength}
\settowidth{\labelsep}{X}
\addtolength{\leftmargin}{\labelsep}
\setlength{\labelwidth}{\tmplength}
}
\item[\textbf{Declaration}\hfill]
\ifpdf
\begin{flushleft}
\fi
\begin{ttfamily}
procedure Register;\end{ttfamily}

\ifpdf
\end{flushleft}
\fi

\par
\item[\textbf{Description}]
Registers the PasDoc components into the IDE.

\end{list}
\section{Authors}
\par
Ralf Junker (delphi@zeitungsjunge.de)

\par
Johannes Berg {$<$}johannes@sipsolutions.de{$>$}

\par
Michalis Kamburelis

\chapter{Unit PasDoc{\_}Scanner}
\label{PasDoc_Scanner}
\index{PasDoc{\_}Scanner}
\section{Description}
Simple Pascal scanner.\hfill\vspace*{1ex}

    



The scanner object \begin{ttfamily}TScanner\end{ttfamily}(\ref{PasDoc_Scanner.TScanner}) returns tokens from a Pascal language character input stream. It uses the \begin{ttfamily}PasDoc{\_}Tokenizer\end{ttfamily}(\ref{PasDoc_Tokenizer}) unit to get tokens, regarding conditional directives that might lead to including another files or will add or delete conditional symbols. Also handles FPC macros (when HandleMacros is true). So, this scanner is a combined tokenizer and pre{-}processor.
\section{Uses}
\begin{itemize}
\item \begin{ttfamily}SysUtils\end{ttfamily}\item \begin{ttfamily}Classes\end{ttfamily}\item \begin{ttfamily}PasDoc{\_}Types\end{ttfamily}(\ref{PasDoc_Types})\item \begin{ttfamily}PasDoc{\_}Tokenizer\end{ttfamily}(\ref{PasDoc_Tokenizer})\item \begin{ttfamily}PasDoc{\_}StringVector\end{ttfamily}(\ref{PasDoc_StringVector})\item \begin{ttfamily}PasDoc{\_}StreamUtils\end{ttfamily}(\ref{PasDoc_StreamUtils})\item \begin{ttfamily}PasDoc{\_}StringPairVector\end{ttfamily}(\ref{PasDoc_StringPairVector})\end{itemize}
\section{Overview}
\begin{description}
\item[\texttt{\begin{ttfamily}ETokenizerStreamEnd\end{ttfamily} Class}]
\item[\texttt{\begin{ttfamily}EInvalidIfCondition\end{ttfamily} Class}]
\item[\texttt{\begin{ttfamily}TScanner\end{ttfamily} Class}]This class scans one unit using one or more \begin{ttfamily}TTokenizer\end{ttfamily}(\ref{PasDoc_Tokenizer.TTokenizer}) objects to scan the unit and all nested include files.
\end{description}
\section{Classes, Interfaces, Objects and Records}
\ifpdf
\subsection*{\large{\textbf{ETokenizerStreamEnd Class}}\normalsize\hspace{1ex}\hrulefill}
\else
\subsection*{ETokenizerStreamEnd Class}
\fi
\label{PasDoc_Scanner.ETokenizerStreamEnd}
\index{ETokenizerStreamEnd}
\subsubsection*{\large{\textbf{Hierarchy}}\normalsize\hspace{1ex}\hfill}
ETokenizerStreamEnd {$>$} \begin{ttfamily}EPasDoc\end{ttfamily}(\ref{PasDoc_Types.EPasDoc}) {$>$} 
Exception
%%%%Description
\ifpdf
\subsection*{\large{\textbf{EInvalidIfCondition Class}}\normalsize\hspace{1ex}\hrulefill}
\else
\subsection*{EInvalidIfCondition Class}
\fi
\label{PasDoc_Scanner.EInvalidIfCondition}
\index{EInvalidIfCondition}
\subsubsection*{\large{\textbf{Hierarchy}}\normalsize\hspace{1ex}\hfill}
EInvalidIfCondition {$>$} \begin{ttfamily}EPasDoc\end{ttfamily}(\ref{PasDoc_Types.EPasDoc}) {$>$} 
Exception
%%%%Description
\ifpdf
\subsection*{\large{\textbf{TScanner Class}}\normalsize\hspace{1ex}\hrulefill}
\else
\subsection*{TScanner Class}
\fi
\label{PasDoc_Scanner.TScanner}
\index{TScanner}
\subsubsection*{\large{\textbf{Hierarchy}}\normalsize\hspace{1ex}\hfill}
TScanner {$>$} TObject
\subsubsection*{\large{\textbf{Description}}\normalsize\hspace{1ex}\hfill}
This class scans one unit using one or more \begin{ttfamily}TTokenizer\end{ttfamily}(\ref{PasDoc_Tokenizer.TTokenizer}) objects to scan the unit and all nested include files.\subsubsection*{\large{\textbf{Properties}}\normalsize\hspace{1ex}\hfill}
\begin{list}{}{
\settowidth{\tmplength}{\textbf{IncludeFilePaths}}
\setlength{\itemindent}{0cm}
\setlength{\listparindent}{0cm}
\setlength{\leftmargin}{\evensidemargin}
\addtolength{\leftmargin}{\tmplength}
\settowidth{\labelsep}{X}
\addtolength{\leftmargin}{\labelsep}
\setlength{\labelwidth}{\tmplength}
}
\label{PasDoc_Scanner.TScanner-IncludeFilePaths}
\index{IncludeFilePaths}
\item[\textbf{IncludeFilePaths}\hfill]
\ifpdf
\begin{flushleft}
\fi
\begin{ttfamily}
public property IncludeFilePaths: TStringVector read FIncludeFilePaths
      write SetIncludeFilePaths;\end{ttfamily}

\ifpdf
\end{flushleft}
\fi


\par Paths to search for include files. When you assign something to this property it causes Assign(Value) call, not a real reference copy.\label{PasDoc_Scanner.TScanner-OnMessage}
\index{OnMessage}
\item[\textbf{OnMessage}\hfill]
\ifpdf
\begin{flushleft}
\fi
\begin{ttfamily}
public property OnMessage: TPasDocMessageEvent read FOnMessage write FOnMessage;\end{ttfamily}

\ifpdf
\end{flushleft}
\fi


\par  \label{PasDoc_Scanner.TScanner-Verbosity}
\index{Verbosity}
\item[\textbf{Verbosity}\hfill]
\ifpdf
\begin{flushleft}
\fi
\begin{ttfamily}
public property Verbosity: Cardinal read FVerbosity write FVerbosity;\end{ttfamily}

\ifpdf
\end{flushleft}
\fi


\par  \label{PasDoc_Scanner.TScanner-SwitchOptions}
\index{SwitchOptions}
\item[\textbf{SwitchOptions}\hfill]
\ifpdf
\begin{flushleft}
\fi
\begin{ttfamily}
public property SwitchOptions: TSwitchOptions read FSwitchOptions;\end{ttfamily}

\ifpdf
\end{flushleft}
\fi


\par  \label{PasDoc_Scanner.TScanner-HandleMacros}
\index{HandleMacros}
\item[\textbf{HandleMacros}\hfill]
\ifpdf
\begin{flushleft}
\fi
\begin{ttfamily}
public property HandleMacros: boolean read FHandleMacros;\end{ttfamily}

\ifpdf
\end{flushleft}
\fi


\par  \end{list}
\subsubsection*{\large{\textbf{Methods}}\normalsize\hspace{1ex}\hfill}
\paragraph*{DoError}\hspace*{\fill}

\label{PasDoc_Scanner.TScanner-DoError}
\index{DoError}
\begin{list}{}{
\settowidth{\tmplength}{\textbf{Description}}
\setlength{\itemindent}{0cm}
\setlength{\listparindent}{0cm}
\setlength{\leftmargin}{\evensidemargin}
\addtolength{\leftmargin}{\tmplength}
\settowidth{\labelsep}{X}
\addtolength{\leftmargin}{\labelsep}
\setlength{\labelwidth}{\tmplength}
}
\item[\textbf{Declaration}\hfill]
\ifpdf
\begin{flushleft}
\fi
\begin{ttfamily}
protected procedure DoError(const AMessage: string; const AArguments: array of const);\end{ttfamily}

\ifpdf
\end{flushleft}
\fi

\end{list}
\paragraph*{DoMessage}\hspace*{\fill}

\label{PasDoc_Scanner.TScanner-DoMessage}
\index{DoMessage}
\begin{list}{}{
\settowidth{\tmplength}{\textbf{Description}}
\setlength{\itemindent}{0cm}
\setlength{\listparindent}{0cm}
\setlength{\leftmargin}{\evensidemargin}
\addtolength{\leftmargin}{\tmplength}
\settowidth{\labelsep}{X}
\addtolength{\leftmargin}{\labelsep}
\setlength{\labelwidth}{\tmplength}
}
\item[\textbf{Declaration}\hfill]
\ifpdf
\begin{flushleft}
\fi
\begin{ttfamily}
protected procedure DoMessage(const AVerbosity: Cardinal; const MessageType: TPasDocMessageType; const AMessage: string; const AArguments: array of const);\end{ttfamily}

\ifpdf
\end{flushleft}
\fi

\end{list}
\paragraph*{Create}\hspace*{\fill}

\label{PasDoc_Scanner.TScanner-Create}
\index{Create}
\begin{list}{}{
\settowidth{\tmplength}{\textbf{Description}}
\setlength{\itemindent}{0cm}
\setlength{\listparindent}{0cm}
\setlength{\leftmargin}{\evensidemargin}
\addtolength{\leftmargin}{\tmplength}
\settowidth{\labelsep}{X}
\addtolength{\leftmargin}{\labelsep}
\setlength{\labelwidth}{\tmplength}
}
\item[\textbf{Declaration}\hfill]
\ifpdf
\begin{flushleft}
\fi
\begin{ttfamily}
public constructor Create( const s: TStream; const OnMessageEvent: TPasDocMessageEvent; const VerbosityLevel: Cardinal; const AStreamName, AStreamPath: string; const AHandleMacros: boolean);\end{ttfamily}

\ifpdf
\end{flushleft}
\fi

\par
\item[\textbf{Description}]
Creates a TScanner object that scans the given input stream.

Note that the stream S will be freed by this object (at destruction or when we will read all it's tokens), so after creating TScanner you should leave the stream to be managed completely by this TScanner.

\end{list}
\paragraph*{Destroy}\hspace*{\fill}

\label{PasDoc_Scanner.TScanner-Destroy}
\index{Destroy}
\begin{list}{}{
\settowidth{\tmplength}{\textbf{Description}}
\setlength{\itemindent}{0cm}
\setlength{\listparindent}{0cm}
\setlength{\leftmargin}{\evensidemargin}
\addtolength{\leftmargin}{\tmplength}
\settowidth{\labelsep}{X}
\addtolength{\leftmargin}{\labelsep}
\setlength{\labelwidth}{\tmplength}
}
\item[\textbf{Declaration}\hfill]
\ifpdf
\begin{flushleft}
\fi
\begin{ttfamily}
public destructor Destroy; override;\end{ttfamily}

\ifpdf
\end{flushleft}
\fi

\end{list}
\paragraph*{AddSymbol}\hspace*{\fill}

\label{PasDoc_Scanner.TScanner-AddSymbol}
\index{AddSymbol}
\begin{list}{}{
\settowidth{\tmplength}{\textbf{Description}}
\setlength{\itemindent}{0cm}
\setlength{\listparindent}{0cm}
\setlength{\leftmargin}{\evensidemargin}
\addtolength{\leftmargin}{\tmplength}
\settowidth{\labelsep}{X}
\addtolength{\leftmargin}{\labelsep}
\setlength{\labelwidth}{\tmplength}
}
\item[\textbf{Declaration}\hfill]
\ifpdf
\begin{flushleft}
\fi
\begin{ttfamily}
public procedure AddSymbol(const Name: string);\end{ttfamily}

\ifpdf
\end{flushleft}
\fi

\par
\item[\textbf{Description}]
Adds Name to the list of symbols (as a normal symbol, not macro).

\end{list}
\paragraph*{AddSymbols}\hspace*{\fill}

\label{PasDoc_Scanner.TScanner-AddSymbols}
\index{AddSymbols}
\begin{list}{}{
\settowidth{\tmplength}{\textbf{Description}}
\setlength{\itemindent}{0cm}
\setlength{\listparindent}{0cm}
\setlength{\leftmargin}{\evensidemargin}
\addtolength{\leftmargin}{\tmplength}
\settowidth{\labelsep}{X}
\addtolength{\leftmargin}{\labelsep}
\setlength{\labelwidth}{\tmplength}
}
\item[\textbf{Declaration}\hfill]
\ifpdf
\begin{flushleft}
\fi
\begin{ttfamily}
public procedure AddSymbols(const NewSymbols: TStringVector);\end{ttfamily}

\ifpdf
\end{flushleft}
\fi

\par
\item[\textbf{Description}]
Adds all symbols in the NewSymbols collection by calling \begin{ttfamily}AddSymbol\end{ttfamily}(\ref{PasDoc_Scanner.TScanner-AddSymbol}) for each of the strings in that collection.

\end{list}
\paragraph*{AddMacro}\hspace*{\fill}

\label{PasDoc_Scanner.TScanner-AddMacro}
\index{AddMacro}
\begin{list}{}{
\settowidth{\tmplength}{\textbf{Description}}
\setlength{\itemindent}{0cm}
\setlength{\listparindent}{0cm}
\setlength{\leftmargin}{\evensidemargin}
\addtolength{\leftmargin}{\tmplength}
\settowidth{\labelsep}{X}
\addtolength{\leftmargin}{\labelsep}
\setlength{\labelwidth}{\tmplength}
}
\item[\textbf{Declaration}\hfill]
\ifpdf
\begin{flushleft}
\fi
\begin{ttfamily}
public procedure AddMacro(const Name, Value: string);\end{ttfamily}

\ifpdf
\end{flushleft}
\fi

\par
\item[\textbf{Description}]
Adds Name as a symbol that is a macro, that expands to Value.

\end{list}
\paragraph*{ConsumeToken}\hspace*{\fill}

\label{PasDoc_Scanner.TScanner-ConsumeToken}
\index{ConsumeToken}
\begin{list}{}{
\settowidth{\tmplength}{\textbf{Description}}
\setlength{\itemindent}{0cm}
\setlength{\listparindent}{0cm}
\setlength{\leftmargin}{\evensidemargin}
\addtolength{\leftmargin}{\tmplength}
\settowidth{\labelsep}{X}
\addtolength{\leftmargin}{\labelsep}
\setlength{\labelwidth}{\tmplength}
}
\item[\textbf{Declaration}\hfill]
\ifpdf
\begin{flushleft}
\fi
\begin{ttfamily}
public procedure ConsumeToken;\end{ttfamily}

\ifpdf
\end{flushleft}
\fi

\par
\item[\textbf{Description}]
Gets next token and throws it away.

\end{list}
\paragraph*{GetToken}\hspace*{\fill}

\label{PasDoc_Scanner.TScanner-GetToken}
\index{GetToken}
\begin{list}{}{
\settowidth{\tmplength}{\textbf{Description}}
\setlength{\itemindent}{0cm}
\setlength{\listparindent}{0cm}
\setlength{\leftmargin}{\evensidemargin}
\addtolength{\leftmargin}{\tmplength}
\settowidth{\labelsep}{X}
\addtolength{\leftmargin}{\labelsep}
\setlength{\labelwidth}{\tmplength}
}
\item[\textbf{Declaration}\hfill]
\ifpdf
\begin{flushleft}
\fi
\begin{ttfamily}
public function GetToken: TToken;\end{ttfamily}

\ifpdf
\end{flushleft}
\fi

\par
\item[\textbf{Description}]
Returns next token. Always non{-}nil (will raise exception in case of any problem).

\end{list}
\paragraph*{GetStreamInfo}\hspace*{\fill}

\label{PasDoc_Scanner.TScanner-GetStreamInfo}
\index{GetStreamInfo}
\begin{list}{}{
\settowidth{\tmplength}{\textbf{Description}}
\setlength{\itemindent}{0cm}
\setlength{\listparindent}{0cm}
\setlength{\leftmargin}{\evensidemargin}
\addtolength{\leftmargin}{\tmplength}
\settowidth{\labelsep}{X}
\addtolength{\leftmargin}{\labelsep}
\setlength{\labelwidth}{\tmplength}
}
\item[\textbf{Declaration}\hfill]
\ifpdf
\begin{flushleft}
\fi
\begin{ttfamily}
public function GetStreamInfo: string;\end{ttfamily}

\ifpdf
\end{flushleft}
\fi

\par
\item[\textbf{Description}]
Returns the name of the file that is currently processed and the line number. Good for meaningful error messages.

\end{list}
\paragraph*{PeekToken}\hspace*{\fill}

\label{PasDoc_Scanner.TScanner-PeekToken}
\index{PeekToken}
\begin{list}{}{
\settowidth{\tmplength}{\textbf{Description}}
\setlength{\itemindent}{0cm}
\setlength{\listparindent}{0cm}
\setlength{\leftmargin}{\evensidemargin}
\addtolength{\leftmargin}{\tmplength}
\settowidth{\labelsep}{X}
\addtolength{\leftmargin}{\labelsep}
\setlength{\labelwidth}{\tmplength}
}
\item[\textbf{Declaration}\hfill]
\ifpdf
\begin{flushleft}
\fi
\begin{ttfamily}
public function PeekToken: TToken;\end{ttfamily}

\ifpdf
\end{flushleft}
\fi

\end{list}
\paragraph*{UnGetToken}\hspace*{\fill}

\label{PasDoc_Scanner.TScanner-UnGetToken}
\index{UnGetToken}
\begin{list}{}{
\settowidth{\tmplength}{\textbf{Description}}
\setlength{\itemindent}{0cm}
\setlength{\listparindent}{0cm}
\setlength{\leftmargin}{\evensidemargin}
\addtolength{\leftmargin}{\tmplength}
\settowidth{\labelsep}{X}
\addtolength{\leftmargin}{\labelsep}
\setlength{\labelwidth}{\tmplength}
}
\item[\textbf{Declaration}\hfill]
\ifpdf
\begin{flushleft}
\fi
\begin{ttfamily}
public procedure UnGetToken(var t: TToken);\end{ttfamily}

\ifpdf
\end{flushleft}
\fi

\par
\item[\textbf{Description}]
Place T in the buffer. Next time you will call GetToken you will get T. This also sets T to nil (because you shouldn't free T anymore after ungetting it). Note that the buffer has room only for 1 token, so you have to make sure that you will never unget more than two tokens. Practically, always call UnGetToken right after some GetToken.

\end{list}
\section{Types}
\ifpdf
\subsection*{\large{\textbf{TUpperCaseLetter}}\normalsize\hspace{1ex}\hrulefill}
\else
\subsection*{TUpperCaseLetter}
\fi
\label{PasDoc_Scanner-TUpperCaseLetter}
\index{TUpperCaseLetter}
\begin{list}{}{
\settowidth{\tmplength}{\textbf{Description}}
\setlength{\itemindent}{0cm}
\setlength{\listparindent}{0cm}
\setlength{\leftmargin}{\evensidemargin}
\addtolength{\leftmargin}{\tmplength}
\settowidth{\labelsep}{X}
\addtolength{\leftmargin}{\labelsep}
\setlength{\labelwidth}{\tmplength}
}
\item[\textbf{Declaration}\hfill]
\ifpdf
\begin{flushleft}
\fi
\begin{ttfamily}
TUpperCaseLetter = 'A'..'Z';\end{ttfamily}

\ifpdf
\end{flushleft}
\fi

\par
\item[\textbf{Description}]
subrange type that has the 26 lower case letters from a to z

\end{list}
\ifpdf
\subsection*{\large{\textbf{TSwitchOptions}}\normalsize\hspace{1ex}\hrulefill}
\else
\subsection*{TSwitchOptions}
\fi
\label{PasDoc_Scanner-TSwitchOptions}
\index{TSwitchOptions}
\begin{list}{}{
\settowidth{\tmplength}{\textbf{Description}}
\setlength{\itemindent}{0cm}
\setlength{\listparindent}{0cm}
\setlength{\leftmargin}{\evensidemargin}
\addtolength{\leftmargin}{\tmplength}
\settowidth{\labelsep}{X}
\addtolength{\leftmargin}{\labelsep}
\setlength{\labelwidth}{\tmplength}
}
\item[\textbf{Declaration}\hfill]
\ifpdf
\begin{flushleft}
\fi
\begin{ttfamily}
TSwitchOptions = array[TUpperCaseLetter] of Boolean;\end{ttfamily}

\ifpdf
\end{flushleft}
\fi

\par
\item[\textbf{Description}]
an array of boolean values, index type is \begin{ttfamily}TUpperCaseLetter\end{ttfamily}(\ref{PasDoc_Scanner-TUpperCaseLetter})

\end{list}
\ifpdf
\subsection*{\large{\textbf{TDirectiveType}}\normalsize\hspace{1ex}\hrulefill}
\else
\subsection*{TDirectiveType}
\fi
\label{PasDoc_Scanner-TDirectiveType}
\index{TDirectiveType}
\begin{list}{}{
\settowidth{\tmplength}{\textbf{Description}}
\setlength{\itemindent}{0cm}
\setlength{\listparindent}{0cm}
\setlength{\leftmargin}{\evensidemargin}
\addtolength{\leftmargin}{\tmplength}
\settowidth{\labelsep}{X}
\addtolength{\leftmargin}{\labelsep}
\setlength{\labelwidth}{\tmplength}
}
\item[\textbf{Declaration}\hfill]
\ifpdf
\begin{flushleft}
\fi
\begin{ttfamily}
TDirectiveType = (...);\end{ttfamily}

\ifpdf
\end{flushleft}
\fi

\par
\item[\textbf{Description}]
All directives a scanner is going to regard.\item[\textbf{Values}]
\begin{description}
\item[\texttt{DT{\_}UNKNOWN}] \label{PasDoc_Scanner-DT_UNKNOWN}
\index{}
 
\item[\texttt{DT{\_}DEFINE}] \label{PasDoc_Scanner-DT_DEFINE}
\index{}
 
\item[\texttt{DT{\_}ELSE}] \label{PasDoc_Scanner-DT_ELSE}
\index{}
 
\item[\texttt{DT{\_}ENDIF}] \label{PasDoc_Scanner-DT_ENDIF}
\index{}
 
\item[\texttt{DT{\_}IFDEF}] \label{PasDoc_Scanner-DT_IFDEF}
\index{}
 
\item[\texttt{DT{\_}IFNDEF}] \label{PasDoc_Scanner-DT_IFNDEF}
\index{}
 
\item[\texttt{DT{\_}IFOPT}] \label{PasDoc_Scanner-DT_IFOPT}
\index{}
 
\item[\texttt{DT{\_}INCLUDE{\_}FILE}] \label{PasDoc_Scanner-DT_INCLUDE_FILE}
\index{}
 
\item[\texttt{DT{\_}UNDEF}] \label{PasDoc_Scanner-DT_UNDEF}
\index{}
 
\item[\texttt{DT{\_}INCLUDE{\_}FILE{\_}2}] \label{PasDoc_Scanner-DT_INCLUDE_FILE_2}
\index{}
 
\item[\texttt{DT{\_}IF}] \label{PasDoc_Scanner-DT_IF}
\index{}
 
\item[\texttt{DT{\_}ELSEIF}] \label{PasDoc_Scanner-DT_ELSEIF}
\index{}
 
\item[\texttt{DT{\_}IFEND}] \label{PasDoc_Scanner-DT_IFEND}
\index{}
 
\end{description}


\end{list}
\section{Constants}
\ifpdf
\subsection*{\large{\textbf{MAX{\_}TOKENIZERS}}\normalsize\hspace{1ex}\hrulefill}
\else
\subsection*{MAX{\_}TOKENIZERS}
\fi
\label{PasDoc_Scanner-MAX_TOKENIZERS}
\index{MAX{\_}TOKENIZERS}
\begin{list}{}{
\settowidth{\tmplength}{\textbf{Description}}
\setlength{\itemindent}{0cm}
\setlength{\listparindent}{0cm}
\setlength{\leftmargin}{\evensidemargin}
\addtolength{\leftmargin}{\tmplength}
\settowidth{\labelsep}{X}
\addtolength{\leftmargin}{\labelsep}
\setlength{\labelwidth}{\tmplength}
}
\item[\textbf{Declaration}\hfill]
\ifpdf
\begin{flushleft}
\fi
\begin{ttfamily}
MAX{\_}TOKENIZERS = 32;\end{ttfamily}

\ifpdf
\end{flushleft}
\fi

\par
\item[\textbf{Description}]
maximum number of streams we can recurse into; first one is the unit stream, any other stream an include file; current value is 32, increase this if you have more include files recursively including others

\end{list}
\section{Authors}
\par
Johannes Berg {$<$}johannes@sipsolutions.de{$>$}

\par
Ralf Junker (delphi@zeitungsjunge.de)

\par
Marco Schmidt (marcoschmidt@geocities.com)

\par
Michalis Kamburelis

\par
Arno Garrels {$<$}first name.name@nospamgmx.de{$>$}

\chapter{Unit PasDoc{\_}Serialize}
\label{PasDoc_Serialize}
\index{PasDoc{\_}Serialize}
\section{Description}
Serializing/deserializing cached information.\hfill\vspace*{1ex}

 
\section{Uses}
\begin{itemize}
\item \begin{ttfamily}Classes\end{ttfamily}\item \begin{ttfamily}SysUtils\end{ttfamily}\item \begin{ttfamily}PasDoc{\_}StreamUtils\end{ttfamily}(\ref{PasDoc_StreamUtils})\end{itemize}
\section{Overview}
\begin{description}
\item[\texttt{\begin{ttfamily}EInvalidCacheFileVersion\end{ttfamily} Class}]
\item[\texttt{\begin{ttfamily}TSerializable\end{ttfamily} Class}]
\item[\texttt{\begin{ttfamily}ESerializedException\end{ttfamily} Class}]
\end{description}
\section{Classes, Interfaces, Objects and Records}
\ifpdf
\subsection*{\large{\textbf{EInvalidCacheFileVersion Class}}\normalsize\hspace{1ex}\hrulefill}
\else
\subsection*{EInvalidCacheFileVersion Class}
\fi
\label{PasDoc_Serialize.EInvalidCacheFileVersion}
\index{EInvalidCacheFileVersion}
\subsubsection*{\large{\textbf{Hierarchy}}\normalsize\hspace{1ex}\hfill}
EInvalidCacheFileVersion {$>$} Exception
%%%%Description
\ifpdf
\subsection*{\large{\textbf{TSerializable Class}}\normalsize\hspace{1ex}\hrulefill}
\else
\subsection*{TSerializable Class}
\fi
\label{PasDoc_Serialize.TSerializable}
\index{TSerializable}
\subsubsection*{\large{\textbf{Hierarchy}}\normalsize\hspace{1ex}\hfill}
TSerializable {$>$} TObject
%%%%Description
\subsubsection*{\large{\textbf{Properties}}\normalsize\hspace{1ex}\hfill}
\begin{list}{}{
\settowidth{\tmplength}{\textbf{WasDeserialized}}
\setlength{\itemindent}{0cm}
\setlength{\listparindent}{0cm}
\setlength{\leftmargin}{\evensidemargin}
\addtolength{\leftmargin}{\tmplength}
\settowidth{\labelsep}{X}
\addtolength{\leftmargin}{\labelsep}
\setlength{\labelwidth}{\tmplength}
}
\label{PasDoc_Serialize.TSerializable-WasDeserialized}
\index{WasDeserialized}
\item[\textbf{WasDeserialized}\hfill]
\ifpdf
\begin{flushleft}
\fi
\begin{ttfamily}
public property WasDeserialized: boolean read FWasDeserialized;\end{ttfamily}

\ifpdf
\end{flushleft}
\fi


\par  \end{list}
\subsubsection*{\large{\textbf{Methods}}\normalsize\hspace{1ex}\hfill}
\paragraph*{Serialize}\hspace*{\fill}

\label{PasDoc_Serialize.TSerializable-Serialize}
\index{Serialize}
\begin{list}{}{
\settowidth{\tmplength}{\textbf{Description}}
\setlength{\itemindent}{0cm}
\setlength{\listparindent}{0cm}
\setlength{\leftmargin}{\evensidemargin}
\addtolength{\leftmargin}{\tmplength}
\settowidth{\labelsep}{X}
\addtolength{\leftmargin}{\labelsep}
\setlength{\labelwidth}{\tmplength}
}
\item[\textbf{Declaration}\hfill]
\ifpdf
\begin{flushleft}
\fi
\begin{ttfamily}
protected procedure Serialize(const ADestination: TStream); virtual;\end{ttfamily}

\ifpdf
\end{flushleft}
\fi

\end{list}
\paragraph*{Deserialize}\hspace*{\fill}

\label{PasDoc_Serialize.TSerializable-Deserialize}
\index{Deserialize}
\begin{list}{}{
\settowidth{\tmplength}{\textbf{Description}}
\setlength{\itemindent}{0cm}
\setlength{\listparindent}{0cm}
\setlength{\leftmargin}{\evensidemargin}
\addtolength{\leftmargin}{\tmplength}
\settowidth{\labelsep}{X}
\addtolength{\leftmargin}{\labelsep}
\setlength{\labelwidth}{\tmplength}
}
\item[\textbf{Declaration}\hfill]
\ifpdf
\begin{flushleft}
\fi
\begin{ttfamily}
protected procedure Deserialize(const ASource: TStream); virtual;\end{ttfamily}

\ifpdf
\end{flushleft}
\fi

\end{list}
\paragraph*{Read7BitEncodedInt}\hspace*{\fill}

\label{PasDoc_Serialize.TSerializable-Read7BitEncodedInt}
\index{Read7BitEncodedInt}
\begin{list}{}{
\settowidth{\tmplength}{\textbf{Description}}
\setlength{\itemindent}{0cm}
\setlength{\listparindent}{0cm}
\setlength{\leftmargin}{\evensidemargin}
\addtolength{\leftmargin}{\tmplength}
\settowidth{\labelsep}{X}
\addtolength{\leftmargin}{\labelsep}
\setlength{\labelwidth}{\tmplength}
}
\item[\textbf{Declaration}\hfill]
\ifpdf
\begin{flushleft}
\fi
\begin{ttfamily}
public class function Read7BitEncodedInt(const ASource: TStream): Integer;\end{ttfamily}

\ifpdf
\end{flushleft}
\fi

\end{list}
\paragraph*{Write7BitEncodedInt}\hspace*{\fill}

\label{PasDoc_Serialize.TSerializable-Write7BitEncodedInt}
\index{Write7BitEncodedInt}
\begin{list}{}{
\settowidth{\tmplength}{\textbf{Description}}
\setlength{\itemindent}{0cm}
\setlength{\listparindent}{0cm}
\setlength{\leftmargin}{\evensidemargin}
\addtolength{\leftmargin}{\tmplength}
\settowidth{\labelsep}{X}
\addtolength{\leftmargin}{\labelsep}
\setlength{\labelwidth}{\tmplength}
}
\item[\textbf{Declaration}\hfill]
\ifpdf
\begin{flushleft}
\fi
\begin{ttfamily}
public class procedure Write7BitEncodedInt(Value: Integer; const ADestination: TStream);\end{ttfamily}

\ifpdf
\end{flushleft}
\fi

\end{list}
\paragraph*{LoadStringFromStream}\hspace*{\fill}

\label{PasDoc_Serialize.TSerializable-LoadStringFromStream}
\index{LoadStringFromStream}
\begin{list}{}{
\settowidth{\tmplength}{\textbf{Description}}
\setlength{\itemindent}{0cm}
\setlength{\listparindent}{0cm}
\setlength{\leftmargin}{\evensidemargin}
\addtolength{\leftmargin}{\tmplength}
\settowidth{\labelsep}{X}
\addtolength{\leftmargin}{\labelsep}
\setlength{\labelwidth}{\tmplength}
}
\item[\textbf{Declaration}\hfill]
\ifpdf
\begin{flushleft}
\fi
\begin{ttfamily}
public class function LoadStringFromStream(const ASource: TStream): string;\end{ttfamily}

\ifpdf
\end{flushleft}
\fi

\end{list}
\paragraph*{SaveStringToStream}\hspace*{\fill}

\label{PasDoc_Serialize.TSerializable-SaveStringToStream}
\index{SaveStringToStream}
\begin{list}{}{
\settowidth{\tmplength}{\textbf{Description}}
\setlength{\itemindent}{0cm}
\setlength{\listparindent}{0cm}
\setlength{\leftmargin}{\evensidemargin}
\addtolength{\leftmargin}{\tmplength}
\settowidth{\labelsep}{X}
\addtolength{\leftmargin}{\labelsep}
\setlength{\labelwidth}{\tmplength}
}
\item[\textbf{Declaration}\hfill]
\ifpdf
\begin{flushleft}
\fi
\begin{ttfamily}
public class procedure SaveStringToStream(const AValue: string; const ADestination: TStream);\end{ttfamily}

\ifpdf
\end{flushleft}
\fi

\end{list}
\paragraph*{LoadDoubleFromStream}\hspace*{\fill}

\label{PasDoc_Serialize.TSerializable-LoadDoubleFromStream}
\index{LoadDoubleFromStream}
\begin{list}{}{
\settowidth{\tmplength}{\textbf{Description}}
\setlength{\itemindent}{0cm}
\setlength{\listparindent}{0cm}
\setlength{\leftmargin}{\evensidemargin}
\addtolength{\leftmargin}{\tmplength}
\settowidth{\labelsep}{X}
\addtolength{\leftmargin}{\labelsep}
\setlength{\labelwidth}{\tmplength}
}
\item[\textbf{Declaration}\hfill]
\ifpdf
\begin{flushleft}
\fi
\begin{ttfamily}
public class function LoadDoubleFromStream(const ASource: TStream): double;\end{ttfamily}

\ifpdf
\end{flushleft}
\fi

\end{list}
\paragraph*{SaveDoubleToStream}\hspace*{\fill}

\label{PasDoc_Serialize.TSerializable-SaveDoubleToStream}
\index{SaveDoubleToStream}
\begin{list}{}{
\settowidth{\tmplength}{\textbf{Description}}
\setlength{\itemindent}{0cm}
\setlength{\listparindent}{0cm}
\setlength{\leftmargin}{\evensidemargin}
\addtolength{\leftmargin}{\tmplength}
\settowidth{\labelsep}{X}
\addtolength{\leftmargin}{\labelsep}
\setlength{\labelwidth}{\tmplength}
}
\item[\textbf{Declaration}\hfill]
\ifpdf
\begin{flushleft}
\fi
\begin{ttfamily}
public class procedure SaveDoubleToStream(const AValue: double; const ADestination: TStream);\end{ttfamily}

\ifpdf
\end{flushleft}
\fi

\end{list}
\paragraph*{LoadIntegerFromStream}\hspace*{\fill}

\label{PasDoc_Serialize.TSerializable-LoadIntegerFromStream}
\index{LoadIntegerFromStream}
\begin{list}{}{
\settowidth{\tmplength}{\textbf{Description}}
\setlength{\itemindent}{0cm}
\setlength{\listparindent}{0cm}
\setlength{\leftmargin}{\evensidemargin}
\addtolength{\leftmargin}{\tmplength}
\settowidth{\labelsep}{X}
\addtolength{\leftmargin}{\labelsep}
\setlength{\labelwidth}{\tmplength}
}
\item[\textbf{Declaration}\hfill]
\ifpdf
\begin{flushleft}
\fi
\begin{ttfamily}
public class function LoadIntegerFromStream(const ASource: TStream): Longint;\end{ttfamily}

\ifpdf
\end{flushleft}
\fi

\end{list}
\paragraph*{SaveIntegerToStream}\hspace*{\fill}

\label{PasDoc_Serialize.TSerializable-SaveIntegerToStream}
\index{SaveIntegerToStream}
\begin{list}{}{
\settowidth{\tmplength}{\textbf{Description}}
\setlength{\itemindent}{0cm}
\setlength{\listparindent}{0cm}
\setlength{\leftmargin}{\evensidemargin}
\addtolength{\leftmargin}{\tmplength}
\settowidth{\labelsep}{X}
\addtolength{\leftmargin}{\labelsep}
\setlength{\labelwidth}{\tmplength}
}
\item[\textbf{Declaration}\hfill]
\ifpdf
\begin{flushleft}
\fi
\begin{ttfamily}
public class procedure SaveIntegerToStream(const AValue: Longint; const ADestination: TStream);\end{ttfamily}

\ifpdf
\end{flushleft}
\fi

\end{list}
\paragraph*{Create}\hspace*{\fill}

\label{PasDoc_Serialize.TSerializable-Create}
\index{Create}
\begin{list}{}{
\settowidth{\tmplength}{\textbf{Description}}
\setlength{\itemindent}{0cm}
\setlength{\listparindent}{0cm}
\setlength{\leftmargin}{\evensidemargin}
\addtolength{\leftmargin}{\tmplength}
\settowidth{\labelsep}{X}
\addtolength{\leftmargin}{\labelsep}
\setlength{\labelwidth}{\tmplength}
}
\item[\textbf{Declaration}\hfill]
\ifpdf
\begin{flushleft}
\fi
\begin{ttfamily}
public constructor Create; virtual;\end{ttfamily}

\ifpdf
\end{flushleft}
\fi

\end{list}
\paragraph*{SerializeObject}\hspace*{\fill}

\label{PasDoc_Serialize.TSerializable-SerializeObject}
\index{SerializeObject}
\begin{list}{}{
\settowidth{\tmplength}{\textbf{Description}}
\setlength{\itemindent}{0cm}
\setlength{\listparindent}{0cm}
\setlength{\leftmargin}{\evensidemargin}
\addtolength{\leftmargin}{\tmplength}
\settowidth{\labelsep}{X}
\addtolength{\leftmargin}{\labelsep}
\setlength{\labelwidth}{\tmplength}
}
\item[\textbf{Declaration}\hfill]
\ifpdf
\begin{flushleft}
\fi
\begin{ttfamily}
public class procedure SerializeObject(const AObject: TSerializable; const ADestination: TStream);\end{ttfamily}

\ifpdf
\end{flushleft}
\fi

\end{list}
\paragraph*{DeserializeObject}\hspace*{\fill}

\label{PasDoc_Serialize.TSerializable-DeserializeObject}
\index{DeserializeObject}
\begin{list}{}{
\settowidth{\tmplength}{\textbf{Description}}
\setlength{\itemindent}{0cm}
\setlength{\listparindent}{0cm}
\setlength{\leftmargin}{\evensidemargin}
\addtolength{\leftmargin}{\tmplength}
\settowidth{\labelsep}{X}
\addtolength{\leftmargin}{\labelsep}
\setlength{\labelwidth}{\tmplength}
}
\item[\textbf{Declaration}\hfill]
\ifpdf
\begin{flushleft}
\fi
\begin{ttfamily}
public class function DeserializeObject(const ASource: TStream): TSerializable;\end{ttfamily}

\ifpdf
\end{flushleft}
\fi

\end{list}
\paragraph*{Register}\hspace*{\fill}

\label{PasDoc_Serialize.TSerializable-Register}
\index{Register}
\begin{list}{}{
\settowidth{\tmplength}{\textbf{Description}}
\setlength{\itemindent}{0cm}
\setlength{\listparindent}{0cm}
\setlength{\leftmargin}{\evensidemargin}
\addtolength{\leftmargin}{\tmplength}
\settowidth{\labelsep}{X}
\addtolength{\leftmargin}{\labelsep}
\setlength{\labelwidth}{\tmplength}
}
\item[\textbf{Declaration}\hfill]
\ifpdf
\begin{flushleft}
\fi
\begin{ttfamily}
public class procedure Register(const AClass: TSerializableClass);\end{ttfamily}

\ifpdf
\end{flushleft}
\fi

\end{list}
\paragraph*{SerializeToFile}\hspace*{\fill}

\label{PasDoc_Serialize.TSerializable-SerializeToFile}
\index{SerializeToFile}
\begin{list}{}{
\settowidth{\tmplength}{\textbf{Description}}
\setlength{\itemindent}{0cm}
\setlength{\listparindent}{0cm}
\setlength{\leftmargin}{\evensidemargin}
\addtolength{\leftmargin}{\tmplength}
\settowidth{\labelsep}{X}
\addtolength{\leftmargin}{\labelsep}
\setlength{\labelwidth}{\tmplength}
}
\item[\textbf{Declaration}\hfill]
\ifpdf
\begin{flushleft}
\fi
\begin{ttfamily}
public procedure SerializeToFile(const AFileName: string);\end{ttfamily}

\ifpdf
\end{flushleft}
\fi

\end{list}
\paragraph*{DeserializeFromFile}\hspace*{\fill}

\label{PasDoc_Serialize.TSerializable-DeserializeFromFile}
\index{DeserializeFromFile}
\begin{list}{}{
\settowidth{\tmplength}{\textbf{Description}}
\setlength{\itemindent}{0cm}
\setlength{\listparindent}{0cm}
\setlength{\leftmargin}{\evensidemargin}
\addtolength{\leftmargin}{\tmplength}
\settowidth{\labelsep}{X}
\addtolength{\leftmargin}{\labelsep}
\setlength{\labelwidth}{\tmplength}
}
\item[\textbf{Declaration}\hfill]
\ifpdf
\begin{flushleft}
\fi
\begin{ttfamily}
public class function DeserializeFromFile(const AFileName: string): TSerializable;\end{ttfamily}

\ifpdf
\end{flushleft}
\fi

\par
\item[\textbf{Description}]
Read back from file. \par
\item[\textbf{Exceptions}]
\begin{description}
\item[\begin{ttfamily}EInvalidCacheFileVersion\end{ttfamily}(\ref{PasDoc_Serialize.EInvalidCacheFileVersion})] When the cached file contents are from an old pasdoc version (or invalid).
\end{description}


\end{list}
\ifpdf
\subsection*{\large{\textbf{ESerializedException Class}}\normalsize\hspace{1ex}\hrulefill}
\else
\subsection*{ESerializedException Class}
\fi
\label{PasDoc_Serialize.ESerializedException}
\index{ESerializedException}
\subsubsection*{\large{\textbf{Hierarchy}}\normalsize\hspace{1ex}\hfill}
ESerializedException {$>$} Exception
%%%%Description
\section{Types}
\ifpdf
\subsection*{\large{\textbf{TSerializableClass}}\normalsize\hspace{1ex}\hrulefill}
\else
\subsection*{TSerializableClass}
\fi
\label{PasDoc_Serialize-TSerializableClass}
\index{TSerializableClass}
\begin{list}{}{
\settowidth{\tmplength}{\textbf{Description}}
\setlength{\itemindent}{0cm}
\setlength{\listparindent}{0cm}
\setlength{\leftmargin}{\evensidemargin}
\addtolength{\leftmargin}{\tmplength}
\settowidth{\labelsep}{X}
\addtolength{\leftmargin}{\labelsep}
\setlength{\labelwidth}{\tmplength}
}
\item[\textbf{Declaration}\hfill]
\ifpdf
\begin{flushleft}
\fi
\begin{ttfamily}
TSerializableClass = class of TSerializable;\end{ttfamily}

\ifpdf
\end{flushleft}
\fi

\end{list}
\section{Author}
\par
Arno Garrels {$<$}first name.name@nospamgmx.de{$>$}

\chapter{Unit PasDoc{\_}SortSettings}
\label{PasDoc_SortSettings}
\index{PasDoc{\_}SortSettings}
\section{Description}
Sorting settings types and names.
\section{Uses}
\begin{itemize}
\item \begin{ttfamily}SysUtils\end{ttfamily}\end{itemize}
\section{Overview}
\begin{description}
\item[\texttt{\begin{ttfamily}EInvalidSortSetting\end{ttfamily} Class}]
\end{description}
\begin{description}
\item[\texttt{SortSettingFromName}]
\item[\texttt{SortSettingsToName}]Comma{-}separated list
\end{description}
\section{Classes, Interfaces, Objects and Records}
\ifpdf
\subsection*{\large{\textbf{EInvalidSortSetting Class}}\normalsize\hspace{1ex}\hrulefill}
\else
\subsection*{EInvalidSortSetting Class}
\fi
\label{PasDoc_SortSettings.EInvalidSortSetting}
\index{EInvalidSortSetting}
\subsubsection*{\large{\textbf{Hierarchy}}\normalsize\hspace{1ex}\hfill}
EInvalidSortSetting {$>$} Exception
%%%%Description
\section{Functions and Procedures}
\ifpdf
\subsection*{\large{\textbf{SortSettingFromName}}\normalsize\hspace{1ex}\hrulefill}
\else
\subsection*{SortSettingFromName}
\fi
\label{PasDoc_SortSettings-SortSettingFromName}
\index{SortSettingFromName}
\begin{list}{}{
\settowidth{\tmplength}{\textbf{Description}}
\setlength{\itemindent}{0cm}
\setlength{\listparindent}{0cm}
\setlength{\leftmargin}{\evensidemargin}
\addtolength{\leftmargin}{\tmplength}
\settowidth{\labelsep}{X}
\addtolength{\leftmargin}{\labelsep}
\setlength{\labelwidth}{\tmplength}
}
\item[\textbf{Declaration}\hfill]
\ifpdf
\begin{flushleft}
\fi
\begin{ttfamily}
function SortSettingFromName(const SortSettingName: string): TSortSetting;\end{ttfamily}

\ifpdf
\end{flushleft}
\fi

\par
\item[\textbf{Description}]
 \par
\item[\textbf{Exceptions}]
\begin{description}
\item[\begin{ttfamily}EInvalidSortSetting\end{ttfamily}(\ref{PasDoc_SortSettings.EInvalidSortSetting})] if ASortSettingName does not match (case ignored) to any SortSettingNames.
\end{description}


\end{list}
\ifpdf
\subsection*{\large{\textbf{SortSettingsToName}}\normalsize\hspace{1ex}\hrulefill}
\else
\subsection*{SortSettingsToName}
\fi
\label{PasDoc_SortSettings-SortSettingsToName}
\index{SortSettingsToName}
\begin{list}{}{
\settowidth{\tmplength}{\textbf{Description}}
\setlength{\itemindent}{0cm}
\setlength{\listparindent}{0cm}
\setlength{\leftmargin}{\evensidemargin}
\addtolength{\leftmargin}{\tmplength}
\settowidth{\labelsep}{X}
\addtolength{\leftmargin}{\labelsep}
\setlength{\labelwidth}{\tmplength}
}
\item[\textbf{Declaration}\hfill]
\ifpdf
\begin{flushleft}
\fi
\begin{ttfamily}
function SortSettingsToName(const SortSettings: TSortSettings): string;\end{ttfamily}

\ifpdf
\end{flushleft}
\fi

\par
\item[\textbf{Description}]
Comma{-}separated list

\end{list}
\section{Types}
\ifpdf
\subsection*{\large{\textbf{TSortSetting}}\normalsize\hspace{1ex}\hrulefill}
\else
\subsection*{TSortSetting}
\fi
\label{PasDoc_SortSettings-TSortSetting}
\index{TSortSetting}
\begin{list}{}{
\settowidth{\tmplength}{\textbf{Description}}
\setlength{\itemindent}{0cm}
\setlength{\listparindent}{0cm}
\setlength{\leftmargin}{\evensidemargin}
\addtolength{\leftmargin}{\tmplength}
\settowidth{\labelsep}{X}
\addtolength{\leftmargin}{\labelsep}
\setlength{\labelwidth}{\tmplength}
}
\item[\textbf{Declaration}\hfill]
\ifpdf
\begin{flushleft}
\fi
\begin{ttfamily}
TSortSetting = (...);\end{ttfamily}

\ifpdf
\end{flushleft}
\fi

\par
\item[\textbf{Description}]
 \item[\textbf{Values}]
\begin{description}
\item[\texttt{ssCIOs}] \label{PasDoc_SortSettings-ssCIOs}
\index{}
 
\item[\texttt{ssConstants}] \label{PasDoc_SortSettings-ssConstants}
\index{}
 
\item[\texttt{ssFuncsProcs}] \label{PasDoc_SortSettings-ssFuncsProcs}
\index{}
 
\item[\texttt{ssTypes}] \label{PasDoc_SortSettings-ssTypes}
\index{}
 
\item[\texttt{ssVariables}] \label{PasDoc_SortSettings-ssVariables}
\index{}
 
\item[\texttt{ssUsesClauses}] \label{PasDoc_SortSettings-ssUsesClauses}
\index{}
 
\item[\texttt{ssRecordFields}] \label{PasDoc_SortSettings-ssRecordFields}
\index{}
 
\item[\texttt{ssNonRecordFields}] \label{PasDoc_SortSettings-ssNonRecordFields}
\index{}
 
\item[\texttt{ssMethods}] \label{PasDoc_SortSettings-ssMethods}
\index{}
 
\item[\texttt{ssProperties}] \label{PasDoc_SortSettings-ssProperties}
\index{}
 
\end{description}


\end{list}
\ifpdf
\subsection*{\large{\textbf{TSortSettings}}\normalsize\hspace{1ex}\hrulefill}
\else
\subsection*{TSortSettings}
\fi
\label{PasDoc_SortSettings-TSortSettings}
\index{TSortSettings}
\begin{list}{}{
\settowidth{\tmplength}{\textbf{Description}}
\setlength{\itemindent}{0cm}
\setlength{\listparindent}{0cm}
\setlength{\leftmargin}{\evensidemargin}
\addtolength{\leftmargin}{\tmplength}
\settowidth{\labelsep}{X}
\addtolength{\leftmargin}{\labelsep}
\setlength{\labelwidth}{\tmplength}
}
\item[\textbf{Declaration}\hfill]
\ifpdf
\begin{flushleft}
\fi
\begin{ttfamily}
TSortSettings = set of TSortSetting;\end{ttfamily}

\ifpdf
\end{flushleft}
\fi

\end{list}
\section{Constants}
\ifpdf
\subsection*{\large{\textbf{AllSortSettings}}\normalsize\hspace{1ex}\hrulefill}
\else
\subsection*{AllSortSettings}
\fi
\label{PasDoc_SortSettings-AllSortSettings}
\index{AllSortSettings}
\begin{list}{}{
\settowidth{\tmplength}{\textbf{Description}}
\setlength{\itemindent}{0cm}
\setlength{\listparindent}{0cm}
\setlength{\leftmargin}{\evensidemargin}
\addtolength{\leftmargin}{\tmplength}
\settowidth{\labelsep}{X}
\addtolength{\leftmargin}{\labelsep}
\setlength{\labelwidth}{\tmplength}
}
\item[\textbf{Declaration}\hfill]
\ifpdf
\begin{flushleft}
\fi
\begin{ttfamily}
AllSortSettings: TSortSettings = [Low(TSortSetting) .. High(TSortSetting)];\end{ttfamily}

\ifpdf
\end{flushleft}
\fi

\end{list}
\ifpdf
\subsection*{\large{\textbf{SortSettingNames}}\normalsize\hspace{1ex}\hrulefill}
\else
\subsection*{SortSettingNames}
\fi
\label{PasDoc_SortSettings-SortSettingNames}
\index{SortSettingNames}
\begin{list}{}{
\settowidth{\tmplength}{\textbf{Description}}
\setlength{\itemindent}{0cm}
\setlength{\listparindent}{0cm}
\setlength{\leftmargin}{\evensidemargin}
\addtolength{\leftmargin}{\tmplength}
\settowidth{\labelsep}{X}
\addtolength{\leftmargin}{\labelsep}
\setlength{\labelwidth}{\tmplength}
}
\item[\textbf{Declaration}\hfill]
\ifpdf
\begin{flushleft}
\fi
\begin{ttfamily}
SortSettingNames: array[TSortSetting] of string = (
    'structures', 'constants', 'functions', 'types', 'variables', 'uses-clauses',
    'record-fields', 'non-record-fields', 'methods', 'properties' );\end{ttfamily}

\ifpdf
\end{flushleft}
\fi

\par
\item[\textbf{Description}]
Must be lowercase. Used in \begin{ttfamily}SortSettingsToName\end{ttfamily}(\ref{PasDoc_SortSettings-SortSettingsToName}), \begin{ttfamily}SortSettingFromName\end{ttfamily}(\ref{PasDoc_SortSettings-SortSettingFromName}).

\end{list}
\chapter{Unit PasDoc{\_}StreamUtils}
\label{PasDoc_StreamUtils}
\index{PasDoc{\_}StreamUtils}
\section{Description}
A few stream utility functions.\hfill\vspace*{1ex}

   TBufferedStream, TStreamReader and TStreamWriter by Arno Garrels.
\section{Uses}
\begin{itemize}
\item \begin{ttfamily}SysUtils\end{ttfamily}\item \begin{ttfamily}Classes\end{ttfamily}\item \begin{ttfamily}PasDoc{\_}Types\end{ttfamily}(\ref{PasDoc_Types})\end{itemize}
\section{Overview}
\begin{description}
\item[\texttt{\begin{ttfamily}TBufferedStream\end{ttfamily} Class}]
\end{description}
\begin{description}
\item[\texttt{StreamReadLine}]
\item[\texttt{StreamWriteLine}]Write AString contents, then LineEnding to AStream
\item[\texttt{StreamWriteString}]Just write AString contents to AStream
\end{description}
\section{Classes, Interfaces, Objects and Records}
\ifpdf
\subsection*{\large{\textbf{TBufferedStream Class}}\normalsize\hspace{1ex}\hrulefill}
\else
\subsection*{TBufferedStream Class}
\fi
\label{PasDoc_StreamUtils.TBufferedStream}
\index{TBufferedStream}
\subsubsection*{\large{\textbf{Hierarchy}}\normalsize\hspace{1ex}\hfill}
TBufferedStream {$>$} TStream
%%%%Description
\subsubsection*{\large{\textbf{Properties}}\normalsize\hspace{1ex}\hfill}
\begin{list}{}{
\settowidth{\tmplength}{\textbf{IsReadOnly}}
\setlength{\itemindent}{0cm}
\setlength{\listparindent}{0cm}
\setlength{\leftmargin}{\evensidemargin}
\addtolength{\leftmargin}{\tmplength}
\settowidth{\labelsep}{X}
\addtolength{\leftmargin}{\labelsep}
\setlength{\labelwidth}{\tmplength}
}
\label{PasDoc_StreamUtils.TBufferedStream-IsReadOnly}
\index{IsReadOnly}
\item[\textbf{IsReadOnly}\hfill]
\ifpdf
\begin{flushleft}
\fi
\begin{ttfamily}
public property IsReadOnly: Boolean read FIsReadOnly write SetIsReadOnly;\end{ttfamily}

\ifpdf
\end{flushleft}
\fi


\par Set IsReadOnly if you are sure you will never write to the stream and nobody else will do, this speeds up getter Size and in turn Seeks as well. IsReadOnly is set to TRUE if a constructor with filename is called with a read only mode and a share lock.\label{PasDoc_StreamUtils.TBufferedStream-FastSize}
\index{FastSize}
\item[\textbf{FastSize}\hfill]
\ifpdf
\begin{flushleft}
\fi
\begin{ttfamily}
public property FastSize: Int64 read GetSize;\end{ttfamily}

\ifpdf
\end{flushleft}
\fi


\par  \end{list}
\subsubsection*{\large{\textbf{Methods}}\normalsize\hspace{1ex}\hfill}
\paragraph*{SetIsReadOnly}\hspace*{\fill}

\label{PasDoc_StreamUtils.TBufferedStream-SetIsReadOnly}
\index{SetIsReadOnly}
\begin{list}{}{
\settowidth{\tmplength}{\textbf{Description}}
\setlength{\itemindent}{0cm}
\setlength{\listparindent}{0cm}
\setlength{\leftmargin}{\evensidemargin}
\addtolength{\leftmargin}{\tmplength}
\settowidth{\labelsep}{X}
\addtolength{\leftmargin}{\labelsep}
\setlength{\labelwidth}{\tmplength}
}
\item[\textbf{Declaration}\hfill]
\ifpdf
\begin{flushleft}
\fi
\begin{ttfamily}
protected procedure SetIsReadOnly(const Value: Boolean);\end{ttfamily}

\ifpdf
\end{flushleft}
\fi

\par
\item[\textbf{Description}]
See property IsReadOnly below

\end{list}
\paragraph*{SetSize}\hspace*{\fill}

\label{PasDoc_StreamUtils.TBufferedStream-SetSize}
\index{SetSize}
\begin{list}{}{
\settowidth{\tmplength}{\textbf{Description}}
\setlength{\itemindent}{0cm}
\setlength{\listparindent}{0cm}
\setlength{\leftmargin}{\evensidemargin}
\addtolength{\leftmargin}{\tmplength}
\settowidth{\labelsep}{X}
\addtolength{\leftmargin}{\labelsep}
\setlength{\labelwidth}{\tmplength}
}
\item[\textbf{Declaration}\hfill]
\ifpdf
\begin{flushleft}
\fi
\begin{ttfamily}
protected procedure SetSize(NewSize: Integer); override;\end{ttfamily}

\ifpdf
\end{flushleft}
\fi

\end{list}
\paragraph*{SetSize}\hspace*{\fill}

\label{PasDoc_StreamUtils.TBufferedStream-SetSize}
\index{SetSize}
\begin{list}{}{
\settowidth{\tmplength}{\textbf{Description}}
\setlength{\itemindent}{0cm}
\setlength{\listparindent}{0cm}
\setlength{\leftmargin}{\evensidemargin}
\addtolength{\leftmargin}{\tmplength}
\settowidth{\labelsep}{X}
\addtolength{\leftmargin}{\labelsep}
\setlength{\labelwidth}{\tmplength}
}
\item[\textbf{Declaration}\hfill]
\ifpdf
\begin{flushleft}
\fi
\begin{ttfamily}
protected procedure SetSize(const NewSize: Int64); override;\end{ttfamily}

\ifpdf
\end{flushleft}
\fi

\end{list}
\paragraph*{InternalGetSize}\hspace*{\fill}

\label{PasDoc_StreamUtils.TBufferedStream-InternalGetSize}
\index{InternalGetSize}
\begin{list}{}{
\settowidth{\tmplength}{\textbf{Description}}
\setlength{\itemindent}{0cm}
\setlength{\listparindent}{0cm}
\setlength{\leftmargin}{\evensidemargin}
\addtolength{\leftmargin}{\tmplength}
\settowidth{\labelsep}{X}
\addtolength{\leftmargin}{\labelsep}
\setlength{\labelwidth}{\tmplength}
}
\item[\textbf{Declaration}\hfill]
\ifpdf
\begin{flushleft}
\fi
\begin{ttfamily}
protected function InternalGetSize: Int64; inline;\end{ttfamily}

\ifpdf
\end{flushleft}
\fi

\end{list}
\paragraph*{GetSize}\hspace*{\fill}

\label{PasDoc_StreamUtils.TBufferedStream-GetSize}
\index{GetSize}
\begin{list}{}{
\settowidth{\tmplength}{\textbf{Description}}
\setlength{\itemindent}{0cm}
\setlength{\listparindent}{0cm}
\setlength{\leftmargin}{\evensidemargin}
\addtolength{\leftmargin}{\tmplength}
\settowidth{\labelsep}{X}
\addtolength{\leftmargin}{\labelsep}
\setlength{\labelwidth}{\tmplength}
}
\item[\textbf{Declaration}\hfill]
\ifpdf
\begin{flushleft}
\fi
\begin{ttfamily}
protected function GetSize: Int64; override;\end{ttfamily}

\ifpdf
\end{flushleft}
\fi

\end{list}
\paragraph*{Init}\hspace*{\fill}

\label{PasDoc_StreamUtils.TBufferedStream-Init}
\index{Init}
\begin{list}{}{
\settowidth{\tmplength}{\textbf{Description}}
\setlength{\itemindent}{0cm}
\setlength{\listparindent}{0cm}
\setlength{\leftmargin}{\evensidemargin}
\addtolength{\leftmargin}{\tmplength}
\settowidth{\labelsep}{X}
\addtolength{\leftmargin}{\labelsep}
\setlength{\labelwidth}{\tmplength}
}
\item[\textbf{Declaration}\hfill]
\ifpdf
\begin{flushleft}
\fi
\begin{ttfamily}
protected procedure Init; virtual;\end{ttfamily}

\ifpdf
\end{flushleft}
\fi

\end{list}
\paragraph*{FillBuffer}\hspace*{\fill}

\label{PasDoc_StreamUtils.TBufferedStream-FillBuffer}
\index{FillBuffer}
\begin{list}{}{
\settowidth{\tmplength}{\textbf{Description}}
\setlength{\itemindent}{0cm}
\setlength{\listparindent}{0cm}
\setlength{\leftmargin}{\evensidemargin}
\addtolength{\leftmargin}{\tmplength}
\settowidth{\labelsep}{X}
\addtolength{\leftmargin}{\labelsep}
\setlength{\labelwidth}{\tmplength}
}
\item[\textbf{Declaration}\hfill]
\ifpdf
\begin{flushleft}
\fi
\begin{ttfamily}
protected function FillBuffer: Boolean; inline;\end{ttfamily}

\ifpdf
\end{flushleft}
\fi

\end{list}
\paragraph*{Create}\hspace*{\fill}

\label{PasDoc_StreamUtils.TBufferedStream-Create}
\index{Create}
\begin{list}{}{
\settowidth{\tmplength}{\textbf{Description}}
\setlength{\itemindent}{0cm}
\setlength{\listparindent}{0cm}
\setlength{\leftmargin}{\evensidemargin}
\addtolength{\leftmargin}{\tmplength}
\settowidth{\labelsep}{X}
\addtolength{\leftmargin}{\labelsep}
\setlength{\labelwidth}{\tmplength}
}
\item[\textbf{Declaration}\hfill]
\ifpdf
\begin{flushleft}
\fi
\begin{ttfamily}
public constructor Create; overload;\end{ttfamily}

\ifpdf
\end{flushleft}
\fi

\end{list}
\paragraph*{Create}\hspace*{\fill}

\label{PasDoc_StreamUtils.TBufferedStream-Create}
\index{Create}
\begin{list}{}{
\settowidth{\tmplength}{\textbf{Description}}
\setlength{\itemindent}{0cm}
\setlength{\listparindent}{0cm}
\setlength{\leftmargin}{\evensidemargin}
\addtolength{\leftmargin}{\tmplength}
\settowidth{\labelsep}{X}
\addtolength{\leftmargin}{\labelsep}
\setlength{\labelwidth}{\tmplength}
}
\item[\textbf{Declaration}\hfill]
\ifpdf
\begin{flushleft}
\fi
\begin{ttfamily}
public constructor Create(Stream : TStream; BufferSize : Integer = DEFAULT{\_}BUFSIZE; OwnsStream : Boolean = FALSE); overload; virtual;\end{ttfamily}

\ifpdf
\end{flushleft}
\fi

\par
\item[\textbf{Description}]
Dummy, don't call!

\end{list}
\paragraph*{Create}\hspace*{\fill}

\label{PasDoc_StreamUtils.TBufferedStream-Create}
\index{Create}
\begin{list}{}{
\settowidth{\tmplength}{\textbf{Description}}
\setlength{\itemindent}{0cm}
\setlength{\listparindent}{0cm}
\setlength{\leftmargin}{\evensidemargin}
\addtolength{\leftmargin}{\tmplength}
\settowidth{\labelsep}{X}
\addtolength{\leftmargin}{\labelsep}
\setlength{\labelwidth}{\tmplength}
}
\item[\textbf{Declaration}\hfill]
\ifpdf
\begin{flushleft}
\fi
\begin{ttfamily}
public constructor Create(const FileName : String; Mode : Word; BufferSize : Integer = DEFAULT{\_}BUFSIZE); overload; virtual;\end{ttfamily}

\ifpdf
\end{flushleft}
\fi

\end{list}
\paragraph*{Destroy}\hspace*{\fill}

\label{PasDoc_StreamUtils.TBufferedStream-Destroy}
\index{Destroy}
\begin{list}{}{
\settowidth{\tmplength}{\textbf{Description}}
\setlength{\itemindent}{0cm}
\setlength{\listparindent}{0cm}
\setlength{\leftmargin}{\evensidemargin}
\addtolength{\leftmargin}{\tmplength}
\settowidth{\labelsep}{X}
\addtolength{\leftmargin}{\labelsep}
\setlength{\labelwidth}{\tmplength}
}
\item[\textbf{Declaration}\hfill]
\ifpdf
\begin{flushleft}
\fi
\begin{ttfamily}
public destructor Destroy; override;\end{ttfamily}

\ifpdf
\end{flushleft}
\fi

\end{list}
\paragraph*{Flush}\hspace*{\fill}

\label{PasDoc_StreamUtils.TBufferedStream-Flush}
\index{Flush}
\begin{list}{}{
\settowidth{\tmplength}{\textbf{Description}}
\setlength{\itemindent}{0cm}
\setlength{\listparindent}{0cm}
\setlength{\leftmargin}{\evensidemargin}
\addtolength{\leftmargin}{\tmplength}
\settowidth{\labelsep}{X}
\addtolength{\leftmargin}{\labelsep}
\setlength{\labelwidth}{\tmplength}
}
\item[\textbf{Declaration}\hfill]
\ifpdf
\begin{flushleft}
\fi
\begin{ttfamily}
public procedure Flush; inline;\end{ttfamily}

\ifpdf
\end{flushleft}
\fi

\end{list}
\paragraph*{Read}\hspace*{\fill}

\label{PasDoc_StreamUtils.TBufferedStream-Read}
\index{Read}
\begin{list}{}{
\settowidth{\tmplength}{\textbf{Description}}
\setlength{\itemindent}{0cm}
\setlength{\listparindent}{0cm}
\setlength{\leftmargin}{\evensidemargin}
\addtolength{\leftmargin}{\tmplength}
\settowidth{\labelsep}{X}
\addtolength{\leftmargin}{\labelsep}
\setlength{\labelwidth}{\tmplength}
}
\item[\textbf{Declaration}\hfill]
\ifpdf
\begin{flushleft}
\fi
\begin{ttfamily}
public function Read(var Buffer; Count: Integer): Integer; override;\end{ttfamily}

\ifpdf
\end{flushleft}
\fi

\end{list}
\paragraph*{Seek}\hspace*{\fill}

\label{PasDoc_StreamUtils.TBufferedStream-Seek}
\index{Seek}
\begin{list}{}{
\settowidth{\tmplength}{\textbf{Description}}
\setlength{\itemindent}{0cm}
\setlength{\listparindent}{0cm}
\setlength{\leftmargin}{\evensidemargin}
\addtolength{\leftmargin}{\tmplength}
\settowidth{\labelsep}{X}
\addtolength{\leftmargin}{\labelsep}
\setlength{\labelwidth}{\tmplength}
}
\item[\textbf{Declaration}\hfill]
\ifpdf
\begin{flushleft}
\fi
\begin{ttfamily}
public function Seek(Offset: Integer; Origin: Word): Integer; override;\end{ttfamily}

\ifpdf
\end{flushleft}
\fi

\end{list}
\paragraph*{Seek}\hspace*{\fill}

\label{PasDoc_StreamUtils.TBufferedStream-Seek}
\index{Seek}
\begin{list}{}{
\settowidth{\tmplength}{\textbf{Description}}
\setlength{\itemindent}{0cm}
\setlength{\listparindent}{0cm}
\setlength{\leftmargin}{\evensidemargin}
\addtolength{\leftmargin}{\tmplength}
\settowidth{\labelsep}{X}
\addtolength{\leftmargin}{\labelsep}
\setlength{\labelwidth}{\tmplength}
}
\item[\textbf{Declaration}\hfill]
\ifpdf
\begin{flushleft}
\fi
\begin{ttfamily}
public function Seek(const Offset: Int64; Origin: TSeekOrigin): Int64; override;\end{ttfamily}

\ifpdf
\end{flushleft}
\fi

\end{list}
\paragraph*{Write}\hspace*{\fill}

\label{PasDoc_StreamUtils.TBufferedStream-Write}
\index{Write}
\begin{list}{}{
\settowidth{\tmplength}{\textbf{Description}}
\setlength{\itemindent}{0cm}
\setlength{\listparindent}{0cm}
\setlength{\leftmargin}{\evensidemargin}
\addtolength{\leftmargin}{\tmplength}
\settowidth{\labelsep}{X}
\addtolength{\leftmargin}{\labelsep}
\setlength{\labelwidth}{\tmplength}
}
\item[\textbf{Declaration}\hfill]
\ifpdf
\begin{flushleft}
\fi
\begin{ttfamily}
public function Write(const Buffer; Count: Integer): Integer; override;\end{ttfamily}

\ifpdf
\end{flushleft}
\fi

\end{list}
\section{Functions and Procedures}
\ifpdf
\subsection*{\large{\textbf{StreamReadLine}}\normalsize\hspace{1ex}\hrulefill}
\else
\subsection*{StreamReadLine}
\fi
\label{PasDoc_StreamUtils-StreamReadLine}
\index{StreamReadLine}
\begin{list}{}{
\settowidth{\tmplength}{\textbf{Description}}
\setlength{\itemindent}{0cm}
\setlength{\listparindent}{0cm}
\setlength{\leftmargin}{\evensidemargin}
\addtolength{\leftmargin}{\tmplength}
\settowidth{\labelsep}{X}
\addtolength{\leftmargin}{\labelsep}
\setlength{\labelwidth}{\tmplength}
}
\item[\textbf{Declaration}\hfill]
\ifpdf
\begin{flushleft}
\fi
\begin{ttfamily}
function StreamReadLine(const AStream: TStream): AnsiString;\end{ttfamily}

\ifpdf
\end{flushleft}
\fi

\end{list}
\ifpdf
\subsection*{\large{\textbf{StreamWriteLine}}\normalsize\hspace{1ex}\hrulefill}
\else
\subsection*{StreamWriteLine}
\fi
\label{PasDoc_StreamUtils-StreamWriteLine}
\index{StreamWriteLine}
\begin{list}{}{
\settowidth{\tmplength}{\textbf{Description}}
\setlength{\itemindent}{0cm}
\setlength{\listparindent}{0cm}
\setlength{\leftmargin}{\evensidemargin}
\addtolength{\leftmargin}{\tmplength}
\settowidth{\labelsep}{X}
\addtolength{\leftmargin}{\labelsep}
\setlength{\labelwidth}{\tmplength}
}
\item[\textbf{Declaration}\hfill]
\ifpdf
\begin{flushleft}
\fi
\begin{ttfamily}
procedure StreamWriteLine(const AStream: TStream; const AString: AnsiString);\end{ttfamily}

\ifpdf
\end{flushleft}
\fi

\par
\item[\textbf{Description}]
Write AString contents, then LineEnding to AStream

\end{list}
\ifpdf
\subsection*{\large{\textbf{StreamWriteString}}\normalsize\hspace{1ex}\hrulefill}
\else
\subsection*{StreamWriteString}
\fi
\label{PasDoc_StreamUtils-StreamWriteString}
\index{StreamWriteString}
\begin{list}{}{
\settowidth{\tmplength}{\textbf{Description}}
\setlength{\itemindent}{0cm}
\setlength{\listparindent}{0cm}
\setlength{\leftmargin}{\evensidemargin}
\addtolength{\leftmargin}{\tmplength}
\settowidth{\labelsep}{X}
\addtolength{\leftmargin}{\labelsep}
\setlength{\labelwidth}{\tmplength}
}
\item[\textbf{Declaration}\hfill]
\ifpdf
\begin{flushleft}
\fi
\begin{ttfamily}
procedure StreamWriteString(const AStream: TStream; const AString: AnsiString);\end{ttfamily}

\ifpdf
\end{flushleft}
\fi

\par
\item[\textbf{Description}]
Just write AString contents to AStream

\end{list}
\section{Constants}
\ifpdf
\subsection*{\large{\textbf{DEFAULT{\_}BUFSIZE}}\normalsize\hspace{1ex}\hrulefill}
\else
\subsection*{DEFAULT{\_}BUFSIZE}
\fi
\label{PasDoc_StreamUtils-DEFAULT_BUFSIZE}
\index{DEFAULT{\_}BUFSIZE}
\begin{list}{}{
\settowidth{\tmplength}{\textbf{Description}}
\setlength{\itemindent}{0cm}
\setlength{\listparindent}{0cm}
\setlength{\leftmargin}{\evensidemargin}
\addtolength{\leftmargin}{\tmplength}
\settowidth{\labelsep}{X}
\addtolength{\leftmargin}{\labelsep}
\setlength{\labelwidth}{\tmplength}
}
\item[\textbf{Declaration}\hfill]
\ifpdf
\begin{flushleft}
\fi
\begin{ttfamily}
DEFAULT{\_}BUFSIZE = 4096;\end{ttfamily}

\ifpdf
\end{flushleft}
\fi

\end{list}
\ifpdf
\subsection*{\large{\textbf{MIN{\_}BUFSIZE}}\normalsize\hspace{1ex}\hrulefill}
\else
\subsection*{MIN{\_}BUFSIZE}
\fi
\label{PasDoc_StreamUtils-MIN_BUFSIZE}
\index{MIN{\_}BUFSIZE}
\begin{list}{}{
\settowidth{\tmplength}{\textbf{Description}}
\setlength{\itemindent}{0cm}
\setlength{\listparindent}{0cm}
\setlength{\leftmargin}{\evensidemargin}
\addtolength{\leftmargin}{\tmplength}
\settowidth{\labelsep}{X}
\addtolength{\leftmargin}{\labelsep}
\setlength{\labelwidth}{\tmplength}
}
\item[\textbf{Declaration}\hfill]
\ifpdf
\begin{flushleft}
\fi
\begin{ttfamily}
MIN{\_}BUFSIZE     = 128;\end{ttfamily}

\ifpdf
\end{flushleft}
\fi

\end{list}
\ifpdf
\subsection*{\large{\textbf{MAX{\_}BUFSIZE}}\normalsize\hspace{1ex}\hrulefill}
\else
\subsection*{MAX{\_}BUFSIZE}
\fi
\label{PasDoc_StreamUtils-MAX_BUFSIZE}
\index{MAX{\_}BUFSIZE}
\begin{list}{}{
\settowidth{\tmplength}{\textbf{Description}}
\setlength{\itemindent}{0cm}
\setlength{\listparindent}{0cm}
\setlength{\leftmargin}{\evensidemargin}
\addtolength{\leftmargin}{\tmplength}
\settowidth{\labelsep}{X}
\addtolength{\leftmargin}{\labelsep}
\setlength{\labelwidth}{\tmplength}
}
\item[\textbf{Declaration}\hfill]
\ifpdf
\begin{flushleft}
\fi
\begin{ttfamily}
MAX{\_}BUFSIZE     = 1024 * 64;\end{ttfamily}

\ifpdf
\end{flushleft}
\fi

\end{list}
\section{Authors}
\par
Johannes Berg {$<$}johannes@sipsolutions.de{$>$}

\par
Arno Garrels {$<$}first name.name@nospamgmx.de{$>$}

\chapter{Unit PasDoc{\_}StringPairVector}
\label{PasDoc_StringPairVector}
\index{PasDoc{\_}StringPairVector}
\section{Description}
Simple container for a pair of strings.
\section{Uses}
\begin{itemize}
\item \begin{ttfamily}Classes\end{ttfamily}\item \begin{ttfamily}PasDoc{\_}ObjectVector\end{ttfamily}(\ref{PasDoc_ObjectVector})\end{itemize}
\section{Overview}
\begin{description}
\item[\texttt{\begin{ttfamily}TStringPair\end{ttfamily} Class}]
\item[\texttt{\begin{ttfamily}TStringPairVector\end{ttfamily} Class}]List of string pairs.
\end{description}
\section{Classes, Interfaces, Objects and Records}
\ifpdf
\subsection*{\large{\textbf{TStringPair Class}}\normalsize\hspace{1ex}\hrulefill}
\else
\subsection*{TStringPair Class}
\fi
\label{PasDoc_StringPairVector.TStringPair}
\index{TStringPair}
\subsubsection*{\large{\textbf{Hierarchy}}\normalsize\hspace{1ex}\hfill}
TStringPair {$>$} TObject
%%%%Description
\subsubsection*{\large{\textbf{Fields}}\normalsize\hspace{1ex}\hfill}
\begin{list}{}{
\settowidth{\tmplength}{\textbf{Value}}
\setlength{\itemindent}{0cm}
\setlength{\listparindent}{0cm}
\setlength{\leftmargin}{\evensidemargin}
\addtolength{\leftmargin}{\tmplength}
\settowidth{\labelsep}{X}
\addtolength{\leftmargin}{\labelsep}
\setlength{\labelwidth}{\tmplength}
}
\label{PasDoc_StringPairVector.TStringPair-Name}
\index{Name}
\item[\textbf{Name}\hfill]
\ifpdf
\begin{flushleft}
\fi
\begin{ttfamily}
public Name: string;\end{ttfamily}

\ifpdf
\end{flushleft}
\fi


\par  \label{PasDoc_StringPairVector.TStringPair-Value}
\index{Value}
\item[\textbf{Value}\hfill]
\ifpdf
\begin{flushleft}
\fi
\begin{ttfamily}
public Value: string;\end{ttfamily}

\ifpdf
\end{flushleft}
\fi


\par  \label{PasDoc_StringPairVector.TStringPair-Data}
\index{Data}
\item[\textbf{Data}\hfill]
\ifpdf
\begin{flushleft}
\fi
\begin{ttfamily}
public Data: Pointer;\end{ttfamily}

\ifpdf
\end{flushleft}
\fi


\par  \end{list}
\subsubsection*{\large{\textbf{Methods}}\normalsize\hspace{1ex}\hfill}
\paragraph*{CreateExtractFirstWord}\hspace*{\fill}

\label{PasDoc_StringPairVector.TStringPair-CreateExtractFirstWord}
\index{CreateExtractFirstWord}
\begin{list}{}{
\settowidth{\tmplength}{\textbf{Description}}
\setlength{\itemindent}{0cm}
\setlength{\listparindent}{0cm}
\setlength{\leftmargin}{\evensidemargin}
\addtolength{\leftmargin}{\tmplength}
\settowidth{\labelsep}{X}
\addtolength{\leftmargin}{\labelsep}
\setlength{\labelwidth}{\tmplength}
}
\item[\textbf{Declaration}\hfill]
\ifpdf
\begin{flushleft}
\fi
\begin{ttfamily}
public constructor CreateExtractFirstWord(const S: string);\end{ttfamily}

\ifpdf
\end{flushleft}
\fi

\par
\item[\textbf{Description}]
Init Name and Value by \begin{ttfamily}ExtractFirstWord\end{ttfamily}(\ref{PasDoc_Utils-ExtractFirstWord}) from S.

\end{list}
\paragraph*{Create}\hspace*{\fill}

\label{PasDoc_StringPairVector.TStringPair-Create}
\index{Create}
\begin{list}{}{
\settowidth{\tmplength}{\textbf{Description}}
\setlength{\itemindent}{0cm}
\setlength{\listparindent}{0cm}
\setlength{\leftmargin}{\evensidemargin}
\addtolength{\leftmargin}{\tmplength}
\settowidth{\labelsep}{X}
\addtolength{\leftmargin}{\labelsep}
\setlength{\labelwidth}{\tmplength}
}
\item[\textbf{Declaration}\hfill]
\ifpdf
\begin{flushleft}
\fi
\begin{ttfamily}
public constructor Create; overload;\end{ttfamily}

\ifpdf
\end{flushleft}
\fi

\end{list}
\paragraph*{Create}\hspace*{\fill}

\label{PasDoc_StringPairVector.TStringPair-Create}
\index{Create}
\begin{list}{}{
\settowidth{\tmplength}{\textbf{Description}}
\setlength{\itemindent}{0cm}
\setlength{\listparindent}{0cm}
\setlength{\leftmargin}{\evensidemargin}
\addtolength{\leftmargin}{\tmplength}
\settowidth{\labelsep}{X}
\addtolength{\leftmargin}{\labelsep}
\setlength{\labelwidth}{\tmplength}
}
\item[\textbf{Declaration}\hfill]
\ifpdf
\begin{flushleft}
\fi
\begin{ttfamily}
public constructor Create(const AName, AValue: string; AData: Pointer = nil); overload;\end{ttfamily}

\ifpdf
\end{flushleft}
\fi

\end{list}
\ifpdf
\subsection*{\large{\textbf{TStringPairVector Class}}\normalsize\hspace{1ex}\hrulefill}
\else
\subsection*{TStringPairVector Class}
\fi
\label{PasDoc_StringPairVector.TStringPairVector}
\index{TStringPairVector}
\subsubsection*{\large{\textbf{Hierarchy}}\normalsize\hspace{1ex}\hfill}
TStringPairVector {$>$} \begin{ttfamily}TObjectVector\end{ttfamily}(\ref{PasDoc_ObjectVector.TObjectVector}) {$>$} 
TObjectList
\subsubsection*{\large{\textbf{Description}}\normalsize\hspace{1ex}\hfill}
List of string pairs. This class contains only non{-}nil objects of class TStringPair.

Using this class instead of TStringList (with it's Name and Value properties) is often better, because this allows both Name and Value of each pair to safely contain any special characters (including '=' and newline markers). It's also faster, since it doesn't try to encode Name and Value into one string.\subsubsection*{\large{\textbf{Properties}}\normalsize\hspace{1ex}\hfill}
\begin{list}{}{
\settowidth{\tmplength}{\textbf{Items}}
\setlength{\itemindent}{0cm}
\setlength{\listparindent}{0cm}
\setlength{\leftmargin}{\evensidemargin}
\addtolength{\leftmargin}{\tmplength}
\settowidth{\labelsep}{X}
\addtolength{\leftmargin}{\labelsep}
\setlength{\labelwidth}{\tmplength}
}
\label{PasDoc_StringPairVector.TStringPairVector-Items}
\index{Items}
\item[\textbf{Items}\hfill]
\ifpdf
\begin{flushleft}
\fi
\begin{ttfamily}
public property Items[i:Integer]: TStringPair read GetItems write SetItems;\end{ttfamily}

\ifpdf
\end{flushleft}
\fi


\par  \end{list}
\subsubsection*{\large{\textbf{Methods}}\normalsize\hspace{1ex}\hfill}
\paragraph*{Text}\hspace*{\fill}

\label{PasDoc_StringPairVector.TStringPairVector-Text}
\index{Text}
\begin{list}{}{
\settowidth{\tmplength}{\textbf{Description}}
\setlength{\itemindent}{0cm}
\setlength{\listparindent}{0cm}
\setlength{\leftmargin}{\evensidemargin}
\addtolength{\leftmargin}{\tmplength}
\settowidth{\labelsep}{X}
\addtolength{\leftmargin}{\labelsep}
\setlength{\labelwidth}{\tmplength}
}
\item[\textbf{Declaration}\hfill]
\ifpdf
\begin{flushleft}
\fi
\begin{ttfamily}
public function Text(const NameValueSepapator, ItemSeparator: string): string;\end{ttfamily}

\ifpdf
\end{flushleft}
\fi

\par
\item[\textbf{Description}]
Returns all items Names and Values glued together. For every item, string Name + NameValueSepapator + Value is constructed. Then all such strings for every items all concatenated with ItemSeparator.

Remember that the very idea of \begin{ttfamily}TStringPair\end{ttfamily}(\ref{PasDoc_StringPairVector.TStringPair}) and \begin{ttfamily}TStringPairVector\end{ttfamily}(\ref{PasDoc_StringPairVector.TStringPairVector}) is that Name and Value strings may contain any special characters, including things you give here as NameValueSepapator and ItemSeparator. So it's practically impossible to later convert such Text back to items and Names/Value pairs.

\end{list}
\paragraph*{FindName}\hspace*{\fill}

\label{PasDoc_StringPairVector.TStringPairVector-FindName}
\index{FindName}
\begin{list}{}{
\settowidth{\tmplength}{\textbf{Description}}
\setlength{\itemindent}{0cm}
\setlength{\listparindent}{0cm}
\setlength{\leftmargin}{\evensidemargin}
\addtolength{\leftmargin}{\tmplength}
\settowidth{\labelsep}{X}
\addtolength{\leftmargin}{\labelsep}
\setlength{\labelwidth}{\tmplength}
}
\item[\textbf{Declaration}\hfill]
\ifpdf
\begin{flushleft}
\fi
\begin{ttfamily}
public function FindName(const Name: string; IgnoreCase: boolean = true): Integer;\end{ttfamily}

\ifpdf
\end{flushleft}
\fi

\par
\item[\textbf{Description}]
Finds a string pair with given Name. Returns {-}1 if not found.

\end{list}
\paragraph*{DeleteName}\hspace*{\fill}

\label{PasDoc_StringPairVector.TStringPairVector-DeleteName}
\index{DeleteName}
\begin{list}{}{
\settowidth{\tmplength}{\textbf{Description}}
\setlength{\itemindent}{0cm}
\setlength{\listparindent}{0cm}
\setlength{\leftmargin}{\evensidemargin}
\addtolength{\leftmargin}{\tmplength}
\settowidth{\labelsep}{X}
\addtolength{\leftmargin}{\labelsep}
\setlength{\labelwidth}{\tmplength}
}
\item[\textbf{Declaration}\hfill]
\ifpdf
\begin{flushleft}
\fi
\begin{ttfamily}
public function DeleteName(const Name: string; IgnoreCase: boolean = true): boolean;\end{ttfamily}

\ifpdf
\end{flushleft}
\fi

\par
\item[\textbf{Description}]
Removes first string pair with given Name. Returns if some pair was removed.

\end{list}
\paragraph*{LoadFromBinaryStream}\hspace*{\fill}

\label{PasDoc_StringPairVector.TStringPairVector-LoadFromBinaryStream}
\index{LoadFromBinaryStream}
\begin{list}{}{
\settowidth{\tmplength}{\textbf{Description}}
\setlength{\itemindent}{0cm}
\setlength{\listparindent}{0cm}
\setlength{\leftmargin}{\evensidemargin}
\addtolength{\leftmargin}{\tmplength}
\settowidth{\labelsep}{X}
\addtolength{\leftmargin}{\labelsep}
\setlength{\labelwidth}{\tmplength}
}
\item[\textbf{Declaration}\hfill]
\ifpdf
\begin{flushleft}
\fi
\begin{ttfamily}
public procedure LoadFromBinaryStream(Stream: TStream);\end{ttfamily}

\ifpdf
\end{flushleft}
\fi

\par
\item[\textbf{Description}]
Load from a stream using the binary format. For each item, it's Name and Value are saved. (TStringPair.Data pointers are \textit{not} saved.)

\end{list}
\paragraph*{SaveToBinaryStream}\hspace*{\fill}

\label{PasDoc_StringPairVector.TStringPairVector-SaveToBinaryStream}
\index{SaveToBinaryStream}
\begin{list}{}{
\settowidth{\tmplength}{\textbf{Description}}
\setlength{\itemindent}{0cm}
\setlength{\listparindent}{0cm}
\setlength{\leftmargin}{\evensidemargin}
\addtolength{\leftmargin}{\tmplength}
\settowidth{\labelsep}{X}
\addtolength{\leftmargin}{\labelsep}
\setlength{\labelwidth}{\tmplength}
}
\item[\textbf{Declaration}\hfill]
\ifpdf
\begin{flushleft}
\fi
\begin{ttfamily}
public procedure SaveToBinaryStream(Stream: TStream);\end{ttfamily}

\ifpdf
\end{flushleft}
\fi

\par
\item[\textbf{Description}]
Save to a stream, in a format readable by \begin{ttfamily}LoadFromBinaryStream\end{ttfamily}(\ref{PasDoc_StringPairVector.TStringPairVector-LoadFromBinaryStream}).

\end{list}
\paragraph*{FirstName}\hspace*{\fill}

\label{PasDoc_StringPairVector.TStringPairVector-FirstName}
\index{FirstName}
\begin{list}{}{
\settowidth{\tmplength}{\textbf{Description}}
\setlength{\itemindent}{0cm}
\setlength{\listparindent}{0cm}
\setlength{\leftmargin}{\evensidemargin}
\addtolength{\leftmargin}{\tmplength}
\settowidth{\labelsep}{X}
\addtolength{\leftmargin}{\labelsep}
\setlength{\labelwidth}{\tmplength}
}
\item[\textbf{Declaration}\hfill]
\ifpdf
\begin{flushleft}
\fi
\begin{ttfamily}
public function FirstName: string;\end{ttfamily}

\ifpdf
\end{flushleft}
\fi

\par
\item[\textbf{Description}]
Name of first item, or '' if list empty.

\end{list}
\chapter{Unit PasDoc{\_}StringVector}
\label{PasDoc_StringVector}
\index{PasDoc{\_}StringVector}
\section{Description}
String vector based on TStringList.\hfill\vspace*{1ex}

   The string vector is based on TStringList and simply exports a few extra functions {-} I did this so I didn't have to change so much old code, this has only little additional functionality
\section{Uses}
\begin{itemize}
\item \begin{ttfamily}Classes\end{ttfamily}\end{itemize}
\section{Overview}
\begin{description}
\item[\texttt{\begin{ttfamily}TStringVector\end{ttfamily} Class}]
\end{description}
\begin{description}
\item[\texttt{NewStringVector}]
\item[\texttt{IsEmpty}]
\end{description}
\section{Classes, Interfaces, Objects and Records}
\ifpdf
\subsection*{\large{\textbf{TStringVector Class}}\normalsize\hspace{1ex}\hrulefill}
\else
\subsection*{TStringVector Class}
\fi
\label{PasDoc_StringVector.TStringVector}
\index{TStringVector}
\subsubsection*{\large{\textbf{Hierarchy}}\normalsize\hspace{1ex}\hfill}
TStringVector {$>$} TStringList
%%%%Description
\subsubsection*{\large{\textbf{Methods}}\normalsize\hspace{1ex}\hfill}
\paragraph*{FirstName}\hspace*{\fill}

\label{PasDoc_StringVector.TStringVector-FirstName}
\index{FirstName}
\begin{list}{}{
\settowidth{\tmplength}{\textbf{Description}}
\setlength{\itemindent}{0cm}
\setlength{\listparindent}{0cm}
\setlength{\leftmargin}{\evensidemargin}
\addtolength{\leftmargin}{\tmplength}
\settowidth{\labelsep}{X}
\addtolength{\leftmargin}{\labelsep}
\setlength{\labelwidth}{\tmplength}
}
\item[\textbf{Declaration}\hfill]
\ifpdf
\begin{flushleft}
\fi
\begin{ttfamily}
public function FirstName: string;\end{ttfamily}

\ifpdf
\end{flushleft}
\fi

\par
\item[\textbf{Description}]
This is the same thing as Items[0]

\end{list}
\paragraph*{LoadFromTextFileAdd}\hspace*{\fill}

\label{PasDoc_StringVector.TStringVector-LoadFromTextFileAdd}
\index{LoadFromTextFileAdd}
\begin{list}{}{
\settowidth{\tmplength}{\textbf{Description}}
\setlength{\itemindent}{0cm}
\setlength{\listparindent}{0cm}
\setlength{\leftmargin}{\evensidemargin}
\addtolength{\leftmargin}{\tmplength}
\settowidth{\labelsep}{X}
\addtolength{\leftmargin}{\labelsep}
\setlength{\labelwidth}{\tmplength}
}
\item[\textbf{Declaration}\hfill]
\ifpdf
\begin{flushleft}
\fi
\begin{ttfamily}
public procedure LoadFromTextFileAdd(const AFilename: string); overload;\end{ttfamily}

\ifpdf
\end{flushleft}
\fi

\end{list}
\paragraph*{LoadFromTextFileAdd}\hspace*{\fill}

\label{PasDoc_StringVector.TStringVector-LoadFromTextFileAdd}
\index{LoadFromTextFileAdd}
\begin{list}{}{
\settowidth{\tmplength}{\textbf{Description}}
\setlength{\itemindent}{0cm}
\setlength{\listparindent}{0cm}
\setlength{\leftmargin}{\evensidemargin}
\addtolength{\leftmargin}{\tmplength}
\settowidth{\labelsep}{X}
\addtolength{\leftmargin}{\labelsep}
\setlength{\labelwidth}{\tmplength}
}
\item[\textbf{Declaration}\hfill]
\ifpdf
\begin{flushleft}
\fi
\begin{ttfamily}
public procedure LoadFromTextFileAdd(var ATextFile: TextFile); overload;\end{ttfamily}

\ifpdf
\end{flushleft}
\fi

\end{list}
\paragraph*{RemoveAllNamesCI}\hspace*{\fill}

\label{PasDoc_StringVector.TStringVector-RemoveAllNamesCI}
\index{RemoveAllNamesCI}
\begin{list}{}{
\settowidth{\tmplength}{\textbf{Description}}
\setlength{\itemindent}{0cm}
\setlength{\listparindent}{0cm}
\setlength{\leftmargin}{\evensidemargin}
\addtolength{\leftmargin}{\tmplength}
\settowidth{\labelsep}{X}
\addtolength{\leftmargin}{\labelsep}
\setlength{\labelwidth}{\tmplength}
}
\item[\textbf{Declaration}\hfill]
\ifpdf
\begin{flushleft}
\fi
\begin{ttfamily}
public procedure RemoveAllNamesCI(const AName: string);\end{ttfamily}

\ifpdf
\end{flushleft}
\fi

\end{list}
\paragraph*{ExistsNameCI}\hspace*{\fill}

\label{PasDoc_StringVector.TStringVector-ExistsNameCI}
\index{ExistsNameCI}
\begin{list}{}{
\settowidth{\tmplength}{\textbf{Description}}
\setlength{\itemindent}{0cm}
\setlength{\listparindent}{0cm}
\setlength{\leftmargin}{\evensidemargin}
\addtolength{\leftmargin}{\tmplength}
\settowidth{\labelsep}{X}
\addtolength{\leftmargin}{\labelsep}
\setlength{\labelwidth}{\tmplength}
}
\item[\textbf{Declaration}\hfill]
\ifpdf
\begin{flushleft}
\fi
\begin{ttfamily}
public function ExistsNameCI(const AName: string): boolean;\end{ttfamily}

\ifpdf
\end{flushleft}
\fi

\end{list}
\paragraph*{IsEmpty}\hspace*{\fill}

\label{PasDoc_StringVector.TStringVector-IsEmpty}
\index{IsEmpty}
\begin{list}{}{
\settowidth{\tmplength}{\textbf{Description}}
\setlength{\itemindent}{0cm}
\setlength{\listparindent}{0cm}
\setlength{\leftmargin}{\evensidemargin}
\addtolength{\leftmargin}{\tmplength}
\settowidth{\labelsep}{X}
\addtolength{\leftmargin}{\labelsep}
\setlength{\labelwidth}{\tmplength}
}
\item[\textbf{Declaration}\hfill]
\ifpdf
\begin{flushleft}
\fi
\begin{ttfamily}
public function IsEmpty: boolean;\end{ttfamily}

\ifpdf
\end{flushleft}
\fi

\end{list}
\paragraph*{AddNotExisting}\hspace*{\fill}

\label{PasDoc_StringVector.TStringVector-AddNotExisting}
\index{AddNotExisting}
\begin{list}{}{
\settowidth{\tmplength}{\textbf{Description}}
\setlength{\itemindent}{0cm}
\setlength{\listparindent}{0cm}
\setlength{\leftmargin}{\evensidemargin}
\addtolength{\leftmargin}{\tmplength}
\settowidth{\labelsep}{X}
\addtolength{\leftmargin}{\labelsep}
\setlength{\labelwidth}{\tmplength}
}
\item[\textbf{Declaration}\hfill]
\ifpdf
\begin{flushleft}
\fi
\begin{ttfamily}
public function AddNotExisting(const AString: string): Integer;\end{ttfamily}

\ifpdf
\end{flushleft}
\fi

\end{list}
\paragraph*{LoadFromBinaryStream}\hspace*{\fill}

\label{PasDoc_StringVector.TStringVector-LoadFromBinaryStream}
\index{LoadFromBinaryStream}
\begin{list}{}{
\settowidth{\tmplength}{\textbf{Description}}
\setlength{\itemindent}{0cm}
\setlength{\listparindent}{0cm}
\setlength{\leftmargin}{\evensidemargin}
\addtolength{\leftmargin}{\tmplength}
\settowidth{\labelsep}{X}
\addtolength{\leftmargin}{\labelsep}
\setlength{\labelwidth}{\tmplength}
}
\item[\textbf{Declaration}\hfill]
\ifpdf
\begin{flushleft}
\fi
\begin{ttfamily}
public procedure LoadFromBinaryStream(Stream: TStream);\end{ttfamily}

\ifpdf
\end{flushleft}
\fi

\par
\item[\textbf{Description}]
Load from a stream using the binary format.

The binary format is \begin{itemize}
\item Count
\item followed by each string, loaded using \begin{ttfamily}TSerializable.LoadStringFromStream\end{ttfamily}(\ref{PasDoc_Serialize.TSerializable-LoadStringFromStream}).
\end{itemize}

Note that you should never use our Text value to load/save this object from/into a stream, like \begin{ttfamily}Text := TSerializable.LoadStringFromStream(Stream)\end{ttfamily}. Using and assigning to the Text value breaks when some strings have newlines inside that should be preserved.

\end{list}
\paragraph*{SaveToBinaryStream}\hspace*{\fill}

\label{PasDoc_StringVector.TStringVector-SaveToBinaryStream}
\index{SaveToBinaryStream}
\begin{list}{}{
\settowidth{\tmplength}{\textbf{Description}}
\setlength{\itemindent}{0cm}
\setlength{\listparindent}{0cm}
\setlength{\leftmargin}{\evensidemargin}
\addtolength{\leftmargin}{\tmplength}
\settowidth{\labelsep}{X}
\addtolength{\leftmargin}{\labelsep}
\setlength{\labelwidth}{\tmplength}
}
\item[\textbf{Declaration}\hfill]
\ifpdf
\begin{flushleft}
\fi
\begin{ttfamily}
public procedure SaveToBinaryStream(Stream: TStream);\end{ttfamily}

\ifpdf
\end{flushleft}
\fi

\par
\item[\textbf{Description}]
Save to a stream, in a format readable by \begin{ttfamily}LoadFromBinaryStream\end{ttfamily}(\ref{PasDoc_StringVector.TStringVector-LoadFromBinaryStream}).

\end{list}
\section{Functions and Procedures}
\ifpdf
\subsection*{\large{\textbf{NewStringVector}}\normalsize\hspace{1ex}\hrulefill}
\else
\subsection*{NewStringVector}
\fi
\label{PasDoc_StringVector-NewStringVector}
\index{NewStringVector}
\begin{list}{}{
\settowidth{\tmplength}{\textbf{Description}}
\setlength{\itemindent}{0cm}
\setlength{\listparindent}{0cm}
\setlength{\leftmargin}{\evensidemargin}
\addtolength{\leftmargin}{\tmplength}
\settowidth{\labelsep}{X}
\addtolength{\leftmargin}{\labelsep}
\setlength{\labelwidth}{\tmplength}
}
\item[\textbf{Declaration}\hfill]
\ifpdf
\begin{flushleft}
\fi
\begin{ttfamily}
function NewStringVector: TStringVector;\end{ttfamily}

\ifpdf
\end{flushleft}
\fi

\end{list}
\ifpdf
\subsection*{\large{\textbf{IsEmpty}}\normalsize\hspace{1ex}\hrulefill}
\else
\subsection*{IsEmpty}
\fi
\label{PasDoc_StringVector-IsEmpty}
\index{IsEmpty}
\begin{list}{}{
\settowidth{\tmplength}{\textbf{Description}}
\setlength{\itemindent}{0cm}
\setlength{\listparindent}{0cm}
\setlength{\leftmargin}{\evensidemargin}
\addtolength{\leftmargin}{\tmplength}
\settowidth{\labelsep}{X}
\addtolength{\leftmargin}{\labelsep}
\setlength{\labelwidth}{\tmplength}
}
\item[\textbf{Declaration}\hfill]
\ifpdf
\begin{flushleft}
\fi
\begin{ttfamily}
function IsEmpty(const AOV: TStringVector): boolean; overload;\end{ttfamily}

\ifpdf
\end{flushleft}
\fi

\end{list}
\section{Authors}
\par
Johannes Berg {$<$}johannes@sipsolutions.de{$>$}

\par
Michalis Kamburelis

\chapter{Unit PasDoc{\_}TagManager}
\label{PasDoc_TagManager}
\index{PasDoc{\_}TagManager}
\section{Description}
Collects information about available @{-}tags and can parse text with tags.
\section{Uses}
\begin{itemize}
\item \begin{ttfamily}SysUtils\end{ttfamily}\item \begin{ttfamily}Classes\end{ttfamily}\item \begin{ttfamily}PasDoc{\_}Types\end{ttfamily}(\ref{PasDoc_Types})\item \begin{ttfamily}PasDoc{\_}ObjectVector\end{ttfamily}(\ref{PasDoc_ObjectVector})\end{itemize}
\section{Overview}
\begin{description}
\item[\texttt{\begin{ttfamily}TTag\end{ttfamily} Class}]
\item[\texttt{\begin{ttfamily}TTopLevelTag\end{ttfamily} Class}]
\item[\texttt{\begin{ttfamily}TNonSelfTag\end{ttfamily} Class}]
\item[\texttt{\begin{ttfamily}TTagVector\end{ttfamily} Class}]All Items of this list must be non{-}nil TTag objects.
\item[\texttt{\begin{ttfamily}TTagManager\end{ttfamily} Class}]
\end{description}
\section{Classes, Interfaces, Objects and Records}
\ifpdf
\subsection*{\large{\textbf{TTag Class}}\normalsize\hspace{1ex}\hrulefill}
\else
\subsection*{TTag Class}
\fi
\label{PasDoc_TagManager.TTag}
\index{TTag}
\subsubsection*{\large{\textbf{Hierarchy}}\normalsize\hspace{1ex}\hfill}
TTag {$>$} TObject
%%%%Description
\subsubsection*{\large{\textbf{Properties}}\normalsize\hspace{1ex}\hfill}
\begin{list}{}{
\settowidth{\tmplength}{\textbf{OnAllowedInside}}
\setlength{\itemindent}{0cm}
\setlength{\listparindent}{0cm}
\setlength{\leftmargin}{\evensidemargin}
\addtolength{\leftmargin}{\tmplength}
\settowidth{\labelsep}{X}
\addtolength{\leftmargin}{\labelsep}
\setlength{\labelwidth}{\tmplength}
}
\label{PasDoc_TagManager.TTag-TagOptions}
\index{TagOptions}
\item[\textbf{TagOptions}\hfill]
\ifpdf
\begin{flushleft}
\fi
\begin{ttfamily}
public property TagOptions: TTagOptions read FTagOptions write FTagOptions;\end{ttfamily}

\ifpdf
\end{flushleft}
\fi


\par  \label{PasDoc_TagManager.TTag-TagManager}
\index{TagManager}
\item[\textbf{TagManager}\hfill]
\ifpdf
\begin{flushleft}
\fi
\begin{ttfamily}
public property TagManager: TTagManager read FTagManager;\end{ttfamily}

\ifpdf
\end{flushleft}
\fi


\par TagManager that will recognize and handle this tag. Note that the tag instance is owned by this tag manager (i.e. it will be freed inside this tag manager). It can be nil if no tag manager currently owns this tag.

Note that it's very useful in \begin{ttfamily}Execute\end{ttfamily}(\ref{PasDoc_TagManager.TTag-Execute}) or \begin{ttfamily}OnExecute\end{ttfamily}(\ref{PasDoc_TagManager.TTag-OnExecute}) implementations.

E.g. you can use it to report a message by \begin{ttfamily}TagManager.DoMessage(...)\end{ttfamily}, this is e.g. used by implementation of TPasItem.StoreAbstractTag.

You could also use this to manually force recursive behavior of a given tag. I.e let's suppose that you have a tag with TagOptions = [toParameterRequired], so the TagParameter parameter passed to handler was not recursively expanded. Then you can do inside your handler \texttt{NewTagParameter~:=~TagManager.Execute(TagParameter,~...)\\
} and this way you have explicitly recursively expanded the tag.

Scenario above is actually used in implementation of @noAutoLink tag. There I call TagManager.Execute with parameter \begin{ttfamily}AutoLink\end{ttfamily} set to false thus preventing auto{-}linking inside text within @noAutoLink.\label{PasDoc_TagManager.TTag-Name}
\index{Name}
\item[\textbf{Name}\hfill]
\ifpdf
\begin{flushleft}
\fi
\begin{ttfamily}
public property Name: string read FName write FName;\end{ttfamily}

\ifpdf
\end{flushleft}
\fi


\par Name of the tag, that must be specified by user after the "@" sign. Value of this property must always be lowercase.\label{PasDoc_TagManager.TTag-OnPreExecute}
\index{OnPreExecute}
\item[\textbf{OnPreExecute}\hfill]
\ifpdf
\begin{flushleft}
\fi
\begin{ttfamily}
public property OnPreExecute: TTagExecuteEvent
      read FOnPreExecute write FOnPreExecute;\end{ttfamily}

\ifpdf
\end{flushleft}
\fi


\par  \label{PasDoc_TagManager.TTag-OnExecute}
\index{OnExecute}
\item[\textbf{OnExecute}\hfill]
\ifpdf
\begin{flushleft}
\fi
\begin{ttfamily}
public property OnExecute: TTagExecuteEvent
      read FOnExecute write FOnExecute;\end{ttfamily}

\ifpdf
\end{flushleft}
\fi


\par  \label{PasDoc_TagManager.TTag-OnAllowedInside}
\index{OnAllowedInside}
\item[\textbf{OnAllowedInside}\hfill]
\ifpdf
\begin{flushleft}
\fi
\begin{ttfamily}
public property OnAllowedInside: TTagAllowedInsideEvent
      read FOnAllowedInside write FOnAllowedInside;\end{ttfamily}

\ifpdf
\end{flushleft}
\fi


\par  \end{list}
\subsubsection*{\large{\textbf{Methods}}\normalsize\hspace{1ex}\hfill}
\paragraph*{Create}\hspace*{\fill}

\label{PasDoc_TagManager.TTag-Create}
\index{Create}
\begin{list}{}{
\settowidth{\tmplength}{\textbf{Description}}
\setlength{\itemindent}{0cm}
\setlength{\listparindent}{0cm}
\setlength{\leftmargin}{\evensidemargin}
\addtolength{\leftmargin}{\tmplength}
\settowidth{\labelsep}{X}
\addtolength{\leftmargin}{\labelsep}
\setlength{\labelwidth}{\tmplength}
}
\item[\textbf{Declaration}\hfill]
\ifpdf
\begin{flushleft}
\fi
\begin{ttfamily}
public constructor Create(ATagManager: TTagManager; const AName: string; AOnPreExecute: TTagExecuteEvent; AOnExecute: TTagExecuteEvent; const ATagOptions: TTagOptions);\end{ttfamily}

\ifpdf
\end{flushleft}
\fi

\par
\item[\textbf{Description}]
Note that AName will be converted to lowercase before assigning to Name.

\end{list}
\paragraph*{PreExecute}\hspace*{\fill}

\label{PasDoc_TagManager.TTag-PreExecute}
\index{PreExecute}
\begin{list}{}{
\settowidth{\tmplength}{\textbf{Description}}
\setlength{\itemindent}{0cm}
\setlength{\listparindent}{0cm}
\setlength{\leftmargin}{\evensidemargin}
\addtolength{\leftmargin}{\tmplength}
\settowidth{\labelsep}{X}
\addtolength{\leftmargin}{\labelsep}
\setlength{\labelwidth}{\tmplength}
}
\item[\textbf{Declaration}\hfill]
\ifpdf
\begin{flushleft}
\fi
\begin{ttfamily}
public procedure PreExecute(var ThisTagData: TObject; EnclosingTag: TTag; var EnclosingTagData: TObject; const TagParameter: string; var ReplaceStr: string); virtual;\end{ttfamily}

\ifpdf
\end{flushleft}
\fi

\par
\item[\textbf{Description}]
This is completely analogous to \begin{ttfamily}Execute\end{ttfamily}(\ref{PasDoc_TagManager.TTag-Execute}) but used when \begin{ttfamily}TTagManager.PreExecute\end{ttfamily}(\ref{PasDoc_TagManager.TTagManager-PreExecute}) is \begin{ttfamily}True\end{ttfamily}. In this class this simply calls \begin{ttfamily}OnPreExecute\end{ttfamily}(\ref{PasDoc_TagManager.TTag-OnPreExecute}).

\end{list}
\paragraph*{Execute}\hspace*{\fill}

\label{PasDoc_TagManager.TTag-Execute}
\index{Execute}
\begin{list}{}{
\settowidth{\tmplength}{\textbf{Description}}
\setlength{\itemindent}{0cm}
\setlength{\listparindent}{0cm}
\setlength{\leftmargin}{\evensidemargin}
\addtolength{\leftmargin}{\tmplength}
\settowidth{\labelsep}{X}
\addtolength{\leftmargin}{\labelsep}
\setlength{\labelwidth}{\tmplength}
}
\item[\textbf{Declaration}\hfill]
\ifpdf
\begin{flushleft}
\fi
\begin{ttfamily}
public procedure Execute(var ThisTagData: TObject; EnclosingTag: TTag; var EnclosingTagData: TObject; const TagParameter: string; var ReplaceStr: string); virtual;\end{ttfamily}

\ifpdf
\end{flushleft}
\fi

\par
\item[\textbf{Description}]
This will be used to do main work when this @{-}tag occured in description.

EnclosingTag parameter specifies enclosing tag. This is useful for tags that must behave differently in different contexts, e.g. in plain{-}text output @item tag will behave differently inside @orderedList and @unorderedList. EnclosingTag is nil when the tag occured at top level of the description.

ThisTagData and EnclosingTagData form a mechanism to pass arbitraty data between child tags enclosed within one parent tag. Example uses:

\begin{itemize}
\item This is the way for multiple @item tags inside @orderedList tag to count themselves (to provide list item numbers, for pasdoc output formats that can't automatically number list items).
\item This is the way for @itemSpacing tag to communicate with enclosing @orderedList tag to specify list style. 
\item And this is the way for @cell tags to be collected inside rows data and then @rows tags to be collected inside table data. Thanks to such collecting \begin{ttfamily}TDocGenerator.FormatTable\end{ttfamily}(\ref{PasDoc_Gen.TDocGenerator-FormatTable}) receives at once all information about given table, and can use it to format table.
\end{itemize}

How does this XxxTagData mechanism work:

When we start parsing parameter of some tag with toRecursiveTags, we create a new pointer inited to \begin{ttfamily}CreateOccurenceData\end{ttfamily}(\ref{PasDoc_TagManager.TTag-CreateOccurenceData}). When @{-}tags occur inside this parameter, we pass them this pointer as EnclosingTagData (this way all @{-}tags with the same parent can use this pointer to communicate with each other). At the end, when parameter was parsed, we call given tag's Execute method passing the resulting pointer as ThisTagData (this way @{-}tags with the same parent can use this pointer to pass some data to their parent).

In this class this method simply calls \begin{ttfamily}OnExecute\end{ttfamily}(\ref{PasDoc_TagManager.TTag-OnExecute}) (if assigned).

\end{list}
\paragraph*{AllowedInside}\hspace*{\fill}

\label{PasDoc_TagManager.TTag-AllowedInside}
\index{AllowedInside}
\begin{list}{}{
\settowidth{\tmplength}{\textbf{Description}}
\setlength{\itemindent}{0cm}
\setlength{\listparindent}{0cm}
\setlength{\leftmargin}{\evensidemargin}
\addtolength{\leftmargin}{\tmplength}
\settowidth{\labelsep}{X}
\addtolength{\leftmargin}{\labelsep}
\setlength{\labelwidth}{\tmplength}
}
\item[\textbf{Declaration}\hfill]
\ifpdf
\begin{flushleft}
\fi
\begin{ttfamily}
public function AllowedInside(EnclosingTag: TTag): boolean; virtual;\end{ttfamily}

\ifpdf
\end{flushleft}
\fi

\par
\item[\textbf{Description}]
This will be checked always when this tag occurs within description. Given EnclosingTag is enclosing tag, nil if we're in top level. If this returns false then this tag will not be allowed inside EnclosingTag.

In this class this method \begin{enumerate}
\setcounter{enumi}{0} \setcounter{enumii}{0} \setcounter{enumiii}{0} \setcounter{enumiv}{0} 
\item  Assumes that Result = true if we're at top level or EnclosingTag.TagOptions contains toAllowOtherTagsInsideByDefault. Else it assumes Result = false.
\setcounter{enumi}{1} \setcounter{enumii}{1} \setcounter{enumiii}{1} \setcounter{enumiv}{1} 
\item  Then it calls \begin{ttfamily}OnAllowedInside(Self, EnclosingTag, Result)\end{ttfamily}(\ref{PasDoc_TagManager.TTag-OnAllowedInside}) (if OnAllowedInside is assigned).
\end{enumerate}

\end{list}
\paragraph*{CreateOccurenceData}\hspace*{\fill}

\label{PasDoc_TagManager.TTag-CreateOccurenceData}
\index{CreateOccurenceData}
\begin{list}{}{
\settowidth{\tmplength}{\textbf{Description}}
\setlength{\itemindent}{0cm}
\setlength{\listparindent}{0cm}
\setlength{\leftmargin}{\evensidemargin}
\addtolength{\leftmargin}{\tmplength}
\settowidth{\labelsep}{X}
\addtolength{\leftmargin}{\labelsep}
\setlength{\labelwidth}{\tmplength}
}
\item[\textbf{Declaration}\hfill]
\ifpdf
\begin{flushleft}
\fi
\begin{ttfamily}
public function CreateOccurenceData: TObject; virtual;\end{ttfamily}

\ifpdf
\end{flushleft}
\fi

\par
\item[\textbf{Description}]
In this class this simply returns \begin{ttfamily}Nil\end{ttfamily}.

\end{list}
\paragraph*{DestroyOccurenceData}\hspace*{\fill}

\label{PasDoc_TagManager.TTag-DestroyOccurenceData}
\index{DestroyOccurenceData}
\begin{list}{}{
\settowidth{\tmplength}{\textbf{Description}}
\setlength{\itemindent}{0cm}
\setlength{\listparindent}{0cm}
\setlength{\leftmargin}{\evensidemargin}
\addtolength{\leftmargin}{\tmplength}
\settowidth{\labelsep}{X}
\addtolength{\leftmargin}{\labelsep}
\setlength{\labelwidth}{\tmplength}
}
\item[\textbf{Declaration}\hfill]
\ifpdf
\begin{flushleft}
\fi
\begin{ttfamily}
public procedure DestroyOccurenceData(Value: TObject); virtual;\end{ttfamily}

\ifpdf
\end{flushleft}
\fi

\par
\item[\textbf{Description}]
In this class this simply does \begin{ttfamily}Value.Free\end{ttfamily}.

\end{list}
\ifpdf
\subsection*{\large{\textbf{TTopLevelTag Class}}\normalsize\hspace{1ex}\hrulefill}
\else
\subsection*{TTopLevelTag Class}
\fi
\label{PasDoc_TagManager.TTopLevelTag}
\index{TTopLevelTag}
\subsubsection*{\large{\textbf{Hierarchy}}\normalsize\hspace{1ex}\hfill}
TTopLevelTag {$>$} \begin{ttfamily}TTag\end{ttfamily}(\ref{PasDoc_TagManager.TTag}) {$>$} 
TObject
%%%%Description
\subsubsection*{\large{\textbf{Methods}}\normalsize\hspace{1ex}\hfill}
\paragraph*{AllowedInside}\hspace*{\fill}

\label{PasDoc_TagManager.TTopLevelTag-AllowedInside}
\index{AllowedInside}
\begin{list}{}{
\settowidth{\tmplength}{\textbf{Description}}
\setlength{\itemindent}{0cm}
\setlength{\listparindent}{0cm}
\setlength{\leftmargin}{\evensidemargin}
\addtolength{\leftmargin}{\tmplength}
\settowidth{\labelsep}{X}
\addtolength{\leftmargin}{\labelsep}
\setlength{\labelwidth}{\tmplength}
}
\item[\textbf{Declaration}\hfill]
\ifpdf
\begin{flushleft}
\fi
\begin{ttfamily}
public function AllowedInside(EnclosingTag: TTag): boolean; override;\end{ttfamily}

\ifpdf
\end{flushleft}
\fi

\par
\item[\textbf{Description}]
This returns just \begin{ttfamily}EnclosingTag = nil\end{ttfamily}.

Which means that this tag is allowed only at top level of description, never inside parameter of some tag.

\end{list}
\ifpdf
\subsection*{\large{\textbf{TNonSelfTag Class}}\normalsize\hspace{1ex}\hrulefill}
\else
\subsection*{TNonSelfTag Class}
\fi
\label{PasDoc_TagManager.TNonSelfTag}
\index{TNonSelfTag}
\subsubsection*{\large{\textbf{Hierarchy}}\normalsize\hspace{1ex}\hfill}
TNonSelfTag {$>$} \begin{ttfamily}TTag\end{ttfamily}(\ref{PasDoc_TagManager.TTag}) {$>$} 
TObject
%%%%Description
\subsubsection*{\large{\textbf{Methods}}\normalsize\hspace{1ex}\hfill}
\paragraph*{AllowedInside}\hspace*{\fill}

\label{PasDoc_TagManager.TNonSelfTag-AllowedInside}
\index{AllowedInside}
\begin{list}{}{
\settowidth{\tmplength}{\textbf{Description}}
\setlength{\itemindent}{0cm}
\setlength{\listparindent}{0cm}
\setlength{\leftmargin}{\evensidemargin}
\addtolength{\leftmargin}{\tmplength}
\settowidth{\labelsep}{X}
\addtolength{\leftmargin}{\labelsep}
\setlength{\labelwidth}{\tmplength}
}
\item[\textbf{Declaration}\hfill]
\ifpdf
\begin{flushleft}
\fi
\begin{ttfamily}
public function AllowedInside(EnclosingTag: TTag): boolean; override;\end{ttfamily}

\ifpdf
\end{flushleft}
\fi

\par
\item[\textbf{Description}]
This returns just \begin{ttfamily}inherited and (EnclosingTag {$<$}{$>$} Self)\end{ttfamily}.

Which means that (assuming that \begin{ttfamily}OnAllowedInside\end{ttfamily}(\ref{PasDoc_TagManager.TTag-OnAllowedInside}) is not assigned) this tag is allowed at top level of description and inside parameter of any tag \textit{but not within itself and not within tags without toAllowOtherTagsInsideByDefault}.

This is currently not used by any tag.

\end{list}
\ifpdf
\subsection*{\large{\textbf{TTagVector Class}}\normalsize\hspace{1ex}\hrulefill}
\else
\subsection*{TTagVector Class}
\fi
\label{PasDoc_TagManager.TTagVector}
\index{TTagVector}
\subsubsection*{\large{\textbf{Hierarchy}}\normalsize\hspace{1ex}\hfill}
TTagVector {$>$} \begin{ttfamily}TObjectVector\end{ttfamily}(\ref{PasDoc_ObjectVector.TObjectVector}) {$>$} 
TObjectList
\subsubsection*{\large{\textbf{Description}}\normalsize\hspace{1ex}\hfill}
All Items of this list must be non{-}nil TTag objects.\subsubsection*{\large{\textbf{Methods}}\normalsize\hspace{1ex}\hfill}
\paragraph*{FindByName}\hspace*{\fill}

\label{PasDoc_TagManager.TTagVector-FindByName}
\index{FindByName}
\begin{list}{}{
\settowidth{\tmplength}{\textbf{Description}}
\setlength{\itemindent}{0cm}
\setlength{\listparindent}{0cm}
\setlength{\leftmargin}{\evensidemargin}
\addtolength{\leftmargin}{\tmplength}
\settowidth{\labelsep}{X}
\addtolength{\leftmargin}{\labelsep}
\setlength{\labelwidth}{\tmplength}
}
\item[\textbf{Declaration}\hfill]
\ifpdf
\begin{flushleft}
\fi
\begin{ttfamily}
public function FindByName(const Name: string): TTag;\end{ttfamily}

\ifpdf
\end{flushleft}
\fi

\par
\item[\textbf{Description}]
Case of Name does \textit{not} matter (so don't bother converting it to lowercase or something like that before using this method). Returns nil if not found.

Maybe in the future it will use hashlist, for now it's not needed.

\end{list}
\ifpdf
\subsection*{\large{\textbf{TTagManager Class}}\normalsize\hspace{1ex}\hrulefill}
\else
\subsection*{TTagManager Class}
\fi
\label{PasDoc_TagManager.TTagManager}
\index{TTagManager}
\subsubsection*{\large{\textbf{Hierarchy}}\normalsize\hspace{1ex}\hfill}
TTagManager {$>$} TObject
%%%%Description
\subsubsection*{\large{\textbf{Properties}}\normalsize\hspace{1ex}\hfill}
\begin{list}{}{
\settowidth{\tmplength}{\textbf{OnTryAutoLink}}
\setlength{\itemindent}{0cm}
\setlength{\listparindent}{0cm}
\setlength{\leftmargin}{\evensidemargin}
\addtolength{\leftmargin}{\tmplength}
\settowidth{\labelsep}{X}
\addtolength{\leftmargin}{\labelsep}
\setlength{\labelwidth}{\tmplength}
}
\label{PasDoc_TagManager.TTagManager-OnMessage}
\index{OnMessage}
\item[\textbf{OnMessage}\hfill]
\ifpdf
\begin{flushleft}
\fi
\begin{ttfamily}
public property OnMessage: TPasDocMessageEvent read FOnMessage write FOnMessage;\end{ttfamily}

\ifpdf
\end{flushleft}
\fi


\par This will be used to print messages from within \begin{ttfamily}Execute\end{ttfamily}(\ref{PasDoc_TagManager.TTagManager-Execute}).

Note that in this unit we essentialy "don't know" that parsed Description string is probably attached to some TPasItem. It's good that we don't know it (because it makes this class more flexible). But it also means that OnMessage that you assign here may want to add to passed AMessage something like + ' (Expanded{\_}TPasItem{\_}Name)', see e.g. TDocGenerator.DoMessageFromExpandDescription. Maybe in the future we will do some descendant of this class, like TTagManagerForPasItem.\label{PasDoc_TagManager.TTagManager-Paragraph}
\index{Paragraph}
\item[\textbf{Paragraph}\hfill]
\ifpdf
\begin{flushleft}
\fi
\begin{ttfamily}
public property Paragraph: string read FParagraph write FParagraph;\end{ttfamily}

\ifpdf
\end{flushleft}
\fi


\par This will be inserted on paragraph marker (two consecutive newlines, see wiki page WritingDocumentation) in the text. This should specify how paragraphs are marked in particular output format, e.g. html generator may set this to '{$<$}p{$>$}'.

Default value is ' ' (one space).\label{PasDoc_TagManager.TTagManager-Space}
\index{Space}
\item[\textbf{Space}\hfill]
\ifpdf
\begin{flushleft}
\fi
\begin{ttfamily}
public property Space: string read FSpace write FSpace;\end{ttfamily}

\ifpdf
\end{flushleft}
\fi


\par This will be inserted on each whitespace sequence (but not on paragraph break). This is consistent with [\href{https://github.com/pasdoc/pasdoc/wiki/WritingDocumentation}{https://github.com/pasdoc/pasdoc/wiki/WritingDocumentation}] that clearly says that "amount of whitespace does not matter".

Although in some pasdoc output formats amount of whitespace also does not matter (e.g. HTML and LaTeX) but in other (e.g. plain text) it matters, so such space compression is needed. In other output formats (no examples yet) it may need to be expressed by something else than simple space, that's why this property is exposed.

Default value is ' ' (one space).\label{PasDoc_TagManager.TTagManager-ShortDash}
\index{ShortDash}
\item[\textbf{ShortDash}\hfill]
\ifpdf
\begin{flushleft}
\fi
\begin{ttfamily}
public property ShortDash: string read FShortDash write FShortDash;\end{ttfamily}

\ifpdf
\end{flushleft}
\fi


\par This will be inserted on \begin{ttfamily}@{-}\end{ttfamily} in description, and on a normal single dash in description that is not a part of en{-}dash or em{-}dash. This should produce just a short dash.

Default value is '{-}'.

You will never get any '{-}' character to be converted by ConvertString. Convertion of '{-}' is controlled solely by XxxDash properties of tag manager.

 \item[\textbf{See also}]
\begin{description}
\item[\begin{ttfamily}EnDash\end{ttfamily}(\ref{PasDoc_TagManager.TTagManager-EnDash})] 
This will be inserted on \begin{ttfamily}{-}{-}\end{ttfamily} in description.
\item[\begin{ttfamily}EmDash\end{ttfamily}(\ref{PasDoc_TagManager.TTagManager-EmDash})] 
This will be inserted on \begin{ttfamily}{-}{-}{-}\end{ttfamily} in description.
\end{description}
\label{PasDoc_TagManager.TTagManager-EnDash}
\index{EnDash}
\item[\textbf{EnDash}\hfill]
\ifpdf
\begin{flushleft}
\fi
\begin{ttfamily}
public property EnDash: string read FEnDash write FEnDash;\end{ttfamily}

\ifpdf
\end{flushleft}
\fi


\par This will be inserted on \begin{ttfamily}{-}{-}\end{ttfamily} in description. This should produce en{-}dash (as in LaTeX). Default value is '{-}{-}'.\label{PasDoc_TagManager.TTagManager-EmDash}
\index{EmDash}
\item[\textbf{EmDash}\hfill]
\ifpdf
\begin{flushleft}
\fi
\begin{ttfamily}
public property EmDash: string read FEmDash write FEmDash;\end{ttfamily}

\ifpdf
\end{flushleft}
\fi


\par This will be inserted on \begin{ttfamily}{-}{-}{-}\end{ttfamily} in description. This should produce em{-}dash (as in LaTeX). Default value is '{-}{-}{-}'.\label{PasDoc_TagManager.TTagManager-URLLink}
\index{URLLink}
\item[\textbf{URLLink}\hfill]
\ifpdf
\begin{flushleft}
\fi
\begin{ttfamily}
public property URLLink: TStringConverter read FURLLink write FURLLink;\end{ttfamily}

\ifpdf
\end{flushleft}
\fi


\par This will be called from \begin{ttfamily}Execute\end{ttfamily}(\ref{PasDoc_TagManager.TTagManager-Execute}) when URL will be found in Description. Note that passed here URL will \textit{not} be processed by \begin{ttfamily}ConvertString\end{ttfamily}(\ref{PasDoc_TagManager.TTagManager-ConvertString}).

This tells what to put in result on URL. If this is not assigned, then ConvertString(URL) will be appended to Result in \begin{ttfamily}Execute\end{ttfamily}(\ref{PasDoc_TagManager.TTagManager-Execute}).\label{PasDoc_TagManager.TTagManager-OnTryAutoLink}
\index{OnTryAutoLink}
\item[\textbf{OnTryAutoLink}\hfill]
\ifpdf
\begin{flushleft}
\fi
\begin{ttfamily}
public property OnTryAutoLink: TTryAutoLinkEvent
      read FOnTryAutoLink write FOnTryAutoLink;\end{ttfamily}

\ifpdf
\end{flushleft}
\fi


\par This should check does QualifiedIdentifier looks like a name of some existing identifier. If yes, sets AutoLinked to true and sets QualifiedIdentifierReplacement to a link to QualifiedIdentifier (QualifiedIdentifierReplacement should be ready to be put in final documentation, i.e. already in the final output format). By default AutoLinked is false.\label{PasDoc_TagManager.TTagManager-ConvertString}
\index{ConvertString}
\item[\textbf{ConvertString}\hfill]
\ifpdf
\begin{flushleft}
\fi
\begin{ttfamily}
public property ConvertString: TStringConverter
      read FConvertString write FConvertString;\end{ttfamily}

\ifpdf
\end{flushleft}
\fi


\par  \label{PasDoc_TagManager.TTagManager-Abbreviations}
\index{Abbreviations}
\item[\textbf{Abbreviations}\hfill]
\ifpdf
\begin{flushleft}
\fi
\begin{ttfamily}
public property Abbreviations: TStringList read FAbbreviations write FAbbreviations;\end{ttfamily}

\ifpdf
\end{flushleft}
\fi


\par  \label{PasDoc_TagManager.TTagManager-PreExecute}
\index{PreExecute}
\item[\textbf{PreExecute}\hfill]
\ifpdf
\begin{flushleft}
\fi
\begin{ttfamily}
public property PreExecute: boolean
      read FPreExecute write FPreExecute;\end{ttfamily}

\ifpdf
\end{flushleft}
\fi


\par When \begin{ttfamily}PreExecute\end{ttfamily} is \begin{ttfamily}True\end{ttfamily}, tag manager will work a little differently than usual:

\begin{itemize}
\item Instead of \begin{ttfamily}TTag.Execute\end{ttfamily}(\ref{PasDoc_TagManager.TTag-Execute}), \begin{ttfamily}TTag.PreExecute\end{ttfamily}(\ref{PasDoc_TagManager.TTag-PreExecute}) will be called.
\item Various warnings will \textit{not} be reported.

Assumption is that you will later process the same text with \begin{ttfamily}PreExecute\end{ttfamily} set to \begin{ttfamily}False\end{ttfamily} to get all the warnings.
\item AutoLink will not be used (like it was always false). Also the result of \begin{ttfamily}Execute\end{ttfamily}(\ref{PasDoc_TagManager.TTagManager-Execute}) will be pretty much random and meaningless (so you should ignore it). Also this means that the TagParameter for tags with toRecursiveTags should be ignored, because it will be something incorrect. This means that only tags without toRecursiveTags should actually use TagParameter in their OnPreExecute handlers.

Assumption is that you actually don't care about the result of \begin{ttfamily}Execute\end{ttfamily}(\ref{PasDoc_TagManager.TTagManager-Execute}) methods, and you will later process the same text with \begin{ttfamily}PreExecute\end{ttfamily} set to \begin{ttfamily}False\end{ttfamily} to get the proper output.

The goal is to make execution with PreExecute set to \begin{ttfamily}True\end{ttfamily} as fast as possible.
\end{itemize}\label{PasDoc_TagManager.TTagManager-Markdown}
\index{Markdown}
\item[\textbf{Markdown}\hfill]
\ifpdf
\begin{flushleft}
\fi
\begin{ttfamily}
public property Markdown: boolean
      read FMarkdown write FMarkdown default false;\end{ttfamily}

\ifpdf
\end{flushleft}
\fi


\par When \begin{ttfamily}Markdown\end{ttfamily} is \begin{ttfamily}True\end{ttfamily}, Markdown syntax is considered\end{list}
\subsubsection*{\large{\textbf{Methods}}\normalsize\hspace{1ex}\hfill}
\paragraph*{Create}\hspace*{\fill}

\label{PasDoc_TagManager.TTagManager-Create}
\index{Create}
\begin{list}{}{
\settowidth{\tmplength}{\textbf{Description}}
\setlength{\itemindent}{0cm}
\setlength{\listparindent}{0cm}
\setlength{\leftmargin}{\evensidemargin}
\addtolength{\leftmargin}{\tmplength}
\settowidth{\labelsep}{X}
\addtolength{\leftmargin}{\labelsep}
\setlength{\labelwidth}{\tmplength}
}
\item[\textbf{Declaration}\hfill]
\ifpdf
\begin{flushleft}
\fi
\begin{ttfamily}
public constructor Create;\end{ttfamily}

\ifpdf
\end{flushleft}
\fi

\end{list}
\paragraph*{Destroy}\hspace*{\fill}

\label{PasDoc_TagManager.TTagManager-Destroy}
\index{Destroy}
\begin{list}{}{
\settowidth{\tmplength}{\textbf{Description}}
\setlength{\itemindent}{0cm}
\setlength{\listparindent}{0cm}
\setlength{\leftmargin}{\evensidemargin}
\addtolength{\leftmargin}{\tmplength}
\settowidth{\labelsep}{X}
\addtolength{\leftmargin}{\labelsep}
\setlength{\labelwidth}{\tmplength}
}
\item[\textbf{Declaration}\hfill]
\ifpdf
\begin{flushleft}
\fi
\begin{ttfamily}
public destructor Destroy; override;\end{ttfamily}

\ifpdf
\end{flushleft}
\fi

\end{list}
\paragraph*{DoMessage}\hspace*{\fill}

\label{PasDoc_TagManager.TTagManager-DoMessage}
\index{DoMessage}
\begin{list}{}{
\settowidth{\tmplength}{\textbf{Description}}
\setlength{\itemindent}{0cm}
\setlength{\listparindent}{0cm}
\setlength{\leftmargin}{\evensidemargin}
\addtolength{\leftmargin}{\tmplength}
\settowidth{\labelsep}{X}
\addtolength{\leftmargin}{\labelsep}
\setlength{\labelwidth}{\tmplength}
}
\item[\textbf{Declaration}\hfill]
\ifpdf
\begin{flushleft}
\fi
\begin{ttfamily}
public procedure DoMessage(const AVerbosity: Cardinal; const MessageType: TPasDocMessageType; const AMessage: string; const AArguments: array of const);\end{ttfamily}

\ifpdf
\end{flushleft}
\fi

\par
\item[\textbf{Description}]
Call OnMessage (if assigned) with given params.

\end{list}
\paragraph*{DoMessageNonPre}\hspace*{\fill}

\label{PasDoc_TagManager.TTagManager-DoMessageNonPre}
\index{DoMessageNonPre}
\begin{list}{}{
\settowidth{\tmplength}{\textbf{Description}}
\setlength{\itemindent}{0cm}
\setlength{\listparindent}{0cm}
\setlength{\leftmargin}{\evensidemargin}
\addtolength{\leftmargin}{\tmplength}
\settowidth{\labelsep}{X}
\addtolength{\leftmargin}{\labelsep}
\setlength{\labelwidth}{\tmplength}
}
\item[\textbf{Declaration}\hfill]
\ifpdf
\begin{flushleft}
\fi
\begin{ttfamily}
public procedure DoMessageNonPre(const AVerbosity: Cardinal; const MessageType: TPasDocMessageType; const AMessage: string; const AArguments: array of const);\end{ttfamily}

\ifpdf
\end{flushleft}
\fi

\par
\item[\textbf{Description}]
Call \begin{ttfamily}DoMessage\end{ttfamily}(\ref{PasDoc_TagManager.TTagManager-DoMessage}) only if \begin{ttfamily}PreExecute\end{ttfamily}(\ref{PasDoc_TagManager.TTagManager-PreExecute}) is \begin{ttfamily}False\end{ttfamily}.

\end{list}
\paragraph*{Execute}\hspace*{\fill}

\label{PasDoc_TagManager.TTagManager-Execute}
\index{Execute}
\begin{list}{}{
\settowidth{\tmplength}{\textbf{Description}}
\setlength{\itemindent}{0cm}
\setlength{\listparindent}{0cm}
\setlength{\leftmargin}{\evensidemargin}
\addtolength{\leftmargin}{\tmplength}
\settowidth{\labelsep}{X}
\addtolength{\leftmargin}{\labelsep}
\setlength{\labelwidth}{\tmplength}
}
\item[\textbf{Declaration}\hfill]
\ifpdf
\begin{flushleft}
\fi
\begin{ttfamily}
public function Execute(const Description: string; AutoLink: boolean; WantFirstSentenceEnd: boolean; out FirstSentenceEnd: Integer): string; overload;\end{ttfamily}

\ifpdf
\end{flushleft}
\fi

\par
\item[\textbf{Description}]
This method is the very essence of this class and this unit. It expands Description, which means that it processes Description (text supplied by user in some comment in parsed unit) into something ready to be included in output documentation. This means that this handles parsing @{-}tags, inserting paragraph markers, recognizing URLs in Description and correctly translating it, and translating rest of the "normal" text via ConvertString.

If WantFirstSentenceEnd then we will look for '.' char followed by any whitespace in Description. Moreover, this '.' must be outside of any @{-}tags parameter. Under FirstSentenceEnd we will return the number of beginning characters \textit{in the output string} that will include correspong '.' character (note that this definition takes into account that ConvertString may translate '.' into something longer). If no such character exists in Description, FirstSentenceEnd will be set to Length(Result), so the whole Description will be treated as it's first sentence.

If WantFirstSentenceEnd, FirstSentenceEnd will not be set.

\end{list}
\paragraph*{Execute}\hspace*{\fill}

\label{PasDoc_TagManager.TTagManager-Execute}
\index{Execute}
\begin{list}{}{
\settowidth{\tmplength}{\textbf{Description}}
\setlength{\itemindent}{0cm}
\setlength{\listparindent}{0cm}
\setlength{\leftmargin}{\evensidemargin}
\addtolength{\leftmargin}{\tmplength}
\settowidth{\labelsep}{X}
\addtolength{\leftmargin}{\labelsep}
\setlength{\labelwidth}{\tmplength}
}
\item[\textbf{Declaration}\hfill]
\ifpdf
\begin{flushleft}
\fi
\begin{ttfamily}
public function Execute(const Description: string; AutoLink: boolean): string; overload;\end{ttfamily}

\ifpdf
\end{flushleft}
\fi

\par
\item[\textbf{Description}]
This is equivalent to Execute(Description, AutoLink, false, Dummy)

\end{list}
\paragraph*{CoreExecute}\hspace*{\fill}

\label{PasDoc_TagManager.TTagManager-CoreExecute}
\index{CoreExecute}
\begin{list}{}{
\settowidth{\tmplength}{\textbf{Description}}
\setlength{\itemindent}{0cm}
\setlength{\listparindent}{0cm}
\setlength{\leftmargin}{\evensidemargin}
\addtolength{\leftmargin}{\tmplength}
\settowidth{\labelsep}{X}
\addtolength{\leftmargin}{\labelsep}
\setlength{\labelwidth}{\tmplength}
}
\item[\textbf{Declaration}\hfill]
\ifpdf
\begin{flushleft}
\fi
\begin{ttfamily}
public function CoreExecute(const Description: string; AutoLink: boolean; EnclosingTag: TTag; var EnclosingTagData: TObject; WantFirstSentenceEnd: boolean; out FirstSentenceEnd: Integer): string; overload;\end{ttfamily}

\ifpdf
\end{flushleft}
\fi

\par
\item[\textbf{Description}]
This is the underlying version of Execute. Use with caution!

If EnclosingTag = nil then this is understood to be toplevel of description, which means that all tags are allowed inside.

If EnclosingTag {$<$}{$>$} nil then this is not toplevel.

EnclosingTagData returns collected data for given EnclosingTag. You should init it to EnclosingTag.CreateOccurenceData. It will be passed as EnclosingTagData to each of @{-}tags found inside Description.

\end{list}
\paragraph*{CoreExecute}\hspace*{\fill}

\label{PasDoc_TagManager.TTagManager-CoreExecute}
\index{CoreExecute}
\begin{list}{}{
\settowidth{\tmplength}{\textbf{Description}}
\setlength{\itemindent}{0cm}
\setlength{\listparindent}{0cm}
\setlength{\leftmargin}{\evensidemargin}
\addtolength{\leftmargin}{\tmplength}
\settowidth{\labelsep}{X}
\addtolength{\leftmargin}{\labelsep}
\setlength{\labelwidth}{\tmplength}
}
\item[\textbf{Declaration}\hfill]
\ifpdf
\begin{flushleft}
\fi
\begin{ttfamily}
public function CoreExecute(const Description: string; AutoLink: boolean; EnclosingTag: TTag; var EnclosingTagData: TObject): string; overload;\end{ttfamily}

\ifpdf
\end{flushleft}
\fi

\end{list}
\section{Types}
\ifpdf
\subsection*{\large{\textbf{TTagExecuteEvent}}\normalsize\hspace{1ex}\hrulefill}
\else
\subsection*{TTagExecuteEvent}
\fi
\label{PasDoc_TagManager-TTagExecuteEvent}
\index{TTagExecuteEvent}
\begin{list}{}{
\settowidth{\tmplength}{\textbf{Description}}
\setlength{\itemindent}{0cm}
\setlength{\listparindent}{0cm}
\setlength{\leftmargin}{\evensidemargin}
\addtolength{\leftmargin}{\tmplength}
\settowidth{\labelsep}{X}
\addtolength{\leftmargin}{\labelsep}
\setlength{\labelwidth}{\tmplength}
}
\item[\textbf{Declaration}\hfill]
\ifpdf
\begin{flushleft}
\fi
\begin{ttfamily}
TTagExecuteEvent = procedure(ThisTag: TTag; var ThisTagData: TObject; EnclosingTag: TTag; var EnclosingTagData: TObject; const TagParameter: string; var ReplaceStr: string) of object;\end{ttfamily}

\ifpdf
\end{flushleft}
\fi

\par
\item[\textbf{Description}]
 \item[\textbf{See also}]
\begin{description}
\item[\begin{ttfamily}TTag.Execute\end{ttfamily}(\ref{PasDoc_TagManager.TTag-Execute})] 
This will be used to do main work when this @{-}tag occured in description.
\end{description}


\end{list}
\ifpdf
\subsection*{\large{\textbf{TTagAllowedInsideEvent}}\normalsize\hspace{1ex}\hrulefill}
\else
\subsection*{TTagAllowedInsideEvent}
\fi
\label{PasDoc_TagManager-TTagAllowedInsideEvent}
\index{TTagAllowedInsideEvent}
\begin{list}{}{
\settowidth{\tmplength}{\textbf{Description}}
\setlength{\itemindent}{0cm}
\setlength{\listparindent}{0cm}
\setlength{\leftmargin}{\evensidemargin}
\addtolength{\leftmargin}{\tmplength}
\settowidth{\labelsep}{X}
\addtolength{\leftmargin}{\labelsep}
\setlength{\labelwidth}{\tmplength}
}
\item[\textbf{Declaration}\hfill]
\ifpdf
\begin{flushleft}
\fi
\begin{ttfamily}
TTagAllowedInsideEvent = procedure( ThisTag: TTag; EnclosingTag: TTag; var Allowed: boolean) of object;\end{ttfamily}

\ifpdf
\end{flushleft}
\fi

\par
\item[\textbf{Description}]
 \item[\textbf{See also}]
\begin{description}
\item[\begin{ttfamily}TTag.AllowedInside\end{ttfamily}(\ref{PasDoc_TagManager.TTag-AllowedInside})] 
This will be checked always when this tag occurs within description.
\end{description}


\end{list}
\ifpdf
\subsection*{\large{\textbf{TStringConverter}}\normalsize\hspace{1ex}\hrulefill}
\else
\subsection*{TStringConverter}
\fi
\label{PasDoc_TagManager-TStringConverter}
\index{TStringConverter}
\begin{list}{}{
\settowidth{\tmplength}{\textbf{Description}}
\setlength{\itemindent}{0cm}
\setlength{\listparindent}{0cm}
\setlength{\leftmargin}{\evensidemargin}
\addtolength{\leftmargin}{\tmplength}
\settowidth{\labelsep}{X}
\addtolength{\leftmargin}{\labelsep}
\setlength{\labelwidth}{\tmplength}
}
\item[\textbf{Declaration}\hfill]
\ifpdf
\begin{flushleft}
\fi
\begin{ttfamily}
TStringConverter = function(const s: string): string of object;\end{ttfamily}

\ifpdf
\end{flushleft}
\fi

\end{list}
\ifpdf
\subsection*{\large{\textbf{TTagOption}}\normalsize\hspace{1ex}\hrulefill}
\else
\subsection*{TTagOption}
\fi
\label{PasDoc_TagManager-TTagOption}
\index{TTagOption}
\begin{list}{}{
\settowidth{\tmplength}{\textbf{Description}}
\setlength{\itemindent}{0cm}
\setlength{\listparindent}{0cm}
\setlength{\leftmargin}{\evensidemargin}
\addtolength{\leftmargin}{\tmplength}
\settowidth{\labelsep}{X}
\addtolength{\leftmargin}{\labelsep}
\setlength{\labelwidth}{\tmplength}
}
\item[\textbf{Declaration}\hfill]
\ifpdf
\begin{flushleft}
\fi
\begin{ttfamily}
TTagOption = (...);\end{ttfamily}

\ifpdf
\end{flushleft}
\fi

\par
\item[\textbf{Description}]
 \item[\textbf{Values}]
\begin{description}
\item[\texttt{toParameterRequired}] \label{PasDoc_TagManager-toParameterRequired}
\index{}
This means that tag expects parameters. If this is not included in TagOptions then tag should not be given any parameters, i.e. TagParameter passed to \begin{ttfamily}TTag.Execute\end{ttfamily}(\ref{PasDoc_TagManager.TTag-Execute}) should be ''. We will display a warning if user will try to give some parameters for such tag.
\item[\texttt{toRecursiveTags}] \label{PasDoc_TagManager-toRecursiveTags}
\index{}
This means that parameters of this tag will be expanded before passing them to \begin{ttfamily}TTag.Execute\end{ttfamily}(\ref{PasDoc_TagManager.TTag-Execute}). This means that we will expand recursive tags inside parameters, that we will ConvertString inside parameters, that we will handle paragraphs inside parameters etc. --- all that does \begin{ttfamily}TTagManager.Execute\end{ttfamily}(\ref{PasDoc_TagManager.TTagManager-Execute}).

If toParameterRequired is not present in TTagOptions then it's not important whether you included toRecursiveTags.

It's useful for some tags to include toParameterRequired without including toRecursiveTags, e.g. @longcode or @html, that want to get their parameters "verbatim", not processed.

\textbf{If toRecursiveTags is not included in tag options:} Then \textit{everything} is allowed within parameter of this tag, but nothing is interpreted. E.g. you can freely use @ char, and even write various @{-}tags inside @html tag --- this doesn't matter, because @{-}tags will not be interpreted (they will not be even searched !) inside @html tag. In other words, @ character means literally "@" inside @html, nothing more. The only exception are double @@, @( and @): we still treat them specially, to allow escaping the default parenthesis matching rules. Unless toRecursiveTagsManually is present.
\item[\texttt{toRecursiveTagsManually}] \label{PasDoc_TagManager-toRecursiveTagsManually}
\index{}
Use this, instead of toRecursiveTags, if the implementation of your tag calls (always!) TagManager.CoreExecute on given TagParameter. This means that your tag is expanded recursively (it handles {-}tags inside), but you do it manually (instead of allowing toRecursiveTags to do the job). In this case, TagParameter given will be really absolutely unmodified (even the special @@, @( and @) will not be handled), because we know that it will be handled later by special CoreExecute call.

Never use both flags toRecursiveTags and toRecursiveTagsManually.
\item[\texttt{toAllowOtherTagsInsideByDefault}] \label{PasDoc_TagManager-toAllowOtherTagsInsideByDefault}
\index{}
This is meaningful only if toRecursiveTags is included. Then toAllowOtherTagsInsideByDefault determines are other tags allowed by the default implementation of \begin{ttfamily}TTag.AllowedInside\end{ttfamily}(\ref{PasDoc_TagManager.TTag-AllowedInside}).
\item[\texttt{toAllowNormalTextInside}] \label{PasDoc_TagManager-toAllowNormalTextInside}
\index{}
This is meaningful only if toRecursiveTags is included. Then \begin{ttfamily}toAllowNormalTextInside\end{ttfamily} says that normal text is allowed inside parameter of this tag. \textit{"Normal text"} is anything except other @{-}tags: normal text, paragraph breaks, various dashes, URLs, and literal @ character (expressed by @@ in descriptions).

If \begin{ttfamily}toAllowNormalTextInside\end{ttfamily} will not be included, then normal text (not enclosed within other @{-}tags) will not be allowed inside. Only whitespace will be allowed, and it will be ignored anyway (i.e. will not be passed to ConvertString, empty line will not produce any Paragraph etc.). This is useful for tags like @orderedList that should only contain other @item tags inside.
\item[\texttt{toFirstWordVerbatim}] \label{PasDoc_TagManager-toFirstWordVerbatim}
\index{}
This is useful for tags like @raises and @param that treat 1st word of their descriptions very specially (where "what exactly is the 1st word" is defined by the \begin{ttfamily}ExtractFirstWord\end{ttfamily}(\ref{PasDoc_Utils-ExtractFirstWord}) function). This tells pasdoc to leave the beginning of tag parameter (the first word and the eventual whitespace before it) as it is in the parameter. Don't search there for @{-}tags, URLs, {-}{-} or other special dashes, don't insert paragraphs, don't try to auto{-}link it.

This is meaningful only if toRecursiveTags is included (otherwise the whole tag parameters are always preserved "verbatim").

TODO: in the future TTagExecuteEvent should just get this "first word" as a separate parameter, separated from TagParameters. Also, this word should not be converted by ConvertString.
\end{description}


\end{list}
\ifpdf
\subsection*{\large{\textbf{TTagOptions}}\normalsize\hspace{1ex}\hrulefill}
\else
\subsection*{TTagOptions}
\fi
\label{PasDoc_TagManager-TTagOptions}
\index{TTagOptions}
\begin{list}{}{
\settowidth{\tmplength}{\textbf{Description}}
\setlength{\itemindent}{0cm}
\setlength{\listparindent}{0cm}
\setlength{\leftmargin}{\evensidemargin}
\addtolength{\leftmargin}{\tmplength}
\settowidth{\labelsep}{X}
\addtolength{\leftmargin}{\labelsep}
\setlength{\labelwidth}{\tmplength}
}
\item[\textbf{Declaration}\hfill]
\ifpdf
\begin{flushleft}
\fi
\begin{ttfamily}
TTagOptions = set of TTagOption;\end{ttfamily}

\ifpdf
\end{flushleft}
\fi

\end{list}
\ifpdf
\subsection*{\large{\textbf{TTryAutoLinkEvent}}\normalsize\hspace{1ex}\hrulefill}
\else
\subsection*{TTryAutoLinkEvent}
\fi
\label{PasDoc_TagManager-TTryAutoLinkEvent}
\index{TTryAutoLinkEvent}
\begin{list}{}{
\settowidth{\tmplength}{\textbf{Description}}
\setlength{\itemindent}{0cm}
\setlength{\listparindent}{0cm}
\setlength{\leftmargin}{\evensidemargin}
\addtolength{\leftmargin}{\tmplength}
\settowidth{\labelsep}{X}
\addtolength{\leftmargin}{\labelsep}
\setlength{\labelwidth}{\tmplength}
}
\item[\textbf{Declaration}\hfill]
\ifpdf
\begin{flushleft}
\fi
\begin{ttfamily}
TTryAutoLinkEvent = procedure(TagManager: TTagManager; const QualifiedIdentifier: TNameParts; out QualifiedIdentifierReplacement: string; var AutoLinked: boolean) of object;\end{ttfamily}

\ifpdf
\end{flushleft}
\fi

\end{list}
\chapter{Unit PasDoc{\_}Tipue}
\label{PasDoc_Tipue}
\index{PasDoc{\_}Tipue}
\section{Description}
Helper unit for integrating tipue [\href{http://www.tipue.com/}{http://www.tipue.com/}] with pasdoc HTML output.
\section{Uses}
\begin{itemize}
\item \begin{ttfamily}PasDoc{\_}Utils\end{ttfamily}(\ref{PasDoc_Utils})\item \begin{ttfamily}PasDoc{\_}Items\end{ttfamily}(\ref{PasDoc_Items})\end{itemize}
\section{Overview}
\begin{description}
\item[\texttt{TipueSearchButtonHead}]Put this in {$<$}head{$>$} of every page with search button.
\item[\texttt{TipueSearchButton}]Put this at a place where Tipue button should appear.
\item[\texttt{TipueAddFiles}]Adds some additional files to html documentation, needed for tipue engine.
\end{description}
\section{Functions and Procedures}
\ifpdf
\subsection*{\large{\textbf{TipueSearchButtonHead}}\normalsize\hspace{1ex}\hrulefill}
\else
\subsection*{TipueSearchButtonHead}
\fi
\label{PasDoc_Tipue-TipueSearchButtonHead}
\index{TipueSearchButtonHead}
\begin{list}{}{
\settowidth{\tmplength}{\textbf{Description}}
\setlength{\itemindent}{0cm}
\setlength{\listparindent}{0cm}
\setlength{\leftmargin}{\evensidemargin}
\addtolength{\leftmargin}{\tmplength}
\settowidth{\labelsep}{X}
\addtolength{\leftmargin}{\labelsep}
\setlength{\labelwidth}{\tmplength}
}
\item[\textbf{Declaration}\hfill]
\ifpdf
\begin{flushleft}
\fi
\begin{ttfamily}
function TipueSearchButtonHead: string;\end{ttfamily}

\ifpdf
\end{flushleft}
\fi

\par
\item[\textbf{Description}]
Put this in {$<$}head{$>$} of every page with search button.

\end{list}
\ifpdf
\subsection*{\large{\textbf{TipueSearchButton}}\normalsize\hspace{1ex}\hrulefill}
\else
\subsection*{TipueSearchButton}
\fi
\label{PasDoc_Tipue-TipueSearchButton}
\index{TipueSearchButton}
\begin{list}{}{
\settowidth{\tmplength}{\textbf{Description}}
\setlength{\itemindent}{0cm}
\setlength{\listparindent}{0cm}
\setlength{\leftmargin}{\evensidemargin}
\addtolength{\leftmargin}{\tmplength}
\settowidth{\labelsep}{X}
\addtolength{\leftmargin}{\labelsep}
\setlength{\labelwidth}{\tmplength}
}
\item[\textbf{Declaration}\hfill]
\ifpdf
\begin{flushleft}
\fi
\begin{ttfamily}
function TipueSearchButton: string;\end{ttfamily}

\ifpdf
\end{flushleft}
\fi

\par
\item[\textbf{Description}]
Put this at a place where Tipue button should appear. It will make a form with search button. You will need to use Format to insert the localized word for "Search", e.g.: Format(TipueSearchButton, ['Search']) for English.

\end{list}
\ifpdf
\subsection*{\large{\textbf{TipueAddFiles}}\normalsize\hspace{1ex}\hrulefill}
\else
\subsection*{TipueAddFiles}
\fi
\label{PasDoc_Tipue-TipueAddFiles}
\index{TipueAddFiles}
\begin{list}{}{
\settowidth{\tmplength}{\textbf{Description}}
\setlength{\itemindent}{0cm}
\setlength{\listparindent}{0cm}
\setlength{\leftmargin}{\evensidemargin}
\addtolength{\leftmargin}{\tmplength}
\settowidth{\labelsep}{X}
\addtolength{\leftmargin}{\labelsep}
\setlength{\labelwidth}{\tmplength}
}
\item[\textbf{Declaration}\hfill]
\ifpdf
\begin{flushleft}
\fi
\begin{ttfamily}
procedure TipueAddFiles(Units: TPasUnits; const Introduction, Conclusion: TExternalItem; const AdditionalFiles: TExternalItemList; const Head, BodyBegin, BodyEnd: string; const LanguageCode: string; const OutputPath: string);\end{ttfamily}

\ifpdf
\end{flushleft}
\fi

\par
\item[\textbf{Description}]
Adds some additional files to html documentation, needed for tipue engine.

OutputPath is our output path, where html output must be placed. Must end with PathDelim.

Units must be non{-}nil. It will be used to generate index data for tipue.

\end{list}
\chapter{Unit PasDoc{\_}Tokenizer}
\label{PasDoc_Tokenizer}
\index{PasDoc{\_}Tokenizer}
\section{Description}
Simple Pascal tokenizer.\hfill\vspace*{1ex}

     

The \begin{ttfamily}TTokenizer\end{ttfamily}(\ref{PasDoc_Tokenizer.TTokenizer}) object creates \begin{ttfamily}TToken\end{ttfamily}(\ref{PasDoc_Tokenizer.TToken}) objects (tokens) for the Pascal programming language from a character input stream.

The \begin{ttfamily}PasDoc{\_}Scanner\end{ttfamily}(\ref{PasDoc_Scanner}) unit does the same (it actually uses this unit's tokenizer), with the exception that it evaluates compiler directives, which are comments that start with a dollar sign.
\section{Uses}
\begin{itemize}
\item \begin{ttfamily}Classes\end{ttfamily}\item \begin{ttfamily}PasDoc{\_}Utils\end{ttfamily}(\ref{PasDoc_Utils})\item \begin{ttfamily}PasDoc{\_}Types\end{ttfamily}(\ref{PasDoc_Types})\item \begin{ttfamily}PasDoc{\_}StreamUtils\end{ttfamily}(\ref{PasDoc_StreamUtils})\end{itemize}
\section{Overview}
\begin{description}
\item[\texttt{\begin{ttfamily}TToken\end{ttfamily} Class}]Stores the exact type and additional information on one token.
\item[\texttt{\begin{ttfamily}TTokenizer\end{ttfamily} Class}]Converts an input TStream to a sequence of \begin{ttfamily}TToken\end{ttfamily}(\ref{PasDoc_Tokenizer.TToken}) objects.
\end{description}
\begin{description}
\item[\texttt{StandardDirectiveByName}]Checks is Name (case ignored) some Pascal keyword.
\item[\texttt{KeyWordByName}]Checks is Name (case ignored) some Pascal standard directive.
\end{description}
\section{Classes, Interfaces, Objects and Records}
\ifpdf
\subsection*{\large{\textbf{TToken Class}}\normalsize\hspace{1ex}\hrulefill}
\else
\subsection*{TToken Class}
\fi
\label{PasDoc_Tokenizer.TToken}
\index{TToken}
\subsubsection*{\large{\textbf{Hierarchy}}\normalsize\hspace{1ex}\hfill}
TToken {$>$} TObject
\subsubsection*{\large{\textbf{Description}}\normalsize\hspace{1ex}\hfill}
Stores the exact type and additional information on one token.\subsubsection*{\large{\textbf{Properties}}\normalsize\hspace{1ex}\hfill}
\begin{list}{}{
\settowidth{\tmplength}{\textbf{BeginPosition}}
\setlength{\itemindent}{0cm}
\setlength{\listparindent}{0cm}
\setlength{\leftmargin}{\evensidemargin}
\addtolength{\leftmargin}{\tmplength}
\settowidth{\labelsep}{X}
\addtolength{\leftmargin}{\labelsep}
\setlength{\labelwidth}{\tmplength}
}
\label{PasDoc_Tokenizer.TToken-StreamName}
\index{StreamName}
\item[\textbf{StreamName}\hfill]
\ifpdf
\begin{flushleft}
\fi
\begin{ttfamily}
public property StreamName: string read FStreamName;\end{ttfamily}

\ifpdf
\end{flushleft}
\fi


\par \begin{ttfamily}StreamName\end{ttfamily} is the name of the TStream from which this \begin{ttfamily}TToken\end{ttfamily} was read. It is currently used to set \begin{ttfamily}TRawDescriptionInfo.StreamName\end{ttfamily}(\ref{PasDoc_Items.TRawDescriptionInfo-StreamName}).\label{PasDoc_Tokenizer.TToken-BeginPosition}
\index{BeginPosition}
\item[\textbf{BeginPosition}\hfill]
\ifpdf
\begin{flushleft}
\fi
\begin{ttfamily}
public property BeginPosition: Int64 read FBeginPosition;\end{ttfamily}

\ifpdf
\end{flushleft}
\fi


\par \begin{ttfamily}BeginPosition\end{ttfamily} is the position in the stream of the start of the token. It is currently used to set \begin{ttfamily}TRawDescriptionInfo.BeginPosition\end{ttfamily}(\ref{PasDoc_Items.TRawDescriptionInfo-BeginPosition}).\label{PasDoc_Tokenizer.TToken-EndPosition}
\index{EndPosition}
\item[\textbf{EndPosition}\hfill]
\ifpdf
\begin{flushleft}
\fi
\begin{ttfamily}
public property EndPosition: Int64 read FEndPosition;\end{ttfamily}

\ifpdf
\end{flushleft}
\fi


\par \begin{ttfamily}EndPosition\end{ttfamily} is the position in the stream of the character immediately after the end of the token. It is currently used to set \begin{ttfamily}TRawDescriptionInfo.EndPosition\end{ttfamily}(\ref{PasDoc_Items.TRawDescriptionInfo-EndPosition}).\end{list}
\subsubsection*{\large{\textbf{Fields}}\normalsize\hspace{1ex}\hfill}
\begin{list}{}{
\settowidth{\tmplength}{\textbf{CommentContent}}
\setlength{\itemindent}{0cm}
\setlength{\listparindent}{0cm}
\setlength{\leftmargin}{\evensidemargin}
\addtolength{\leftmargin}{\tmplength}
\settowidth{\labelsep}{X}
\addtolength{\leftmargin}{\labelsep}
\setlength{\labelwidth}{\tmplength}
}
\label{PasDoc_Tokenizer.TToken-Data}
\index{Data}
\item[\textbf{Data}\hfill]
\ifpdf
\begin{flushleft}
\fi
\begin{ttfamily}
public Data: string;\end{ttfamily}

\ifpdf
\end{flushleft}
\fi


\par the exact character representation of this token as it was found in the input file\label{PasDoc_Tokenizer.TToken-MyType}
\index{MyType}
\item[\textbf{MyType}\hfill]
\ifpdf
\begin{flushleft}
\fi
\begin{ttfamily}
public MyType: TTokenType;\end{ttfamily}

\ifpdf
\end{flushleft}
\fi


\par the type of this token as \begin{ttfamily}TTokenType\end{ttfamily}(\ref{PasDoc_Tokenizer-TTokenType})\label{PasDoc_Tokenizer.TToken-Info}
\index{Info}
\item[\textbf{Info}\hfill]
\ifpdf
\begin{flushleft}
\fi
\begin{ttfamily}
public Info: record\end{ttfamily}

\ifpdf
\end{flushleft}
\fi


\par additional information on this token as a variant record depending on the token's MyType\label{PasDoc_Tokenizer.TToken-CommentContent}
\index{CommentContent}
\item[\textbf{CommentContent}\hfill]
\ifpdf
\begin{flushleft}
\fi
\begin{ttfamily}
public CommentContent: string;\end{ttfamily}

\ifpdf
\end{flushleft}
\fi


\par Contents of a comment token. This is defined only when MyType is in TokenCommentTypes or is TOK{\_}DIRECTIVE. This is the text within the comment \textit{without} comment delimiters. For TOK{\_}DIRECTIVE you can safely assume that CommentContent[1] = '{\$}'.\label{PasDoc_Tokenizer.TToken-StringContent}
\index{StringContent}
\item[\textbf{StringContent}\hfill]
\ifpdf
\begin{flushleft}
\fi
\begin{ttfamily}
public StringContent: string;\end{ttfamily}

\ifpdf
\end{flushleft}
\fi


\par Contents of the string token, that is: the value of the string literal. D only when MyType is TOK{\_}STRING.\end{list}
\subsubsection*{\large{\textbf{Methods}}\normalsize\hspace{1ex}\hfill}
\paragraph*{Create}\hspace*{\fill}

\label{PasDoc_Tokenizer.TToken-Create}
\index{Create}
\begin{list}{}{
\settowidth{\tmplength}{\textbf{Description}}
\setlength{\itemindent}{0cm}
\setlength{\listparindent}{0cm}
\setlength{\leftmargin}{\evensidemargin}
\addtolength{\leftmargin}{\tmplength}
\settowidth{\labelsep}{X}
\addtolength{\leftmargin}{\labelsep}
\setlength{\labelwidth}{\tmplength}
}
\item[\textbf{Declaration}\hfill]
\ifpdf
\begin{flushleft}
\fi
\begin{ttfamily}
public constructor Create(const TT: TTokenType);\end{ttfamily}

\ifpdf
\end{flushleft}
\fi

\par
\item[\textbf{Description}]
Create a token of and assign the argument token type to \begin{ttfamily}MyType\end{ttfamily}(\ref{PasDoc_Tokenizer.TToken-MyType})

\end{list}
\paragraph*{GetTypeName}\hspace*{\fill}

\label{PasDoc_Tokenizer.TToken-GetTypeName}
\index{GetTypeName}
\begin{list}{}{
\settowidth{\tmplength}{\textbf{Description}}
\setlength{\itemindent}{0cm}
\setlength{\listparindent}{0cm}
\setlength{\leftmargin}{\evensidemargin}
\addtolength{\leftmargin}{\tmplength}
\settowidth{\labelsep}{X}
\addtolength{\leftmargin}{\labelsep}
\setlength{\labelwidth}{\tmplength}
}
\item[\textbf{Declaration}\hfill]
\ifpdf
\begin{flushleft}
\fi
\begin{ttfamily}
public function GetTypeName: string;\end{ttfamily}

\ifpdf
\end{flushleft}
\fi

\end{list}
\paragraph*{IsSymbol}\hspace*{\fill}

\label{PasDoc_Tokenizer.TToken-IsSymbol}
\index{IsSymbol}
\begin{list}{}{
\settowidth{\tmplength}{\textbf{Description}}
\setlength{\itemindent}{0cm}
\setlength{\listparindent}{0cm}
\setlength{\leftmargin}{\evensidemargin}
\addtolength{\leftmargin}{\tmplength}
\settowidth{\labelsep}{X}
\addtolength{\leftmargin}{\labelsep}
\setlength{\labelwidth}{\tmplength}
}
\item[\textbf{Declaration}\hfill]
\ifpdf
\begin{flushleft}
\fi
\begin{ttfamily}
public function IsSymbol(const ASymbolType: TSymbolType): Boolean;\end{ttfamily}

\ifpdf
\end{flushleft}
\fi

\par
\item[\textbf{Description}]
Does \begin{ttfamily}MyType\end{ttfamily}(\ref{PasDoc_Tokenizer.TToken-MyType}) is TOK{\_}SYMBOL and Info.SymbolType is ASymbolType ?

\end{list}
\paragraph*{IsKeyWord}\hspace*{\fill}

\label{PasDoc_Tokenizer.TToken-IsKeyWord}
\index{IsKeyWord}
\begin{list}{}{
\settowidth{\tmplength}{\textbf{Description}}
\setlength{\itemindent}{0cm}
\setlength{\listparindent}{0cm}
\setlength{\leftmargin}{\evensidemargin}
\addtolength{\leftmargin}{\tmplength}
\settowidth{\labelsep}{X}
\addtolength{\leftmargin}{\labelsep}
\setlength{\labelwidth}{\tmplength}
}
\item[\textbf{Declaration}\hfill]
\ifpdf
\begin{flushleft}
\fi
\begin{ttfamily}
public function IsKeyWord(const AKeyWord: TKeyWord): Boolean;\end{ttfamily}

\ifpdf
\end{flushleft}
\fi

\par
\item[\textbf{Description}]
Does \begin{ttfamily}MyType\end{ttfamily}(\ref{PasDoc_Tokenizer.TToken-MyType}) is TOK{\_}KEYWORD and Info.KeyWord is AKeyWord ?

\end{list}
\paragraph*{IsStandardDirective}\hspace*{\fill}

\label{PasDoc_Tokenizer.TToken-IsStandardDirective}
\index{IsStandardDirective}
\begin{list}{}{
\settowidth{\tmplength}{\textbf{Description}}
\setlength{\itemindent}{0cm}
\setlength{\listparindent}{0cm}
\setlength{\leftmargin}{\evensidemargin}
\addtolength{\leftmargin}{\tmplength}
\settowidth{\labelsep}{X}
\addtolength{\leftmargin}{\labelsep}
\setlength{\labelwidth}{\tmplength}
}
\item[\textbf{Declaration}\hfill]
\ifpdf
\begin{flushleft}
\fi
\begin{ttfamily}
public function IsStandardDirective( const AStandardDirective: TStandardDirective): Boolean;\end{ttfamily}

\ifpdf
\end{flushleft}
\fi

\par
\item[\textbf{Description}]
Does \begin{ttfamily}MyType\end{ttfamily}(\ref{PasDoc_Tokenizer.TToken-MyType}) is TOK{\_}IDENTIFIER and Info.StandardDirective is AStandardDirective ?

\end{list}
\paragraph*{Description}\hspace*{\fill}

\label{PasDoc_Tokenizer.TToken-Description}
\index{Description}
\begin{list}{}{
\settowidth{\tmplength}{\textbf{Description}}
\setlength{\itemindent}{0cm}
\setlength{\listparindent}{0cm}
\setlength{\leftmargin}{\evensidemargin}
\addtolength{\leftmargin}{\tmplength}
\settowidth{\labelsep}{X}
\addtolength{\leftmargin}{\labelsep}
\setlength{\labelwidth}{\tmplength}
}
\item[\textbf{Declaration}\hfill]
\ifpdf
\begin{flushleft}
\fi
\begin{ttfamily}
public function Description: string;\end{ttfamily}

\ifpdf
\end{flushleft}
\fi

\par
\item[\textbf{Description}]
Few words long description of this token. Describes MyType and Data (for those tokens that tend to have short Data). Starts with lower letter.

\end{list}
\ifpdf
\subsection*{\large{\textbf{TTokenizer Class}}\normalsize\hspace{1ex}\hrulefill}
\else
\subsection*{TTokenizer Class}
\fi
\label{PasDoc_Tokenizer.TTokenizer}
\index{TTokenizer}
\subsubsection*{\large{\textbf{Hierarchy}}\normalsize\hspace{1ex}\hfill}
TTokenizer {$>$} TObject
\subsubsection*{\large{\textbf{Description}}\normalsize\hspace{1ex}\hfill}
Converts an input TStream to a sequence of \begin{ttfamily}TToken\end{ttfamily}(\ref{PasDoc_Tokenizer.TToken}) objects.\subsubsection*{\large{\textbf{Properties}}\normalsize\hspace{1ex}\hfill}
\begin{list}{}{
\settowidth{\tmplength}{\textbf{StreamName}}
\setlength{\itemindent}{0cm}
\setlength{\listparindent}{0cm}
\setlength{\leftmargin}{\evensidemargin}
\addtolength{\leftmargin}{\tmplength}
\settowidth{\labelsep}{X}
\addtolength{\leftmargin}{\labelsep}
\setlength{\labelwidth}{\tmplength}
}
\label{PasDoc_Tokenizer.TTokenizer-OnMessage}
\index{OnMessage}
\item[\textbf{OnMessage}\hfill]
\ifpdf
\begin{flushleft}
\fi
\begin{ttfamily}
public property OnMessage: TPasDocMessageEvent read FOnMessage write FOnMessage;\end{ttfamily}

\ifpdf
\end{flushleft}
\fi


\par  \label{PasDoc_Tokenizer.TTokenizer-Verbosity}
\index{Verbosity}
\item[\textbf{Verbosity}\hfill]
\ifpdf
\begin{flushleft}
\fi
\begin{ttfamily}
public property Verbosity: Cardinal read FVerbosity write FVerbosity;\end{ttfamily}

\ifpdf
\end{flushleft}
\fi


\par  \label{PasDoc_Tokenizer.TTokenizer-StreamName}
\index{StreamName}
\item[\textbf{StreamName}\hfill]
\ifpdf
\begin{flushleft}
\fi
\begin{ttfamily}
public property StreamName: string read FStreamName;\end{ttfamily}

\ifpdf
\end{flushleft}
\fi


\par  \label{PasDoc_Tokenizer.TTokenizer-StreamPath}
\index{StreamPath}
\item[\textbf{StreamPath}\hfill]
\ifpdf
\begin{flushleft}
\fi
\begin{ttfamily}
public property StreamPath: string read FStreamPath;\end{ttfamily}

\ifpdf
\end{flushleft}
\fi


\par This is the path where the underlying file of this stream is located.

It may be an absolute path or a relative path. Relative paths are always resolved vs pasdoc current directory. This way user can give relative paths in command{-}line when writing Pascal source filenames to parse.

In particular, this may be '' to indicate current dir.

It's always specified like it was processed by IncludeTrailingPathDelimiter, so it has trailing PathDelim included (unless it was '', in which case it remains empty).\end{list}
\subsubsection*{\large{\textbf{Fields}}\normalsize\hspace{1ex}\hfill}
\begin{list}{}{
\settowidth{\tmplength}{\textbf{IsCharBuffered}}
\setlength{\itemindent}{0cm}
\setlength{\listparindent}{0cm}
\setlength{\leftmargin}{\evensidemargin}
\addtolength{\leftmargin}{\tmplength}
\settowidth{\labelsep}{X}
\addtolength{\leftmargin}{\labelsep}
\setlength{\labelwidth}{\tmplength}
}
\label{PasDoc_Tokenizer.TTokenizer-FOnMessage}
\index{FOnMessage}
\item[\textbf{FOnMessage}\hfill]
\ifpdf
\begin{flushleft}
\fi
\begin{ttfamily}
protected FOnMessage: TPasDocMessageEvent;\end{ttfamily}

\ifpdf
\end{flushleft}
\fi


\par  \label{PasDoc_Tokenizer.TTokenizer-FVerbosity}
\index{FVerbosity}
\item[\textbf{FVerbosity}\hfill]
\ifpdf
\begin{flushleft}
\fi
\begin{ttfamily}
protected FVerbosity: Cardinal;\end{ttfamily}

\ifpdf
\end{flushleft}
\fi


\par  \label{PasDoc_Tokenizer.TTokenizer-BufferedChar}
\index{BufferedChar}
\item[\textbf{BufferedChar}\hfill]
\ifpdf
\begin{flushleft}
\fi
\begin{ttfamily}
protected BufferedChar: Char;\end{ttfamily}

\ifpdf
\end{flushleft}
\fi


\par if \begin{ttfamily}IsCharBuffered\end{ttfamily}(\ref{PasDoc_Tokenizer.TTokenizer-IsCharBuffered}) is true, this field contains the buffered character\label{PasDoc_Tokenizer.TTokenizer-EOS}
\index{EOS}
\item[\textbf{EOS}\hfill]
\ifpdf
\begin{flushleft}
\fi
\begin{ttfamily}
protected EOS: Boolean;\end{ttfamily}

\ifpdf
\end{flushleft}
\fi


\par true if end of stream \begin{ttfamily}Stream\end{ttfamily}(\ref{PasDoc_Tokenizer.TTokenizer-Stream}) has been reached, false otherwise\label{PasDoc_Tokenizer.TTokenizer-IsCharBuffered}
\index{IsCharBuffered}
\item[\textbf{IsCharBuffered}\hfill]
\ifpdf
\begin{flushleft}
\fi
\begin{ttfamily}
protected IsCharBuffered: Boolean;\end{ttfamily}

\ifpdf
\end{flushleft}
\fi


\par if this is true, \begin{ttfamily}BufferedChar\end{ttfamily}(\ref{PasDoc_Tokenizer.TTokenizer-BufferedChar}) contains a buffered character; the next call to \begin{ttfamily}GetChar\end{ttfamily}(\ref{PasDoc_Tokenizer.TTokenizer-GetChar}) or \begin{ttfamily}PeekChar\end{ttfamily}(\ref{PasDoc_Tokenizer.TTokenizer-PeekChar}) will return this character, not the next in the associated stream \begin{ttfamily}Stream\end{ttfamily}(\ref{PasDoc_Tokenizer.TTokenizer-Stream})\label{PasDoc_Tokenizer.TTokenizer-Row}
\index{Row}
\item[\textbf{Row}\hfill]
\ifpdf
\begin{flushleft}
\fi
\begin{ttfamily}
protected Row: Integer;\end{ttfamily}

\ifpdf
\end{flushleft}
\fi


\par current row in stream \begin{ttfamily}Stream\end{ttfamily}(\ref{PasDoc_Tokenizer.TTokenizer-Stream}); useful when giving error messages\label{PasDoc_Tokenizer.TTokenizer-Stream}
\index{Stream}
\item[\textbf{Stream}\hfill]
\ifpdf
\begin{flushleft}
\fi
\begin{ttfamily}
protected Stream: TStream;\end{ttfamily}

\ifpdf
\end{flushleft}
\fi


\par the input stream this tokenizer is working on\label{PasDoc_Tokenizer.TTokenizer-FStreamName}
\index{FStreamName}
\item[\textbf{FStreamName}\hfill]
\ifpdf
\begin{flushleft}
\fi
\begin{ttfamily}
protected FStreamName: string;\end{ttfamily}

\ifpdf
\end{flushleft}
\fi


\par  \label{PasDoc_Tokenizer.TTokenizer-FStreamPath}
\index{FStreamPath}
\item[\textbf{FStreamPath}\hfill]
\ifpdf
\begin{flushleft}
\fi
\begin{ttfamily}
protected FStreamPath: string;\end{ttfamily}

\ifpdf
\end{flushleft}
\fi


\par  \end{list}
\subsubsection*{\large{\textbf{Methods}}\normalsize\hspace{1ex}\hfill}
\paragraph*{DoError}\hspace*{\fill}

\label{PasDoc_Tokenizer.TTokenizer-DoError}
\index{DoError}
\begin{list}{}{
\settowidth{\tmplength}{\textbf{Description}}
\setlength{\itemindent}{0cm}
\setlength{\listparindent}{0cm}
\setlength{\leftmargin}{\evensidemargin}
\addtolength{\leftmargin}{\tmplength}
\settowidth{\labelsep}{X}
\addtolength{\leftmargin}{\labelsep}
\setlength{\labelwidth}{\tmplength}
}
\item[\textbf{Declaration}\hfill]
\ifpdf
\begin{flushleft}
\fi
\begin{ttfamily}
protected procedure DoError(const AMessage: string; const AArguments: array of const);\end{ttfamily}

\ifpdf
\end{flushleft}
\fi

\end{list}
\paragraph*{DoMessage}\hspace*{\fill}

\label{PasDoc_Tokenizer.TTokenizer-DoMessage}
\index{DoMessage}
\begin{list}{}{
\settowidth{\tmplength}{\textbf{Description}}
\setlength{\itemindent}{0cm}
\setlength{\listparindent}{0cm}
\setlength{\leftmargin}{\evensidemargin}
\addtolength{\leftmargin}{\tmplength}
\settowidth{\labelsep}{X}
\addtolength{\leftmargin}{\labelsep}
\setlength{\labelwidth}{\tmplength}
}
\item[\textbf{Declaration}\hfill]
\ifpdf
\begin{flushleft}
\fi
\begin{ttfamily}
protected procedure DoMessage(const AVerbosity: Cardinal; const MessageType: TPasDocMessageType; const AMessage: string; const AArguments: array of const);\end{ttfamily}

\ifpdf
\end{flushleft}
\fi

\end{list}
\paragraph*{CheckForDirective}\hspace*{\fill}

\label{PasDoc_Tokenizer.TTokenizer-CheckForDirective}
\index{CheckForDirective}
\begin{list}{}{
\settowidth{\tmplength}{\textbf{Description}}
\setlength{\itemindent}{0cm}
\setlength{\listparindent}{0cm}
\setlength{\leftmargin}{\evensidemargin}
\addtolength{\leftmargin}{\tmplength}
\settowidth{\labelsep}{X}
\addtolength{\leftmargin}{\labelsep}
\setlength{\labelwidth}{\tmplength}
}
\item[\textbf{Declaration}\hfill]
\ifpdf
\begin{flushleft}
\fi
\begin{ttfamily}
protected procedure CheckForDirective(const t: TToken);\end{ttfamily}

\ifpdf
\end{flushleft}
\fi

\end{list}
\paragraph*{ConsumeChar}\hspace*{\fill}

\label{PasDoc_Tokenizer.TTokenizer-ConsumeChar}
\index{ConsumeChar}
\begin{list}{}{
\settowidth{\tmplength}{\textbf{Description}}
\setlength{\itemindent}{0cm}
\setlength{\listparindent}{0cm}
\setlength{\leftmargin}{\evensidemargin}
\addtolength{\leftmargin}{\tmplength}
\settowidth{\labelsep}{X}
\addtolength{\leftmargin}{\labelsep}
\setlength{\labelwidth}{\tmplength}
}
\item[\textbf{Declaration}\hfill]
\ifpdf
\begin{flushleft}
\fi
\begin{ttfamily}
protected procedure ConsumeChar;\end{ttfamily}

\ifpdf
\end{flushleft}
\fi

\end{list}
\paragraph*{CreateSymbolToken}\hspace*{\fill}

\label{PasDoc_Tokenizer.TTokenizer-CreateSymbolToken}
\index{CreateSymbolToken}
\begin{list}{}{
\settowidth{\tmplength}{\textbf{Description}}
\setlength{\itemindent}{0cm}
\setlength{\listparindent}{0cm}
\setlength{\leftmargin}{\evensidemargin}
\addtolength{\leftmargin}{\tmplength}
\settowidth{\labelsep}{X}
\addtolength{\leftmargin}{\labelsep}
\setlength{\labelwidth}{\tmplength}
}
\item[\textbf{Declaration}\hfill]
\ifpdf
\begin{flushleft}
\fi
\begin{ttfamily}
protected function CreateSymbolToken(const st: TSymbolType; const s: string): TToken; overload;\end{ttfamily}

\ifpdf
\end{flushleft}
\fi

\end{list}
\paragraph*{CreateSymbolToken}\hspace*{\fill}

\label{PasDoc_Tokenizer.TTokenizer-CreateSymbolToken}
\index{CreateSymbolToken}
\begin{list}{}{
\settowidth{\tmplength}{\textbf{Description}}
\setlength{\itemindent}{0cm}
\setlength{\listparindent}{0cm}
\setlength{\leftmargin}{\evensidemargin}
\addtolength{\leftmargin}{\tmplength}
\settowidth{\labelsep}{X}
\addtolength{\leftmargin}{\labelsep}
\setlength{\labelwidth}{\tmplength}
}
\item[\textbf{Declaration}\hfill]
\ifpdf
\begin{flushleft}
\fi
\begin{ttfamily}
protected function CreateSymbolToken(const st: TSymbolType): TToken; overload;\end{ttfamily}

\ifpdf
\end{flushleft}
\fi

\par
\item[\textbf{Description}]
Uses default symbol representation, from SymbolNames[st]

\end{list}
\paragraph*{GetChar}\hspace*{\fill}

\label{PasDoc_Tokenizer.TTokenizer-GetChar}
\index{GetChar}
\begin{list}{}{
\settowidth{\tmplength}{\textbf{Description}}
\setlength{\itemindent}{0cm}
\setlength{\listparindent}{0cm}
\setlength{\leftmargin}{\evensidemargin}
\addtolength{\leftmargin}{\tmplength}
\settowidth{\labelsep}{X}
\addtolength{\leftmargin}{\labelsep}
\setlength{\labelwidth}{\tmplength}
}
\item[\textbf{Declaration}\hfill]
\ifpdf
\begin{flushleft}
\fi
\begin{ttfamily}
protected function GetChar(out c: AnsiChar): Integer;\end{ttfamily}

\ifpdf
\end{flushleft}
\fi

\par
\item[\textbf{Description}]
Returns 1 on success or 0 on failure

\end{list}
\paragraph*{PeekChar}\hspace*{\fill}

\label{PasDoc_Tokenizer.TTokenizer-PeekChar}
\index{PeekChar}
\begin{list}{}{
\settowidth{\tmplength}{\textbf{Description}}
\setlength{\itemindent}{0cm}
\setlength{\listparindent}{0cm}
\setlength{\leftmargin}{\evensidemargin}
\addtolength{\leftmargin}{\tmplength}
\settowidth{\labelsep}{X}
\addtolength{\leftmargin}{\labelsep}
\setlength{\labelwidth}{\tmplength}
}
\item[\textbf{Declaration}\hfill]
\ifpdf
\begin{flushleft}
\fi
\begin{ttfamily}
protected function PeekChar(out c: Char): Boolean;\end{ttfamily}

\ifpdf
\end{flushleft}
\fi

\end{list}
\paragraph*{ReadCommentType1}\hspace*{\fill}

\label{PasDoc_Tokenizer.TTokenizer-ReadCommentType1}
\index{ReadCommentType1}
\begin{list}{}{
\settowidth{\tmplength}{\textbf{Description}}
\setlength{\itemindent}{0cm}
\setlength{\listparindent}{0cm}
\setlength{\leftmargin}{\evensidemargin}
\addtolength{\leftmargin}{\tmplength}
\settowidth{\labelsep}{X}
\addtolength{\leftmargin}{\labelsep}
\setlength{\labelwidth}{\tmplength}
}
\item[\textbf{Declaration}\hfill]
\ifpdf
\begin{flushleft}
\fi
\begin{ttfamily}
protected function ReadCommentType1: TToken;\end{ttfamily}

\ifpdf
\end{flushleft}
\fi

\end{list}
\paragraph*{ReadCommentType2}\hspace*{\fill}

\label{PasDoc_Tokenizer.TTokenizer-ReadCommentType2}
\index{ReadCommentType2}
\begin{list}{}{
\settowidth{\tmplength}{\textbf{Description}}
\setlength{\itemindent}{0cm}
\setlength{\listparindent}{0cm}
\setlength{\leftmargin}{\evensidemargin}
\addtolength{\leftmargin}{\tmplength}
\settowidth{\labelsep}{X}
\addtolength{\leftmargin}{\labelsep}
\setlength{\labelwidth}{\tmplength}
}
\item[\textbf{Declaration}\hfill]
\ifpdf
\begin{flushleft}
\fi
\begin{ttfamily}
protected function ReadCommentType2: TToken;\end{ttfamily}

\ifpdf
\end{flushleft}
\fi

\end{list}
\paragraph*{ReadCommentType3}\hspace*{\fill}

\label{PasDoc_Tokenizer.TTokenizer-ReadCommentType3}
\index{ReadCommentType3}
\begin{list}{}{
\settowidth{\tmplength}{\textbf{Description}}
\setlength{\itemindent}{0cm}
\setlength{\listparindent}{0cm}
\setlength{\leftmargin}{\evensidemargin}
\addtolength{\leftmargin}{\tmplength}
\settowidth{\labelsep}{X}
\addtolength{\leftmargin}{\labelsep}
\setlength{\labelwidth}{\tmplength}
}
\item[\textbf{Declaration}\hfill]
\ifpdf
\begin{flushleft}
\fi
\begin{ttfamily}
protected function ReadCommentType3: TToken;\end{ttfamily}

\ifpdf
\end{flushleft}
\fi

\end{list}
\paragraph*{ReadAttAssemblerRegister}\hspace*{\fill}

\label{PasDoc_Tokenizer.TTokenizer-ReadAttAssemblerRegister}
\index{ReadAttAssemblerRegister}
\begin{list}{}{
\settowidth{\tmplength}{\textbf{Description}}
\setlength{\itemindent}{0cm}
\setlength{\listparindent}{0cm}
\setlength{\leftmargin}{\evensidemargin}
\addtolength{\leftmargin}{\tmplength}
\settowidth{\labelsep}{X}
\addtolength{\leftmargin}{\labelsep}
\setlength{\labelwidth}{\tmplength}
}
\item[\textbf{Declaration}\hfill]
\ifpdf
\begin{flushleft}
\fi
\begin{ttfamily}
protected function ReadAttAssemblerRegister: TToken;\end{ttfamily}

\ifpdf
\end{flushleft}
\fi

\end{list}
\paragraph*{ReadLiteralString}\hspace*{\fill}

\label{PasDoc_Tokenizer.TTokenizer-ReadLiteralString}
\index{ReadLiteralString}
\begin{list}{}{
\settowidth{\tmplength}{\textbf{Description}}
\setlength{\itemindent}{0cm}
\setlength{\listparindent}{0cm}
\setlength{\leftmargin}{\evensidemargin}
\addtolength{\leftmargin}{\tmplength}
\settowidth{\labelsep}{X}
\addtolength{\leftmargin}{\labelsep}
\setlength{\labelwidth}{\tmplength}
}
\item[\textbf{Declaration}\hfill]
\ifpdf
\begin{flushleft}
\fi
\begin{ttfamily}
protected function ReadLiteralString(var t: TToken): Boolean;\end{ttfamily}

\ifpdf
\end{flushleft}
\fi

\end{list}
\paragraph*{ReadToken}\hspace*{\fill}

\label{PasDoc_Tokenizer.TTokenizer-ReadToken}
\index{ReadToken}
\begin{list}{}{
\settowidth{\tmplength}{\textbf{Description}}
\setlength{\itemindent}{0cm}
\setlength{\listparindent}{0cm}
\setlength{\leftmargin}{\evensidemargin}
\addtolength{\leftmargin}{\tmplength}
\settowidth{\labelsep}{X}
\addtolength{\leftmargin}{\labelsep}
\setlength{\labelwidth}{\tmplength}
}
\item[\textbf{Declaration}\hfill]
\ifpdf
\begin{flushleft}
\fi
\begin{ttfamily}
protected function ReadToken(c: Char; const s: TCharSet; const TT: TTokenType; var t: TToken): Boolean;\end{ttfamily}

\ifpdf
\end{flushleft}
\fi

\end{list}
\paragraph*{Create}\hspace*{\fill}

\label{PasDoc_Tokenizer.TTokenizer-Create}
\index{Create}
\begin{list}{}{
\settowidth{\tmplength}{\textbf{Description}}
\setlength{\itemindent}{0cm}
\setlength{\listparindent}{0cm}
\setlength{\leftmargin}{\evensidemargin}
\addtolength{\leftmargin}{\tmplength}
\settowidth{\labelsep}{X}
\addtolength{\leftmargin}{\labelsep}
\setlength{\labelwidth}{\tmplength}
}
\item[\textbf{Declaration}\hfill]
\ifpdf
\begin{flushleft}
\fi
\begin{ttfamily}
public constructor Create( const AStream: TStream; const OnMessageEvent: TPasDocMessageEvent; const VerbosityLevel: Cardinal; const AStreamName, AStreamPath: string);\end{ttfamily}

\ifpdf
\end{flushleft}
\fi

\par
\item[\textbf{Description}]
Creates a TTokenizer and associates it with given input TStream. Note that AStream will be freed when this object will be freed.

\end{list}
\paragraph*{Destroy}\hspace*{\fill}

\label{PasDoc_Tokenizer.TTokenizer-Destroy}
\index{Destroy}
\begin{list}{}{
\settowidth{\tmplength}{\textbf{Description}}
\setlength{\itemindent}{0cm}
\setlength{\listparindent}{0cm}
\setlength{\leftmargin}{\evensidemargin}
\addtolength{\leftmargin}{\tmplength}
\settowidth{\labelsep}{X}
\addtolength{\leftmargin}{\labelsep}
\setlength{\labelwidth}{\tmplength}
}
\item[\textbf{Declaration}\hfill]
\ifpdf
\begin{flushleft}
\fi
\begin{ttfamily}
public destructor Destroy; override;\end{ttfamily}

\ifpdf
\end{flushleft}
\fi

\par
\item[\textbf{Description}]
Releases all dynamically allocated memory.

\end{list}
\paragraph*{HasData}\hspace*{\fill}

\label{PasDoc_Tokenizer.TTokenizer-HasData}
\index{HasData}
\begin{list}{}{
\settowidth{\tmplength}{\textbf{Description}}
\setlength{\itemindent}{0cm}
\setlength{\listparindent}{0cm}
\setlength{\leftmargin}{\evensidemargin}
\addtolength{\leftmargin}{\tmplength}
\settowidth{\labelsep}{X}
\addtolength{\leftmargin}{\labelsep}
\setlength{\labelwidth}{\tmplength}
}
\item[\textbf{Declaration}\hfill]
\ifpdf
\begin{flushleft}
\fi
\begin{ttfamily}
public function HasData: Boolean;\end{ttfamily}

\ifpdf
\end{flushleft}
\fi

\end{list}
\paragraph*{GetStreamInfo}\hspace*{\fill}

\label{PasDoc_Tokenizer.TTokenizer-GetStreamInfo}
\index{GetStreamInfo}
\begin{list}{}{
\settowidth{\tmplength}{\textbf{Description}}
\setlength{\itemindent}{0cm}
\setlength{\listparindent}{0cm}
\setlength{\leftmargin}{\evensidemargin}
\addtolength{\leftmargin}{\tmplength}
\settowidth{\labelsep}{X}
\addtolength{\leftmargin}{\labelsep}
\setlength{\labelwidth}{\tmplength}
}
\item[\textbf{Declaration}\hfill]
\ifpdf
\begin{flushleft}
\fi
\begin{ttfamily}
public function GetStreamInfo: string;\end{ttfamily}

\ifpdf
\end{flushleft}
\fi

\end{list}
\paragraph*{GetToken}\hspace*{\fill}

\label{PasDoc_Tokenizer.TTokenizer-GetToken}
\index{GetToken}
\begin{list}{}{
\settowidth{\tmplength}{\textbf{Description}}
\setlength{\itemindent}{0cm}
\setlength{\listparindent}{0cm}
\setlength{\leftmargin}{\evensidemargin}
\addtolength{\leftmargin}{\tmplength}
\settowidth{\labelsep}{X}
\addtolength{\leftmargin}{\labelsep}
\setlength{\labelwidth}{\tmplength}
}
\item[\textbf{Declaration}\hfill]
\ifpdf
\begin{flushleft}
\fi
\begin{ttfamily}
public function GetToken(const NilOnEnd: Boolean = false): TToken;\end{ttfamily}

\ifpdf
\end{flushleft}
\fi

\end{list}
\paragraph*{UnGetToken}\hspace*{\fill}

\label{PasDoc_Tokenizer.TTokenizer-UnGetToken}
\index{UnGetToken}
\begin{list}{}{
\settowidth{\tmplength}{\textbf{Description}}
\setlength{\itemindent}{0cm}
\setlength{\listparindent}{0cm}
\setlength{\leftmargin}{\evensidemargin}
\addtolength{\leftmargin}{\tmplength}
\settowidth{\labelsep}{X}
\addtolength{\leftmargin}{\labelsep}
\setlength{\labelwidth}{\tmplength}
}
\item[\textbf{Declaration}\hfill]
\ifpdf
\begin{flushleft}
\fi
\begin{ttfamily}
public procedure UnGetToken(var T: TToken);\end{ttfamily}

\ifpdf
\end{flushleft}
\fi

\par
\item[\textbf{Description}]
Makes the token T next to be returned by GetToken. Also sets T to \begin{ttfamily}Nil\end{ttfamily}, to prevent you from freeing it accidentally.

You cannot have more than one "unget" token. If you only call UnGetToken after some GetToken, you are safe.

\end{list}
\paragraph*{SkipUntilCompilerDirective}\hspace*{\fill}

\label{PasDoc_Tokenizer.TTokenizer-SkipUntilCompilerDirective}
\index{SkipUntilCompilerDirective}
\begin{list}{}{
\settowidth{\tmplength}{\textbf{Description}}
\setlength{\itemindent}{0cm}
\setlength{\listparindent}{0cm}
\setlength{\leftmargin}{\evensidemargin}
\addtolength{\leftmargin}{\tmplength}
\settowidth{\labelsep}{X}
\addtolength{\leftmargin}{\labelsep}
\setlength{\labelwidth}{\tmplength}
}
\item[\textbf{Declaration}\hfill]
\ifpdf
\begin{flushleft}
\fi
\begin{ttfamily}
public function SkipUntilCompilerDirective: TToken;\end{ttfamily}

\ifpdf
\end{flushleft}
\fi

\par
\item[\textbf{Description}]
Skip all chars until it encounters some compiler directive, like {\$}ELSE or {\$}ENDIF. Returns either \begin{ttfamily}Nil\end{ttfamily} or a token with MyType = TOK{\_}DIRECTIVE.

\end{list}
\section{Functions and Procedures}
\ifpdf
\subsection*{\large{\textbf{StandardDirectiveByName}}\normalsize\hspace{1ex}\hrulefill}
\else
\subsection*{StandardDirectiveByName}
\fi
\label{PasDoc_Tokenizer-StandardDirectiveByName}
\index{StandardDirectiveByName}
\begin{list}{}{
\settowidth{\tmplength}{\textbf{Description}}
\setlength{\itemindent}{0cm}
\setlength{\listparindent}{0cm}
\setlength{\leftmargin}{\evensidemargin}
\addtolength{\leftmargin}{\tmplength}
\settowidth{\labelsep}{X}
\addtolength{\leftmargin}{\labelsep}
\setlength{\labelwidth}{\tmplength}
}
\item[\textbf{Declaration}\hfill]
\ifpdf
\begin{flushleft}
\fi
\begin{ttfamily}
function StandardDirectiveByName(const Name: string): TStandardDirective;\end{ttfamily}

\ifpdf
\end{flushleft}
\fi

\par
\item[\textbf{Description}]
Checks is Name (case ignored) some Pascal keyword. Returns SD{\_}INVALIDSTANDARDDIRECTIVE if not.

\end{list}
\ifpdf
\subsection*{\large{\textbf{KeyWordByName}}\normalsize\hspace{1ex}\hrulefill}
\else
\subsection*{KeyWordByName}
\fi
\label{PasDoc_Tokenizer-KeyWordByName}
\index{KeyWordByName}
\begin{list}{}{
\settowidth{\tmplength}{\textbf{Description}}
\setlength{\itemindent}{0cm}
\setlength{\listparindent}{0cm}
\setlength{\leftmargin}{\evensidemargin}
\addtolength{\leftmargin}{\tmplength}
\settowidth{\labelsep}{X}
\addtolength{\leftmargin}{\labelsep}
\setlength{\labelwidth}{\tmplength}
}
\item[\textbf{Declaration}\hfill]
\ifpdf
\begin{flushleft}
\fi
\begin{ttfamily}
function KeyWordByName(const Name: string): TKeyword;\end{ttfamily}

\ifpdf
\end{flushleft}
\fi

\par
\item[\textbf{Description}]
Checks is Name (case ignored) some Pascal standard directive. Returns KEY{\_}INVALIDKEYWORD if not.

\end{list}
\section{Types}
\ifpdf
\subsection*{\large{\textbf{TTokenType}}\normalsize\hspace{1ex}\hrulefill}
\else
\subsection*{TTokenType}
\fi
\label{PasDoc_Tokenizer-TTokenType}
\index{TTokenType}
\begin{list}{}{
\settowidth{\tmplength}{\textbf{Description}}
\setlength{\itemindent}{0cm}
\setlength{\listparindent}{0cm}
\setlength{\leftmargin}{\evensidemargin}
\addtolength{\leftmargin}{\tmplength}
\settowidth{\labelsep}{X}
\addtolength{\leftmargin}{\labelsep}
\setlength{\labelwidth}{\tmplength}
}
\item[\textbf{Declaration}\hfill]
\ifpdf
\begin{flushleft}
\fi
\begin{ttfamily}
TTokenType = (...);\end{ttfamily}

\ifpdf
\end{flushleft}
\fi

\par
\item[\textbf{Description}]
enumeration type that provides all types of tokens; each token's name starts with TOK{\_}.

TOK{\_}DIRECTIVE is a compiler directive (like {\$}ifdef, {\$}define).

Note that tokenizer is not able to tell whether you used standard directive (e.g. 'Register') as an identifier (e.g. you're declaring procedure named 'Register') or as a real standard directive (e.g. a calling specifier 'register'). So there is \textit{no} value like TOK{\_}STANDARD{\_}DIRECTIVE here, standard directives are always reported as TOK{\_}IDENTIFIER. You can check TToken.Info.StandardDirective to know whether this identifier is \textit{maybe} used as real standard directive.\item[\textbf{Values}]
\begin{description}
\item[\texttt{TOK{\_}WHITESPACE}] \label{PasDoc_Tokenizer-TOK_WHITESPACE}
\index{}
 
\item[\texttt{TOK{\_}COMMENT{\_}PAS}] \label{PasDoc_Tokenizer-TOK_COMMENT_PAS}
\index{}
 
\item[\texttt{TOK{\_}COMMENT{\_}EXT}] \label{PasDoc_Tokenizer-TOK_COMMENT_EXT}
\index{}
 
\item[\texttt{TOK{\_}COMMENT{\_}HELPINSIGHT}] \label{PasDoc_Tokenizer-TOK_COMMENT_HELPINSIGHT}
\index{}
 
\item[\texttt{TOK{\_}COMMENT{\_}CSTYLE}] \label{PasDoc_Tokenizer-TOK_COMMENT_CSTYLE}
\index{}
 
\item[\texttt{TOK{\_}IDENTIFIER}] \label{PasDoc_Tokenizer-TOK_IDENTIFIER}
\index{}
 
\item[\texttt{TOK{\_}NUMBER}] \label{PasDoc_Tokenizer-TOK_NUMBER}
\index{}
 
\item[\texttt{TOK{\_}STRING}] \label{PasDoc_Tokenizer-TOK_STRING}
\index{}
 
\item[\texttt{TOK{\_}SYMBOL}] \label{PasDoc_Tokenizer-TOK_SYMBOL}
\index{}
 
\item[\texttt{TOK{\_}DIRECTIVE}] \label{PasDoc_Tokenizer-TOK_DIRECTIVE}
\index{}
 
\item[\texttt{TOK{\_}KEYWORD}] \label{PasDoc_Tokenizer-TOK_KEYWORD}
\index{}
 
\item[\texttt{TOK{\_}ATT{\_}ASSEMBLER{\_}REGISTER}] \label{PasDoc_Tokenizer-TOK_ATT_ASSEMBLER_REGISTER}
\index{}
 
\end{description}


\end{list}
\ifpdf
\subsection*{\large{\textbf{TKeyword}}\normalsize\hspace{1ex}\hrulefill}
\else
\subsection*{TKeyword}
\fi
\label{PasDoc_Tokenizer-TKeyword}
\index{TKeyword}
\begin{list}{}{
\settowidth{\tmplength}{\textbf{Description}}
\setlength{\itemindent}{0cm}
\setlength{\listparindent}{0cm}
\setlength{\leftmargin}{\evensidemargin}
\addtolength{\leftmargin}{\tmplength}
\settowidth{\labelsep}{X}
\addtolength{\leftmargin}{\labelsep}
\setlength{\labelwidth}{\tmplength}
}
\item[\textbf{Declaration}\hfill]
\ifpdf
\begin{flushleft}
\fi
\begin{ttfamily}
TKeyword = (...);\end{ttfamily}

\ifpdf
\end{flushleft}
\fi

\par
\item[\textbf{Description}]
 \item[\textbf{Values}]
\begin{description}
\item[\texttt{KEY{\_}INVALIDKEYWORD}] \label{PasDoc_Tokenizer-KEY_INVALIDKEYWORD}
\index{}
 
\item[\texttt{KEY{\_}AND}] \label{PasDoc_Tokenizer-KEY_AND}
\index{}
 
\item[\texttt{KEY{\_}ARRAY}] \label{PasDoc_Tokenizer-KEY_ARRAY}
\index{}
 
\item[\texttt{KEY{\_}AS}] \label{PasDoc_Tokenizer-KEY_AS}
\index{}
 
\item[\texttt{KEY{\_}ASM}] \label{PasDoc_Tokenizer-KEY_ASM}
\index{}
 
\item[\texttt{KEY{\_}BEGIN}] \label{PasDoc_Tokenizer-KEY_BEGIN}
\index{}
 
\item[\texttt{KEY{\_}CASE}] \label{PasDoc_Tokenizer-KEY_CASE}
\index{}
 
\item[\texttt{KEY{\_}CLASS}] \label{PasDoc_Tokenizer-KEY_CLASS}
\index{}
 
\item[\texttt{KEY{\_}CONST}] \label{PasDoc_Tokenizer-KEY_CONST}
\index{}
 
\item[\texttt{KEY{\_}CONSTRUCTOR}] \label{PasDoc_Tokenizer-KEY_CONSTRUCTOR}
\index{}
 
\item[\texttt{KEY{\_}DESTRUCTOR}] \label{PasDoc_Tokenizer-KEY_DESTRUCTOR}
\index{}
 
\item[\texttt{KEY{\_}DISPINTERFACE}] \label{PasDoc_Tokenizer-KEY_DISPINTERFACE}
\index{}
 
\item[\texttt{KEY{\_}DIV}] \label{PasDoc_Tokenizer-KEY_DIV}
\index{}
 
\item[\texttt{KEY{\_}DO}] \label{PasDoc_Tokenizer-KEY_DO}
\index{}
 
\item[\texttt{KEY{\_}DOWNTO}] \label{PasDoc_Tokenizer-KEY_DOWNTO}
\index{}
 
\item[\texttt{KEY{\_}ELSE}] \label{PasDoc_Tokenizer-KEY_ELSE}
\index{}
 
\item[\texttt{KEY{\_}END}] \label{PasDoc_Tokenizer-KEY_END}
\index{}
 
\item[\texttt{KEY{\_}EXCEPT}] \label{PasDoc_Tokenizer-KEY_EXCEPT}
\index{}
 
\item[\texttt{KEY{\_}EXPORTS}] \label{PasDoc_Tokenizer-KEY_EXPORTS}
\index{}
 
\item[\texttt{KEY{\_}FILE}] \label{PasDoc_Tokenizer-KEY_FILE}
\index{}
 
\item[\texttt{KEY{\_}FINALIZATION}] \label{PasDoc_Tokenizer-KEY_FINALIZATION}
\index{}
 
\item[\texttt{KEY{\_}FINALLY}] \label{PasDoc_Tokenizer-KEY_FINALLY}
\index{}
 
\item[\texttt{KEY{\_}FOR}] \label{PasDoc_Tokenizer-KEY_FOR}
\index{}
 
\item[\texttt{KEY{\_}FUNCTION}] \label{PasDoc_Tokenizer-KEY_FUNCTION}
\index{}
 
\item[\texttt{KEY{\_}GOTO}] \label{PasDoc_Tokenizer-KEY_GOTO}
\index{}
 
\item[\texttt{KEY{\_}IF}] \label{PasDoc_Tokenizer-KEY_IF}
\index{}
 
\item[\texttt{KEY{\_}IMPLEMENTATION}] \label{PasDoc_Tokenizer-KEY_IMPLEMENTATION}
\index{}
 
\item[\texttt{KEY{\_}IN}] \label{PasDoc_Tokenizer-KEY_IN}
\index{}
 
\item[\texttt{KEY{\_}INHERITED}] \label{PasDoc_Tokenizer-KEY_INHERITED}
\index{}
 
\item[\texttt{KEY{\_}INITIALIZATION}] \label{PasDoc_Tokenizer-KEY_INITIALIZATION}
\index{}
 
\item[\texttt{KEY{\_}INLINE}] \label{PasDoc_Tokenizer-KEY_INLINE}
\index{}
 
\item[\texttt{KEY{\_}INTERFACE}] \label{PasDoc_Tokenizer-KEY_INTERFACE}
\index{}
 
\item[\texttt{KEY{\_}IS}] \label{PasDoc_Tokenizer-KEY_IS}
\index{}
 
\item[\texttt{KEY{\_}LABEL}] \label{PasDoc_Tokenizer-KEY_LABEL}
\index{}
 
\item[\texttt{KEY{\_}LIBRARY}] \label{PasDoc_Tokenizer-KEY_LIBRARY}
\index{}
 
\item[\texttt{KEY{\_}MOD}] \label{PasDoc_Tokenizer-KEY_MOD}
\index{}
 
\item[\texttt{KEY{\_}NIL}] \label{PasDoc_Tokenizer-KEY_NIL}
\index{}
 
\item[\texttt{KEY{\_}NOT}] \label{PasDoc_Tokenizer-KEY_NOT}
\index{}
 
\item[\texttt{KEY{\_}OBJECT}] \label{PasDoc_Tokenizer-KEY_OBJECT}
\index{}
 
\item[\texttt{KEY{\_}OF}] \label{PasDoc_Tokenizer-KEY_OF}
\index{}
 
\item[\texttt{KEY{\_}ON}] \label{PasDoc_Tokenizer-KEY_ON}
\index{}
 
\item[\texttt{KEY{\_}OR}] \label{PasDoc_Tokenizer-KEY_OR}
\index{}
 
\item[\texttt{KEY{\_}PACKED}] \label{PasDoc_Tokenizer-KEY_PACKED}
\index{}
 
\item[\texttt{KEY{\_}PROCEDURE}] \label{PasDoc_Tokenizer-KEY_PROCEDURE}
\index{}
 
\item[\texttt{KEY{\_}PROGRAM}] \label{PasDoc_Tokenizer-KEY_PROGRAM}
\index{}
 
\item[\texttt{KEY{\_}PROPERTY}] \label{PasDoc_Tokenizer-KEY_PROPERTY}
\index{}
 
\item[\texttt{KEY{\_}RAISE}] \label{PasDoc_Tokenizer-KEY_RAISE}
\index{}
 
\item[\texttt{KEY{\_}RECORD}] \label{PasDoc_Tokenizer-KEY_RECORD}
\index{}
 
\item[\texttt{KEY{\_}REPEAT}] \label{PasDoc_Tokenizer-KEY_REPEAT}
\index{}
 
\item[\texttt{KEY{\_}RESOURCESTRING}] \label{PasDoc_Tokenizer-KEY_RESOURCESTRING}
\index{}
 
\item[\texttt{KEY{\_}SET}] \label{PasDoc_Tokenizer-KEY_SET}
\index{}
 
\item[\texttt{KEY{\_}SHL}] \label{PasDoc_Tokenizer-KEY_SHL}
\index{}
 
\item[\texttt{KEY{\_}SHR}] \label{PasDoc_Tokenizer-KEY_SHR}
\index{}
 
\item[\texttt{KEY{\_}STRING}] \label{PasDoc_Tokenizer-KEY_STRING}
\index{}
 
\item[\texttt{KEY{\_}THEN}] \label{PasDoc_Tokenizer-KEY_THEN}
\index{}
 
\item[\texttt{KEY{\_}THREADVAR}] \label{PasDoc_Tokenizer-KEY_THREADVAR}
\index{}
 
\item[\texttt{KEY{\_}TO}] \label{PasDoc_Tokenizer-KEY_TO}
\index{}
 
\item[\texttt{KEY{\_}TRY}] \label{PasDoc_Tokenizer-KEY_TRY}
\index{}
 
\item[\texttt{KEY{\_}TYPE}] \label{PasDoc_Tokenizer-KEY_TYPE}
\index{}
 
\item[\texttt{KEY{\_}UNIT}] \label{PasDoc_Tokenizer-KEY_UNIT}
\index{}
 
\item[\texttt{KEY{\_}UNTIL}] \label{PasDoc_Tokenizer-KEY_UNTIL}
\index{}
 
\item[\texttt{KEY{\_}USES}] \label{PasDoc_Tokenizer-KEY_USES}
\index{}
 
\item[\texttt{KEY{\_}VAR}] \label{PasDoc_Tokenizer-KEY_VAR}
\index{}
 
\item[\texttt{KEY{\_}WHILE}] \label{PasDoc_Tokenizer-KEY_WHILE}
\index{}
 
\item[\texttt{KEY{\_}WITH}] \label{PasDoc_Tokenizer-KEY_WITH}
\index{}
 
\item[\texttt{KEY{\_}XOR}] \label{PasDoc_Tokenizer-KEY_XOR}
\index{}
 
\end{description}


\end{list}
\ifpdf
\subsection*{\large{\textbf{TStandardDirective}}\normalsize\hspace{1ex}\hrulefill}
\else
\subsection*{TStandardDirective}
\fi
\label{PasDoc_Tokenizer-TStandardDirective}
\index{TStandardDirective}
\begin{list}{}{
\settowidth{\tmplength}{\textbf{Description}}
\setlength{\itemindent}{0cm}
\setlength{\listparindent}{0cm}
\setlength{\leftmargin}{\evensidemargin}
\addtolength{\leftmargin}{\tmplength}
\settowidth{\labelsep}{X}
\addtolength{\leftmargin}{\labelsep}
\setlength{\labelwidth}{\tmplength}
}
\item[\textbf{Declaration}\hfill]
\ifpdf
\begin{flushleft}
\fi
\begin{ttfamily}
TStandardDirective = (...);\end{ttfamily}

\ifpdf
\end{flushleft}
\fi

\par
\item[\textbf{Description}]
 \item[\textbf{Values}]
\begin{description}
\item[\texttt{SD{\_}INVALIDSTANDARDDIRECTIVE}] \label{PasDoc_Tokenizer-SD_INVALIDSTANDARDDIRECTIVE}
\index{}
 
\item[\texttt{SD{\_}ABSOLUTE}] \label{PasDoc_Tokenizer-SD_ABSOLUTE}
\index{}
 
\item[\texttt{SD{\_}ABSTRACT}] \label{PasDoc_Tokenizer-SD_ABSTRACT}
\index{}
 
\item[\texttt{SD{\_}APIENTRY}] \label{PasDoc_Tokenizer-SD_APIENTRY}
\index{}
 
\item[\texttt{SD{\_}ASSEMBLER}] \label{PasDoc_Tokenizer-SD_ASSEMBLER}
\index{}
 
\item[\texttt{SD{\_}AUTOMATED}] \label{PasDoc_Tokenizer-SD_AUTOMATED}
\index{}
 
\item[\texttt{SD{\_}CDECL}] \label{PasDoc_Tokenizer-SD_CDECL}
\index{}
 
\item[\texttt{SD{\_}CVAR}] \label{PasDoc_Tokenizer-SD_CVAR}
\index{}
 
\item[\texttt{SD{\_}DEFAULT}] \label{PasDoc_Tokenizer-SD_DEFAULT}
\index{}
 
\item[\texttt{SD{\_}DISPID}] \label{PasDoc_Tokenizer-SD_DISPID}
\index{}
 
\item[\texttt{SD{\_}DYNAMIC}] \label{PasDoc_Tokenizer-SD_DYNAMIC}
\index{}
 
\item[\texttt{SD{\_}EXPERIMENTAL}] \label{PasDoc_Tokenizer-SD_EXPERIMENTAL}
\index{}
 
\item[\texttt{SD{\_}EXPORT}] \label{PasDoc_Tokenizer-SD_EXPORT}
\index{}
 
\item[\texttt{SD{\_}EXTERNAL}] \label{PasDoc_Tokenizer-SD_EXTERNAL}
\index{}
 
\item[\texttt{SD{\_}FAR}] \label{PasDoc_Tokenizer-SD_FAR}
\index{}
 
\item[\texttt{SD{\_}FORWARD}] \label{PasDoc_Tokenizer-SD_FORWARD}
\index{}
 
\item[\texttt{SD{\_}GENERIC}] \label{PasDoc_Tokenizer-SD_GENERIC}
\index{}
 
\item[\texttt{SD{\_}HELPER}] \label{PasDoc_Tokenizer-SD_HELPER}
\index{}
 
\item[\texttt{SD{\_}INDEX}] \label{PasDoc_Tokenizer-SD_INDEX}
\index{}
 
\item[\texttt{SD{\_}INLINE}] \label{PasDoc_Tokenizer-SD_INLINE}
\index{}
 
\item[\texttt{SD{\_}MESSAGE}] \label{PasDoc_Tokenizer-SD_MESSAGE}
\index{}
 
\item[\texttt{SD{\_}NAME}] \label{PasDoc_Tokenizer-SD_NAME}
\index{}
 
\item[\texttt{SD{\_}NEAR}] \label{PasDoc_Tokenizer-SD_NEAR}
\index{}
 
\item[\texttt{SD{\_}NODEFAULT}] \label{PasDoc_Tokenizer-SD_NODEFAULT}
\index{}
 
\item[\texttt{SD{\_}OPERATOR}] \label{PasDoc_Tokenizer-SD_OPERATOR}
\index{}
 
\item[\texttt{SD{\_}OUT}] \label{PasDoc_Tokenizer-SD_OUT}
\index{}
 
\item[\texttt{SD{\_}OVERLOAD}] \label{PasDoc_Tokenizer-SD_OVERLOAD}
\index{}
 
\item[\texttt{SD{\_}OVERRIDE}] \label{PasDoc_Tokenizer-SD_OVERRIDE}
\index{}
 
\item[\texttt{SD{\_}PASCAL}] \label{PasDoc_Tokenizer-SD_PASCAL}
\index{}
 
\item[\texttt{SD{\_}PRIVATE}] \label{PasDoc_Tokenizer-SD_PRIVATE}
\index{}
 
\item[\texttt{SD{\_}PROTECTED}] \label{PasDoc_Tokenizer-SD_PROTECTED}
\index{}
 
\item[\texttt{SD{\_}PUBLIC}] \label{PasDoc_Tokenizer-SD_PUBLIC}
\index{}
 
\item[\texttt{SD{\_}PUBLISHED}] \label{PasDoc_Tokenizer-SD_PUBLISHED}
\index{}
 
\item[\texttt{SD{\_}READ}] \label{PasDoc_Tokenizer-SD_READ}
\index{}
 
\item[\texttt{SD{\_}REFERENCE}] \label{PasDoc_Tokenizer-SD_REFERENCE}
\index{}
 
\item[\texttt{SD{\_}REGISTER}] \label{PasDoc_Tokenizer-SD_REGISTER}
\index{}
 
\item[\texttt{SD{\_}REINTRODUCE}] \label{PasDoc_Tokenizer-SD_REINTRODUCE}
\index{}
 
\item[\texttt{SD{\_}RESIDENT}] \label{PasDoc_Tokenizer-SD_RESIDENT}
\index{}
 
\item[\texttt{SD{\_}SEALED}] \label{PasDoc_Tokenizer-SD_SEALED}
\index{}
 
\item[\texttt{SD{\_}SPECIALIZE}] \label{PasDoc_Tokenizer-SD_SPECIALIZE}
\index{}
 
\item[\texttt{SD{\_}STATIC}] \label{PasDoc_Tokenizer-SD_STATIC}
\index{}
 
\item[\texttt{SD{\_}STDCALL}] \label{PasDoc_Tokenizer-SD_STDCALL}
\index{}
 
\item[\texttt{SD{\_}STORED}] \label{PasDoc_Tokenizer-SD_STORED}
\index{}
 
\item[\texttt{SD{\_}STRICT}] \label{PasDoc_Tokenizer-SD_STRICT}
\index{}
 
\item[\texttt{SD{\_}VIRTUAL}] \label{PasDoc_Tokenizer-SD_VIRTUAL}
\index{}
 
\item[\texttt{SD{\_}WRITE}] \label{PasDoc_Tokenizer-SD_WRITE}
\index{}
 
\item[\texttt{SD{\_}DEPRECATED}] \label{PasDoc_Tokenizer-SD_DEPRECATED}
\index{}
 
\item[\texttt{SD{\_}SAFECALL}] \label{PasDoc_Tokenizer-SD_SAFECALL}
\index{}
 
\item[\texttt{SD{\_}PLATFORM}] \label{PasDoc_Tokenizer-SD_PLATFORM}
\index{}
 
\item[\texttt{SD{\_}VARARGS}] \label{PasDoc_Tokenizer-SD_VARARGS}
\index{}
 
\item[\texttt{SD{\_}FINAL}] \label{PasDoc_Tokenizer-SD_FINAL}
\index{}
 
\end{description}


\end{list}
\ifpdf
\subsection*{\large{\textbf{TStandardDirectives}}\normalsize\hspace{1ex}\hrulefill}
\else
\subsection*{TStandardDirectives}
\fi
\label{PasDoc_Tokenizer-TStandardDirectives}
\index{TStandardDirectives}
\begin{list}{}{
\settowidth{\tmplength}{\textbf{Description}}
\setlength{\itemindent}{0cm}
\setlength{\listparindent}{0cm}
\setlength{\leftmargin}{\evensidemargin}
\addtolength{\leftmargin}{\tmplength}
\settowidth{\labelsep}{X}
\addtolength{\leftmargin}{\labelsep}
\setlength{\labelwidth}{\tmplength}
}
\item[\textbf{Declaration}\hfill]
\ifpdf
\begin{flushleft}
\fi
\begin{ttfamily}
TStandardDirectives = set of TStandardDirective;\end{ttfamily}

\ifpdf
\end{flushleft}
\fi

\end{list}
\ifpdf
\subsection*{\large{\textbf{TSymbolType}}\normalsize\hspace{1ex}\hrulefill}
\else
\subsection*{TSymbolType}
\fi
\label{PasDoc_Tokenizer-TSymbolType}
\index{TSymbolType}
\begin{list}{}{
\settowidth{\tmplength}{\textbf{Description}}
\setlength{\itemindent}{0cm}
\setlength{\listparindent}{0cm}
\setlength{\leftmargin}{\evensidemargin}
\addtolength{\leftmargin}{\tmplength}
\settowidth{\labelsep}{X}
\addtolength{\leftmargin}{\labelsep}
\setlength{\labelwidth}{\tmplength}
}
\item[\textbf{Declaration}\hfill]
\ifpdf
\begin{flushleft}
\fi
\begin{ttfamily}
TSymbolType = (...);\end{ttfamily}

\ifpdf
\end{flushleft}
\fi

\par
\item[\textbf{Description}]
enumeration type that provides all types of symbols; each symbol's name starts with SYM{\_}\item[\textbf{Values}]
\begin{description}
\item[\texttt{SYM{\_}PLUS}] \label{PasDoc_Tokenizer-SYM_PLUS}
\index{}
 
\item[\texttt{SYM{\_}MINUS}] \label{PasDoc_Tokenizer-SYM_MINUS}
\index{}
 
\item[\texttt{SYM{\_}ASTERISK}] \label{PasDoc_Tokenizer-SYM_ASTERISK}
\index{}
 
\item[\texttt{SYM{\_}SLASH}] \label{PasDoc_Tokenizer-SYM_SLASH}
\index{}
 
\item[\texttt{SYM{\_}EQUAL}] \label{PasDoc_Tokenizer-SYM_EQUAL}
\index{}
 
\item[\texttt{SYM{\_}LESS{\_}THAN}] \label{PasDoc_Tokenizer-SYM_LESS_THAN}
\index{}
 
\item[\texttt{SYM{\_}LESS{\_}THAN{\_}EQUAL}] \label{PasDoc_Tokenizer-SYM_LESS_THAN_EQUAL}
\index{}
 
\item[\texttt{SYM{\_}GREATER{\_}THAN}] \label{PasDoc_Tokenizer-SYM_GREATER_THAN}
\index{}
 
\item[\texttt{SYM{\_}GREATER{\_}THAN{\_}EQUAL}] \label{PasDoc_Tokenizer-SYM_GREATER_THAN_EQUAL}
\index{}
 
\item[\texttt{SYM{\_}LEFT{\_}BRACKET}] \label{PasDoc_Tokenizer-SYM_LEFT_BRACKET}
\index{}
 
\item[\texttt{SYM{\_}RIGHT{\_}BRACKET}] \label{PasDoc_Tokenizer-SYM_RIGHT_BRACKET}
\index{}
 
\item[\texttt{SYM{\_}COMMA}] \label{PasDoc_Tokenizer-SYM_COMMA}
\index{}
 
\item[\texttt{SYM{\_}LEFT{\_}PARENTHESIS}] \label{PasDoc_Tokenizer-SYM_LEFT_PARENTHESIS}
\index{}
 
\item[\texttt{SYM{\_}RIGHT{\_}PARENTHESIS}] \label{PasDoc_Tokenizer-SYM_RIGHT_PARENTHESIS}
\index{}
 
\item[\texttt{SYM{\_}COLON}] \label{PasDoc_Tokenizer-SYM_COLON}
\index{}
 
\item[\texttt{SYM{\_}SEMICOLON}] \label{PasDoc_Tokenizer-SYM_SEMICOLON}
\index{}
 
\item[\texttt{SYM{\_}DEREFERENCE}] \label{PasDoc_Tokenizer-SYM_DEREFERENCE}
\index{}
 
\item[\texttt{SYM{\_}PERIOD}] \label{PasDoc_Tokenizer-SYM_PERIOD}
\index{}
 
\item[\texttt{SYM{\_}AT}] \label{PasDoc_Tokenizer-SYM_AT}
\index{}
 
\item[\texttt{SYM{\_}DOLLAR}] \label{PasDoc_Tokenizer-SYM_DOLLAR}
\index{}
 
\item[\texttt{SYM{\_}ASSIGN}] \label{PasDoc_Tokenizer-SYM_ASSIGN}
\index{}
 
\item[\texttt{SYM{\_}RANGE}] \label{PasDoc_Tokenizer-SYM_RANGE}
\index{}
 
\item[\texttt{SYM{\_}POWER}] \label{PasDoc_Tokenizer-SYM_POWER}
\index{}
 
\item[\texttt{SYM{\_}BACKSLASH}] \label{PasDoc_Tokenizer-SYM_BACKSLASH}
\index{}
SYM{\_}BACKSLASH may occur when writing char constant "{\^{}}{\textbackslash}", see ../../tests/ok{\_}caret{\_}character.pas
\end{description}


\end{list}
\section{Constants}
\ifpdf
\subsection*{\large{\textbf{TOKEN{\_}TYPE{\_}NAMES}}\normalsize\hspace{1ex}\hrulefill}
\else
\subsection*{TOKEN{\_}TYPE{\_}NAMES}
\fi
\label{PasDoc_Tokenizer-TOKEN_TYPE_NAMES}
\index{TOKEN{\_}TYPE{\_}NAMES}
\begin{list}{}{
\settowidth{\tmplength}{\textbf{Description}}
\setlength{\itemindent}{0cm}
\setlength{\listparindent}{0cm}
\setlength{\leftmargin}{\evensidemargin}
\addtolength{\leftmargin}{\tmplength}
\settowidth{\labelsep}{X}
\addtolength{\leftmargin}{\labelsep}
\setlength{\labelwidth}{\tmplength}
}
\item[\textbf{Declaration}\hfill]
\ifpdf
\begin{flushleft}
\fi
\begin{ttfamily}
TOKEN{\_}TYPE{\_}NAMES: array[TTokenType] of string =
  ( 'whitespace', 'comment ((**)-style)', 'comment ({\{}{\}}-style)',
    'comment (///-style)',
    'comment (//-style)', 'identifier', 'number', 'string', 'symbol',
    'directive', 'reserved word', 'AT{\&}T assembler register name');\end{ttfamily}

\ifpdf
\end{flushleft}
\fi

\par
\item[\textbf{Description}]
Names of the token types. All start with lower letter. They should somehow describe (in a few short words) given TTokenType.

\end{list}
\ifpdf
\subsection*{\large{\textbf{TokenCommentTypes}}\normalsize\hspace{1ex}\hrulefill}
\else
\subsection*{TokenCommentTypes}
\fi
\label{PasDoc_Tokenizer-TokenCommentTypes}
\index{TokenCommentTypes}
\begin{list}{}{
\settowidth{\tmplength}{\textbf{Description}}
\setlength{\itemindent}{0cm}
\setlength{\listparindent}{0cm}
\setlength{\leftmargin}{\evensidemargin}
\addtolength{\leftmargin}{\tmplength}
\settowidth{\labelsep}{X}
\addtolength{\leftmargin}{\labelsep}
\setlength{\labelwidth}{\tmplength}
}
\item[\textbf{Declaration}\hfill]
\ifpdf
\begin{flushleft}
\fi
\begin{ttfamily}
TokenCommentTypes: set of TTokenType =
  [ TOK{\_}COMMENT{\_}PAS, TOK{\_}COMMENT{\_}EXT,
  TOK{\_}COMMENT{\_}HELPINSIGHT,
  TOK{\_}COMMENT{\_}CSTYLE ];\end{ttfamily}

\ifpdf
\end{flushleft}
\fi

\end{list}
\ifpdf
\subsection*{\large{\textbf{SymbolNames}}\normalsize\hspace{1ex}\hrulefill}
\else
\subsection*{SymbolNames}
\fi
\label{PasDoc_Tokenizer-SymbolNames}
\index{SymbolNames}
\begin{list}{}{
\settowidth{\tmplength}{\textbf{Description}}
\setlength{\itemindent}{0cm}
\setlength{\listparindent}{0cm}
\setlength{\leftmargin}{\evensidemargin}
\addtolength{\leftmargin}{\tmplength}
\settowidth{\labelsep}{X}
\addtolength{\leftmargin}{\labelsep}
\setlength{\labelwidth}{\tmplength}
}
\item[\textbf{Declaration}\hfill]
\ifpdf
\begin{flushleft}
\fi
\begin{ttfamily}
SymbolNames: array[TSymbolType] of string =
  ( '+', '-', '*', '/', '=', '{$<$}', '{$<$}=', '{$>$}', '{$>$}=', '[', ']', ',',
    '(', ')', ':', ';', '{\^{}}', '.', '@', '{\$}', ':=', '..', '**', '{\textbackslash}' );\end{ttfamily}

\ifpdf
\end{flushleft}
\fi

\par
\item[\textbf{Description}]
Symbols as strings. They can be useful to have some mapping TSymbolType {-}{$>$} string, but remember that actually some symbols in tokenizer have multiple possible representations, e.g. "right bracket" is usually given as "]" but can also be written as ".)".

\end{list}
\ifpdf
\subsection*{\large{\textbf{KeyWordArray}}\normalsize\hspace{1ex}\hrulefill}
\else
\subsection*{KeyWordArray}
\fi
\label{PasDoc_Tokenizer-KeyWordArray}
\index{KeyWordArray}
\begin{list}{}{
\settowidth{\tmplength}{\textbf{Description}}
\setlength{\itemindent}{0cm}
\setlength{\listparindent}{0cm}
\setlength{\leftmargin}{\evensidemargin}
\addtolength{\leftmargin}{\tmplength}
\settowidth{\labelsep}{X}
\addtolength{\leftmargin}{\labelsep}
\setlength{\labelwidth}{\tmplength}
}
\item[\textbf{Declaration}\hfill]
\ifpdf
\begin{flushleft}
\fi
\begin{ttfamily}
KeyWordArray: array[Low(TKeyword)..High(TKeyword)] of string =
  ('x', 
    'AND', 'ARRAY', 'AS', 'ASM', 'BEGIN', 'CASE', 'CLASS', 'CONST',
    'CONSTRUCTOR', 'DESTRUCTOR', 'DISPINTERFACE', 'DIV',  'DO', 'DOWNTO',
    'ELSE', 'END', 'EXCEPT', 'EXPORTS', 'FILE', 'FINALIZATION',
    'FINALLY', 'FOR', 'FUNCTION', 'GOTO', 'IF', 'IMPLEMENTATION',
    'IN', 'INHERITED', 'INITIALIZATION', 'INLINE', 'INTERFACE',
    'IS', 'LABEL', 'LIBRARY', 'MOD', 'NIL', 'NOT', 'OBJECT', 'OF',
    'ON', 'OR', 'PACKED', 'PROCEDURE', 'PROGRAM', 'PROPERTY',
    'RAISE', 'RECORD', 'REPEAT', 'RESOURCESTRING', 'SET', 'SHL',
    'SHR', 'STRING', 'THEN', 'THREADVAR', 'TO', 'TRY', 'TYPE',
    'UNIT', 'UNTIL', 'USES', 'VAR', 'WHILE', 'WITH', 'XOR');\end{ttfamily}

\ifpdf
\end{flushleft}
\fi

\par
\item[\textbf{Description}]
all Object Pascal keywords

\end{list}
\ifpdf
\subsection*{\large{\textbf{StandardDirectiveArray}}\normalsize\hspace{1ex}\hrulefill}
\else
\subsection*{StandardDirectiveArray}
\fi
\label{PasDoc_Tokenizer-StandardDirectiveArray}
\index{StandardDirectiveArray}
\begin{list}{}{
\settowidth{\tmplength}{\textbf{Description}}
\setlength{\itemindent}{0cm}
\setlength{\listparindent}{0cm}
\setlength{\leftmargin}{\evensidemargin}
\addtolength{\leftmargin}{\tmplength}
\settowidth{\labelsep}{X}
\addtolength{\leftmargin}{\labelsep}
\setlength{\labelwidth}{\tmplength}
}
\item[\textbf{Declaration}\hfill]
\ifpdf
\begin{flushleft}
\fi
\begin{ttfamily}
StandardDirectiveArray:
    array[Low(TStandardDirective)..High(TStandardDirective)] of PChar =
  ('x', 
    'ABSOLUTE', 'ABSTRACT', 'APIENTRY', 'ASSEMBLER', 'AUTOMATED',
    'CDECL', 'CVAR', 'DEFAULT', 'DISPID', 'DYNAMIC', 'EXPERIMENTAL', 'EXPORT', 'EXTERNAL',
    'FAR', 'FORWARD', 'GENERIC', 'HELPER', 'INDEX', 'INLINE', 'MESSAGE', 'NAME', 'NEAR',
    'NODEFAULT', 'OPERATOR', 'OUT', 'OVERLOAD', 'OVERRIDE', 'PASCAL', 'PRIVATE',
    'PROTECTED', 'PUBLIC', 'PUBLISHED', 'READ', 'REFERENCE', 'REGISTER',
    'REINTRODUCE', 'RESIDENT', 'SEALED', 'SPECIALIZE', 'STATIC',
    'STDCALL', 'STORED', 'STRICT', 'VIRTUAL',
    'WRITE', 'DEPRECATED', 'SAFECALL', 'PLATFORM', 'VARARGS', 'FINAL');\end{ttfamily}

\ifpdf
\end{flushleft}
\fi

\par
\item[\textbf{Description}]
Object Pascal directives

\end{list}
\section{Authors}
\par
Johannes Berg {$<$}johannes@sipsolutions.de{$>$}

\par
Ralf Junker (delphi@zeitungsjunge.de)

\par
Marco Schmidt (marcoschmidt@geocities.com)

\par
Michalis Kamburelis

\par
Arno Garrels {$<$}first name.name@nospamgmx.de{$>$}

\chapter{Unit PasDoc{\_}Types}
\label{PasDoc_Types}
\index{PasDoc{\_}Types}
\section{Description}
Basic types.\hfill\vspace*{1ex}

   
\section{Uses}
\begin{itemize}
\item \begin{ttfamily}SysUtils\end{ttfamily}\item \begin{ttfamily}StrUtils\end{ttfamily}\item \begin{ttfamily}Types\end{ttfamily}\end{itemize}
\section{Overview}
\begin{description}
\item[\texttt{\begin{ttfamily}EPasDoc\end{ttfamily} Class}]
\end{description}
\begin{description}
\item[\texttt{SplitNameParts}]Splits S, which can be made of any number of parts, separated by dots (Delphi namespaces, like PasDoc.Output.HTML.TWriter.Write).
\item[\texttt{OneNamePart}]Simply returns an array with Length = 1 and one item = S.
\item[\texttt{GlueNameParts}]Simply concatenates all NameParts with dot.
\end{description}
\section{Classes, Interfaces, Objects and Records}
\ifpdf
\subsection*{\large{\textbf{EPasDoc Class}}\normalsize\hspace{1ex}\hrulefill}
\else
\subsection*{EPasDoc Class}
\fi
\label{PasDoc_Types.EPasDoc}
\index{EPasDoc}
\subsubsection*{\large{\textbf{Hierarchy}}\normalsize\hspace{1ex}\hfill}
EPasDoc {$>$} Exception
%%%%Description
\subsubsection*{\large{\textbf{Methods}}\normalsize\hspace{1ex}\hfill}
\paragraph*{Create}\hspace*{\fill}

\label{PasDoc_Types.EPasDoc-Create}
\index{Create}
\begin{list}{}{
\settowidth{\tmplength}{\textbf{Description}}
\setlength{\itemindent}{0cm}
\setlength{\listparindent}{0cm}
\setlength{\leftmargin}{\evensidemargin}
\addtolength{\leftmargin}{\tmplength}
\settowidth{\labelsep}{X}
\addtolength{\leftmargin}{\labelsep}
\setlength{\labelwidth}{\tmplength}
}
\item[\textbf{Declaration}\hfill]
\ifpdf
\begin{flushleft}
\fi
\begin{ttfamily}
public constructor Create(const AMessage: string; const AArguments: array of const; const AExitCode: Word = 3);\end{ttfamily}

\ifpdf
\end{flushleft}
\fi

\end{list}
\section{Functions and Procedures}
\ifpdf
\subsection*{\large{\textbf{SplitNameParts}}\normalsize\hspace{1ex}\hrulefill}
\else
\subsection*{SplitNameParts}
\fi
\label{PasDoc_Types-SplitNameParts}
\index{SplitNameParts}
\begin{list}{}{
\settowidth{\tmplength}{\textbf{Description}}
\setlength{\itemindent}{0cm}
\setlength{\listparindent}{0cm}
\setlength{\leftmargin}{\evensidemargin}
\addtolength{\leftmargin}{\tmplength}
\settowidth{\labelsep}{X}
\addtolength{\leftmargin}{\labelsep}
\setlength{\labelwidth}{\tmplength}
}
\item[\textbf{Declaration}\hfill]
\ifpdf
\begin{flushleft}
\fi
\begin{ttfamily}
function SplitNameParts(S: string; out NameParts: TNameParts): Boolean;\end{ttfamily}

\ifpdf
\end{flushleft}
\fi

\par
\item[\textbf{Description}]
Splits S, which can be made of any number of parts, separated by dots (Delphi namespaces, like PasDoc.Output.HTML.TWriter.Write). If S is not a valid identifier, \begin{ttfamily}False\end{ttfamily} is returned, otherwise \begin{ttfamily}True\end{ttfamily} is returned and splitted name is returned as NameParts.

\end{list}
\ifpdf
\subsection*{\large{\textbf{OneNamePart}}\normalsize\hspace{1ex}\hrulefill}
\else
\subsection*{OneNamePart}
\fi
\label{PasDoc_Types-OneNamePart}
\index{OneNamePart}
\begin{list}{}{
\settowidth{\tmplength}{\textbf{Description}}
\setlength{\itemindent}{0cm}
\setlength{\listparindent}{0cm}
\setlength{\leftmargin}{\evensidemargin}
\addtolength{\leftmargin}{\tmplength}
\settowidth{\labelsep}{X}
\addtolength{\leftmargin}{\labelsep}
\setlength{\labelwidth}{\tmplength}
}
\item[\textbf{Declaration}\hfill]
\ifpdf
\begin{flushleft}
\fi
\begin{ttfamily}
function OneNamePart(const S: string): TNameParts;\end{ttfamily}

\ifpdf
\end{flushleft}
\fi

\par
\item[\textbf{Description}]
Simply returns an array with Length = 1 and one item = S.

\end{list}
\ifpdf
\subsection*{\large{\textbf{GlueNameParts}}\normalsize\hspace{1ex}\hrulefill}
\else
\subsection*{GlueNameParts}
\fi
\label{PasDoc_Types-GlueNameParts}
\index{GlueNameParts}
\begin{list}{}{
\settowidth{\tmplength}{\textbf{Description}}
\setlength{\itemindent}{0cm}
\setlength{\listparindent}{0cm}
\setlength{\leftmargin}{\evensidemargin}
\addtolength{\leftmargin}{\tmplength}
\settowidth{\labelsep}{X}
\addtolength{\leftmargin}{\labelsep}
\setlength{\labelwidth}{\tmplength}
}
\item[\textbf{Declaration}\hfill]
\ifpdf
\begin{flushleft}
\fi
\begin{ttfamily}
function GlueNameParts(const NameParts: TNameParts): string;\end{ttfamily}

\ifpdf
\end{flushleft}
\fi

\par
\item[\textbf{Description}]
Simply concatenates all NameParts with dot.

\end{list}
\section{Types}
\ifpdf
\subsection*{\large{\textbf{TBytes}}\normalsize\hspace{1ex}\hrulefill}
\else
\subsection*{TBytes}
\fi
\label{PasDoc_Types-TBytes}
\index{TBytes}
\begin{list}{}{
\settowidth{\tmplength}{\textbf{Description}}
\setlength{\itemindent}{0cm}
\setlength{\listparindent}{0cm}
\setlength{\leftmargin}{\evensidemargin}
\addtolength{\leftmargin}{\tmplength}
\settowidth{\labelsep}{X}
\addtolength{\leftmargin}{\labelsep}
\setlength{\labelwidth}{\tmplength}
}
\item[\textbf{Declaration}\hfill]
\ifpdf
\begin{flushleft}
\fi
\begin{ttfamily}
TBytes = array of Byte;\end{ttfamily}

\ifpdf
\end{flushleft}
\fi

\end{list}
\ifpdf
\subsection*{\large{\textbf{UnicodeString}}\normalsize\hspace{1ex}\hrulefill}
\else
\subsection*{UnicodeString}
\fi
\label{PasDoc_Types-UnicodeString}
\index{UnicodeString}
\begin{list}{}{
\settowidth{\tmplength}{\textbf{Description}}
\setlength{\itemindent}{0cm}
\setlength{\listparindent}{0cm}
\setlength{\leftmargin}{\evensidemargin}
\addtolength{\leftmargin}{\tmplength}
\settowidth{\labelsep}{X}
\addtolength{\leftmargin}{\labelsep}
\setlength{\labelwidth}{\tmplength}
}
\item[\textbf{Declaration}\hfill]
\ifpdf
\begin{flushleft}
\fi
\begin{ttfamily}
UnicodeString = WideString;\end{ttfamily}

\ifpdf
\end{flushleft}
\fi

\end{list}
\ifpdf
\subsection*{\large{\textbf{RawByteString}}\normalsize\hspace{1ex}\hrulefill}
\else
\subsection*{RawByteString}
\fi
\label{PasDoc_Types-RawByteString}
\index{RawByteString}
\begin{list}{}{
\settowidth{\tmplength}{\textbf{Description}}
\setlength{\itemindent}{0cm}
\setlength{\listparindent}{0cm}
\setlength{\leftmargin}{\evensidemargin}
\addtolength{\leftmargin}{\tmplength}
\settowidth{\labelsep}{X}
\addtolength{\leftmargin}{\labelsep}
\setlength{\labelwidth}{\tmplength}
}
\item[\textbf{Declaration}\hfill]
\ifpdf
\begin{flushleft}
\fi
\begin{ttfamily}
RawByteString = AnsiString;\end{ttfamily}

\ifpdf
\end{flushleft}
\fi

\end{list}
\ifpdf
\subsection*{\large{\textbf{TStringArray}}\normalsize\hspace{1ex}\hrulefill}
\else
\subsection*{TStringArray}
\fi
\label{PasDoc_Types-TStringArray}
\index{TStringArray}
\begin{list}{}{
\settowidth{\tmplength}{\textbf{Description}}
\setlength{\itemindent}{0cm}
\setlength{\listparindent}{0cm}
\setlength{\leftmargin}{\evensidemargin}
\addtolength{\leftmargin}{\tmplength}
\settowidth{\labelsep}{X}
\addtolength{\leftmargin}{\labelsep}
\setlength{\labelwidth}{\tmplength}
}
\item[\textbf{Declaration}\hfill]
\ifpdf
\begin{flushleft}
\fi
\begin{ttfamily}
TStringArray = TStringDynArray;\end{ttfamily}

\ifpdf
\end{flushleft}
\fi

\end{list}
\ifpdf
\subsection*{\large{\textbf{TNameParts}}\normalsize\hspace{1ex}\hrulefill}
\else
\subsection*{TNameParts}
\fi
\label{PasDoc_Types-TNameParts}
\index{TNameParts}
\begin{list}{}{
\settowidth{\tmplength}{\textbf{Description}}
\setlength{\itemindent}{0cm}
\setlength{\listparindent}{0cm}
\setlength{\leftmargin}{\evensidemargin}
\addtolength{\leftmargin}{\tmplength}
\settowidth{\labelsep}{X}
\addtolength{\leftmargin}{\labelsep}
\setlength{\labelwidth}{\tmplength}
}
\item[\textbf{Declaration}\hfill]
\ifpdf
\begin{flushleft}
\fi
\begin{ttfamily}
TNameParts = TStringArray;\end{ttfamily}

\ifpdf
\end{flushleft}
\fi

\par
\item[\textbf{Description}]
This represents parts of a qualified name of some item.

User supplies such name by separating each part with dot, e.g. 'UnitName.ClassName.ProcedureName', then \begin{ttfamily}SplitNameParts\end{ttfamily}(\ref{PasDoc_Types-SplitNameParts}) converts it to TNameParts like ['UnitName', 'ClassName', 'ProcedureName']. Length must be \textit{always} between 1 and \begin{ttfamily}MaxNameParts\end{ttfamily}(\ref{PasDoc_Types-MaxNameParts}).

\end{list}
\ifpdf
\subsection*{\large{\textbf{TPasDocMessageType}}\normalsize\hspace{1ex}\hrulefill}
\else
\subsection*{TPasDocMessageType}
\fi
\label{PasDoc_Types-TPasDocMessageType}
\index{TPasDocMessageType}
\begin{list}{}{
\settowidth{\tmplength}{\textbf{Description}}
\setlength{\itemindent}{0cm}
\setlength{\listparindent}{0cm}
\setlength{\leftmargin}{\evensidemargin}
\addtolength{\leftmargin}{\tmplength}
\settowidth{\labelsep}{X}
\addtolength{\leftmargin}{\labelsep}
\setlength{\labelwidth}{\tmplength}
}
\item[\textbf{Declaration}\hfill]
\ifpdf
\begin{flushleft}
\fi
\begin{ttfamily}
TPasDocMessageType = (...);\end{ttfamily}

\ifpdf
\end{flushleft}
\fi

\par
\item[\textbf{Description}]
 \item[\textbf{Values}]
\begin{description}
\item[\texttt{pmtPlainText}] \label{PasDoc_Types-pmtPlainText}
\index{}
 
\item[\texttt{pmtInformation}] \label{PasDoc_Types-pmtInformation}
\index{}
 
\item[\texttt{pmtWarning}] \label{PasDoc_Types-pmtWarning}
\index{}
 
\item[\texttt{pmtError}] \label{PasDoc_Types-pmtError}
\index{}
 
\end{description}


\end{list}
\ifpdf
\subsection*{\large{\textbf{TPasDocMessageEvent}}\normalsize\hspace{1ex}\hrulefill}
\else
\subsection*{TPasDocMessageEvent}
\fi
\label{PasDoc_Types-TPasDocMessageEvent}
\index{TPasDocMessageEvent}
\begin{list}{}{
\settowidth{\tmplength}{\textbf{Description}}
\setlength{\itemindent}{0cm}
\setlength{\listparindent}{0cm}
\setlength{\leftmargin}{\evensidemargin}
\addtolength{\leftmargin}{\tmplength}
\settowidth{\labelsep}{X}
\addtolength{\leftmargin}{\labelsep}
\setlength{\labelwidth}{\tmplength}
}
\item[\textbf{Declaration}\hfill]
\ifpdf
\begin{flushleft}
\fi
\begin{ttfamily}
TPasDocMessageEvent = procedure(const MessageType: TPasDocMessageType; const AMessage: string; const AVerbosity: Cardinal) of object;\end{ttfamily}

\ifpdf
\end{flushleft}
\fi

\end{list}
\ifpdf
\subsection*{\large{\textbf{TCharSet}}\normalsize\hspace{1ex}\hrulefill}
\else
\subsection*{TCharSet}
\fi
\label{PasDoc_Types-TCharSet}
\index{TCharSet}
\begin{list}{}{
\settowidth{\tmplength}{\textbf{Description}}
\setlength{\itemindent}{0cm}
\setlength{\listparindent}{0cm}
\setlength{\leftmargin}{\evensidemargin}
\addtolength{\leftmargin}{\tmplength}
\settowidth{\labelsep}{X}
\addtolength{\leftmargin}{\labelsep}
\setlength{\labelwidth}{\tmplength}
}
\item[\textbf{Declaration}\hfill]
\ifpdf
\begin{flushleft}
\fi
\begin{ttfamily}
TCharSet = set of AnsiChar;\end{ttfamily}

\ifpdf
\end{flushleft}
\fi

\end{list}
\ifpdf
\subsection*{\large{\textbf{TImplicitVisibility}}\normalsize\hspace{1ex}\hrulefill}
\else
\subsection*{TImplicitVisibility}
\fi
\label{PasDoc_Types-TImplicitVisibility}
\index{TImplicitVisibility}
\begin{list}{}{
\settowidth{\tmplength}{\textbf{Description}}
\setlength{\itemindent}{0cm}
\setlength{\listparindent}{0cm}
\setlength{\leftmargin}{\evensidemargin}
\addtolength{\leftmargin}{\tmplength}
\settowidth{\labelsep}{X}
\addtolength{\leftmargin}{\labelsep}
\setlength{\labelwidth}{\tmplength}
}
\item[\textbf{Declaration}\hfill]
\ifpdf
\begin{flushleft}
\fi
\begin{ttfamily}
TImplicitVisibility = (...);\end{ttfamily}

\ifpdf
\end{flushleft}
\fi

\par
\item[\textbf{Description}]
See command{-}line option {-}{-}implicit{-}visibility documentation at [\href{https://github.com/pasdoc/pasdoc/wiki/ImplicitVisibilityOption}{https://github.com/pasdoc/pasdoc/wiki/ImplicitVisibilityOption}]\item[\textbf{Values}]
\begin{description}
\item[\texttt{ivPublic}] \label{PasDoc_Types-ivPublic}
\index{}
 
\item[\texttt{ivPublished}] \label{PasDoc_Types-ivPublished}
\index{}
 
\item[\texttt{ivImplicit}] \label{PasDoc_Types-ivImplicit}
\index{}
 
\end{description}


\end{list}
\section{Constants}
\ifpdf
\subsection*{\large{\textbf{MaxNameParts}}\normalsize\hspace{1ex}\hrulefill}
\else
\subsection*{MaxNameParts}
\fi
\label{PasDoc_Types-MaxNameParts}
\index{MaxNameParts}
\begin{list}{}{
\settowidth{\tmplength}{\textbf{Description}}
\setlength{\itemindent}{0cm}
\setlength{\listparindent}{0cm}
\setlength{\leftmargin}{\evensidemargin}
\addtolength{\leftmargin}{\tmplength}
\settowidth{\labelsep}{X}
\addtolength{\leftmargin}{\labelsep}
\setlength{\labelwidth}{\tmplength}
}
\item[\textbf{Declaration}\hfill]
\ifpdf
\begin{flushleft}
\fi
\begin{ttfamily}
MaxNameParts = 3;\end{ttfamily}

\ifpdf
\end{flushleft}
\fi

\end{list}
\ifpdf
\subsection*{\large{\textbf{CP{\_}UTF16}}\normalsize\hspace{1ex}\hrulefill}
\else
\subsection*{CP{\_}UTF16}
\fi
\label{PasDoc_Types-CP_UTF16}
\index{CP{\_}UTF16}
\begin{list}{}{
\settowidth{\tmplength}{\textbf{Description}}
\setlength{\itemindent}{0cm}
\setlength{\listparindent}{0cm}
\setlength{\leftmargin}{\evensidemargin}
\addtolength{\leftmargin}{\tmplength}
\settowidth{\labelsep}{X}
\addtolength{\leftmargin}{\labelsep}
\setlength{\labelwidth}{\tmplength}
}
\item[\textbf{Declaration}\hfill]
\ifpdf
\begin{flushleft}
\fi
\begin{ttfamily}
CP{\_}UTF16      = 1200;\end{ttfamily}

\ifpdf
\end{flushleft}
\fi

\par
\item[\textbf{Description}]
Windows Unicode code page ID

\end{list}
\ifpdf
\subsection*{\large{\textbf{CP{\_}UTF16Be}}\normalsize\hspace{1ex}\hrulefill}
\else
\subsection*{CP{\_}UTF16Be}
\fi
\label{PasDoc_Types-CP_UTF16Be}
\index{CP{\_}UTF16Be}
\begin{list}{}{
\settowidth{\tmplength}{\textbf{Description}}
\setlength{\itemindent}{0cm}
\setlength{\listparindent}{0cm}
\setlength{\leftmargin}{\evensidemargin}
\addtolength{\leftmargin}{\tmplength}
\settowidth{\labelsep}{X}
\addtolength{\leftmargin}{\labelsep}
\setlength{\labelwidth}{\tmplength}
}
\item[\textbf{Declaration}\hfill]
\ifpdf
\begin{flushleft}
\fi
\begin{ttfamily}
CP{\_}UTF16Be    = 1201;\end{ttfamily}

\ifpdf
\end{flushleft}
\fi

\end{list}
\ifpdf
\subsection*{\large{\textbf{CP{\_}UTF32}}\normalsize\hspace{1ex}\hrulefill}
\else
\subsection*{CP{\_}UTF32}
\fi
\label{PasDoc_Types-CP_UTF32}
\index{CP{\_}UTF32}
\begin{list}{}{
\settowidth{\tmplength}{\textbf{Description}}
\setlength{\itemindent}{0cm}
\setlength{\listparindent}{0cm}
\setlength{\leftmargin}{\evensidemargin}
\addtolength{\leftmargin}{\tmplength}
\settowidth{\labelsep}{X}
\addtolength{\leftmargin}{\labelsep}
\setlength{\labelwidth}{\tmplength}
}
\item[\textbf{Declaration}\hfill]
\ifpdf
\begin{flushleft}
\fi
\begin{ttfamily}
CP{\_}UTF32      = 12000;\end{ttfamily}

\ifpdf
\end{flushleft}
\fi

\end{list}
\ifpdf
\subsection*{\large{\textbf{CP{\_}UTF32Be}}\normalsize\hspace{1ex}\hrulefill}
\else
\subsection*{CP{\_}UTF32Be}
\fi
\label{PasDoc_Types-CP_UTF32Be}
\index{CP{\_}UTF32Be}
\begin{list}{}{
\settowidth{\tmplength}{\textbf{Description}}
\setlength{\itemindent}{0cm}
\setlength{\listparindent}{0cm}
\setlength{\leftmargin}{\evensidemargin}
\addtolength{\leftmargin}{\tmplength}
\settowidth{\labelsep}{X}
\addtolength{\leftmargin}{\labelsep}
\setlength{\labelwidth}{\tmplength}
}
\item[\textbf{Declaration}\hfill]
\ifpdf
\begin{flushleft}
\fi
\begin{ttfamily}
CP{\_}UTF32Be    = 12001;\end{ttfamily}

\ifpdf
\end{flushleft}
\fi

\end{list}
\section{Authors}
\par
Johannes Berg {$<$}johannes@sipsolutions.de{$>$}

\par
Michalis Kamburelis

\par
Arno Garrels {$<$}first name.name@nospamgmx.de{$>$}

\chapter{Unit PasDoc{\_}Utils}
\label{PasDoc_Utils}
\index{PasDoc{\_}Utils}
\section{Description}
Utility functions.\hfill\vspace*{1ex}

   
\section{Uses}
\begin{itemize}
\item \begin{ttfamily}SysUtils\end{ttfamily}\item \begin{ttfamily}PasDoc{\_}Types\end{ttfamily}(\ref{PasDoc_Types})\end{itemize}
\section{Overview}
\begin{description}
\item[\texttt{\begin{ttfamily}TCharReplacement\end{ttfamily} Record}]
\end{description}
\begin{description}
\item[\texttt{IsStrEmptyA}]string empty means it contains only whitespace
\item[\texttt{StrCountCharA}]count occurences of AChar in AString
\item[\texttt{StrPosIA}]Position of the ASub in AString.
\item[\texttt{MakeMethod}]creates a "method pointer"
\item[\texttt{StringReplaceChars}]Returns S with each char from ReplacementArray[].cChar replaced with ReplacementArray[].sSpec.
\item[\texttt{SCharIs}]Comfortable shortcut for Index {$<$}= Length(S) and S[Index] = C.
\item[\texttt{SCharIs}]Comfortable shortcut for Index {$<$}= Length(S) and S[Index] in Chars.
\item[\texttt{ExtractFirstWord}]Extracts all characters up to the first white{-}space encountered (ignoring white{-}space at the very beginning of the string) from the string specified by S.
\item[\texttt{ExtractFirstWord}]Another version of ExtractFirstWord.
\item[\texttt{FileToString}]
\item[\texttt{StringToFile}]
\item[\texttt{DataToFile}]
\item[\texttt{SCharsReplace}]Returns S with all Chars replaced by ReplacementChar
\item[\texttt{CopyFile}]
\item[\texttt{IsPrefix}]Checks is Prefix a prefix of S.
\item[\texttt{RemovePrefix}]If IsPrefix(Prefix, S), then remove the prefix, otherwise return unmodifed S.
\item[\texttt{SEnding}]SEnding returns S contents starting from position P.
\item[\texttt{IsPathAbsolute}]Check is the given Path absolute.
\item[\texttt{IsPathAbsoluteOnDrive}]Just like IsPathAbsolute, but on Windows accepts also paths that specify full directory tree without drive letter.
\item[\texttt{CombinePaths}]Combines BasePath with RelPath.
\item[\texttt{DeleteFileExt}]Remove from the FileName the last extension (including the dot).
\item[\texttt{RemoveIndentation}]Remove common indentation (whitespace prefix) from a multiline string.
\item[\texttt{Swap16Buf}]
\item[\texttt{IsCharInSet}]
\item[\texttt{IsCharInSet}]
\item[\texttt{IsUtf8LeadByte}]
\item[\texttt{IsUtf8TrailByte}]
\item[\texttt{Utf8Size}]
\item[\texttt{IsLeadChar}]
\item[\texttt{StripHtml}]Strip HTML elements from the string.
\item[\texttt{SAppendPart}]If S = '' then returns NextPart, else returns S + PartSeparator + NextPart.
\end{description}
\section{Classes, Interfaces, Objects and Records}
\ifpdf
\subsection*{\large{\textbf{TCharReplacement Record}}\normalsize\hspace{1ex}\hrulefill}
\else
\subsection*{TCharReplacement Record}
\fi
\label{PasDoc_Utils.TCharReplacement}
\index{TCharReplacement}
%%%%Description
\subsubsection*{\large{\textbf{Fields}}\normalsize\hspace{1ex}\hfill}
\begin{list}{}{
\settowidth{\tmplength}{\textbf{cChar}}
\setlength{\itemindent}{0cm}
\setlength{\listparindent}{0cm}
\setlength{\leftmargin}{\evensidemargin}
\addtolength{\leftmargin}{\tmplength}
\settowidth{\labelsep}{X}
\addtolength{\leftmargin}{\labelsep}
\setlength{\labelwidth}{\tmplength}
}
\label{PasDoc_Utils.TCharReplacement-cChar}
\index{cChar}
\item[\textbf{cChar}\hfill]
\ifpdf
\begin{flushleft}
\fi
\begin{ttfamily}
public cChar: Char;\end{ttfamily}

\ifpdf
\end{flushleft}
\fi


\par  \label{PasDoc_Utils.TCharReplacement-sSpec}
\index{sSpec}
\item[\textbf{sSpec}\hfill]
\ifpdf
\begin{flushleft}
\fi
\begin{ttfamily}
public sSpec: string;\end{ttfamily}

\ifpdf
\end{flushleft}
\fi


\par  \end{list}
\section{Functions and Procedures}
\ifpdf
\subsection*{\large{\textbf{IsStrEmptyA}}\normalsize\hspace{1ex}\hrulefill}
\else
\subsection*{IsStrEmptyA}
\fi
\label{PasDoc_Utils-IsStrEmptyA}
\index{IsStrEmptyA}
\begin{list}{}{
\settowidth{\tmplength}{\textbf{Description}}
\setlength{\itemindent}{0cm}
\setlength{\listparindent}{0cm}
\setlength{\leftmargin}{\evensidemargin}
\addtolength{\leftmargin}{\tmplength}
\settowidth{\labelsep}{X}
\addtolength{\leftmargin}{\labelsep}
\setlength{\labelwidth}{\tmplength}
}
\item[\textbf{Declaration}\hfill]
\ifpdf
\begin{flushleft}
\fi
\begin{ttfamily}
function IsStrEmptyA(const AString: string): boolean;\end{ttfamily}

\ifpdf
\end{flushleft}
\fi

\par
\item[\textbf{Description}]
string empty means it contains only whitespace

\end{list}
\ifpdf
\subsection*{\large{\textbf{StrCountCharA}}\normalsize\hspace{1ex}\hrulefill}
\else
\subsection*{StrCountCharA}
\fi
\label{PasDoc_Utils-StrCountCharA}
\index{StrCountCharA}
\begin{list}{}{
\settowidth{\tmplength}{\textbf{Description}}
\setlength{\itemindent}{0cm}
\setlength{\listparindent}{0cm}
\setlength{\leftmargin}{\evensidemargin}
\addtolength{\leftmargin}{\tmplength}
\settowidth{\labelsep}{X}
\addtolength{\leftmargin}{\labelsep}
\setlength{\labelwidth}{\tmplength}
}
\item[\textbf{Declaration}\hfill]
\ifpdf
\begin{flushleft}
\fi
\begin{ttfamily}
function StrCountCharA(const AString: string; const AChar: Char): Integer;\end{ttfamily}

\ifpdf
\end{flushleft}
\fi

\par
\item[\textbf{Description}]
count occurences of AChar in AString

\end{list}
\ifpdf
\subsection*{\large{\textbf{StrPosIA}}\normalsize\hspace{1ex}\hrulefill}
\else
\subsection*{StrPosIA}
\fi
\label{PasDoc_Utils-StrPosIA}
\index{StrPosIA}
\begin{list}{}{
\settowidth{\tmplength}{\textbf{Description}}
\setlength{\itemindent}{0cm}
\setlength{\listparindent}{0cm}
\setlength{\leftmargin}{\evensidemargin}
\addtolength{\leftmargin}{\tmplength}
\settowidth{\labelsep}{X}
\addtolength{\leftmargin}{\labelsep}
\setlength{\labelwidth}{\tmplength}
}
\item[\textbf{Declaration}\hfill]
\ifpdf
\begin{flushleft}
\fi
\begin{ttfamily}
function StrPosIA(const ASub, AString: string): Integer;\end{ttfamily}

\ifpdf
\end{flushleft}
\fi

\par
\item[\textbf{Description}]
Position of the ASub in AString. Return 0 if not found

\end{list}
\ifpdf
\subsection*{\large{\textbf{MakeMethod}}\normalsize\hspace{1ex}\hrulefill}
\else
\subsection*{MakeMethod}
\fi
\label{PasDoc_Utils-MakeMethod}
\index{MakeMethod}
\begin{list}{}{
\settowidth{\tmplength}{\textbf{Description}}
\setlength{\itemindent}{0cm}
\setlength{\listparindent}{0cm}
\setlength{\leftmargin}{\evensidemargin}
\addtolength{\leftmargin}{\tmplength}
\settowidth{\labelsep}{X}
\addtolength{\leftmargin}{\labelsep}
\setlength{\labelwidth}{\tmplength}
}
\item[\textbf{Declaration}\hfill]
\ifpdf
\begin{flushleft}
\fi
\begin{ttfamily}
function MakeMethod(const AObject: Pointer; AMethod: Pointer): TMethod;\end{ttfamily}

\ifpdf
\end{flushleft}
\fi

\par
\item[\textbf{Description}]
creates a "method pointer"

\end{list}
\ifpdf
\subsection*{\large{\textbf{StringReplaceChars}}\normalsize\hspace{1ex}\hrulefill}
\else
\subsection*{StringReplaceChars}
\fi
\label{PasDoc_Utils-StringReplaceChars}
\index{StringReplaceChars}
\begin{list}{}{
\settowidth{\tmplength}{\textbf{Description}}
\setlength{\itemindent}{0cm}
\setlength{\listparindent}{0cm}
\setlength{\leftmargin}{\evensidemargin}
\addtolength{\leftmargin}{\tmplength}
\settowidth{\labelsep}{X}
\addtolength{\leftmargin}{\labelsep}
\setlength{\labelwidth}{\tmplength}
}
\item[\textbf{Declaration}\hfill]
\ifpdf
\begin{flushleft}
\fi
\begin{ttfamily}
function StringReplaceChars(const S: string; const ReplacementArray: array of TCharReplacement): string;\end{ttfamily}

\ifpdf
\end{flushleft}
\fi

\par
\item[\textbf{Description}]
Returns S with each char from ReplacementArray[].cChar replaced with ReplacementArray[].sSpec.

\end{list}
\ifpdf
\subsection*{\large{\textbf{SCharIs}}\normalsize\hspace{1ex}\hrulefill}
\else
\subsection*{SCharIs}
\fi
\label{PasDoc_Utils-SCharIs}
\index{SCharIs}
\begin{list}{}{
\settowidth{\tmplength}{\textbf{Description}}
\setlength{\itemindent}{0cm}
\setlength{\listparindent}{0cm}
\setlength{\leftmargin}{\evensidemargin}
\addtolength{\leftmargin}{\tmplength}
\settowidth{\labelsep}{X}
\addtolength{\leftmargin}{\labelsep}
\setlength{\labelwidth}{\tmplength}
}
\item[\textbf{Declaration}\hfill]
\ifpdf
\begin{flushleft}
\fi
\begin{ttfamily}
function SCharIs(const S: string; Index: integer; C: char): boolean; overload;\end{ttfamily}

\ifpdf
\end{flushleft}
\fi

\par
\item[\textbf{Description}]
Comfortable shortcut for Index {$<$}= Length(S) and S[Index] = C.

\end{list}
\ifpdf
\subsection*{\large{\textbf{SCharIs}}\normalsize\hspace{1ex}\hrulefill}
\else
\subsection*{SCharIs}
\fi
\label{PasDoc_Utils-SCharIs}
\index{SCharIs}
\begin{list}{}{
\settowidth{\tmplength}{\textbf{Description}}
\setlength{\itemindent}{0cm}
\setlength{\listparindent}{0cm}
\setlength{\leftmargin}{\evensidemargin}
\addtolength{\leftmargin}{\tmplength}
\settowidth{\labelsep}{X}
\addtolength{\leftmargin}{\labelsep}
\setlength{\labelwidth}{\tmplength}
}
\item[\textbf{Declaration}\hfill]
\ifpdf
\begin{flushleft}
\fi
\begin{ttfamily}
function SCharIs(const S: string; Index: integer; const Chars: TCharSet): boolean; overload;\end{ttfamily}

\ifpdf
\end{flushleft}
\fi

\par
\item[\textbf{Description}]
Comfortable shortcut for Index {$<$}= Length(S) and S[Index] in Chars.

\end{list}
\ifpdf
\subsection*{\large{\textbf{ExtractFirstWord}}\normalsize\hspace{1ex}\hrulefill}
\else
\subsection*{ExtractFirstWord}
\fi
\label{PasDoc_Utils-ExtractFirstWord}
\index{ExtractFirstWord}
\begin{list}{}{
\settowidth{\tmplength}{\textbf{Description}}
\setlength{\itemindent}{0cm}
\setlength{\listparindent}{0cm}
\setlength{\leftmargin}{\evensidemargin}
\addtolength{\leftmargin}{\tmplength}
\settowidth{\labelsep}{X}
\addtolength{\leftmargin}{\labelsep}
\setlength{\labelwidth}{\tmplength}
}
\item[\textbf{Declaration}\hfill]
\ifpdf
\begin{flushleft}
\fi
\begin{ttfamily}
function ExtractFirstWord(var s: string): string; overload;\end{ttfamily}

\ifpdf
\end{flushleft}
\fi

\par
\item[\textbf{Description}]
Extracts all characters up to the first white{-}space encountered (ignoring white{-}space at the very beginning of the string) from the string specified by S.

If there is no white{-}space in S (or there is white{-}space only at the beginning of S, in which case it is ignored) then the whole S is regarded as it's first word.

Both S and result are trimmed, i.e. they don't have any excessive white{-}space at the beginning or end.

\end{list}
\ifpdf
\subsection*{\large{\textbf{ExtractFirstWord}}\normalsize\hspace{1ex}\hrulefill}
\else
\subsection*{ExtractFirstWord}
\fi
\label{PasDoc_Utils-ExtractFirstWord}
\index{ExtractFirstWord}
\begin{list}{}{
\settowidth{\tmplength}{\textbf{Description}}
\setlength{\itemindent}{0cm}
\setlength{\listparindent}{0cm}
\setlength{\leftmargin}{\evensidemargin}
\addtolength{\leftmargin}{\tmplength}
\settowidth{\labelsep}{X}
\addtolength{\leftmargin}{\labelsep}
\setlength{\labelwidth}{\tmplength}
}
\item[\textbf{Declaration}\hfill]
\ifpdf
\begin{flushleft}
\fi
\begin{ttfamily}
procedure ExtractFirstWord(const S: string; out FirstWord, Rest: string); overload;\end{ttfamily}

\ifpdf
\end{flushleft}
\fi

\par
\item[\textbf{Description}]
Another version of ExtractFirstWord.

Splits S by it's first white{-}space (ignoring white{-}space at the very beginning of the string). No such white{-}space means that whole S is regarded as the FirstWord.

Both FirstWord and Rest are trimmed.

\end{list}
\ifpdf
\subsection*{\large{\textbf{FileToString}}\normalsize\hspace{1ex}\hrulefill}
\else
\subsection*{FileToString}
\fi
\label{PasDoc_Utils-FileToString}
\index{FileToString}
\begin{list}{}{
\settowidth{\tmplength}{\textbf{Description}}
\setlength{\itemindent}{0cm}
\setlength{\listparindent}{0cm}
\setlength{\leftmargin}{\evensidemargin}
\addtolength{\leftmargin}{\tmplength}
\settowidth{\labelsep}{X}
\addtolength{\leftmargin}{\labelsep}
\setlength{\labelwidth}{\tmplength}
}
\item[\textbf{Declaration}\hfill]
\ifpdf
\begin{flushleft}
\fi
\begin{ttfamily}
function FileToString(const FileName: string): string;\end{ttfamily}

\ifpdf
\end{flushleft}
\fi

\end{list}
\ifpdf
\subsection*{\large{\textbf{StringToFile}}\normalsize\hspace{1ex}\hrulefill}
\else
\subsection*{StringToFile}
\fi
\label{PasDoc_Utils-StringToFile}
\index{StringToFile}
\begin{list}{}{
\settowidth{\tmplength}{\textbf{Description}}
\setlength{\itemindent}{0cm}
\setlength{\listparindent}{0cm}
\setlength{\leftmargin}{\evensidemargin}
\addtolength{\leftmargin}{\tmplength}
\settowidth{\labelsep}{X}
\addtolength{\leftmargin}{\labelsep}
\setlength{\labelwidth}{\tmplength}
}
\item[\textbf{Declaration}\hfill]
\ifpdf
\begin{flushleft}
\fi
\begin{ttfamily}
procedure StringToFile(const FileName, S: string);\end{ttfamily}

\ifpdf
\end{flushleft}
\fi

\end{list}
\ifpdf
\subsection*{\large{\textbf{DataToFile}}\normalsize\hspace{1ex}\hrulefill}
\else
\subsection*{DataToFile}
\fi
\label{PasDoc_Utils-DataToFile}
\index{DataToFile}
\begin{list}{}{
\settowidth{\tmplength}{\textbf{Description}}
\setlength{\itemindent}{0cm}
\setlength{\listparindent}{0cm}
\setlength{\leftmargin}{\evensidemargin}
\addtolength{\leftmargin}{\tmplength}
\settowidth{\labelsep}{X}
\addtolength{\leftmargin}{\labelsep}
\setlength{\labelwidth}{\tmplength}
}
\item[\textbf{Declaration}\hfill]
\ifpdf
\begin{flushleft}
\fi
\begin{ttfamily}
procedure DataToFile(const FileName: string; const Data: array of Byte);\end{ttfamily}

\ifpdf
\end{flushleft}
\fi

\end{list}
\ifpdf
\subsection*{\large{\textbf{SCharsReplace}}\normalsize\hspace{1ex}\hrulefill}
\else
\subsection*{SCharsReplace}
\fi
\label{PasDoc_Utils-SCharsReplace}
\index{SCharsReplace}
\begin{list}{}{
\settowidth{\tmplength}{\textbf{Description}}
\setlength{\itemindent}{0cm}
\setlength{\listparindent}{0cm}
\setlength{\leftmargin}{\evensidemargin}
\addtolength{\leftmargin}{\tmplength}
\settowidth{\labelsep}{X}
\addtolength{\leftmargin}{\labelsep}
\setlength{\labelwidth}{\tmplength}
}
\item[\textbf{Declaration}\hfill]
\ifpdf
\begin{flushleft}
\fi
\begin{ttfamily}
function SCharsReplace(const S: string; const Chars: TCharSet; ReplacementChar: char): string;\end{ttfamily}

\ifpdf
\end{flushleft}
\fi

\par
\item[\textbf{Description}]
Returns S with all Chars replaced by ReplacementChar

\end{list}
\ifpdf
\subsection*{\large{\textbf{CopyFile}}\normalsize\hspace{1ex}\hrulefill}
\else
\subsection*{CopyFile}
\fi
\label{PasDoc_Utils-CopyFile}
\index{CopyFile}
\begin{list}{}{
\settowidth{\tmplength}{\textbf{Description}}
\setlength{\itemindent}{0cm}
\setlength{\listparindent}{0cm}
\setlength{\leftmargin}{\evensidemargin}
\addtolength{\leftmargin}{\tmplength}
\settowidth{\labelsep}{X}
\addtolength{\leftmargin}{\labelsep}
\setlength{\labelwidth}{\tmplength}
}
\item[\textbf{Declaration}\hfill]
\ifpdf
\begin{flushleft}
\fi
\begin{ttfamily}
procedure CopyFile(const SourceFileName, DestinationFileName: string);\end{ttfamily}

\ifpdf
\end{flushleft}
\fi

\end{list}
\ifpdf
\subsection*{\large{\textbf{IsPrefix}}\normalsize\hspace{1ex}\hrulefill}
\else
\subsection*{IsPrefix}
\fi
\label{PasDoc_Utils-IsPrefix}
\index{IsPrefix}
\begin{list}{}{
\settowidth{\tmplength}{\textbf{Description}}
\setlength{\itemindent}{0cm}
\setlength{\listparindent}{0cm}
\setlength{\leftmargin}{\evensidemargin}
\addtolength{\leftmargin}{\tmplength}
\settowidth{\labelsep}{X}
\addtolength{\leftmargin}{\labelsep}
\setlength{\labelwidth}{\tmplength}
}
\item[\textbf{Declaration}\hfill]
\ifpdf
\begin{flushleft}
\fi
\begin{ttfamily}
function IsPrefix(const Prefix, S: string): boolean;\end{ttfamily}

\ifpdf
\end{flushleft}
\fi

\par
\item[\textbf{Description}]
Checks is Prefix a prefix of S. Not case{-}sensitive.

\end{list}
\ifpdf
\subsection*{\large{\textbf{RemovePrefix}}\normalsize\hspace{1ex}\hrulefill}
\else
\subsection*{RemovePrefix}
\fi
\label{PasDoc_Utils-RemovePrefix}
\index{RemovePrefix}
\begin{list}{}{
\settowidth{\tmplength}{\textbf{Description}}
\setlength{\itemindent}{0cm}
\setlength{\listparindent}{0cm}
\setlength{\leftmargin}{\evensidemargin}
\addtolength{\leftmargin}{\tmplength}
\settowidth{\labelsep}{X}
\addtolength{\leftmargin}{\labelsep}
\setlength{\labelwidth}{\tmplength}
}
\item[\textbf{Declaration}\hfill]
\ifpdf
\begin{flushleft}
\fi
\begin{ttfamily}
function RemovePrefix(const Prefix, S: string): string;\end{ttfamily}

\ifpdf
\end{flushleft}
\fi

\par
\item[\textbf{Description}]
If IsPrefix(Prefix, S), then remove the prefix, otherwise return unmodifed S.

\end{list}
\ifpdf
\subsection*{\large{\textbf{SEnding}}\normalsize\hspace{1ex}\hrulefill}
\else
\subsection*{SEnding}
\fi
\label{PasDoc_Utils-SEnding}
\index{SEnding}
\begin{list}{}{
\settowidth{\tmplength}{\textbf{Description}}
\setlength{\itemindent}{0cm}
\setlength{\listparindent}{0cm}
\setlength{\leftmargin}{\evensidemargin}
\addtolength{\leftmargin}{\tmplength}
\settowidth{\labelsep}{X}
\addtolength{\leftmargin}{\labelsep}
\setlength{\labelwidth}{\tmplength}
}
\item[\textbf{Declaration}\hfill]
\ifpdf
\begin{flushleft}
\fi
\begin{ttfamily}
function SEnding(const s: string; P: integer): string;\end{ttfamily}

\ifpdf
\end{flushleft}
\fi

\par
\item[\textbf{Description}]
SEnding returns S contents starting from position P. Returns '' if P {$>$} length(S). Yes, this is simply equivalent to Copy(S, P, MaxInt).

\end{list}
\ifpdf
\subsection*{\large{\textbf{IsPathAbsolute}}\normalsize\hspace{1ex}\hrulefill}
\else
\subsection*{IsPathAbsolute}
\fi
\label{PasDoc_Utils-IsPathAbsolute}
\index{IsPathAbsolute}
\begin{list}{}{
\settowidth{\tmplength}{\textbf{Description}}
\setlength{\itemindent}{0cm}
\setlength{\listparindent}{0cm}
\setlength{\leftmargin}{\evensidemargin}
\addtolength{\leftmargin}{\tmplength}
\settowidth{\labelsep}{X}
\addtolength{\leftmargin}{\labelsep}
\setlength{\labelwidth}{\tmplength}
}
\item[\textbf{Declaration}\hfill]
\ifpdf
\begin{flushleft}
\fi
\begin{ttfamily}
function IsPathAbsolute(const Path: string): boolean;\end{ttfamily}

\ifpdf
\end{flushleft}
\fi

\par
\item[\textbf{Description}]
Check is the given Path absolute.

Path may point to directory or normal file, it doesn't matter. Also it doesn't matter whether Path ends with PathDelim or not.

Note for Windows: while it's obvious that \begin{ttfamily}'c:{\textbackslash}autoexec.bat'\end{ttfamily} is an absolute path, and \begin{ttfamily}'autoexec.bat'\end{ttfamily} is not, there's a question whether path like \begin{ttfamily}'{\textbackslash}autoexec.bat'\end{ttfamily} is absolute? It doesn't specify drive letter, but it does specify full directory hierarchy on some drive. This function treats this as \textit{not absolute}, on the reasoning that "not all information is contained in Path".

\item[\textbf{See also}]
\begin{description}
\item[\begin{ttfamily}IsPathAbsoluteOnDrive\end{ttfamily}(\ref{PasDoc_Utils-IsPathAbsoluteOnDrive})] 
Just like IsPathAbsolute, but on Windows accepts also paths that specify full directory tree without drive letter.
\end{description}


\end{list}
\ifpdf
\subsection*{\large{\textbf{IsPathAbsoluteOnDrive}}\normalsize\hspace{1ex}\hrulefill}
\else
\subsection*{IsPathAbsoluteOnDrive}
\fi
\label{PasDoc_Utils-IsPathAbsoluteOnDrive}
\index{IsPathAbsoluteOnDrive}
\begin{list}{}{
\settowidth{\tmplength}{\textbf{Description}}
\setlength{\itemindent}{0cm}
\setlength{\listparindent}{0cm}
\setlength{\leftmargin}{\evensidemargin}
\addtolength{\leftmargin}{\tmplength}
\settowidth{\labelsep}{X}
\addtolength{\leftmargin}{\labelsep}
\setlength{\labelwidth}{\tmplength}
}
\item[\textbf{Declaration}\hfill]
\ifpdf
\begin{flushleft}
\fi
\begin{ttfamily}
function IsPathAbsoluteOnDrive(const Path: string): boolean;\end{ttfamily}

\ifpdf
\end{flushleft}
\fi

\par
\item[\textbf{Description}]
Just like IsPathAbsolute, but on Windows accepts also paths that specify full directory tree without drive letter.

\item[\textbf{See also}]
\begin{description}
\item[\begin{ttfamily}IsPathAbsolute\end{ttfamily}(\ref{PasDoc_Utils-IsPathAbsolute})] 
Check is the given Path absolute.
\end{description}


\end{list}
\ifpdf
\subsection*{\large{\textbf{CombinePaths}}\normalsize\hspace{1ex}\hrulefill}
\else
\subsection*{CombinePaths}
\fi
\label{PasDoc_Utils-CombinePaths}
\index{CombinePaths}
\begin{list}{}{
\settowidth{\tmplength}{\textbf{Description}}
\setlength{\itemindent}{0cm}
\setlength{\listparindent}{0cm}
\setlength{\leftmargin}{\evensidemargin}
\addtolength{\leftmargin}{\tmplength}
\settowidth{\labelsep}{X}
\addtolength{\leftmargin}{\labelsep}
\setlength{\labelwidth}{\tmplength}
}
\item[\textbf{Declaration}\hfill]
\ifpdf
\begin{flushleft}
\fi
\begin{ttfamily}
function CombinePaths(BasePath, RelPath: string): string;\end{ttfamily}

\ifpdf
\end{flushleft}
\fi

\par
\item[\textbf{Description}]
Combines BasePath with RelPath. BasePath MUST be an absolute path, on Windows it must contain at least drive specifier (like 'c:'), on Unix it must begin with "/". RelPath can be relative and can be absolute. If RelPath is absolute, result is RelPath. Else the result is an absolute path calculated by combining RelPath with BasePath.

\end{list}
\ifpdf
\subsection*{\large{\textbf{DeleteFileExt}}\normalsize\hspace{1ex}\hrulefill}
\else
\subsection*{DeleteFileExt}
\fi
\label{PasDoc_Utils-DeleteFileExt}
\index{DeleteFileExt}
\begin{list}{}{
\settowidth{\tmplength}{\textbf{Description}}
\setlength{\itemindent}{0cm}
\setlength{\listparindent}{0cm}
\setlength{\leftmargin}{\evensidemargin}
\addtolength{\leftmargin}{\tmplength}
\settowidth{\labelsep}{X}
\addtolength{\leftmargin}{\labelsep}
\setlength{\labelwidth}{\tmplength}
}
\item[\textbf{Declaration}\hfill]
\ifpdf
\begin{flushleft}
\fi
\begin{ttfamily}
function DeleteFileExt(const FileName: string): string;\end{ttfamily}

\ifpdf
\end{flushleft}
\fi

\par
\item[\textbf{Description}]
Remove from the FileName the last extension (including the dot). Note that if the FileName had a couple of extensions (e.g. \begin{ttfamily}blah.x3d.gz\end{ttfamily}) this will remove only the last one. Will remove nothing if filename has no extension.

\end{list}
\ifpdf
\subsection*{\large{\textbf{RemoveIndentation}}\normalsize\hspace{1ex}\hrulefill}
\else
\subsection*{RemoveIndentation}
\fi
\label{PasDoc_Utils-RemoveIndentation}
\index{RemoveIndentation}
\begin{list}{}{
\settowidth{\tmplength}{\textbf{Description}}
\setlength{\itemindent}{0cm}
\setlength{\listparindent}{0cm}
\setlength{\leftmargin}{\evensidemargin}
\addtolength{\leftmargin}{\tmplength}
\settowidth{\labelsep}{X}
\addtolength{\leftmargin}{\labelsep}
\setlength{\labelwidth}{\tmplength}
}
\item[\textbf{Declaration}\hfill]
\ifpdf
\begin{flushleft}
\fi
\begin{ttfamily}
function RemoveIndentation(const Code: string): string;\end{ttfamily}

\ifpdf
\end{flushleft}
\fi

\par
\item[\textbf{Description}]
Remove common indentation (whitespace prefix) from a multiline string.

\end{list}
\ifpdf
\subsection*{\large{\textbf{Swap16Buf}}\normalsize\hspace{1ex}\hrulefill}
\else
\subsection*{Swap16Buf}
\fi
\label{PasDoc_Utils-Swap16Buf}
\index{Swap16Buf}
\begin{list}{}{
\settowidth{\tmplength}{\textbf{Description}}
\setlength{\itemindent}{0cm}
\setlength{\listparindent}{0cm}
\setlength{\leftmargin}{\evensidemargin}
\addtolength{\leftmargin}{\tmplength}
\settowidth{\labelsep}{X}
\addtolength{\leftmargin}{\labelsep}
\setlength{\labelwidth}{\tmplength}
}
\item[\textbf{Declaration}\hfill]
\ifpdf
\begin{flushleft}
\fi
\begin{ttfamily}
procedure Swap16Buf(Src, Dst: PWord; WordCount: Integer);\end{ttfamily}

\ifpdf
\end{flushleft}
\fi

\end{list}
\ifpdf
\subsection*{\large{\textbf{IsCharInSet}}\normalsize\hspace{1ex}\hrulefill}
\else
\subsection*{IsCharInSet}
\fi
\label{PasDoc_Utils-IsCharInSet}
\index{IsCharInSet}
\begin{list}{}{
\settowidth{\tmplength}{\textbf{Description}}
\setlength{\itemindent}{0cm}
\setlength{\listparindent}{0cm}
\setlength{\leftmargin}{\evensidemargin}
\addtolength{\leftmargin}{\tmplength}
\settowidth{\labelsep}{X}
\addtolength{\leftmargin}{\labelsep}
\setlength{\labelwidth}{\tmplength}
}
\item[\textbf{Declaration}\hfill]
\ifpdf
\begin{flushleft}
\fi
\begin{ttfamily}
function IsCharInSet(C: AnsiChar; const CharSet: TCharSet): Boolean; overload; inline;\end{ttfamily}

\ifpdf
\end{flushleft}
\fi

\end{list}
\ifpdf
\subsection*{\large{\textbf{IsCharInSet}}\normalsize\hspace{1ex}\hrulefill}
\else
\subsection*{IsCharInSet}
\fi
\label{PasDoc_Utils-IsCharInSet}
\index{IsCharInSet}
\begin{list}{}{
\settowidth{\tmplength}{\textbf{Description}}
\setlength{\itemindent}{0cm}
\setlength{\listparindent}{0cm}
\setlength{\leftmargin}{\evensidemargin}
\addtolength{\leftmargin}{\tmplength}
\settowidth{\labelsep}{X}
\addtolength{\leftmargin}{\labelsep}
\setlength{\labelwidth}{\tmplength}
}
\item[\textbf{Declaration}\hfill]
\ifpdf
\begin{flushleft}
\fi
\begin{ttfamily}
function IsCharInSet(C: WideChar; const CharSet: TCharSet): Boolean; overload; inline;\end{ttfamily}

\ifpdf
\end{flushleft}
\fi

\end{list}
\ifpdf
\subsection*{\large{\textbf{IsUtf8LeadByte}}\normalsize\hspace{1ex}\hrulefill}
\else
\subsection*{IsUtf8LeadByte}
\fi
\label{PasDoc_Utils-IsUtf8LeadByte}
\index{IsUtf8LeadByte}
\begin{list}{}{
\settowidth{\tmplength}{\textbf{Description}}
\setlength{\itemindent}{0cm}
\setlength{\listparindent}{0cm}
\setlength{\leftmargin}{\evensidemargin}
\addtolength{\leftmargin}{\tmplength}
\settowidth{\labelsep}{X}
\addtolength{\leftmargin}{\labelsep}
\setlength{\labelwidth}{\tmplength}
}
\item[\textbf{Declaration}\hfill]
\ifpdf
\begin{flushleft}
\fi
\begin{ttfamily}
function IsUtf8LeadByte(const B: Byte): Boolean; inline;\end{ttfamily}

\ifpdf
\end{flushleft}
\fi

\end{list}
\ifpdf
\subsection*{\large{\textbf{IsUtf8TrailByte}}\normalsize\hspace{1ex}\hrulefill}
\else
\subsection*{IsUtf8TrailByte}
\fi
\label{PasDoc_Utils-IsUtf8TrailByte}
\index{IsUtf8TrailByte}
\begin{list}{}{
\settowidth{\tmplength}{\textbf{Description}}
\setlength{\itemindent}{0cm}
\setlength{\listparindent}{0cm}
\setlength{\leftmargin}{\evensidemargin}
\addtolength{\leftmargin}{\tmplength}
\settowidth{\labelsep}{X}
\addtolength{\leftmargin}{\labelsep}
\setlength{\labelwidth}{\tmplength}
}
\item[\textbf{Declaration}\hfill]
\ifpdf
\begin{flushleft}
\fi
\begin{ttfamily}
function IsUtf8TrailByte(const B: Byte): Boolean; inline;\end{ttfamily}

\ifpdf
\end{flushleft}
\fi

\end{list}
\ifpdf
\subsection*{\large{\textbf{Utf8Size}}\normalsize\hspace{1ex}\hrulefill}
\else
\subsection*{Utf8Size}
\fi
\label{PasDoc_Utils-Utf8Size}
\index{Utf8Size}
\begin{list}{}{
\settowidth{\tmplength}{\textbf{Description}}
\setlength{\itemindent}{0cm}
\setlength{\listparindent}{0cm}
\setlength{\leftmargin}{\evensidemargin}
\addtolength{\leftmargin}{\tmplength}
\settowidth{\labelsep}{X}
\addtolength{\leftmargin}{\labelsep}
\setlength{\labelwidth}{\tmplength}
}
\item[\textbf{Declaration}\hfill]
\ifpdf
\begin{flushleft}
\fi
\begin{ttfamily}
function Utf8Size(const LeadByte: Byte): Integer; inline;\end{ttfamily}

\ifpdf
\end{flushleft}
\fi

\end{list}
\ifpdf
\subsection*{\large{\textbf{IsLeadChar}}\normalsize\hspace{1ex}\hrulefill}
\else
\subsection*{IsLeadChar}
\fi
\label{PasDoc_Utils-IsLeadChar}
\index{IsLeadChar}
\begin{list}{}{
\settowidth{\tmplength}{\textbf{Description}}
\setlength{\itemindent}{0cm}
\setlength{\listparindent}{0cm}
\setlength{\leftmargin}{\evensidemargin}
\addtolength{\leftmargin}{\tmplength}
\settowidth{\labelsep}{X}
\addtolength{\leftmargin}{\labelsep}
\setlength{\labelwidth}{\tmplength}
}
\item[\textbf{Declaration}\hfill]
\ifpdf
\begin{flushleft}
\fi
\begin{ttfamily}
function IsLeadChar(Ch: WideChar): Boolean; overload; inline;\end{ttfamily}

\ifpdf
\end{flushleft}
\fi

\end{list}
\ifpdf
\subsection*{\large{\textbf{StripHtml}}\normalsize\hspace{1ex}\hrulefill}
\else
\subsection*{StripHtml}
\fi
\label{PasDoc_Utils-StripHtml}
\index{StripHtml}
\begin{list}{}{
\settowidth{\tmplength}{\textbf{Description}}
\setlength{\itemindent}{0cm}
\setlength{\listparindent}{0cm}
\setlength{\leftmargin}{\evensidemargin}
\addtolength{\leftmargin}{\tmplength}
\settowidth{\labelsep}{X}
\addtolength{\leftmargin}{\labelsep}
\setlength{\labelwidth}{\tmplength}
}
\item[\textbf{Declaration}\hfill]
\ifpdf
\begin{flushleft}
\fi
\begin{ttfamily}
function StripHtml(const S: string): string;\end{ttfamily}

\ifpdf
\end{flushleft}
\fi

\par
\item[\textbf{Description}]
Strip HTML elements from the string.

Assumes that the HTML content is correct (all elements are nicely closed, all {$<$} {$>$} inside attributes are escaped to {\&}lt; {\&}gt;, all {$<$} {$>$} outside elements are escaped to {\&}lt; {\&}gt;). It doesn't try very hard to deal with incorrect HTML context (it will not crash, but results are undefined). It's designed to strip HTML from PasDoc{-}generated HTML, which should always be correct.

\end{list}
\ifpdf
\subsection*{\large{\textbf{SAppendPart}}\normalsize\hspace{1ex}\hrulefill}
\else
\subsection*{SAppendPart}
\fi
\label{PasDoc_Utils-SAppendPart}
\index{SAppendPart}
\begin{list}{}{
\settowidth{\tmplength}{\textbf{Description}}
\setlength{\itemindent}{0cm}
\setlength{\listparindent}{0cm}
\setlength{\leftmargin}{\evensidemargin}
\addtolength{\leftmargin}{\tmplength}
\settowidth{\labelsep}{X}
\addtolength{\leftmargin}{\labelsep}
\setlength{\labelwidth}{\tmplength}
}
\item[\textbf{Declaration}\hfill]
\ifpdf
\begin{flushleft}
\fi
\begin{ttfamily}
function SAppendPart(const s, PartSeparator, NextPart: String): String;\end{ttfamily}

\ifpdf
\end{flushleft}
\fi

\par
\item[\textbf{Description}]
If S = '' then returns NextPart, else returns S + PartSeparator + NextPart.

\end{list}
\section{Constants}
\ifpdf
\subsection*{\large{\textbf{AllChars}}\normalsize\hspace{1ex}\hrulefill}
\else
\subsection*{AllChars}
\fi
\label{PasDoc_Utils-AllChars}
\index{AllChars}
\begin{list}{}{
\settowidth{\tmplength}{\textbf{Description}}
\setlength{\itemindent}{0cm}
\setlength{\listparindent}{0cm}
\setlength{\leftmargin}{\evensidemargin}
\addtolength{\leftmargin}{\tmplength}
\settowidth{\labelsep}{X}
\addtolength{\leftmargin}{\labelsep}
\setlength{\labelwidth}{\tmplength}
}
\item[\textbf{Declaration}\hfill]
\ifpdf
\begin{flushleft}
\fi
\begin{ttfamily}
AllChars = [Low(AnsiChar)..High(AnsiChar)];\end{ttfamily}

\ifpdf
\end{flushleft}
\fi

\end{list}
\ifpdf
\subsection*{\large{\textbf{WhiteSpaceNotNL}}\normalsize\hspace{1ex}\hrulefill}
\else
\subsection*{WhiteSpaceNotNL}
\fi
\label{PasDoc_Utils-WhiteSpaceNotNL}
\index{WhiteSpaceNotNL}
\begin{list}{}{
\settowidth{\tmplength}{\textbf{Description}}
\setlength{\itemindent}{0cm}
\setlength{\listparindent}{0cm}
\setlength{\leftmargin}{\evensidemargin}
\addtolength{\leftmargin}{\tmplength}
\settowidth{\labelsep}{X}
\addtolength{\leftmargin}{\labelsep}
\setlength{\labelwidth}{\tmplength}
}
\item[\textbf{Declaration}\hfill]
\ifpdf
\begin{flushleft}
\fi
\begin{ttfamily}
WhiteSpaceNotNL = [' ', {\#}9];\end{ttfamily}

\ifpdf
\end{flushleft}
\fi

\par
\item[\textbf{Description}]
Whitespace that is not any part of newline.

\end{list}
\ifpdf
\subsection*{\large{\textbf{WhiteSpaceNL}}\normalsize\hspace{1ex}\hrulefill}
\else
\subsection*{WhiteSpaceNL}
\fi
\label{PasDoc_Utils-WhiteSpaceNL}
\index{WhiteSpaceNL}
\begin{list}{}{
\settowidth{\tmplength}{\textbf{Description}}
\setlength{\itemindent}{0cm}
\setlength{\listparindent}{0cm}
\setlength{\leftmargin}{\evensidemargin}
\addtolength{\leftmargin}{\tmplength}
\settowidth{\labelsep}{X}
\addtolength{\leftmargin}{\labelsep}
\setlength{\labelwidth}{\tmplength}
}
\item[\textbf{Declaration}\hfill]
\ifpdf
\begin{flushleft}
\fi
\begin{ttfamily}
WhiteSpaceNL = [{\#}10, {\#}13];\end{ttfamily}

\ifpdf
\end{flushleft}
\fi

\par
\item[\textbf{Description}]
Whitespace that is some part of newline.

\end{list}
\ifpdf
\subsection*{\large{\textbf{WhiteSpace}}\normalsize\hspace{1ex}\hrulefill}
\else
\subsection*{WhiteSpace}
\fi
\label{PasDoc_Utils-WhiteSpace}
\index{WhiteSpace}
\begin{list}{}{
\settowidth{\tmplength}{\textbf{Description}}
\setlength{\itemindent}{0cm}
\setlength{\listparindent}{0cm}
\setlength{\leftmargin}{\evensidemargin}
\addtolength{\leftmargin}{\tmplength}
\settowidth{\labelsep}{X}
\addtolength{\leftmargin}{\labelsep}
\setlength{\labelwidth}{\tmplength}
}
\item[\textbf{Declaration}\hfill]
\ifpdf
\begin{flushleft}
\fi
\begin{ttfamily}
WhiteSpace = WhiteSpaceNotNL + WhiteSpaceNL;\end{ttfamily}

\ifpdf
\end{flushleft}
\fi

\par
\item[\textbf{Description}]
Any whitespace (that may indicate newline or not)

\end{list}
\ifpdf
\subsection*{\large{\textbf{FlagStartSigns}}\normalsize\hspace{1ex}\hrulefill}
\else
\subsection*{FlagStartSigns}
\fi
\label{PasDoc_Utils-FlagStartSigns}
\index{FlagStartSigns}
\begin{list}{}{
\settowidth{\tmplength}{\textbf{Description}}
\setlength{\itemindent}{0cm}
\setlength{\listparindent}{0cm}
\setlength{\leftmargin}{\evensidemargin}
\addtolength{\leftmargin}{\tmplength}
\settowidth{\labelsep}{X}
\addtolength{\leftmargin}{\labelsep}
\setlength{\labelwidth}{\tmplength}
}
\item[\textbf{Declaration}\hfill]
\ifpdf
\begin{flushleft}
\fi
\begin{ttfamily}
FlagStartSigns = ['['];\end{ttfamily}

\ifpdf
\end{flushleft}
\fi

\par
\item[\textbf{Description}]
Flag Start{-} and Endsigns for parameters (Feature request "direction of parameter": \href{https://github.com/pasdoc/pasdoc/issues/8}{https://github.com/pasdoc/pasdoc/issues/8})

\end{list}
\ifpdf
\subsection*{\large{\textbf{FlagEndSigns}}\normalsize\hspace{1ex}\hrulefill}
\else
\subsection*{FlagEndSigns}
\fi
\label{PasDoc_Utils-FlagEndSigns}
\index{FlagEndSigns}
\begin{list}{}{
\settowidth{\tmplength}{\textbf{Description}}
\setlength{\itemindent}{0cm}
\setlength{\listparindent}{0cm}
\setlength{\leftmargin}{\evensidemargin}
\addtolength{\leftmargin}{\tmplength}
\settowidth{\labelsep}{X}
\addtolength{\leftmargin}{\labelsep}
\setlength{\labelwidth}{\tmplength}
}
\item[\textbf{Declaration}\hfill]
\ifpdf
\begin{flushleft}
\fi
\begin{ttfamily}
FlagEndSigns = [']'];\end{ttfamily}

\ifpdf
\end{flushleft}
\fi

\end{list}
\section{Authors}
\par
Johannes Berg {$<$}johannes@sipsolutions.de{$>$}

\par
Michalis Kamburelis

\par
Arno Garrels {$<$}first name.name@nospamgmx.de{$>$}

\chapter{Unit PasDoc{\_}Versions}
\label{PasDoc_Versions}
\index{PasDoc{\_}Versions}
\section{Description}
Information about PasDoc and compilers version.
\section{Overview}
\begin{description}
\item[\texttt{COMPILER{\_}NAME}]Nice compiler name.
\item[\texttt{PASDOC{\_}FULL{\_}INFO}]Returns pasdoc name, version, used compiler version, etc.
\end{description}
\section{Functions and Procedures}
\ifpdf
\subsection*{\large{\textbf{COMPILER{\_}NAME}}\normalsize\hspace{1ex}\hrulefill}
\else
\subsection*{COMPILER{\_}NAME}
\fi
\label{PasDoc_Versions-COMPILER_NAME}
\index{COMPILER{\_}NAME}
\begin{list}{}{
\settowidth{\tmplength}{\textbf{Description}}
\setlength{\itemindent}{0cm}
\setlength{\listparindent}{0cm}
\setlength{\leftmargin}{\evensidemargin}
\addtolength{\leftmargin}{\tmplength}
\settowidth{\labelsep}{X}
\addtolength{\leftmargin}{\labelsep}
\setlength{\labelwidth}{\tmplength}
}
\item[\textbf{Declaration}\hfill]
\ifpdf
\begin{flushleft}
\fi
\begin{ttfamily}
function COMPILER{\_}NAME: string;\end{ttfamily}

\ifpdf
\end{flushleft}
\fi

\par
\item[\textbf{Description}]
Nice compiler name. This is a function only because we can't nicely declare it as a constant. But this behaves like a constant, i.e. every time you call it it returns the same thing (as long as this is the same binary).

\end{list}
\ifpdf
\subsection*{\large{\textbf{PASDOC{\_}FULL{\_}INFO}}\normalsize\hspace{1ex}\hrulefill}
\else
\subsection*{PASDOC{\_}FULL{\_}INFO}
\fi
\label{PasDoc_Versions-PASDOC_FULL_INFO}
\index{PASDOC{\_}FULL{\_}INFO}
\begin{list}{}{
\settowidth{\tmplength}{\textbf{Description}}
\setlength{\itemindent}{0cm}
\setlength{\listparindent}{0cm}
\setlength{\leftmargin}{\evensidemargin}
\addtolength{\leftmargin}{\tmplength}
\settowidth{\labelsep}{X}
\addtolength{\leftmargin}{\labelsep}
\setlength{\labelwidth}{\tmplength}
}
\item[\textbf{Declaration}\hfill]
\ifpdf
\begin{flushleft}
\fi
\begin{ttfamily}
function PASDOC{\_}FULL{\_}INFO: string;\end{ttfamily}

\ifpdf
\end{flushleft}
\fi

\par
\item[\textbf{Description}]
Returns pasdoc name, version, used compiler version, etc.

This is a function only because we can't nicely declare it as a constant. But this behaves like a constant, i.e. every time you call it it returns the same thing (as long as this is the same binary).

\end{list}
\section{Constants}
\ifpdf
\subsection*{\large{\textbf{COMPILER{\_}BITS}}\normalsize\hspace{1ex}\hrulefill}
\else
\subsection*{COMPILER{\_}BITS}
\fi
\label{PasDoc_Versions-COMPILER_BITS}
\index{COMPILER{\_}BITS}
\begin{list}{}{
\settowidth{\tmplength}{\textbf{Description}}
\setlength{\itemindent}{0cm}
\setlength{\listparindent}{0cm}
\setlength{\leftmargin}{\evensidemargin}
\addtolength{\leftmargin}{\tmplength}
\settowidth{\labelsep}{X}
\addtolength{\leftmargin}{\labelsep}
\setlength{\labelwidth}{\tmplength}
}
\item[\textbf{Declaration}\hfill]
\ifpdf
\begin{flushleft}
\fi
\begin{ttfamily}
COMPILER{\_}BITS =   '32' ;\end{ttfamily}

\ifpdf
\end{flushleft}
\fi

\end{list}
\ifpdf
\subsection*{\large{\textbf{PASDOC{\_}NAME}}\normalsize\hspace{1ex}\hrulefill}
\else
\subsection*{PASDOC{\_}NAME}
\fi
\label{PasDoc_Versions-PASDOC_NAME}
\index{PASDOC{\_}NAME}
\begin{list}{}{
\settowidth{\tmplength}{\textbf{Description}}
\setlength{\itemindent}{0cm}
\setlength{\listparindent}{0cm}
\setlength{\leftmargin}{\evensidemargin}
\addtolength{\leftmargin}{\tmplength}
\settowidth{\labelsep}{X}
\addtolength{\leftmargin}{\labelsep}
\setlength{\labelwidth}{\tmplength}
}
\item[\textbf{Declaration}\hfill]
\ifpdf
\begin{flushleft}
\fi
\begin{ttfamily}
PASDOC{\_}NAME = 'PasDoc';\end{ttfamily}

\ifpdf
\end{flushleft}
\fi

\end{list}
\ifpdf
\subsection*{\large{\textbf{PASDOC{\_}DATE}}\normalsize\hspace{1ex}\hrulefill}
\else
\subsection*{PASDOC{\_}DATE}
\fi
\label{PasDoc_Versions-PASDOC_DATE}
\index{PASDOC{\_}DATE}
\begin{list}{}{
\settowidth{\tmplength}{\textbf{Description}}
\setlength{\itemindent}{0cm}
\setlength{\listparindent}{0cm}
\setlength{\leftmargin}{\evensidemargin}
\addtolength{\leftmargin}{\tmplength}
\settowidth{\labelsep}{X}
\addtolength{\leftmargin}{\labelsep}
\setlength{\labelwidth}{\tmplength}
}
\item[\textbf{Declaration}\hfill]
\ifpdf
\begin{flushleft}
\fi
\begin{ttfamily}
PASDOC{\_}DATE = '2021-02-07';\end{ttfamily}

\ifpdf
\end{flushleft}
\fi

\par
\item[\textbf{Description}]
Date of last pasdoc release.

We used to have this constant set to CVS/SVN \begin{ttfamily}{\$} Date\end{ttfamily} keyword, but: \begin{itemize}
\item That's not a really correct indication of pasdoc release. \begin{ttfamily}{\$} Date\end{ttfamily} is only the date when this file, \begin{ttfamily}PasDoc{\_}Base.pas\end{ttfamily}, was last modified.

As it happens, always when you make an official release you have to manually change PASDOC{\_}VERSION constant in this file below. So PASDOC{\_}DATE was (at the time when the official release was made) updated to current date. But, since you have to change PASDOC{\_}VERSION constant manually anyway, then it's not much of a problem to also update PASDOC{\_}DATE manually.

For unofficial releases (i.e. when pasdoc is simply compiled from SVN by anyone, or when it's packaged for [\href{https://github.com/pasdoc/pasdoc/wiki/DevelopmentSnapshots}{https://github.com/pasdoc/pasdoc/wiki/DevelopmentSnapshots}]), PASDOC{\_}DATE has no clear meaning. It's not the date of this release (since you don't update the PASDOC{\_}VERSION constant) and it's not the date of last official release (since some commits possibly happened to \begin{ttfamily}PasDoc{\_}Base.pas\end{ttfamily} since last release). 
\item SVN makes this date look bad for the purpose of PASDOC{\_}FULL{\_}INFO. It's too long: contains the time, day of the week, and a descriptive version. Like \begin{verbatim}2006-11-15 07:12:34 +0100 (Wed, 15 Nov 2006)\end{verbatim}

Moreover, it contains indication of local user's system time, and the words (day of the week and month's name) are localized. So it depends on the locale developer has set (you can avoid localization of the words by doing things like \begin{ttfamily}export LANG=C\end{ttfamily} before SVN operations, but it's too error{-}prone). 
\end{itemize}

\end{list}
\ifpdf
\subsection*{\large{\textbf{PASDOC{\_}VERSION}}\normalsize\hspace{1ex}\hrulefill}
\else
\subsection*{PASDOC{\_}VERSION}
\fi
\label{PasDoc_Versions-PASDOC_VERSION}
\index{PASDOC{\_}VERSION}
\begin{list}{}{
\settowidth{\tmplength}{\textbf{Description}}
\setlength{\itemindent}{0cm}
\setlength{\listparindent}{0cm}
\setlength{\leftmargin}{\evensidemargin}
\addtolength{\leftmargin}{\tmplength}
\settowidth{\labelsep}{X}
\addtolength{\leftmargin}{\labelsep}
\setlength{\labelwidth}{\tmplength}
}
\item[\textbf{Declaration}\hfill]
\ifpdf
\begin{flushleft}
\fi
\begin{ttfamily}
PASDOC{\_}VERSION = '0.16.0';\end{ttfamily}

\ifpdf
\end{flushleft}
\fi

\end{list}
\ifpdf
\subsection*{\large{\textbf{PASDOC{\_}NAME{\_}AND{\_}VERSION}}\normalsize\hspace{1ex}\hrulefill}
\else
\subsection*{PASDOC{\_}NAME{\_}AND{\_}VERSION}
\fi
\label{PasDoc_Versions-PASDOC_NAME_AND_VERSION}
\index{PASDOC{\_}NAME{\_}AND{\_}VERSION}
\begin{list}{}{
\settowidth{\tmplength}{\textbf{Description}}
\setlength{\itemindent}{0cm}
\setlength{\listparindent}{0cm}
\setlength{\leftmargin}{\evensidemargin}
\addtolength{\leftmargin}{\tmplength}
\settowidth{\labelsep}{X}
\addtolength{\leftmargin}{\labelsep}
\setlength{\labelwidth}{\tmplength}
}
\item[\textbf{Declaration}\hfill]
\ifpdf
\begin{flushleft}
\fi
\begin{ttfamily}
PASDOC{\_}NAME{\_}AND{\_}VERSION = PASDOC{\_}NAME + ' ' + PASDOC{\_}VERSION;\end{ttfamily}

\ifpdf
\end{flushleft}
\fi

\end{list}
\ifpdf
\subsection*{\large{\textbf{PASDOC{\_}HOMEPAGE}}\normalsize\hspace{1ex}\hrulefill}
\else
\subsection*{PASDOC{\_}HOMEPAGE}
\fi
\label{PasDoc_Versions-PASDOC_HOMEPAGE}
\index{PASDOC{\_}HOMEPAGE}
\begin{list}{}{
\settowidth{\tmplength}{\textbf{Description}}
\setlength{\itemindent}{0cm}
\setlength{\listparindent}{0cm}
\setlength{\leftmargin}{\evensidemargin}
\addtolength{\leftmargin}{\tmplength}
\settowidth{\labelsep}{X}
\addtolength{\leftmargin}{\labelsep}
\setlength{\labelwidth}{\tmplength}
}
\item[\textbf{Declaration}\hfill]
\ifpdf
\begin{flushleft}
\fi
\begin{ttfamily}
PASDOC{\_}HOMEPAGE = 'https://pasdoc.github.io/';\end{ttfamily}

\ifpdf
\end{flushleft}
\fi

\end{list}
\end{document}
